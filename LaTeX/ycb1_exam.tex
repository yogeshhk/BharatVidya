%%%%%%%%%%%%%%%%%%%%%%%%%%%%%%%%%%%%%%%%%%%%%%%%%%%%%%%%%%%%%%%%%%%%%%%%%%%%%%%%%%
\begin{frame}[fragile]\frametitle{}
\begin{center}
{\Large Examination}
\end{center}
\end{frame}


%%%%%%%%%%%%%%%%%%%%%%%%%%%%%%%%%%%%%%%%%%%%%%%%%%%%%%%%%%%%%%%%%%%%%%%%%%%%%%%%%%
\begin{frame}[fragile]\frametitle{}
\begin{center}
{\Large Smayjak Yoga School}
\end{center}
\end{frame}

%%%%%%%%%%%%%%%%%%%%%%%%%%%%%%%%%%%%%%%%%%%%%%%%%%%%%%%%%%%
\begin{frame}[fragile]\frametitle{Instructions}
    \begin{itemize}
        \item Put mat horizontally to be visible for standing as well as horizontal asanas
        \item Keep enrollment number and Aadhar card ready
        \item Good internet connection and good voice quality
        \item Say \textbf{“No”} if you don’t know. Don’t answer anything extra
        \item Stay calm, without any stress
    \end{itemize}
\end{frame}

%%%%%%%%%%%%%%%%%%%%%%%%%%%%%%%%%%%%%%%%%%%%%%%%%%%%%%%%%%%
\begin{frame}[fragile]\frametitle{Preparatory Steps}
    \begin{itemize}
        \item 3 times Omkar (ॐकार), then 3 times ‘Shaanti’ (शान्ति). Sit in any sitting asana, say Sukhasana (सुखासन), eyes closed.
        \item Prarthana (प्रार्थना): \textbf{“गुरुर्ब्रह्मा गुरुर्विष्णुः गुरुर्देवो महेश्वरः”}
        \item Sukshma Vyayam (सूक्ष्म व्यायाम): Greeva Shakti (ग्रीवा शक्ति) (any from I to IV), theory, like who started, his guru.
        \item Sthul Vyayam (स्थूल व्यायाम): (any Hrid (हृदय)/Sarvanga Pushti (सर्वांग पुष्टि))
    \end{itemize}
\end{frame}


%%%%%%%%%%%%%%%%%%%%%%%%%%%%%%%%%%%%%%%%%%%%%%%%%%%%%%%%%%%
\begin{frame}[fragile]\frametitle{Asanas}
    \begin{itemize}
        \item Asana (आसन): Standing (Trikonasana (त्रिकोणासन), hold for 10-20 seconds), do counter posture i.e., do it on the opposite side. Stand in Sama Sthiti (सम स्थिति)
        \item Surya Namaskar (सूर्य नमस्कार), 1 time, meaning 2 times
        \item Shavasana (शवासन) and counterpose Viparita Karani (विपरीत करणी) (legs up, then chest up)
        \item Abdomen Asana: Bhujangasana (भुजंगासन), then rest in Makarasana (मकरासन)
        \item Sitting Asana: Vakrasana (वक्रासन) and counter pose on the opposite side
        \item Meditation Asana, then Pranayama (प्राणायाम), Anuloma Viloma (अनुलोम विलोम)
        \item Bandhas (बंध): Uddiyana Bandha (उड्डीयान बंध) (male), Jalandhara Bandha (जालंधर बंध) (female)
		\item Types of Meditations
			\begin{itemize}
				\item Vipassana (विपस्सना) (by Buddha)
				\item Preksha (प्रेक्षा) by Mahaveer (महावीर)
				\item Body Awareness
			\end{itemize}
    \end{itemize}
		
\end{frame}


%%%%%%%%%%%%%%%%%%%%%%%%%%%%%%%%%%%%%%%%%%%%%%%%%%%%%%%%%%%
\begin{frame}[fragile]\frametitle{Viva}
    \begin{itemize}
        \item Shatkarma (शट्कर्म) (Gherand Samhita (घेरंड संहिता)) in sequence: Dhauti (धौती), Basti (बस्ति), Nauli (नौली), Neti (नेटि), Tratak (त्राटक), Kapalbhati (कपालभाति)
        \item Type of Kapalbhati (or any other Shatkarma): 3 types: Vatakrama (वातकर्मा) (wind cleansing, the usual), Vyutkrama (व्यूत्कर्मा) (sinus cleansing), and Sheetkrama (शीतकर्मा) (mucus cleansing)
        \item Details of Sutra Neti (सूत्र नेति), length of the thread (7-8 inches), precautions to be taken, Mudra (मुद्रा) to be done (Kagasan (कागासन)), breath length differs
        \item Asanas (आसना) for Diabetes Recommended (lower abdomen): Vakrasana (वक्रासन), Mandukasana (मंडूकासन)
        \item Give instructions for Paschimottanasana (पश्चिमोत्तानासन) in the class
        \item Tell benefits, contraindications, and counter posture
    \end{itemize}
\end{frame}

%%%%%%%%%%%%%%%%%%%%%%%%%%%%%%%%%%%%%%%%%%%%%%%%%%%%%%%%%%%%%%%%%%%%%%%%%%%%%%%%%%
\begin{frame}[fragile]\frametitle{}
\begin{center}
{\Large Yog Aurora}
\end{center}
\end{frame}

%%%%%%%%%%%%%%%%%%%%%%%%%%%%%%%%%%%%%%%%%%%%%%%%%%%%%%%%%%%%%%%%%%%%%%%%%%%%%%%%%%
\begin{frame}[fragile]\frametitle{Question Topics and Expectations}
    \begin{itemize}
        \item \textbf{Prayer}
        \begin{itemize}
            \item Recite Yoga Prayers like Patanjali Prayer, Shanti Mantra.
            \item Understanding of Prayer and some knowledge on background/history.
        \end{itemize}
        \item \textbf{Cleansing Techniques}
        \begin{itemize}
            \item Should know complete details on Dhauti (धौती) and Neti (नेटि) - Types, Process, Benefits, Contradictions.
            \item Should be able to perform and show Kapalabhati (कपालभाति), and know the details as above.
        \end{itemize}
        \item \textbf{Sukshma Vyayama}
        \begin{itemize}
            \item Should know Sanskrit names of movements like Neck movement is called \textit{Griva Shakti Vikasaka} (ग्रीवा शक्ति विकासक) and types of all as mentioned in syllabus.
            \item Should be able to perform all mentioned Sukshma Vyayama (सूक्ष्म व्यायाम).
            \item Should understand - Pros, Cons.
        \end{itemize}
        \item \textbf{Sthula Vyayama}
        \begin{itemize}
            \item Should know Sanskrit names of movements.
            \item Should be able to perform all mentioned Sthula Vyayama (स्थूल व्यायाम).
            \item Should understand - Pros, Cons.
        \end{itemize}
        \item \textbf{Surya Namaskar}
        \begin{itemize}
            \item Expected to know Asanas (आसना) names, sequence.
            \item Benefits - Physical and Psychological Level, Contradictions.
            \item Should be able to perform (in their own capacity).
        \end{itemize}
    \end{itemize}
\end{frame}


%%%%%%%%%%%%%%%%%%%%%%%%%%%%%%%%%%%%%%%%%%%%%%%%%%%%%%%%%%%%%%%%%%%%%%%%%%%%%%%%%%
\begin{frame}[fragile]\frametitle{Question Topics and Expectations}
    \begin{itemize}
        \item \textbf{Asanas}
        \begin{itemize}
            \item You should know about all Asanas (आसना) mentioned in syllabus - Technique, Benefits, and Contradictions.
            \item Perform 2 Asanas of Examiner's choice - Only Perform.
            \item 1 Asana of your own choice - Perform and Instruct as well.
            \item Sanskrit names of Asanas - Logic behind names.
        \end{itemize}
        \item \textbf{Breathing Practices}
        \begin{itemize}
            \item You should know 3 types of sectional breathing - Technique, Benefits, Contradictions.
            \item Should be able to perform and instruct.
            \item Should know the pattern followed in Yogic Breathing, Advantages.
        \end{itemize}

    \end{itemize}
\end{frame}

%%%%%%%%%%%%%%%%%%%%%%%%%%%%%%%%%%%%%%%%%%%%%%%%%%%%%%%%%%%%%%%%%%%%%%%%%%%%%%%%%%
\begin{frame}[fragile]\frametitle{Question Topics and Expectations}
    \begin{itemize}
        \item \textbf{Pranayama}
        \begin{itemize}
            \item You should know the meaning of \textit{Puraka} (पूरक), \textit{Rechaka} (रेचक), and \textit{Kumbhaka} (कुम्भक).
            \item Should know how to perform \textit{Anulom Vilom Pranayama} (अनुलोम विलोम प्राणायाम) - Hand Mudra, Technique, Benefits, Contradictions.
            \item How many breaths for beginners: \textit{Puraka:Rechaka} (Ratio).
            \item \textit{Nadi Shodhana} (नाड़ी शोधन) - \textit{Puraka:Rechaka:Kumbhaka} - Hand Mudra, Technique, Benefits, and Contradictions.
            \item \textit{Sheetali} (शीतालि) - Perform without Kumbhaka. However, should know about Kumbhaka technique - Benefits and Contradictions.
            \item \textit{Bhramari} (भ्रामरी) - Perform without Kumbhaka. However, should know about Kumbhaka technique - Benefits and Contradictions.
            \item From theory syllabus - How many types of Pranayama.
            \item Hatha Yoga, Gheranda Samhita (घेरंड संहिता) - Which Pranayama are mentioned.
            \item Pranayama for particular diseases/particular age-groups etc.
        \end{itemize}		
    \end{itemize}
\end{frame}



%%%%%%%%%%%%%%%%%%%%%%%%%%%%%%%%%%%%%%%%%%%%%%%%%%%%%%%%%%%%%%%%%%%%%%%%%%%%%%%%%%
\begin{frame}[fragile]\frametitle{Question Topics and Expectations}
    \begin{itemize}
        \item \textbf{Bandha}
        \begin{itemize}
            \item You should be able to perform all 3 \textbf{Bandhas} (बन्ध) - Should be able to Instruct.
            \item Understand technique, benefits, and contradictions.
            \item Theory - Hatha Yoga Pradipika (हठ योग प्रदीपिका) how many Bandhas are there.
            \item Logic behind Sanskrit names of Bandhas.
            \item On which \textbf{Chakras} (चक्र) these Bandhas are working.
        \end{itemize}
		
        \item \textbf{Mudras}
        \begin{itemize}
            \item What are \textbf{Mudras} (मुद्रा), How do they work?
            \item Should be able to show mentioned Mudras, know benefits of them.
            \item Mudras during sectional breathing and their importance.
        \end{itemize}
        
        \item \textbf{Meditation}
        \begin{itemize}
            \item Application of knowledge-based questions - expected to know different techniques of meditations.
            \item 8 limbs of Yoga - \textit{Dhyana} (ध्यान), \textit{Dharana} (धारणा), \textit{Samadhi} (समाधि).
            \item \textit{Pratyahara} (प्रत्याहार).
            \item You should know stages of \textit{Yoga Nidra} (योग निद्रा). Sequence, benefits, contradictions (anxiety, High BP).
        \end{itemize}
    \end{itemize}
\end{frame}


%%%%%%%%%%%%%%%%%%%%%%%%%%%%%%%%%%%%%%%%%%%%%%%%%%%%%%%%%%%%%%%%%%%%%%%%%%%%%%%%%%
\begin{frame}[fragile]\frametitle{Question Topics and Expectations}
    \begin{itemize}
        \item \textbf{YOGA SESSIONS}
        \begin{itemize}
            \item Expected to know how to teach/communicate effectively.
            \item How to plan Yoga sessions for \textbf{Disabled} (विकलांग), \textbf{Sick} (बीमार) - Particular ailments such as \textbf{Diabetes} (मधुमेह), \textbf{BP} (रक्तचाप), \textbf{Cholesterol} (कोलेस्ट्रॉल), \textbf{Back Pain} (कमर दर्द).
            \item Yoga session for \textbf{Students} (छात्र), \textbf{Ladies} (महिलाएँ), etc. (Different categories).
        \end{itemize}
        \item \textbf{THEORY SYLLABUS}
        \begin{itemize}
            \item Expected to know what all is covered in the theory syllabus.
            \item Recite \textbf{Yoga Sutras} (योग सूत्र) (1-12), Recite \textbf{Bhagavad Gita} (भगवद गीता) Shloka 70 and its meaning. What do you understand from this.
            \item Yoga Definitions - \textbf{Yoga Vashishta} (योग वाशिष्ठ), \textbf{Patanjali} (पतंजलि), \textbf{Bhagavad Gita} (भगवद गीता).
            \item What is \textbf{Hatha Yoga Pradipika} (हठ योग प्रदीपिका), how many chapters are there in \textbf{HYP}. What is \textbf{Gheranda Samhita} (घेरंड संहिता).
            \item Concept of \textbf{Chakras} (चक्र).
            \item \textbf{Panchkosha} (पंचकोश)? \textbf{Tridosha} (त्रिदोष)?
        \end{itemize}
    \end{itemize}
\end{frame}

