%%%%%%%%%%%%%%%%%%%%%%%%%%%%%%%%%%%%%%%%%%%%%%%%%%%%%%%%%%%%%%%%%%%%%%%%%%%%%%%%%%
\begin{frame}[fragile]\frametitle{}
\begin{center}
{\Large Yoganidra Course by Shrimath  -  Krishna Prakash}
\end{center}
\end{frame}

%%%%%%%%%%%%%%%%%%%%%%%%%%%%%%%%%%%%%%%%%%%%%%%%%%%%%%%%%%%%%%%%%%%%%%%%%%%%%%%%%%
\begin{frame}[fragile]\frametitle{}
\begin{center}
{\Large Day 1}
\end{center}
\end{frame}

%%%%%%%%%%%%%%%%%%%%%%%%%%%%%%%%%%%%%%%%%%%%%%%%%%%%%%%%%%%
\begin{frame}[fragile]\frametitle{Yoga Nidra  योगनिद्रा  -  A Tool for Self-Discovery}
      \begin{itemize}
        \item As per Swami Satyananda स्वामी सत्यानंद , Yoganidra is the first step towards transcendence (Samadhi समाधी )
		\item Yoganidra is mentioned in Yoga-Taravali योग तारावली  by Adi Shankara आदी शंकर 	  
        \item Helps navigate day-to-day challenges and seek the constant.
        \item Assists in self-inquiry - ``Who am I?'' and ``Why am I doing what I do?''.
        \item Bridges the material world (Mahamaya महामाया ) and pure consciousness.
        \item Supports achieving goals and ultimate self-realization, which is knowing who we are.
      \end{itemize}
\end{frame}

%%%%%%%%%%%%%%%%%%%%%%%%%%%%%%%%%%%%%%%%%%%%%%%%%%%%%%%%%%%
\begin{frame}[fragile]\frametitle{Aastika Darshana आस्तिक दर्शन  - The Knowledge Tradition}
      \begin{itemize}
        \item Aastika आस्तिक does not mean belief in God but belief in knowledge \& tradition.
        \item Knowledge (Veda वेद ) is born with creation.
        \item Veda is not a textbook but a body of knowledge passed through tradition.
        \item Pursuit of knowledge transcends caste, creed, and religion.
      \end{itemize}
\end{frame}

%%%%%%%%%%%%%%%%%%%%%%%%%%%%%%%%%%%%%%%%%%%%%%%%%%%%%%%%%%%
\begin{frame}[fragile]\frametitle{Trust in Tradition}
      \begin{itemize}
        \item When in doubt, trust the tradition of teachers, then
you are Aastika.
        \item A guide who has already experienced it can lead us beyond our intellect.
        \item Yoga Nidra belongs to the Aastika Darshana, derived from Tantra तंत्र , not classical Yoga.
      \end{itemize}
\end{frame}

%%%%%%%%%%%%%%%%%%%%%%%%%%%%%%%%%%%%%%%%%%%%%%%%%%%%%%%%%%%
\begin{frame}[fragile]\frametitle{Yoga Sutra योगसूत्र  \& Living Traditions}
      \begin{itemize}
        \item Yoga Sutra does not outline processes beyond meditation on OM.
        \item Processes are determined by the living traditions of the era.
        \item The goal is to reach the root of why one seeks Yoga Nidra.
      \end{itemize}
\end{frame}

%%%%%%%%%%%%%%%%%%%%%%%%%%%%%%%%%%%%%%%%%%%%%%%%%%%%%%%%%%%
\begin{frame}[fragile]\frametitle{Journey of Self-Discovery}
      \begin{itemize}
        \item ``Who am I?'' is a personal journey; no need for external validation.
        \item Understand yourself first; do not expect others to empathize.
        \item Doubts are welcome, but resolve them through understanding concepts.
      \end{itemize}
\end{frame}

%%%%%%%%%%%%%%%%%%%%%%%%%%%%%%%%%%%%%%%%%%%%%%%%%%%%%%%%%%%
\begin{frame}[fragile]\frametitle{Understanding Desire}
      \begin{itemize}
        \item Desire drives daily actions;  an emotion is energy in motion, similarly a thought that propels action is desire.
        \item Life is a series of desires - understanding them is crucial.
        \item Automated living leads to stress, insomnia, and rage.
        \item Exercise: Write a list of desires; do not share or judge them.
      \end{itemize}
\end{frame}

%%%%%%%%%%%%%%%%%%%%%%%%%%%%%%%%%%%%%%%%%%%%%%%%%%%%%%%%%%%
\begin{frame}[fragile]\frametitle{Tantra तंत्र - Channelizing Desires}
      \begin{itemize}
        \item Tantra provides methods to channel desires, not suppress them.
        \item Use Dharma धर्म  as a filter to refine desires.
        \item Infinite desires exist but can be classified into four categories (Purushartha पुरुषार्थ).
      \end{itemize}
\end{frame}

%%%%%%%%%%%%%%%%%%%%%%%%%%%%%%%%%%%%%%%%%%%%%%%%%%%%%%%%%%%
\begin{frame}[fragile]\frametitle{Purushartha - The Four Desires}
Buckets for ideas, desires , goals:
      \begin{itemize}
        \item Dharma धर्म : Role clarity, duties, righteousness, understanding right or wrong.
        \item Artha अर्थ : Wealth generation, using intellect for decisions. This appreciates with time. You use mind to take this type of decision. Need to generate wealth
for the other three categories (bramacharya ब्रह्मचर्य , vanaprastha वानप्रस्थ , sanyasa सन्यास). Getting fooled to think if
you are spiritual you should be poor these are all pedal ideas don't believe that. Health is probably wealth (wellness because of it appreciates with time)
        \item Kaama काम : Sense pleasures, using body \& senses for decisions. This depreciates with time.
        \item Moksha मोक्ष : Ultimate self-realization, the common goal. We can keep it aside as one common desire out of infinite desires. So all remaining desires can now be classified into remaining 3 categories. That's the authority of Indian Traditions.
        \item Rule: Artha \& Kaama are valid if aligned with Dharma.
      \end{itemize}
\end{frame}

%%%%%%%%%%%%%%%%%%%%%%%%%%%%%%%%%%%%%%%%%%%%%%%%%%%%%%%%%%%
\begin{frame}[fragile]\frametitle{Antahkarana अंत : करण - The Inner Instrument}

Where does the thinking/processing is happening?: Antahkarana अंत : करण (inner instrument, 'M'ind), works in 4 different modes:
      \begin{itemize}
        \item Manas मनस (Mind): Collects thoughts, generates desires.
        \item Buddhi बुद्धी (Intellect): Decision making, applying filters.
        \item Chitta चित्त (Memory): Storage of past experiences.
        \item Ahamkara अहंकार (Ego): Self-identity, action initiator. Self Arrogating Principle, helps us take action. Inferiority \& superiority complexes stem from a sense of lack. 
        \item Ego should be used mindfully to implement what is intellectually right.
        \item Willpower plays a key role in executing our decisions.
        \item Balance between intellect and action is necessary for growth.
      \end{itemize}
\end{frame}

%%%%%%%%%%%%%%%%%%%%%%%%%%%%%%%%%%%%%%%%%%%%%%%%%%%%%%%%%%%
\begin{frame}[fragile]\frametitle{Processes for Inner Clarity}
      \begin{itemize}
        \item Yoga Nidra योगनिद्रा : Calms the mind and aids goal realization.
        \item Antarmouna अंतरमौन : Inner silence practice for self-reflection.
        \item Bhramari Pranayama भ्रामरी प्राणायाम : Breathing technique for calming the mind.
        \item Mantra Sadhana मंत्र साधना : Chanting practice to enhance focus and awareness.
      \end{itemize}
\end{frame}

%%%%%%%%%%%%%%%%%%%%%%%%%%%%%%%%%%%%%%%%%%%%%%%%%%%%%%%%%%%%%%%%%%%%%%%%%%%%%%%%%%
\begin{frame}[fragile]\frametitle{}
\begin{center}
{\Large Day 2}
\end{center}
\end{frame}
