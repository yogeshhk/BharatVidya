%%%%%%%%%%%%%%%%%%%%%%%%%%%%%%%%%%%%%%%%%%%%%%%%%%%%%%%%%%%%%%%%%%%%%%%%%%%%%%%%%%
\begin{frame}[fragile]\frametitle{}
\begin{center}
{\Large Introduction to Upanishada}
\end{center}
\end{frame}

%%%%%%%%%%%%%%%%%%%%%%%%%%%%%%%%%%%%%%%%%%%%%%%%%%%%%%%%%%%%%%%%%%%%%%%%%%%%%%%%%%
\begin{frame}[fragile]\frametitle{What are the Upanishads?}
    \begin{itemize}
        \item The Upanishads are a collection of ancient Sanskrit texts
        \item They are considered the culmination of the Vedic scriptural tradition
        \item Upanishads are philosophical treatises that explore the nature of reality and the human condition
        \item They are considered the most important part of the Vedas, the ancient scriptures of Hinduism
        \item The Upanishads are believed to have been composed between 800 BCE and 200 CE
    \end{itemize}
\end{frame}

%%%%%%%%%%%%%%%%%%%%%%%%%%%%%%%%%%%%%%%%%%%%%%%%%%%%%%%%%%%%%%%%%%%%%%%%%%%%%%%%%%
\begin{frame}[fragile]\frametitle{Key Themes and Concepts}
    \begin{itemize}
        \item The concept of Brahman: the ultimate, transcendent reality
        \item The idea of Atman: the individual soul or self
        \item The relationship between Brahman and Atman
        \item The search for self-knowledge and spiritual enlightenment
        \item The theory of Karma and the cycle of rebirth (Samsara)
        \item The importance of meditation and contemplation
        \item The rejection of rituals and external religious practices
    \end{itemize}
\end{frame}

%%%%%%%%%%%%%%%%%%%%%%%%%%%%%%%%%%%%%%%%%%%%%%%%%%%%%%%%%%%%%%%%%%%%%%%%%%%%%%%%%%
\begin{frame}[fragile]\frametitle{Major Upanishads}
    \begin{itemize}
        \item The Isha Upanishad
        \item The Kena Upanishad
        \item The Katha Upanishad
        \item The Mundaka Upanishad
        \item The Chandogya Upanishad
        \item The Brihadaranyaka Upanishad
        \item The Taittiriya Upanishad
        \item The Aitareya Upanishad
        \item The Svetasvatara Upanishad
    \end{itemize}
\end{frame}

%%%%%%%%%%%%%%%%%%%%%%%%%%%%%%%%%%%%%%%%%%%%%%%%%%%%%%%%%%%%%%%%%%%%%%%%%%%%%%%%%%
\begin{frame}[fragile]\frametitle{Influence and Legacy}
    \begin{itemize}
        \item The Upanishads have had a profound influence on Indian philosophy and spirituality
        \item They have shaped the development of Vedanta, one of the six major schools of Hindu philosophy
        \item The ideas and concepts of the Upanishads have been widely studied and interpreted by scholars and spiritual leaders
        \item The Upanishads have also influenced other Asian philosophies, such as Taoism and Buddhism
        \item They continue to be an important source of inspiration and study for Hindus and others interested in Indian spiritual traditions
    \end{itemize}
\end{frame}
