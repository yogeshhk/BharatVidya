%%%%%%%%%%%%%%%%%%%%%%%%%%%%%%%%%%%%%%%%%%%%%%%%%%%%%%%%%%%%%%%%%%%%%%%%%%%%%%%%%%
\begin{frame}[fragile]\frametitle{}
\begin{center}
{\Large Introduction to Yogashastra}
\end{center}
\end{frame}

%%%%%%%%%%%%%%%%%%%%%%%%%%%%%%%%%%%%%%%%%%%%%%%%%%%%%%%%%%%%%%%%%%%%%%%%%%%%%%%%%%
\begin{frame}[fragile]\frametitle{What is Yogashastra?}
    \begin{itemize}
        \item Yogashastra refers to the ancient Indian system of yoga, which encompasses a wide range of physical, mental, and spiritual practices
        \item It is a comprehensive body of knowledge and teachings that have been preserved and transmitted through various textual and oral traditions
        \item Yogashastra is considered one of the six orthodox schools of Hindu philosophy, known as the Shad-Darshana
        \item The core of Yogashastra is the practice and mastery of yoga, which aims to achieve physical, mental, and spiritual well-being
    \end{itemize}
\end{frame}

%%%%%%%%%%%%%%%%%%%%%%%%%%%%%%%%%%%%%%%%%%%%%%%%%%%%%%%%%%%%%%%%%%%%%%%%%%%%%%%%%%
\begin{frame}[fragile]\frametitle{Philosophical Foundations}
    \begin{itemize}
        \item The foundation of Yogashastra is the Yoga Sutras, attributed to the sage Patanjali
        \item The Yoga Sutras outline the eight limbs of yoga, known as Ashtanga Yoga, which include ethical principles, physical postures, breath control, and meditation
        \item Yogashastra is based on the principles of Samkhya philosophy, which provides a metaphysical framework for understanding the nature of reality and the human condition
        \item Key concepts in Yogashastra include Purusha (the absolute, eternal self) and Prakriti (the primordial matter or nature)
    \end{itemize}
\end{frame}

%%%%%%%%%%%%%%%%%%%%%%%%%%%%%%%%%%%%%%%%%%%%%%%%%%%%%%%%%%%%%%%%%%%%%%%%%%%%%%%%%%
\begin{frame}[fragile]\frametitle{Major Branches and Practices}
    \begin{itemize}
        \item Hatha Yoga: Focuses on physical postures (asanas), breath control (pranayama), and purification techniques
        \item Raja Yoga: Emphasizes the mental and spiritual aspects of yoga, such as meditation and the control of the mind
        \item Karma Yoga: Involves the cultivation of selfless action and the integration of spiritual wisdom into daily life
        \item Bhakti Yoga: Emphasizes the path of devotion and the cultivation of love and devotion towards the divine
        \item Jnana Yoga: Focuses on the path of knowledge and the attainment of self-realization through philosophical inquiry
    \end{itemize}
\end{frame}

%%%%%%%%%%%%%%%%%%%%%%%%%%%%%%%%%%%%%%%%%%%%%%%%%%%%%%%%%%%%%%%%%%%%%%%%%%%%%%%%%%
\begin{frame}[fragile]\frametitle{Texts and Traditions}
    \begin{itemize}
        \item Key texts in Yogashastra include the Yoga Sutras, the Bhagavad Gita, the Upanishads, and various Tantric and Puranic texts
        \item Yogashastra has been transmitted through diverse lineages and schools, each with its own unique emphasis and practices
        \item Prominent yoga masters and teachers, such as Patanjali, Swami Vivekananda, and B.K.S. Iyengar, have contributed to the evolution and dissemination of Yogashastra
        \item The practice and study of Yogashastra continue to be an integral part of the spiritual and cultural landscape of India
    \end{itemize}
\end{frame}

%%%%%%%%%%%%%%%%%%%%%%%%%%%%%%%%%%%%%%%%%%%%%%%%%%%%%%%%%%%%%%%%%%%%%%%%%%%%%%%%%%
\begin{frame}[fragile]\frametitle{Global Influence and Contemporary Relevance}
    \begin{itemize}
        \item Yogashastra has had a significant impact on the global health and wellness industry, with the widespread adoption of yoga and meditation practices
        \item The philosophical and practical insights of Yogashastra have been integrated into various fields, such as psychology, neuroscience, and holistic medicine
        \item Contemporary practitioners and scholars continue to explore the relevance and applicability of Yogashastra in the modern context, adapting its teachings to address contemporary needs and challenges
        \item The preservation and transmission of Yogashastra's ancient wisdom remain essential for maintaining the rich heritage of Indian spiritual and cultural traditions
    \end{itemize}
\end{frame}
