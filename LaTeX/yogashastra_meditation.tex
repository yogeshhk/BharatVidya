%%%%%%%%%%%%%%%%%%%%%%%%%%%%%%%%%%%%%%%%%%%%%%%%%%%%%%%%%%%%%%%%
\begin{frame}[fragile]\frametitle{}
\begin{center}
{\Large Learning Meditation}


{\tiny (Based on Rakesh Deshpande's Anahat Meditation Course)}
\end{center}
\end{frame}

%%%%%%%%%%%%%%%%%%%%%%%%%%%%%%%%%%%%%%%%%%%%%%%%%%%%%%%%%%%
\begin{frame}[fragile]\frametitle{Day 1: Myths of Meditation}
      \begin{itemize}
        \item Thoughts will always arise in meditation; do not resist or force them away.
        \item The aim of meditation is a thoughtless state, which comes naturally.
        \item Being thoughtless is not a process, but a result of right conditions.
        \item Like sowing a seed and watering, thoughtlessness arises automatically.
        \item A calm mind and a thoughtless mind are not the same.
        \item Calmness can remain throughout the day, thoughtlessness is temporary.
        \item Mental peace means less chatter, fewer unwanted and negative thoughts.
        \item Overthinking causes stress and poor performance; inner peace solves it.
      \end{itemize}
\end{frame}

%%%%%%%%%%%%%%%%%%%%%%%%%%%%%%%%%%%%%%%%%%%%%%%%%%%%%%%%%%%
\begin{frame}[fragile]\frametitle{Guided vs Real Meditation}
      \begin{itemize}
        \item Guided meditation or music meditation is not real meditation.
        \item True meditation means turning awareness inward, not following external cues.
        \item External instructions keep the mind engaged, not silent.
        \item Real meditation involves focusing on breath, heartbeat, or subtle sounds.
        \item Guided meditation is relaxation, useful for freshness but not meditation.
        \item Yoga Nidra is relaxation under \textit{pratyahara}, not \textit{dharana-dhyaan-samadhi}.
      \end{itemize}
\end{frame}

%%%%%%%%%%%%%%%%%%%%%%%%%%%%%%%%%%%%%%%%%%%%%%%%%%%%%%%%%%%
\begin{frame}[fragile]\frametitle{False Notions of Meditation}
      \begin{itemize}
        \item Walking, jogging, or driving meditation is a misconception.
        \item These are absent-minded activities or flow states, not meditation.
        \item Meditation requires sitting still at one place with focus.
        \item Aim is to erase psychological patterns, not escape reality.
        \item A fixed time and place strengthen meditation practice.
        \item Sleepiness during meditation indicates lack of proper rest.
        \item 7–8 hours of sound sleep is essential for effective meditation.
      \end{itemize}
\end{frame}

%%%%%%%%%%%%%%%%%%%%%%%%%%%%%%%%%%%%%%%%%%%%%%%%%%%%%%%%%%%
\begin{frame}[fragile]\frametitle{Day 1 Exercise}
      \begin{itemize}
        \item Practice meditation for 20 minutes daily.
        \item Simply observe sensations through all senses.
        \item Notice touch, taste, sight, smell, and sound slowly.
        \item Keep eyes open, do not move or count time.
        \item Avoid mixing with other methods or distractions.
        \item Remain steady and patient throughout the session.
      \end{itemize}
\end{frame}

%%%%%%%%%%%%%%%%%%%%%%%%%%%%%%%%%%%%%%%%%%%%%%%%%%%%%%%%%%%
\begin{frame}[fragile]\frametitle{Day 2: When is really a Meditation?}
      \begin{itemize}
        \item Goal is a thoughtless mind, achieved naturally.
        \item Do not struggle with the mind directly; find the “switch”.
        \item Non-material thoughts are linked to material breath.
        \item Work on the breath to reduce mental activity.
        \item Like idol worship, breath is a gateway to inner silence.
      \end{itemize}
\end{frame}

%%%%%%%%%%%%%%%%%%%%%%%%%%%%%%%%%%%%%%%%%%%%%%%%%%%%%%%%%%%
\begin{frame}[fragile]\frametitle{Breath and Meditation}
      \begin{itemize}
        \item Trial method: sit straight, chant Om mentally 5 times.
        \item Synchronize breath with Omkar length for calmness.
        \item Slowing the breath leads to slowing thoughts.
        \item Stillness of the body helps in slowing breath naturally.
        \item Reverse approach: still body $\rightarrow$ slow breath $\rightarrow$ thoughtlessness.
      \end{itemize}
\end{frame}

%%%%%%%%%%%%%%%%%%%%%%%%%%%%%%%%%%%%%%%%%%%%%%%%%%%%%%%%%%%
\begin{frame}[fragile]\frametitle{Shunyavastha – The Zero State}
      \begin{itemize}
        \item During meditation, breath may stop briefly and mind becomes thoughtless.
        \item This state is called \textit{Shunyavastha} – complete emptiness.
        \item Awareness of this state is subtle and paradoxical.
        \item Beginners experience it for seconds; practice extends duration.
        \item Short glimpses of Shunyavastha bring deep mental peace.
        \item True enlightenment comes when awareness and thoughtlessness merge.
      \end{itemize}
\end{frame}

%%%%%%%%%%%%%%%%%%%%%%%%%%%%%%%%%%%%%%%%%%%%%%%%%%%%%%%%%%%
\begin{frame}[fragile]\frametitle{Meditation vs Sleep}
      \begin{itemize}
        \item Shunyavastha is similar to sleep, but with awareness.
        \item In sleep, body and breath continue but awareness is absent.
        \item Meditation is conscious entry into stillness, not unconscious rest.
        \item Body control is essential in meditation, unlike in sleep.
        \item Mudras are primarily for maintaining body steadiness.
        \item Stable posture: “Sthiram Sukham Asanam” – steady and comfortable seat.
      \end{itemize}
\end{frame}

%%%%%%%%%%%%%%%%%%%%%%%%%%%%%%%%%%%%%%%%%%%%%%%%%%%%%%%%%%%
\begin{frame}[fragile]\frametitle{Broader Yogic Approaches}
      \begin{itemize}
        \item Ashtanga Yoga by Patanjali: mental and spiritual upliftment.
        \item Focuses on meditation stages, not much on physical asanas.
        \item Emphasis on stillness of body to prepare for dhyaan.
        \item Hatha Yoga by Swatmaram (14th century): physical practices.
        \item Includes asanas, mudras, bandhas for energy control.
        \item Both approaches complement each other in the yogic path.
      \end{itemize}
\end{frame}

%%%%%%%%%%%%%%%%%%%%%%%%%%%%%%%%%%%%%%%%%%%%%%%%%%%%%%%%%%%
\begin{frame}[fragile]\frametitle{Day 2 Exercise}
      \begin{itemize}
        \item Sit straight, eyes open, without back support.
        \item Keep body still for 20 minutes.
        \item Allow only eyelid movement and natural swallowing.
        \item Do not scratch or adjust posture despite minor discomforts.
        \item Observe breath and sensations quietly.
        \item This stillness creates the condition for real meditation.
      \end{itemize}
\end{frame}


%%%%%%%%%%%%%%%%%%%%%%%%%%%%%%%%%%%%%%%%%%%%%%%%%%%%%%%%%%%
\begin{frame}[fragile]\frametitle{Day 3: Obstacles in Meditation}
      \begin{itemize}
        \item Recap: Sit still $\rightarrow$ breath slows $\rightarrow$ mind becomes thoughtless.
        \item Thoughts will arise anyway; do not resist them.
        \item Becoming thoughtless takes time and steady practice.
        \item Like building the body, results appear gradually.
        \item Benefits start from day one: relaxation, focus, calmness.
      \end{itemize}
\end{frame}

%%%%%%%%%%%%%%%%%%%%%%%%%%%%%%%%%%%%%%%%%%%%%%%%%%%%%%%%%%%
\begin{frame}[fragile]\frametitle{Changing Nature of Experience}
      \begin{itemize}
        \item Meditation experience does not repeat exactly each time.
        \item It depends on daily mental state, which keeps changing.
        \item Every session is different; accept variety of experiences.
        \item Lifestyle outside meditation also influences meditation quality.
        \item True meditation is not limited to one session but a way of life.
      \end{itemize}
\end{frame}

%%%%%%%%%%%%%%%%%%%%%%%%%%%%%%%%%%%%%%%%%%%%%%%%%%%%%%%%%%%
\begin{frame}[fragile]\frametitle{Avoiding False Expectations}
      \begin{itemize}
        \item Do not expect extraordinary experiences like floating or visions.
        \item Different people experience meditation differently.
        \item Your meditation is not “failing” if results differ from others.
        \item There is no fixed recipe for meditative experiences.
        \item Comparison with others only creates frustration and doubt.
      \end{itemize}
\end{frame}

%%%%%%%%%%%%%%%%%%%%%%%%%%%%%%%%%%%%%%%%%%%%%%%%%%%%%%%%%%%
\begin{frame}[fragile]\frametitle{Dealing with Boredom}
      \begin{itemize}
        \item Meditation may feel boring; avoid rushing through it.
        \item Strong willpower is essential to continue practice daily.
        \item If a practice feels boring for 3 months, it may not suit you.
        \item Choose a genuine practice that feels natural to you.
        \item Boredom is often a sign of restlessness, not failure.
      \end{itemize}
\end{frame}

%%%%%%%%%%%%%%%%%%%%%%%%%%%%%%%%%%%%%%%%%%%%%%%%%%%%%%%%%%%
\begin{frame}[fragile]\frametitle{Consistency Over Novelty}
      \begin{itemize}
        \item Do not change meditation techniques frequently.
        \item Even if the technique feels monotonous, stay with it.
        \item Practice at the same time daily for discipline.
        \item Twice a day meditation brings deeper results.
        \item Stability in practice is more important than variety.
      \end{itemize}
\end{frame}


%%%%%%%%%%%%%%%%%%%%%%%%%%%%%%%%%%%%%%%%%%%%%%%%%%%%%%%%%%%
\begin{frame}[fragile]\frametitle{Practical Exercise: Day 3}
\begin{lstlisting}
Scene: India-Pakistan cricket match.
India needs 4 runs on the last ball.
The entire stadium holds its breath.
Suddenly, the lights go off.
Nobody knows what happened.
Question: What can be used from this for meditation?
\end{lstlisting}
\end{frame}

%%%%%%%%%%%%%%%%%%%%%%%%%%%%%%%%%%%%%%%%%%%%%%%%%%%%%%%%%%%
\begin{frame}[fragile]\frametitle{Day 3: Meditation Insight from the Scene}
      \begin{itemize}
        \item Breath naturally pauses when deep focus arises.
        \item In meditation, focus makes breath slow down.
        \item Still body + focused mind = meditation condition.
        \item Focus on inner elements like breath, heartbeat, or subtle sound.
        \item Close eyes, remain motionless for 20 minutes.
        \item Allow breath to settle naturally, without effort.
      \end{itemize}
\end{frame}

%%%%%%%%%%%%%%%%%%%%%%%%%%%%%%%%%%%%%%%%%%%%%%%%%%%%%%%%%%%
\begin{frame}[fragile]\frametitle{Day 4: Signs Meditation is Helping}
      \begin{itemize}
        \item Inner state becomes independent of outer situations.
        \item Choice lies with you—whether to react or stay calm.
        \item External triggers lose their control over your emotions.
        \item True progress shows as stability amidst challenges.
      \end{itemize}
\end{frame}

%%%%%%%%%%%%%%%%%%%%%%%%%%%%%%%%%%%%%%%%%%%%%%%%%%%%%%%%%%%
\begin{frame}[fragile]\frametitle{Acceptance and Detachment}
      \begin{itemize}
        \item Acceptance means taking things as they are.
        \item Decide what is under control and what is not.
        \item Acceptance is not weakness, but clarity in response.
        \item Detachment reduces overdependence and stabilizes mind.
        \item Like in trading, apply “stop loss” to avoid deeper harm.
        \item Detachment from goals and possessions enables clear vision.
      \end{itemize}
\end{frame}

%%%%%%%%%%%%%%%%%%%%%%%%%%%%%%%%%%%%%%%%%%%%%%%%%%%%%%%%%%%
\begin{frame}[fragile]\frametitle{Other Benefits of Meditation}
      \begin{itemize}
        \item Staying in the present moment most of the day.
        \item Positive thoughts gradually replace negative ones.
        \item Law of Attraction works—positive energy manifests outcomes.
        \item Fewer doubts, more openness to possibilities.
        \item Others around you may change—negative people drift away.
        \item Positive people enter and surround your life.
        \item Emotions become balanced and under control.
      \end{itemize}
\end{frame}




%%%%%%%%%%%%%%%%%%%%%%%%%%%%%%%%%%%%%%%%%%%%%%%%%%%%%%%%%%%%%%%%
\begin{frame}[fragile]\frametitle{}
\begin{center}
{\Large Types of Meditations and their suitability}


{\tiny (Based on Dr. K's Approach To Meditation - HealthyGamerGG
)}
\end{center}
\end{frame}

%%%%%%%%%%%%%%%%%%%%%%%%%%%%%%%%%%%%%%%%%%%%%%%%%%%%%%%%%%%
\begin{frame}[fragile]\frametitle{Introduction: Why Meditation Feels Hard}
      \begin{itemize}
	\item Every human being has a unique cognitive fingerprint
	\item Different people have different preferences for color, cuisine, recreation, and mental activities
	\item We all have different aptitudes - artistic, mathematical, or other cognitive strengths
	\item The biggest mistake in teaching meditation is using one technique for all different brains
	\item Traditional meditation classes often lose 50\% of students because techniques don't match their cognitive style
	\item Most meditation teachers study primarily one tradition, limiting their ability to match techniques to individuals
	  \end{itemize}
\end{frame}

%%%%%%%%%%%%%%%%%%%%%%%%%%%%%%%%%%%%%%%%%%%%%%%%%%%%%%%%%%%
\begin{frame}[fragile]\frametitle{The Problem with One-Size-Fits-All Meditation}
      \begin{itemize}
	\item Traditional meditation teachers rarely make referrals to different traditions
	\item Unlike medical professionals who refer patients to specialists based on specific needs
	\item Teachers typically say "keep practicing and it will work for you" regardless of fit
	\item This approach ignores individual cognitive differences and specific goals
	\item Many effective techniques exist across different traditions (Vigyan Tantra has 112 techniques alone)
	\item Buddhist, Hindu, Zen, and tantric traditions offer diverse approaches for different minds
	\item The key is matching technique to cognitive fingerprint and meditation goals
	  \end{itemize}
\end{frame}

%%%%%%%%%%%%%%%%%%%%%%%%%%%%%%%%%%%%%%%%%%%%%%%%%%%%%%%%%%%
\begin{frame}[fragile]\frametitle{Personalized Meditation Approach}
      \begin{itemize}
	\item Take collection of different techniques from various traditions
	\item Pick the technique that fits your specific cognitive fingerprint
	\item Match technique to your particular goals and mental patterns
	\item When you find the right technique, meditation becomes easy and highly effective
	\item Benefits can be much more dramatic than most meditation teachers realize
	\item This approach mirrors clinical medicine - specific treatments for specific conditions
	\item Research shows customized meditation regimens can be developed for particular diagnoses
	  \end{itemize}
\end{frame}

%%%%%%%%%%%%%%%%%%%%%%%%%%%%%%%%%%%%%%%%%%%%%%%%%%%%%%%%%%%
\begin{frame}[fragile]\frametitle{Pranayama Techniques for Anxiety and Panic}
      \begin{itemize}
	\item Nadi Shodhana: Alternate nostril breathing technique
	\item Kapal Bhati: Rapid forceful exhalation causing controlled hyperventilation
	\item Both techniques physiologically activate the vagus nerve
	\item Reduces sympathetic nervous system activity (stress, adrenaline, cortisol)
	\item Slows heart rate and induces deep physiological relaxation
	\item Perfect for people with active minds, anxiety, or panic tendencies
	\item Effective when your stress response is "flying mind" rather than withdrawal
	\item Works better than passive observation techniques for anxious minds
	  \end{itemize}
\end{frame}

%%%%%%%%%%%%%%%%%%%%%%%%%%%%%%%%%%%%%%%%%%%%%%%%%%%%%%%%%%%
\begin{frame}[fragile]\frametitle{Tumo Meditation for Depression and Low Energy}
      \begin{itemize}
	\item Tumo named after tantric goddess of fire and passion from Buddhist tradition
	\item Practiced by Himalayan monks to generate internal heat and energy
	\item Studies show practitioners can increase digit temperature by 8°C (15°F)
	\item Designed specifically to energize and cultivate inner passion
	\item Effective for sluggish, lethargic, or depressed mental states
	\item Helps with lack of motivation and difficulty getting out of bed
	\item Completely opposite effect from calming techniques - energizes rather than relaxes
	\item Can be practiced outdoors in cold weather to demonstrate energizing effects
	  \end{itemize}
\end{frame}

%%%%%%%%%%%%%%%%%%%%%%%%%%%%%%%%%%%%%%%%%%%%%%%%%%%%%%%%%%%
\begin{frame}[fragile]\frametitle{Shunya Meditation for Ego Dissolution}
      \begin{itemize}
	\item Shunya means "void," "null," or "zero" in Sanskrit
	\item Designed to dissolve ego-based thoughts and self-criticism
	\item Similar mechanism to ketamine and psychedelics in treating depression
	\item Helps disconnect from negative self-talk: "I'm worthless," "I suck," etc.
	\item Creates neutral state free from both narcissistic and self-deprecating thoughts
	\item Effective for narcissistic personality disorder and low self-esteem
	\item Allows you to just be yourself without constant self-judgment
	\item Works mentally rather than physiologically - affects ego structure directly
	  \end{itemize}
\end{frame}

%%%%%%%%%%%%%%%%%%%%%%%%%%%%%%%%%%%%%%%%%%%%%%%%%%%%%%%%%%%
\begin{frame}[fragile]\frametitle{Yoga Nidra for Subconscious Reprogramming}
      \begin{itemize}
	\item Induces autosuggestive or hypnotic state for deep mental programming
	\item Allows implantation of positive thoughts into subconscious mind
	\item Uses "Sankalpa" (resolve) to plant desired mental patterns
	\item More effective than simple positive affirmations or auto-suggestion
	\item Mind must be in specific receptive state for programming to work
	\item Complements psychotherapy: therapy removes negative, Yoga Nidra implants positive
	\item Creates lasting behavioral change by programming subconscious responses
	\item Example: "As long as I don't give up, I can do this" becomes automatic response to setbacks
	  \end{itemize}
\end{frame}

%%%%%%%%%%%%%%%%%%%%%%%%%%%%%%%%%%%%%%%%%%%%%%%%%%%%%%%%%%%
\begin{frame}[fragile]\frametitle{Key Principles for Effective Meditation}
      \begin{itemize}
	\item Recognize that different techniques serve different mental states and goals
	\item Anxious minds need physiological calming (pranayama) not just mental observation
	\item Depressed/sluggish minds need energizing (tumo) not more calming
	\item Ego-driven minds need dissolution techniques (shunya) for neutrality
	\item Behavioral change needs subconscious programming (yoga nidra)
	\item Match technique to your specific cognitive fingerprint and current needs
	\item 80-95\% of people can find a meditation practice that works for them
	\item Don't give up - find the right teacher and technique for your unique mind
	  \end{itemize}
\end{frame}

%%%%%%%%%%%%%%%%%%%%%%%%%%%%%%%%%%%%%%%%%%%%%%%%%%%%%%%%%%%
\begin{frame}[fragile]\frametitle{Conclusion: Finding Your Meditation Path}
      \begin{itemize}
	\item Meditation struggles don't mean you're bad at it or it's not for you
	\item Like exercise, there are many forms - find what resonates with your mind
	\item Understanding your goals is crucial for selecting appropriate technique
	\item Different traditions offer tools for different mental challenges
	\item Personalized approach leads to easier practice and more dramatic benefits
	\item Seek teachers who can assess your needs and recommend suitable practices
	\item Remember: the purpose is to help your mind move in the direction you want
	\item With the right match, meditation becomes a powerful tool for transformation
	  \end{itemize}
\end{frame}