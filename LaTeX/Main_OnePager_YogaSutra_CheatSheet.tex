\documentclass[9pt,portrait]{article}
\usepackage{beamerarticle} % makes slides into article

%% xcolor Option Clash issue
%	Do not include xcolor,, tikz-qtree, todonotes, here do it after beamerarticle
\usepackage{multicol}
\usepackage{booktabs}
\usepackage{calc}
\usepackage{ifthen}
\usepackage[portrait]{geometry}
\usepackage{hyperref}
\usepackage{color}
\usepackage{enumitem}
\usepackage{textcomp} 				% copyleft symbol
\usepackage{verbatim}
\usepackage{adjustbox} 				% for resizebox to adjust table figure content
\usepackage{enumitem}				% margin free lists
\usepackage{amsmath}
\usepackage{mathrsfs}
\usepackage{csvsimple}				% importing csv as table
\usepackage{textcomp} 				% copyleft symbol
\usepackage{graphicx}
\usepackage{etoolbox} % conditional inclusions

\usepackage{media9}
 \usepackage{multimedia}
 \usepackage{makecell}
 \usepackage{listings}
%  \usepackage{color}
 
\definecolor{codegreen}{rgb}{0,0.6,0}
\definecolor{codegray}{rgb}{0.5,0.5,0.5}
\definecolor{codepurple}{rgb}{0.58,0,0.82}
\definecolor{backcolour}{rgb}{.914, .89, .957} % pale purple

\definecolor{mygreen}{rgb}{0,0.6,0}
\definecolor{mygray}{rgb}{0.5,0.5,0.5}
\definecolor{mymauve}{rgb}{0.58,0,0.82}

\lstdefinestyle{mystyle}{
  backgroundcolor=\color{backcolour},   % choose the background color; you must add \usepackage{color} or \usepackage{xcolor}; should come as last argument
  basicstyle=\footnotesize\ttfamily,       % the size of the fonts that are used for the code
  breakatwhitespace=true,          % sets if automatic breaks should only happen at whitespace
  breaklines=true,                 % sets automatic line breaking
  captionpos=b,                    % sets the caption-position to bottom
  commentstyle=\color{mygreen},    % comment style
  deletekeywords={...},            % if you want to delete keywords from the given language
  escapeinside={\%*}{*)},          % if you want to add LaTeX within your code
  extendedchars=true,              % lets you use non-ASCII characters; for 8-bits encodings only, does not work with UTF-8
  frame=single,	                   % adds a frame around the code
  keepspaces=true,                 % keeps spaces in text, useful for keeping indentation of code (possibly needs columns=flexible)
  keywordstyle=\color{blue},       % keyword style
  language=Python,                 % the language of the code
  morekeywords={*,...},            % if you want to add more keywords to the set
  % numbers=left,                    % where to put the line-numbers; possible values are (none, left, right)
  % numbersep=5pt,                   % how far the line-numbers are from the code
  % numberstyle=\tiny\color{mygray}, % the style that is used for the line-numbers
  rulecolor=\color{black},         % if not set, the frame-color may be changed on line-breaks within not-black text (e.g. comments (green here))
  showspaces=false,                % show spaces everywhere adding particular underscores; it overrides 'showstringspaces'
  showstringspaces=false,          % underline spaces within strings only
  showtabs=false,                  % show tabs within strings adding particular underscores
  stepnumber=2,                    % the step between two line-numbers. If it's 1, each line will be numbered
  stringstyle=\color{codepurple},  % string literal style
  tabsize=2,	                   % sets default tabsize to 2 spaces
  columns=fullflexible,
  linewidth=0.98\linewidth,        % Box width
  aboveskip=10pt,	   			   % Space before listing 
  belowskip=-15pt,	   			   % Space after listing  
  xleftmargin=.02\linewidth,  
  title=\lstname                   % show the filename of files included with \lstinputlisting; also try caption instead of title
}


% \definecolor{codegreen}{rgb}{0,0.6,0}
% \definecolor{codegray}{rgb}{0.5,0.5,0.5}
% \definecolor{codepurple}{rgb}{0.58,0,0.82}
% %\definecolor{backcolour}{rgb}{0.95,0.95,0.92} % faint postman color
% \definecolor{backcolour}{rgb}{.914, .89, .957} % pale purple
% %\lstset{basicstyle=\footnotesize\ttfamily}

% \lstdefinestyle{mystyle}{
    % backgroundcolor=\color{backcolour},   
    % commentstyle=\color{codegreen},
    % keywordstyle=\color{magenta},
    % numberstyle=\tiny\color{codegray},
    % stringstyle=\color{codepurple},
    % basicstyle= \tiny\ttfamily %\scriptsize\ttfamily, %\footnotesize,  % the size of the fonts that are used for the code
    % breakatwhitespace=true,  % sets if automatic breaks should only happen at whitespace        
    % breaklines=true, % sets automatic line breaking   
    % linewidth=\linewidth,	
    % captionpos=b,                    
    % keepspaces=true,% keeps spaces in text, useful for keeping indentation                
% %    numbers=left,                  
    % numbers=none,  
% %    numbersep=5pt,                  
    % showspaces=false,                
    % showstringspaces=false,
    % showtabs=false,                  
    % tabsize=2
% }
\lstset{style=mystyle}


%\lstset{basicstyle=\footnotesize\ttfamily}

\newtoggle{VideoFrames}
\togglefalse{VideoFrames}


\newtoggle{CopyrightPictures}
\togglefalse{CopyrightPictures}

\hypersetup{ % remove ugly hyperlink boxes
    colorlinks,
    linkcolor={red!50!black},
    citecolor={blue!50!black}%,
    %urlcolor={green!80!black}
}

% This sets page margins to .5 inch if using letter paper, and to 1cm
% if using A4 paper. (This probably isn't strictly necessary.)
% If using another size paper, use default 1cm margins.
\ifthenelse{\lengthtest { \paperwidth = 8.5in}}
	{ \geometry{top=.2in,left=.25in,right=.25in,bottom=.5in} }
	{\ifthenelse{ \lengthtest{ \paperwidth = 290mm}}
		{\geometry{top=1cm,left=1cm,right=1cm,bottom=2cm} }
		{\geometry{top=1cm,left=1cm,right=1cm,bottom=2cm} }
	}

% % Turn off header and footer
% \pagestyle{empty}

\usepackage{fancyhdr}   % Package to customize headers and footers
% Turn ON header and footer
\pagestyle{fancy}

% Add a line above the footer
\renewcommand{\footrulewidth}{0.4pt}  % Set the line thickness (adjust as needed)
% Set footer content
\fancyfoot[C]{\textit{Yogesh Haribhau Kulkarni}} % Centered footer text
\fancyfoot[R]{\thepage}                       % Page number on the right
\fancyfoot[L]{\textit{\href{http://www.yogeshkulkarni.com}{yogeshkulkarni@yahoo.com}}} % Centered footer text

% Optional: Set header if needed
%\fancyhead[L]{\textit{Your Header Text Here}} % Header text on the left
 

% Redefine section commands to use less space
\makeatletter
\renewcommand{\section}{\@startsection{section}{1}{0mm}%
                                {-1ex plus -.5ex minus -.2ex}%
                                {0.5ex plus .2ex}%x
                                {\normalfont\large\bfseries}}
\renewcommand{\subsection}{\@startsection{subsection}{2}{0mm}%
                                {-1explus -.5ex minus -.2ex}%
                                {0.5ex plus .2ex}%
                                {\normalfont\normalsize\bfseries}}
\renewcommand{\subsubsection}{\@startsection{subsubsection}{3}{0mm}%
                                {-1ex plus -.5ex minus -.2ex}%
                                {1ex plus .2ex}%
                                {\normalfont\small\bfseries}}
\makeatother

% Define BibTeX command
\def\BibTeX{{\rm B\kern-.05em{\sc i\kern-.025em b}\kern-.08em
    T\kern-.1667em\lower.7ex\hbox{E}\kern-.125emX}}

% Don't print section numbers
\setcounter{secnumdepth}{0}

\newcommand{\code}[1]{\par\vskip0pt plus 1filll \footnotesize Code:~\itshape#1}


\setlength{\parindent}{0pt}
\setlength{\parskip}{0pt plus 0.5ex}
\setlength\columnsep{30pt}

\usepackage{tcolorbox}  % For creating fancy boxes
\usepackage{tikz}       % For drawing borders

% Define a fancy style for cover pages
\tcbuselibrary{skins, breakable, theorems}
\tcbset{
    coverstyle/.style={
        enhanced,
        colframe=black,
        colback=white,
        coltitle=black,
        fonttitle=\bfseries\LARGE,
        fontupper=\normalsize,
        boxrule=1mm,
        width=\textwidth,
        arc=4mm,
        boxsep=5mm,
        outer arc=0mm,
        attach boxed title to top center={yshift=-0.5cm},
        boxed title style={colframe=black, colback=white, boxrule=0mm},
    }
}
\usepackage{polyglossia}
\setdefaultlanguage{sanskrit}
\setotherlanguage{english}

\usepackage{fontspec}
\setmainfont{Segoe UI}

% Devanagari Fonts
\newfontfamily\devanagarifont[Script=Devanagari]{Nakula}
\newfontfamily\devanagarifontsf[Script=Devanagari]{Nakula}
\newfontfamily\devanagarifonttt[Script=Devanagari]{Nakula}
\newfontfamily\devtransl[Mapping=DevRom]{Segoe UI}

% Sharada Fonts
\newfontfamily\sharadafont[Script=Sharada]{Noto Sans Sharada}

\graphicspath{{images/}}

\usepackage{geometry}
\geometry{top=1cm, bottom=1cm, left=1cm, right=1cm}
\usepackage{multicol}
\usepackage{titlesec}

% Reduce space before and after sections
\titlespacing*{\section}{0pt}{6pt}{3pt}
\titlespacing*{\subsection}{0pt}{3pt}{2pt}

\begin{document}
%\footnotesize

\begin{center}
\Large{\textbf{पातञ्जलयोगसूत्राणि\\ महर्षि पतञ्जलि प्रणीतं योगदर्शनम्}}  
\end{center}

\begin{multicols}{2}

\section*{॥ प्रथमोऽध्यायः ॥  ॥ समाधि-पादः ॥}
\begin{flushleft}
\devanagarifont
अथ योगानुशासनम् ॥ १.१॥\\
योगश्चित्तवृत्तिनिरोधः ॥ १.२॥\\
तदा द्रष्टुः स्वरूपेऽवस्थानम् ॥ १.३॥\\
वृत्तिसारूप्यमितरत्र ॥ १.४॥\\
वृत्तयः पञ्चतय्यः क्लिष्टाऽक्लिष्टाः ॥ १.५॥\\
प्रमाणविपर्ययविकल्पनिद्रास्मृतयः ॥ १.६॥\\
प्रत्यक्षानुमानागमाः प्रमाणानि ॥ १.७॥\\
विपर्ययो मिथ्याज्ञानमतद्रूपप्रतिष्ठम् ॥ १.८॥\\
शब्दज्ञानानुपाती वस्तुशून्यो विकल्पः ॥ १.९॥\\
अभावप्रत्ययालम्बना वृत्तिर्निद्रा ॥ १.१०॥\\
अनुभूतविषयासम्प्रमोषः स्मृतिः ॥ १.११॥\\
अभ्यासवैराग्याभ्यां तन्निरोधः ॥ १.१२॥\\
तत्र स्थितौ यत्नोऽभ्यासः ॥ १.१३॥\\
स तु दीर्घकालनैरन्तर्यसत्कारासेवितो दृढभूमिः ॥ १.१४॥\\
दृष्टानुश्रविकविषयवितृष्णस्य वशीकारसंज्ञा वैराग्यम् ॥ १.१५॥\\
तत्परं पुरुषख्यातेर्गुणवैतृष्ण्यम् ॥ १.१६॥\\
वितर्कविचारानन्दास्मितारूपानुगमात् सम्प्रज्ञातः ॥ १.१७॥ (स्मितास्वरूपानुगमात्, स्मितानुगमात्) \\
विरामप्रत्ययाभ्यासपूर्वः संस्कारशेषोऽन्यः ॥ १.१८॥\\
भवप्रत्ययो विदेहप्रकृतिलयानाम् ॥ १.१९॥\\
श्रद्धावीर्यस्मृतिसमाधिप्रज्ञापूर्वक इतरेषाम् ॥ १.२०॥\\
तीव्रसंवेगानामासन्नः ॥ १.२१॥\\
मृदुमध्याधिमात्रत्वात् ततोऽपि विशेषः ॥ १.२२॥\\
ईश्वरप्रणिधानाद्वा ॥ १.२३॥\\
क्लेशकर्मविपाकाशयैरपरामृष्टः पुरुषविशेष ईश्वरः ॥ १.२४॥\\
तत्र निरतिशयं सर्वज्ञबीजम् ॥ १.२५॥ (सर्वज्ञत्वबीजम्) \\
स पूर्वेषामपि गुरुः कालेनानवच्छेदात् ॥ १.२६॥\\
तस्य वाचकः प्रणवः ॥ १.२७॥\\
तज्जपस्तदर्थभावनम् ॥ १.२८॥\\
ततः प्रत्यक्चेतनाधिगमोऽप्यन्तरायाभावश्च ॥ १.२९॥\\
व्याधिस्त्यानसंशयप्रमादालस्याविरति-भ्रान्तिदर्शनालब्धभूमिकत्वानवस्थितत्वानिचित्तविक्षेपास्तेऽन्तरायाः ॥ १.३०॥\\
दुःखदौर्मनस्याङ्गमेजयत्वश्वासप्रश्वासा विक्षेपसहभुवः ॥ १.३१॥\\
तत्प्रतिषेधार्थमेकतत्त्वाभ्यासः ॥ १.३२॥\\
मैत्रीकरुणामुदितोपेक्षाणां सुखदुःखपुण्यापुण्यविषयाणां भावनातश्चित्तप्रसादनम् ॥ १.३३॥\\
प्रच्छर्दनविधारणाभ्यां वा प्राणस्य ॥ १.३४॥\\
विषयवती वा प्रवृत्तिरुत्पन्ना मनसः स्थितिनिबन्धिनी ॥ १.३५॥\\
विशोका वा ज्योतिष्मती ॥ १.३६॥\\
वीतरागविषयं वा चित्तम् ॥ १.३७॥\\
स्वप्ननिद्राज्ञानालम्बनं वा ॥ १.३८॥\\
यथाभिमतध्यानाद्वा ॥ १.३९॥\\
परमाणु परममहत्त्वान्तोऽस्य वशीकारः ॥ १.४०॥\\
क्षीणवृत्तेरभिजातस्येव मणेर्ग्रहीतृग्रहणग्राह्येषुतत्स्थतदञ्जनता समापत्तिः ॥ १.४१॥\\
तत्र शब्दार्थज्ञानविकल्पैः सङ्कीर्णा सवितर्का समापत्तिः ॥ १.४२॥\\
स्मृतिपरिशुद्धौ स्वरूपशून्येवार्थमात्रनिर्भासा निर्वितर्का ॥ १.४३॥\\
एतयैव सविचारा निर्विचारा च सूक्ष्मविषया व्याख्याता ॥ १.४४॥\\
सूक्ष्मविषयत्वं चालिङ्गपर्यवसानम् ॥ १.४५॥\\
ता एव सबीजः समाधिः ॥ १.४६॥\\
निर्विचारवैशारद्येऽध्यात्मप्रसादः ॥ १.४७॥\\
ऋतम्भरा तत्र प्रज्ञा ॥ १.४८॥\\
श्रुतानुमानप्रज्ञाभ्यामन्यविषया विशेषार्थत्वात् ॥ १.४९॥\\
तज्जः संस्कारोऽन्यसंस्कारप्रतिबन्धी ॥ १.५०॥\\
तस्यापि निरोधे सर्वनिरोधान्निर्बीजः समाधिः ॥ १.५१॥\\
॥ इति पतञ्जलि-विरचिते योग-सूत्रे प्रथमः समाधि-पादः ॥\\
\end{flushleft}

\section*{॥ द्वितीयोऽध्यायः ॥  ॥ साधन-पादः ॥}
\begin{flushleft}
\devanagarifont
तपःस्वाध्यायेश्वरप्रणिधानानि क्रियायोगः ॥ २.१॥\\
समाधिभावनार्थः क्लेशतनूकरणार्थश्च ॥ २.२॥\\
अविद्यास्मितारागद्वेषाभिनिवेशाः क्लेशाः ॥ २.३॥\\
अविद्या क्षेत्रमुत्तरेषां प्रसुप्ततनुविच्छिन्नोदाराणाम् ॥ २.४॥\\
अनित्याशुचिदुःखानात्मसु नित्यशुचिसुखात्मख्यातिरविद्या ॥ २.५॥\\
दृग्दर्शनशक्त्योरेकात्मतेवास्मिता ॥ २.६॥\\
सुखानुशयी रागः ॥ २.७॥\\
दुःखानुशयी द्वेषः ॥ २.८॥\\
स्वरसवाही विदुषोऽपि तथारूढोऽभिनिवेशः ॥ २.९॥\\
ते प्रतिप्रसवहेयाः सूक्ष्माः ॥ २.१०॥\\
ध्यानहेयास्तद्वृत्तयः ॥ २.११॥\\
क्लेशमूलः कर्माशयो दृष्टादृष्टजन्मवेदनीयः ॥ २.१२॥\\
सति मूले तद्विपाको जात्यायुर्भोगाः ॥ २.१३॥\\
ते ह्लादपरितापफलाः पुण्यापुण्यहेतुत्वात् ॥ २.१४॥\\
परिणामतापसंस्कारदुःखैर्गुणवृत्तिविरोधाच्च दुःखमेव सर्वं विवेकिनः ॥ २.१५॥\\
हेयं दुःखमनागतम् ॥ २.१६॥\\
द्रष्टृदृश्ययोः संयोगो हेयहेतुः ॥ २.१७॥\\
प्रकाशक्रियास्थितिशीलं भूतेन्द्रियात्मकंभोगापवर्गार्थं दृश्यम् ॥ २.१८॥\\
विशेषाविशेषलिङ्गमात्रालिङ्गानि गुणपर्वाणि ॥ २.१९॥\\
द्रष्टा दृशिमात्रः शुद्धोऽपि प्रत्ययानुपश्यः ॥ २.२०॥\\
तदर्थ एव दृश्यस्यात्मा ॥ २.२१॥\\
कृतार्थं प्रति नष्टमप्यनष्टं तदन्यसाधारणत्वात् ॥ २.२२॥\\
स्वस्वामिशक्त्योः स्वरूपोपलब्धिहेतुः संयोगः ॥ २.२३॥\\
तस्य हेतुरविद्या ॥ २.२४॥\\
तदभावात् संयोगाभावो हानं तद्दृशेः कैवल्यम् ॥ २.२५॥\\
विवेकख्यातिरविप्लवा हानोपायः ॥ २.२६॥\\
तस्य सप्तधा प्रान्तभूमिः प्रज्ञा ॥ २.२७॥\\
योगाङ्गानुष्ठानादशुद्धिक्षये ज्ञानदीप्तिरा विवेकख्यातेः ॥ २.२८॥\\
यमनियमासनप्राणायामप्रत्याहारधारणाध्यानसमाधयोऽष्टावङ्गानि ॥ २.२९॥\\
अहिंसासत्यास्तेयब्रह्मचर्यापरिग्रहा यमाः ॥ २.३०॥\\
जातिदेशकालसमयानवच्छिन्नाः सार्वभौमा महाव्रतम् ॥ २.३१॥\\
शौचसन्तोषतपःस्वाध्यायेश्वरप्रणिधानानि नियमाः ॥ २.३२॥\\
वितर्कबाधने प्रतिपक्षभावनम् ॥ २.३३॥\\
वितर्का हिंसादयः कृतकारितानुमोदिता लोभक्रोधमोहपूर्वका मृदुमध्याधिमात्रा दुःखाज्ञानानन्तफला इति प्रतिपक्षभावनम् ॥ २.३४॥\\
अहिंसाप्रतिष्ठायां तत्सन्निधौ वैरत्यागः ॥ २.३५॥\\
सत्यप्रतिष्ठायां क्रियाफलाश्रयत्वम् ॥ २.३६॥\\
अस्तेयप्रतिष्ठायां सर्वरत्नोपस्थानम् ॥ २.३७॥\\
ब्रह्मचर्यप्रतिष्ठायां वीर्यलाभः ॥ २.३८॥\\
अपरिग्रहस्थैर्ये जन्मकथन्तासम्बोधः ॥ २.३९॥\\
शौचात् स्वाङ्गजुगुप्सा परैरसंसर्गः ॥ २.४०॥\\
सत्त्वशुद्धिसौमनस्यैकाग्र्येन्द्रियजयात्मदर्शनयोग्यत्वानि च ॥ २.४१॥\\
सन्तोषादनुत्तमसुखलाभः ॥ २.४२॥\\
कायेन्द्रियसिद्धिरशुद्धिक्षयात् तपसः ॥ २.४३॥\\
स्वाध्यायाद् इष्टदेवतासम्प्रयोगः ॥ २.४४॥\\
समाधिसिद्धिरीश्वरप्रणिधानात् ॥ २.४५॥\\
स्थिरसुखम् आसनम् ॥ २.४६॥\\
प्रयत्नशैथिल्यानन्तसमापत्तिभ्याम् ॥ २.४७॥\\
ततो द्वन्द्वानभिघातः ॥ २.४८॥\\
तस्मिन्सति श्वासप्रश्वासयोर्गतिविच्छेदः प्राणायामः ॥ २.४९॥\\
बाह्याभ्यन्तरस्तम्भवृत्तिर्देशकालसङ्ख्याभिः परिदृष्टो दीर्घसूक्ष्मः ॥ २.५०॥\\
बाह्याभ्यन्तरविषयाक्षेपी चतुर्थः ॥ २.५१॥\\
ततः क्षीयते प्रकाशावरणम् ॥ २.५२॥\\
धारणासु च योग्यता मनसः ॥ २.५३॥\\
स्वविषयासम्प्रयोगे चित्तस्वरूपानुकार इवेन्द्रियाणां प्रत्याहारः ॥ २.५४॥\\
ततः परमा वश्यतेन्द्रियाणाम् ॥ २.५५॥\\
॥ इति पतञ्जलि-विरचिते योग-सूत्रे द्वितीयः साधन-पादः ॥\\
\end{flushleft}

\section*{॥ तृतीयोऽध्यायः ॥  ॥ विभूति-पादः ॥}
\begin{flushleft}
\devanagarifont
देशबन्धश्चित्तस्य धारणा ॥ ३.१॥\\
तत्र प्रत्ययैकतानता ध्यानम् ॥ ३.२॥\\
तदेवार्थमात्रनिर्भासं स्वरूपशून्यमिव समाधिः ॥ ३.३॥\\
त्रयमेकत्र संयमः ॥ ३.४॥\\
तज्जयात्प्रज्ञालोकः ॥ ३.५॥\\
तस्य भूमिषु विनियोगः ॥ ३.६॥\\
त्रयमन्तरङ्गं पूर्वेभ्यः ॥ ३.७॥\\
तदपि बहिरङ्गं निर्बीजस्य ॥ ३.८॥\\
व्युत्थाननिरोधसंस्कारयोरभिभवप्रादुर्भावौ निरोधक्षणचित्तान्वयो निरोधपरिणामः ॥ ३.९॥\\
तस्य प्रशान्तवाहिता संस्कारात् ॥ ३.१०॥\\
सर्वार्थतैकाग्रतयोः क्षयोदयौ चित्तस्य समाधिपरिणामः ॥ ३.११॥\\
ततः पुनः शान्तोदितौ तुल्यप्रत्ययौ चित्तस्यैकाग्रतापरिणामः ॥ ३.१२॥\\
एतेन भूतेन्द्रियेषु धर्मलक्षणावस्थापरिणामा व्याख्याताः ॥ ३.१३॥\\
शान्तोदिताव्यपदेश्यधर्मानुपाती धर्मी ॥ ३.१४॥\\
क्रमान्यत्वं परिणामान्यत्वे हेतुः ॥ ३.१५॥\\
परिणामत्रयसंयमाद् अतीतानागतज्ञानम् ॥ ३.१६॥\\
शब्दार्थप्रत्ययानामितरेतराध्यासात् सङ्करस्तत्प्रविभागसंयमात्सर्वभूतरुतज्ञानम् ॥ ३.१७॥\\
संस्कारसाक्षात्करणात्पूर्वजातिज्ञानम् ॥ ३.१८॥\\
प्रत्ययस्य परचित्तज्ञानम् ॥ ३.१९॥\\
न च तत्सालम्बनं तस्याविषयीभूतत्वात् ॥ ३.२०॥\\
कायरूपसंयमात्तद्ग्राह्यशक्तिस्तम्भे चक्षुःप्रकाशासम्प्रयोगेऽन्तर्धानम् ॥ ३.२१॥\\
सोपक्रमं निरुपक्रमं च कर्म तत्संयमादपरान्तज्ञानमरिष्टेभ्यो वा ॥ ३.२२॥\\
मैत्र्यादिषु बलानि ॥ ३.२३॥\\
बलेषु हस्तिबलादीनि ॥ ३.२४॥\\
प्रवृत्त्यालोकन्यासात्सूक्ष्मव्यवहितविप्रकृष्टज्ञानम् ॥ ३.२५॥\\
भुवनज्ञानं सूर्ये संयमात् ॥ ३.२६॥\\
चन्द्रे ताराव्यूहज्ञानम् ॥ ३.२७॥\\
ध्रुवे तद्गतिज्ञानम् ॥ ३.२८॥\\
नाभिचक्रे कायव्यूहज्ञानम् ॥ ३.२९॥\\
कण्ठकूपे क्षुत्पिपासानिवृत्तिः ॥ ३.३०॥\\
कूर्मनाड्यां स्थैर्यम् ॥ ३.३१॥\\
मूर्धज्योतिषि सिद्धदर्शनम् ॥ ३.३२॥\\
प्रातिभाद्वा सर्वम् ॥ ३.३३॥\\
हृदये चित्तसंवित् ॥ ३.३४॥\\
सत्त्वपुरुषयोरत्यन्तासङ्कीर्णयोः प्रत्ययाविशेषो भोगः परार्थत्वात्स्वार्थसंयमात्पुरुषज्ञानम् ॥ ३.३५॥\\
ततः प्रातिभश्रावणवेदनादर्शास्वादवार्ता जायन्ते ॥ ३.३६॥\\
ते समाधावुपसर्गा व्युत्थाने सिद्धयः ॥ ३.३७॥\\
बन्धकारणशैथिल्यात्प्रचारसंवेदनाच्च चित्तस्य परशरीरावेशः ॥ ३.३८॥\\
उदानजयाज्जलपङ्ककण्टकादिष्वसङ्ग उत्क्रान्तिश्च ॥ ३.३९॥\\
समानजयाज्ज्वलनम्  (समानजयात्प्रज्वलनम्) ॥ ३.४०॥\\
श्रोत्राकाशयोः सम्बन्धसंयमाद्दिव्यं श्रोत्रम् ॥ ३.४१॥\\
कायाकाशयोः सम्बन्धसंयमाल्लघुतूल-समापत्तेश्चाकाशगमनम् ॥ ३.४२॥\\
बहिरकल्पिता वृत्तिर्महाविदेहा ततः प्रकाशावरणक्षयः ॥ ३.४३॥\\
स्थूलस्वरूपसूक्ष्मान्वयार्थवत्त्वसंयमाद्भूतजयः ॥ ३.४४॥\\
ततोऽणिमादिप्रादुर्भावः कायसम्पत्तद्धर्मानभिघातश्च ॥ ३.४५॥\\
रूपलावण्यबलवज्रसंहननत्वानि कायसम्पत् ॥ ३.४६॥\\
ग्रहणस्वरूपास्मितान्वयार्थवत्त्वसंयमादिन्द्रियजयः ॥ ३.४७॥\\
ततो मनोजवित्वं विकरणभावः प्रधानजयश्च ॥ ३.४८॥\\
सत्त्वपुरुषान्यताख्यातिमात्रस्य सर्वभावाधिष्ठातृत्वंसर्वज्ञातृत्वं च ॥ ३.४९॥\\
तद्वैराग्यादपि दोषबीजक्षये कैवल्यम् ॥ ३.५०॥\\
स्थान्युपनिमन्त्रणे सङ्गस्मयाकरणं पुनरनिष्टप्रसङ्गात् ॥ ३.५१॥\\
क्षणतत्क्रमयोः संयमाद्विवेकजं ज्ञानम् ॥ ३.५२॥\\
जातिलक्षणदेशैरन्यतानवच्छेदात् तुल्ययोस्ततः प्रतिपत्तिः ॥ ३.५३॥\\
तारकं सर्वविषयं सर्वथाविषयमक्रमञ्चेति विवेकजं ज्ञानम् ॥ ३.५४॥\\
सत्त्वपुरुषयोः शुद्धिसाम्ये कैवल्यमिति ॥ ३.५५॥\\
॥ इति पतञ्जलि-विरचिते योग-सूत्रे तृतीयो विभूति-पादः ॥\\
\end{flushleft}

\section*{॥ चतुर्थोऽध्यायः ॥  ॥ कैवल्य-पादः ॥}
\begin{flushleft}
\devanagarifont
जन्मौषधिमन्त्रतपःसमाधिजाः सिद्धयः ॥ ४.१॥\\
जात्यन्तरपरिणामः प्रकृत्यापूरात् ॥ ४.२॥\\
निमित्तमप्रयोजकं प्रकृतीनां वरणभेदस्तु ततः क्षेत्रिकवत् ॥ ४.३॥\\
निर्माणचित्तान्यस्मितामात्रात् ॥ ४.४॥\\
प्रवृत्तिभेदे प्रयोजकं चित्तमेकमनेकेषाम् ॥ ४.५॥\\
तत्र ध्यानजमनाशयम् ॥ ४.६॥\\
कर्माशुक्लाकृष्णं योगिनस्त्रिविधमितरेषाम् ॥ ४.७॥\\
ततस्तद्विपाकानुगुणानामेवाभिव्यक्तिर्वासनानाम् ॥ ४.८॥\\
जातिदेशकालव्यवहितानामप्यानन्तर्यं स्मृतिसंस्कारयोरेकरूपत्वात् ॥ ४.९॥\\
तासामनादित्वं चाशिषो नित्यत्वात् ॥ ४.१०॥\\
हेतुफलाश्रयालम्बनैः सङ्गृहीतत्वादेषामभावे तदभावः ॥ ४.११॥\\
अतीतानागतं स्वरूपतोऽस्त्यध्वभेदाद्धर्माणाम् ॥ ४.१२॥\\
ते व्यक्तसूक्ष्मा गुणात्मानः ॥ ४.१३॥\\
परिणामैकत्वाद्वस्तुतत्त्वम् ॥ ४.१४॥\\
वस्तुसाम्ये चित्तभेदात्तयोर्विभक्तः पन्थाः ॥ ४.१५॥\\
न चैकचित्ततन्त्रं वस्तु तदप्रमाणकं तदा किं स्यात् ॥ ४.१६॥\\
तदुपरागापेक्षित्वाच्चित्तस्य वस्तु ज्ञाताज्ञातम् ॥ ४.१७॥\\
सदा ज्ञाताश्चित्तवृत्तयस्तत्प्रभोः पुरुषस्यापरिणामित्वात् ॥ ४.१८॥\\
न तत्स्वाभासं दृश्यत्वात् ॥ ४.१९॥\\
एकसमये चोभयानवधारणम् ॥ ४.२०॥\\
चित्तान्तरदृश्ये बुद्धिबुद्धेरतिप्रसङ्गः स्मृतिसङ्करश्च ॥ ४.२१॥\\
चितेरप्रतिसङ्क्रमायास्तदाकारापत्तौ स्वबुद्धिसंवेदनम् ॥ ४.२२॥\\
द्रष्टृदृश्योपरक्तं चित्तं सर्वार्थम् ॥ ४.२३॥\\
तदसङ्ख्येयवासनाभिश्चित्रमपि परार्थं संहत्यकारित्वात् ॥ ४.२४॥\\
विशेषदर्शिन आत्मभावभावनाविनिवृत्तिः ॥ ४.२५॥\\
तदा विवेकनिम्नं कैवल्यप्राग्भारं चित्तम् ॥ ४.२६॥\\
तच्छिद्रेषु प्रत्ययान्तराणि संस्कारेभ्यः ॥ ४.२७॥\\
हानमेषां क्लेशवदुक्तम् ॥ ४.२८॥\\
प्रसङ्ख्यानेऽप्यकुसीदस्य सर्वथा विवेकख्यातेर्धर्ममेघः समाधिः ॥ ४.२९॥\\
ततः क्लेशकर्मनिवृत्तिः ॥ ४.३०॥\\
तदा सर्वावरणमलापेतस्य ज्ञानस्यानन्त्याज्ज्ञेयमल्पम् ॥ ४.३१॥\\
ततः कृतार्थानां परिणामक्रमसमाप्तिर्गुणानाम् ॥ ४.३२॥\\
क्षणप्रतियोगी परिणामापरान्तनिर्ग्राह्यः क्रमः ॥ ४.३३॥\\
पुरुषार्थशून्यानां गुणानां प्रतिप्रसवः कैवल्यं स्वरूपप्रतिष्ठा वा चितिशक्तिरिति ॥ ४.३४॥\\
॥ इति पतञ्जलि-विरचिते योग-सूत्रे चतुर्थः कैवल्य-पादः ॥\\
॥ इति श्री पातञ्जल-योग-सूत्राणि ॥\\
\end{flushleft}
\end{multicols}

%\rule{\linewidth}{0.25pt}
%\scriptsize
%Copyleft \textcopyleft\  Send suggestions to 
%\href{http://www.yogeshkulkarni.com}{yogeshkulkarni@yahoo.com}

\end{document}
