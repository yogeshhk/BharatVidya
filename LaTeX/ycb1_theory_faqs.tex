%%%%%%%%%%%%%%%%%%%%%%%%%%%%%%%%%%%%%%%%%%%%%%%%%%%%%%%%%%%%%%%%%%%%%%%%%%%%%%%%%%
\begin{frame}[fragile]\frametitle{}
\begin{center}
{\Large FAQs}
\end{center}
\end{frame}

%%%%%%%%%%%%%%%%%%%%%%%%%%%%%%%%%%%%%%%%%%%%%%%%%%%%%%%%%%%%%%%%%%%%%%%%%%%%%%%%%%
\begin{frame}[fragile]\frametitle{}
\begin{itemize}
\item Definition of yoga by Patanjali: Yogas chitta vritti nirodhah (Yoga is the cessation of fluctuations of the mind)
\item Definition of yoga by Vashishtha: Mano nigraha (Control of the mind)
\item 3 principles of sukshma vyayam: Slow movement, synchronization with breath, awareness
\item Difference between sukshma and sthul vyayam: Sukshma involves subtle movements, sthul involves gross movements
\item Two variations of sukshma vyayam for janu and griva: Rotation and flexion/extension
\end{itemize}
\end{frame}

%%%%%%%%%%%%%%%%%%%%%%%%%%%%%%%%%%%%%%%%%%%%%%%%%%%%%%%%%%%%%%%%%%%%%%%%%%%%%%%%%%
\begin{frame}[fragile]\frametitle{}
\begin{itemize}
\item Sthul vyayam of sarvanga pushti: Full body movements for overall strength
\item Demonstrate surya namaskar: A sequence of 12 yoga postures
\item Surya namaskar is an exercise or asana: It's a sequence of asanas, not a single asana
\item 2 benefits of suryanamaskar: Improves flexibility, boosts cardiovascular health
\item Suryanamaskar includes how many steps: 12 steps
\end{itemize}
\end{frame}

%%%%%%%%%%%%%%%%%%%%%%%%%%%%%%%%%%%%%%%%%%%%%%%%%%%%%%%%%%%%%%%%%%%%%%%%%%%%%%%%%%
\begin{frame}[fragile]\frametitle{}
\begin{itemize}
\item Steps not repeated in suryanamaskar: Each step is unique in a single round
\item Definition of asana as per Patanjali: Sthira sukham asanam (Steady and comfortable posture)
\item HYP and GS talk about how many asanas: HYP mentions 4, GS mentions 84
\item 4 important meditative asanas: Padmasana, Siddhasana, Vajrasana, Sukhasana
\item Meaning of pranayama: Control of breath or life force
\end{itemize}
\end{frame}

%%%%%%%%%%%%%%%%%%%%%%%%%%%%%%%%%%%%%%%%%%%%%%%%%%%%%%%%%%%%%%%%%%%%%%%%%%%%%%%%%%
\begin{frame}[fragile]\frametitle{}
\begin{itemize}
\item Definition of pranayama according to Patanjali: Tasmin sati svasa prasvasayoh gati vicchedah pranayamah
\item How many pranayama are described as per Patanjali: 4 (Bahya, Abhyantara, Stambha, and Bahya-abhyantara)
\item Name the pranayamas as per HYP: Surya Bhedana, Ujjayi, Sitkari, Shitali, Bhastrika, Bhramari, Murcha, Plavini
\item Name the pranayama as per GS: Similar to HYP, with some variations
\item In sankhya yoga prakriti comprises of: 3 gunas (Sattva, Rajas, Tamas)
\end{itemize}
\end{frame}

%%%%%%%%%%%%%%%%%%%%%%%%%%%%%%%%%%%%%%%%%%%%%%%%%%%%%%%%%%%%%%%%%%%%%%%%%%%%%%%%%%
\begin{frame}[fragile]\frametitle{}
\begin{itemize}
\item Qualities of Bhakta: Devotion, surrender, love for the divine
\item How many paths of yoga has been described in Bhagvat Geeta: 4 (Karma, Bhakti, Jnana, Raja)
\item Explain pancha kosha: Five sheaths of existence (Annamaya, Pranamaya, Manomaya, Vijnanamaya, Anandamaya)
\item Which are trisharir: Sthula (gross), Sukshma (subtle), Karana (causal)
\item Give yogic principles: Yama, Niyama, Asana, Pranayama, Pratyahara, Dharana, Dhyana, Samadhi
\end{itemize}
\end{frame}

%%%%%%%%%%%%%%%%%%%%%%%%%%%%%%%%%%%%%%%%%%%%%%%%%%%%%%%%%%%%%%%%%%%%%%%%%%%%%%%%%%
\begin{frame}[fragile]\frametitle{}
\begin{itemize}
\item Name 4 Vedas: Rig Veda, Sama Veda, Yajur Veda, Atharva Veda
\item Vedas are divided into which two parts: karmakanda (rituals), jnanakanda (knowledge)
\item What is Vedanta: Philosophical traditions based on the Upanishads
\item Give 9 Indian philosophy: Nyaya, Vaisheshika, Samkhya, Yoga, Mimamsa, Vedanta, Charvaka, Jainism, Buddhism
\item Sources of pains or sufferings: Avidya (ignorance), Asmita (ego), Raga (attachment), Dvesha (aversion), Abhinivesha (fear of death)
\end{itemize}
\end{frame}

%%%%%%%%%%%%%%%%%%%%%%%%%%%%%%%%%%%%%%%%%%%%%%%%%%%%%%%%%%%%%%%%%%%%%%%%%%%%%%%%%%
\begin{frame}[fragile]\frametitle{}
\begin{itemize}
\item HYP talks about how many limbs? Name them: 4 (Asana, Pranayama, Mudra, Nadanusandhana)
\item GS talks about how many limbs? Name them: 7 (Shatkarma, Asana, Mudra, Pratyahara, Pranayama, Dhyana, Samadhi)
\item Patanjali talks about how many limbs? Name them: 8 (Yama, Niyama, Asana, Pranayama, Pratyahara, Dharana, Dhyana, Samadhi)
\item Total sutras in PYS: 196
\item PYS is divided into 4 chapters? And no of sutras in each chapter: Samadhi Pada (51), Sadhana Pada (55), Vibhuti Pada (56), Kaivalya Pada (34)
\end{itemize}
\end{frame}

%%%%%%%%%%%%%%%%%%%%%%%%%%%%%%%%%%%%%%%%%%%%%%%%%%%%%%%%%%%%%%%%%%%%%%%%%%%%%%%%%%
\begin{frame}[fragile]\frametitle{}
\begin{itemize}
\item What is Pranav jap: Repetition of the sacred syllable OM
\item The vibhuti pad deals with: Supernatural powers or siddhis
\item According to Patanjali Kai app happens in which samadhi: Savitarka Samadhi
\item Stages of jnana yoga: Viveka, Vairagya, Shatsampatti, Mumukshutva
\item What is mitahaar: Moderate diet in yoga practice
\end{itemize}
\end{frame}

%%%%%%%%%%%%%%%%%%%%%%%%%%%%%%%%%%%%%%%%%%%%%%%%%%%%%%%%%%%%%%%%%%%%%%%%%%%%%%%%%%
\begin{frame}[fragile]\frametitle{}
\begin{itemize}
\item 7 dhatus as per Ayurveda: Rasa, Rakta, Mamsa, Meda, Asthi, Majja, Shukra
\item Meaning of shatkarma: Six cleansing techniques in Hatha Yoga
\item Process of dhauti: Cleansing of the digestive tract
\item Process of neti: Nasal cleansing technique
\item Names of sadhak tatva as per HYP: Shradha, Virya, Smriti, Samadhi, Prajna
\end{itemize}
\end{frame}

%%%%%%%%%%%%%%%%%%%%%%%%%%%%%%%%%%%%%%%%%%%%%%%%%%%%%%%%%%%%%%%%%%%%%%%%%%%%%%%%%%
\begin{frame}[fragile]\frametitle{}
\begin{itemize}
\item Concept of ghat shuddhi: Purification of the body
\item 5 asanas included in laghu shankaprakshalan: Tadasana, Tiryak Tadasana, Kati Chakrasana, Tiryak Bhujangasana, Udarakarshan
\item Theory of Chitta bhumis elaborated by: Vyasa in his commentary on Yoga Sutras
\item Name the Chita bhumis: Kshipta, Mudha, Vikshipta, Ekagra, Niruddha
\item Other name of antarayas: Obstacles or Vikshepas
\end{itemize}
\end{frame}

%%%%%%%%%%%%%%%%%%%%%%%%%%%%%%%%%%%%%%%%%%%%%%%%%%%%%%%%%%%%%%%%%%%%%%%%%%%%%%%%%%
\begin{frame}[fragile]\frametitle{}
\begin{itemize}
\item Name anatarayas and it's sahbhuvas: Vyadhi, Styana, Samshaya, Pramada, Alasya, Avirati, Bhranti-darshana, Alabdha-bhumikatva, Anavasthitatva
\item 5th sutra in PYS: Vrittayah panchatayah klishta aklishta
\item 5 kleshas tat Patanjali talks about: Avidya, Asmita, Raga, Dvesha, Abhinivesha
\item Patanjali talks about how many samadhis: Two main types - Samprajnata and Asamprajnata
\item Bridge between bahiranga yoga and antaranga yoga: Pratyahara
\end{itemize}
\end{frame}

%%%%%%%%%%%%%%%%%%%%%%%%%%%%%%%%%%%%%%%%%%%%%%%%%%%%%%%%%%%%%%%%%%%%%%%%%%%%%%%%%%
\begin{frame}[fragile]\frametitle{}
\begin{itemize}
\item Types of kapalbhati: Vatakrama, Vyutkrama, Sheetkrama
\item Other name of kapalbhati: Frontal brain cleansing
\item Contraindications of kapalbhati: High blood pressure, heart problems, stroke, epilepsy
\item Concept of pinda and Bramhanda: Microcosm (individual) and macrocosm (universe)
\item Two metaphysical principles of sankhya yoga: Purusha (consciousness) and Prakriti (matter)
\end{itemize}
\end{frame}

%%%%%%%%%%%%%%%%%%%%%%%%%%%%%%%%%%%%%%%%%%%%%%%%%%%%%%%%%%%%%%%%%%%%%%%%%%%%%%%%%%
\begin{frame}[fragile]\frametitle{}

Question: The total number of dhatus in the body are

\begin{itemize}
\item[A)] 5
\item[B)] 7
\item[C)] 9
\item[D)] 11
\end{itemize}

Answer: B) 7
\end{frame}

%%%%%%%%%%%%%%%%%%%%%%%%%%%%%%%%%%%%%%%%%%%%%%%%%%%%%%%%%%%%%%%%%%%%%%%%%%%%%%%%%%
\begin{frame}[fragile]\frametitle{}

Question: How many pairs of qualities are considered to be inherent in all the substances?

\begin{itemize}
\item[A)] 06 pairs
\item[B)] 08 pairs
\item[C)] 10 pairs
\item[D)] 12 pairs
\end{itemize}

Answer: C) 10 pairs
\end{frame}

%%%%%%%%%%%%%%%%%%%%%%%%%%%%%%%%%%%%%%%%%%%%%%%%%%%%%%%%%%%%%%%%%%%%%%%%%%%%%%%%%%
\begin{frame}[fragile]\frametitle{}

Question: Which of the following is not a quality of a sattvic person?

\begin{itemize}
\item[A)] Focused
\item[B)] Committed
\item[C)] Shabby
\item[D)] Positive
\end{itemize}

Answer: C) Shabby
\end{frame}

%%%%%%%%%%%%%%%%%%%%%%%%%%%%%%%%%%%%%%%%%%%%%%%%%%%%%%%%%%%%%%%%%%%%%%%%%%%%%%%%%%
\begin{frame}[fragile]\frametitle{}

Question: The Sarangdhara Samhita belongs to

\begin{itemize}
\item[A)] Assam
\item[B)] Bihar
\item[C)] Karnataka
\item[D)] Rajasthan
\end{itemize}

Answer: D) Rajasthan
\end{frame}

%%%%%%%%%%%%%%%%%%%%%%%%%%%%%%%%%%%%%%%%%%%%%%%%%%%%%%%%%%%%%%%%%%%%%%%%%%%%%%%%%%
\begin{frame}[fragile]\frametitle{}

Question: Which of the following surgeries is mentioned in the Sushruta Samhita?

\begin{itemize}
\item[A)] Cataract Surgery
\item[B)] Vesical calculi surgery
\item[C)] Rhinoplasty
\item[D)] All the above
\end{itemize}

Answer: D) All the above
\end{frame}


%%%%%%%%%%%%%%%%%%%%%%%%%%%%%%%%%%%%%%%%%%%%%%%%%%%%%%%%%%%%%%%%%%%%%%%%%%%%%%%%%%
\begin{frame}[fragile]\frametitle{}

Question: Adhija vyadhi means-

\begin{itemize}
\item[A)] Disturbance in the body at physical level
\item[B)] Disturbance in the body at mental level
\item[C)] Treatment of physical diseases
\item[D)] Treatment of mental diseases
\end{itemize}

Answer: B) Disturbance in the body at mental level
\end{frame}

%%%%%%%%%%%%%%%%%%%%%%%%%%%%%%%%%%%%%%%%%%%%%%%%%%%%%%%%%%%%%%%%%%%%%%%%%%%%%%%%%%
\begin{frame}[fragile]\frametitle{}

Question: Adhija vyadhi are the cause of disturbance in the –

\begin{itemize}
\item[A)] Annamaya kosha
\item[B)] Manomaya kosha
\item[C)] Vijnanmaya kosha
\item[D)] Anandmaya kosha
\end{itemize}

Answer: B) Manomaya kosha
\end{frame}

%%%%%%%%%%%%%%%%%%%%%%%%%%%%%%%%%%%%%%%%%%%%%%%%%%%%%%%%%%%%%%%%%%%%%%%%%%%%%%%%%%
\begin{frame}[fragile]\frametitle{}

Question: Anadhija vyadhi are the-

\begin{itemize}
\item[A)] Mental ailments
\item[B)] Physical ailments
\item[C)] Severe diseases
\item[D)] Viral diseases
\end{itemize}

Answer: B) Physical ailments
\end{frame}

%%%%%%%%%%%%%%%%%%%%%%%%%%%%%%%%%%%%%%%%%%%%%%%%%%%%%%%%%%%%%%%%%%%%%%%%%%%%%%%%%%
\begin{frame}[fragile]\frametitle{}

Question: Anadhija vyadhi mainly affects the-

\begin{itemize}
\item[A)] Annamaya kosha
\item[B)] Pranamaya kosha
\item[C)] Manomaya kosha
\item[D)] Anandmaya kosha
\end{itemize}

Answer: A) Annamaya kosha
\end{frame}

%%%%%%%%%%%%%%%%%%%%%%%%%%%%%%%%%%%%%%%%%%%%%%%%%%%%%%%%%%%%%%%%%%%%%%%%%%%%%%%%%%
\begin{frame}[fragile]\frametitle{}

Question: Match the correct-

1) Dukha                            A) Sadness, Despair
2) Daurmanasya                    B) Mental or physical pain
3) Angamejayatva                  C) Respiration disturbances
4) Shvasa Prashvasa              D) Anxious tremor

\begin{itemize}
\item[A)] 1-B, 2-A, 3-D, 4-C
\item[B)] 1-A, 2-B, 3-C, 4-D
\item[C)] 1-C, 2-B, 3-A, 4-D
\item[D)] 1-D, 2-C, 3-A, 4-B
\end{itemize}

Answer: A) 1-B, 2-A, 3-D, 4-C
\end{frame}

%%%%%%%%%%%%%%%%%%%%%%%%%%%%%%%%%%%%%%%%%%%%%%%%%%%%%%%%%%%%%%%%%%%%%%%%%%%%%%%%%%
\begin{frame}[fragile]\frametitle{}

Question: Disease caused due to the deficiency of protein is-

\begin{itemize}
\item[A)] Scurvy
\item[B)] Kwashiorkor
\item[C)] Pellagra
\item[D)] Cataract
\end{itemize}

Answer: B) Kwashiorkor
\end{frame}

%%%%%%%%%%%%%%%%%%%%%%%%%%%%%%%%%%%%%%%%%%%%%%%%%%%%%%%%%%%%%%%%%%%%%%%%%%%%%%%%%%
\begin{frame}[fragile]\frametitle{}

Question: According to Ayurveda, what is the main cause of various types of diseases in the body?

\begin{itemize}
\item[A)] Attack of germs on the body organs
\item[B)] Weakening of the immune system
\item[C)] Imbalance of vata, pitta \& kapha
\item[D)] Weakening of the pranamaya kosha
\end{itemize}

Answer: C) Imbalance of vata, pitta \& kapha
\end{frame}

%%%%%%%%%%%%%%%%%%%%%%%%%%%%%%%%%%%%%%%%%%%%%%%%%%%%%%%%%%%%%%%%%%%%%%%%%%%%%%%%%%
\begin{frame}[fragile]\frametitle{}

Question: According to Mandukya Upanishad, which of the following matches are correct?

\begin{itemize}
\item[A)] Jagritavastha-Akara
\item[B)] Swapnavastha-Ukara
\item[C)] Sushuptivastha-Makara
\item[D)] All the above
\end{itemize}

Answer: D) All the above
\end{frame}

%%%%%%%%%%%%%%%%%%%%%%%%%%%%%%%%%%%%%%%%%%%%%%%%%%%%%%%%%%%%%%%%%%%%%%%%%%%%%%%%%%
\begin{frame}[fragile]\frametitle{}

Question: In Taittiriya Upanishad, the main theory used in the treatment of diseases through yoga is-

\begin{itemize}
\item[A)] Shatchakra theory
\item[B)] Pancha kosha theory
\item[C)] Meditation theory
\item[D)] Pancha mahabhuta theory
\end{itemize}

Answer: B) Pancha kosha theory
\end{frame}

%%%%%%%%%%%%%%%%%%%%%%%%%%%%%%%%%%%%%%%%%%%%%%%%%%%%%%%%%%%%%%%%%%%%%%%%%%%%%%%%%%
\begin{frame}[fragile]\frametitle{}

Question: Which is the first book of the Charaka Samhita?

\begin{itemize}
\item[A)] Nidana sthana
\item[B)] Sharira sthana
\item[C)] Sutra sthana
\item[D)] Kalpa sthana
\end{itemize}

Answer: C) Sutra sthana
\end{frame}

%%%%%%%%%%%%%%%%%%%%%%%%%%%%%%%%%%%%%%%%%%%%%%%%%%%%%%%%%%%%%%%%%%%%%%%%%%%%%%%%%%
\begin{frame}[fragile]\frametitle{}

Question: Which dosha gets suppressed in the Hemant ritu?

\begin{itemize}
\item[A)] Vata
\item[B)] Pitta
\item[C)] Kapha
\item[D)] Vata, Kapha
\end{itemize}

Answer: B) Pitta
\end{frame}

%%%%%%%%%%%%%%%%%%%%%%%%%%%%%%%%%%%%%%%%%%%%%%%%%%%%%%%%%%%%%%%%%%%%%%%%%%%%%%%%%%
\begin{frame}[fragile]\frametitle{}

Question: Which dosha gets suppressed in the Grishma (summer) ritu?

\begin{itemize}
\item[A)] Vata
\item[B)] Pitta
\item[C)] Kapha
\item[D)] Pitta, Kapha
\end{itemize}

Answer: C) Kapha
\end{frame}

%%%%%%%%%%%%%%%%%%%%%%%%%%%%%%%%%%%%%%%%%%%%%%%%%%%%%%%%%%%%%%%%%%%%%%%%%%%%%%%%%%
\begin{frame}[fragile]\frametitle{}

Question: In which ritu does pitta accumulate in the body?

\begin{itemize}
\item[A)] Grishma ritu
\item[B)] Sharad ritu
\item[C)] Shishira ritu
\item[D)] Varsha ritu
\end{itemize}

Answer: D) Varsha ritu
\end{frame}

%%%%%%%%%%%%%%%%%%%%%%%%%%%%%%%%%%%%%%%%%%%%%%%%%%%%%%%%%%%%%%%%%%%%%%%%%%%%%%%%%%
\begin{frame}[fragile]\frametitle{}

Question: To control diseases with the help of fasting is a type of therapy, known as-

\begin{itemize}
\item[A)] Langhana
\item[B)] Stambhana
\item[C)] Svedana
\item[D)] Snehana
\end{itemize}

Answer: A) Langhana
\end{frame}

%%%%%%%%%%%%%%%%%%%%%%%%%%%%%%%%%%%%%%%%%%%%%%%%%%%%%%%%%%%%%%%%%%%%%%%%%%%%%%%%%%
\begin{frame}[fragile]\frametitle{}

Question: "Swasthya ............... Rakshanam".

\begin{itemize}
\item[A)] Ayurveda
\item[B)] Swastha
\item[C)] Prakriti
\item[D)] Yoga
\end{itemize}

Answer: A) Ayurveda
\end{frame}

%%%%%%%%%%%%%%%%%%%%%%%%%%%%%%%%%%%%%%%%%%%%%%%%%%%%%%%%%%%%%%%%%%%%%%%%%%%%%%%%%%
\begin{frame}[fragile]\frametitle{}

Question: The disorders caused by natural agents are known as-

\begin{itemize}
\item[A)] Adhibhautika
\item[B)] Adhidaivika
\item[C)] Daruna
\item[D)] Daivabala prasrita
\end{itemize}

Answer: A) Adhibhautika
\end{frame}

%%%%%%%%%%%%%%%%%%%%%%%%%%%%%%%%%%%%%%%%%%%%%%%%%%%%%%%%%%%%%%%%%%%%%%%%%%%%%%%%%%
\begin{frame}[fragile]\frametitle{}

Question: The reason for vaikarika nidra is more .......... and less .......... in the body.

\begin{itemize}
\item[A)] Pitta, Kapha
\item[B)] Kapha, Pitta
\item[C)] Vata, Pitta
\item[D)] Pitta, Vata
\end{itemize}

Answer: B) Kapha, Pitta
\end{frame}

%%%%%%%%%%%%%%%%%%%%%%%%%%%%%%%%%%%%%%%%%%%%%%%%%%%%%%%%%%%%%%%%%%%%%%%%%%%%%%%%%%
\begin{frame}[fragile]\frametitle{}

Question: Achamana kriya is related to-

\begin{itemize}
\item[A)] Taking fruits
\item[B)] Taking water
\item[C)] Shatkarma
\item[D)] Fasting
\end{itemize}

Answer: B) Taking water
\end{frame}

%%%%%%%%%%%%%%%%%%%%%%%%%%%%%%%%%%%%%%%%%%%%%%%%%%%%%%%%%%%%%%%%%%%%%%%%%%%%%%%%%%
\begin{frame}[fragile]\frametitle{}

Question: Abhyanga means-

\begin{itemize}
\item[A)] Detoxification
\item[B)] Body massage
\item[C)] Shatkarma
\item[D)] Fasting
\end{itemize}

Answer: B) Body massage
\end{frame}

%%%%%%%%%%%%%%%%%%%%%%%%%%%%%%%%%%%%%%%%%%%%%%%%%%%%%%%%%%%%%%%%%%%%%%%%%%%%%%%%%%
\begin{frame}[fragile]\frametitle{}

Question: The strength of the body in adana kala-

\begin{itemize}
\item[A)] Decreases
\item[B)] Increases
\item[C)] No change
\item[D)] Changes frequently
\end{itemize}

Answer: A) Decreases
\end{frame}

%%%%%%%%%%%%%%%%%%%%%%%%%%%%%%%%%%%%%%%%%%%%%%%%%%%%%%%%%%%%%%%%%%%%%%%%%%%%%%%%%%
\begin{frame}[fragile]\frametitle{}

Question: According to Ayurveda, how many total seasons are there in India?

\begin{itemize}
\item[A)] 3
\item[B)] 4
\item[C)] 6
\item[D)] 7
\end{itemize}

Answer: C) 6
\end{frame}

%%%%%%%%%%%%%%%%%%%%%%%%%%%%%%%%%%%%%%%%%%%%%%%%%%%%%%%%%%%%%%%%%%%%%%%%%%%%%%%%%%
\begin{frame}[fragile]\frametitle{}

Question: In dakshinayana kala the sun is-

\begin{itemize}
\item[A)] Towards tropic of Capricorn
\item[B)] Towards equator
\item[C)] Towards tropic of cancer
\item[D)] None
\end{itemize}

Answer: A) Towards tropic of Capricorn
\end{frame}

%%%%%%%%%%%%%%%%%%%%%%%%%%%%%%%%%%%%%%%%%%%%%%%%%%%%%%%%%%%%%%%%%%%%%%%%%%%%%%%%%%
\begin{frame}[fragile]\frametitle{}

Question: The line 'Aatu Rasa Vikara Parasabddama' is related to-

\begin{itemize}
\item[A)] Yoga
\item[B)] Fast
\item[C)] Fruit
\item[D)] Swasthavritta
\end{itemize}

Answer: D) Swasthavritta
\end{frame}

%%%%%%%%%%%%%%%%%%%%%%%%%%%%%%%%%%%%%%%%%%%%%%%%%%%%%%%%%%%%%%%%%%%%%%%%%%%%%%%%%%
\begin{frame}[fragile]\frametitle{}

Question: In which season is kapha accumulated in the body?

\begin{itemize}
\item[A)] Winter
\item[B)] Late autumn
\item[C)] Spring
\item[D)] Summer
\end{itemize}

Answer: C) Spring
\end{frame}

%%%%%%%%%%%%%%%%%%%%%%%%%%%%%%%%%%%%%%%%%%%%%%%%%%%%%%%%%%%%%%%%%%%%%%%%%%%%%%%%%%
\begin{frame}[fragile]\frametitle{}

Question: In which season is consuming sattu harmful?

\begin{itemize}
\item[A)] Summer
\item[B)] Winter
\item[C)] Spring
\item[D)] Late autumn
\end{itemize}

Answer: A) Summer
\end{frame}

%%%%%%%%%%%%%%%%%%%%%%%%%%%%%%%%%%%%%%%%%%%%%%%%%%%%%%%%%%%%%%%%%%%%%%%%%%%%%%%%%%
\begin{frame}[fragile]\frametitle{}

Question: The much spicy food which takes more time to be digested is known as-

\begin{itemize}
\item[A)] Tamasic food
\item[B)] Sattvic food
\item[C)] Rajasic food
\item[D)] None
\end{itemize}

Answer: C) Rajasic food
\end{frame}

%%%%%%%%%%%%%%%%%%%%%%%%%%%%%%%%%%%%%%%%%%%%%%%%%%%%%%%%%%%%%%%%%%%%%%%%%%%%%%%%%%
\begin{frame}[fragile]\frametitle{}

Question: Which type of food is the main cause of diseases?

\begin{itemize}
\item[A)] Rajasic food
\item[B)] Tamasic food
\item[C)] Sattvic food
\item[D)] Hot food
\end{itemize}

Answer: B) Tamasic food
\end{frame}

%%%%%%%%%%%%%%%%%%%%%%%%%%%%%%%%%%%%%%%%%%%%%%%%%%%%%%%%%%%%%%%%%%%%%%%%%%%%%%%%%%
\begin{frame}[fragile]\frametitle{}

Question: Which of the following statements is correct?

\begin{itemize}
\item[A)] Consuming cloying food is pitta annihilator
\item[B)] Consuming cloying food is kapha creator
\item[C)] Consuming cloying food is vata creator
\item[D)] All the above
\end{itemize}

Answer: B) Consuming cloying food is kapha creator
\end{frame}

%%%%%%%%%%%%%%%%%%%%%%%%%%%%%%%%%%%%%%%%%%%%%%%%%%%%%%%%%%%%%%%%%%%%%%%%%%%%%%%%%%
\begin{frame}[fragile]\frametitle{}

Question: The meaning of 'Prajalapta' is-

\begin{itemize}
\item[A)] Over eating
\item[B)] Talkativeness
\item[C)] Feeble minded
\item[D)] Over sleeping
\end{itemize}

Answer: B) Talkativeness
\end{frame}

%%%%%%%%%%%%%%%%%%%%%%%%%%%%%%%%%%%%%%%%%%%%%%%%%%%%%%%%%%%%%%%%%%%%%%%%%%%%%%%%%%
\begin{frame}[fragile]\frametitle{}

Question: The therapy done by using water is known as-

\begin{itemize}
\item[A)] Hydropathy
\item[B)] Heliotherapy
\item[C)] Pathology
\item[D)] Hydrology
\end{itemize}

Answer: A) Hydropathy
\end{frame}

%%%%%%%%%%%%%%%%%%%%%%%%%%%%%%%%%%%%%%%%%%%%%%%%%%%%%%%%%%%%%%%%%%%%%%%%%%%%%%%%%%
\begin{frame}[fragile]\frametitle{}

Question: The temperature of water for cold spinal bath should be-

\begin{itemize}
\item[A)] 65-85 °F
\item[B)] 55-65 °F
\item[C)] 85-95 °F
\item[D)] 75-95 °F
\end{itemize}

Answer: B) 55-65 °F
\end{frame}

%%%%%%%%%%%%%%%%%%%%%%%%%%%%%%%%%%%%%%%%%%%%%%%%%%%%%%%%%%%%%%%%%%%%%%%%%%%%%%%%%%
\begin{frame}[fragile]\frametitle{}

Question: Who is the writer of 'Heal without medicine'?

\begin{itemize}
\item[A)] Shenwan
\item[B)] Collins
\item[C)] Dr. Frawley
\item[D)] Adler
\end{itemize}

Answer: C) Dr. Frawley
\end{frame}

%%%%%%%%%%%%%%%%%%%%%%%%%%%%%%%%%%%%%%%%%%%%%%%%%%%%%%%%%%%%%%%%%%%%%%%%%%%%%%%%%%
\begin{frame}[fragile]\frametitle{}

Question: Arthritis is mainly

\begin{itemize}
\item[A)] A vata generated disorder
\item[B)] A pitta generated disorder
\item[C)] A Kapha generated disorder
\item[D)] A blood disorder
\end{itemize}

Answer: A) A vata generated disorder
\end{frame}

%%%%%%%%%%%%%%%%%%%%%%%%%%%%%%%%%%%%%%%%%%%%%%%%%%%%%%%%%%%%%%%%%%%%%%%%%%%%%%%%%%
\begin{frame}[fragile]\frametitle{}

Question: How many types of nidras are explained by Maharishi Sushruta?

\begin{itemize}
\item[A)] 3
\item[B)] 4
\item[C)] 5
\item[D)] 6
\end{itemize}

Answer: A) 3
\end{frame}

%%%%%%%%%%%%%%%%%%%%%%%%%%%%%%%%%%%%%%%%%%%%%%%%%%%%%%%%%%%%%%%%%%%%%%%%%%%%%%%%%%
\begin{frame}[fragile]\frametitle{}

Question: Who is the author of the book 'New science of healing'?

\begin{itemize}
\item[A)] Dr. David Frawley
\item[B)] Dr. Luis Kuhne
\item[C)] Edward Thorndike
\item[D)] William Shenwan
\end{itemize}

Answer: B) Dr. Luis Kuhne
\end{frame}

%%%%%%%%%%%%%%%%%%%%%%%%%%%%%%%%%%%%%%%%%%%%%%%%%%%%%%%%%%%%%%%%%%%%%%%%%%%%%%%%%%
\begin{frame}[fragile]\frametitle{}

Question: The meaning of word 'mita' in mitahara is-

\begin{itemize}
\item[A)] One time
\item[B)] Two times
\item[C)] Food
\item[D)] Limited
\end{itemize}

Answer: D) Limited
\end{frame}

%%%%%%%%%%%%%%%%%%%%%%%%%%%%%%%%%%%%%%%%%%%%%%%%%%%%%%%%%%%%%%%%%%%%%%%%%%%%%%%%%%
\begin{frame}[fragile]\frametitle{}

Question: According to Ayurveda, how much part of stomach should be left empty for space during a meal?

\begin{itemize}
\item[A)] One third
\item[B)] One fourth
\item[C)] Half
\item[D)] None
\end{itemize}

Answer: A) One third
\end{frame}

%%%%%%%%%%%%%%%%%%%%%%%%%%%%%%%%%%%%%%%%%%%%%%%%%%%%%%%%%%%%%%%%%%%%%%%%%%%%%%%%%%
\begin{frame}[fragile]\frametitle{}

Question: The water charged in which colour of bottle is useful for a person suffering from chronic skin disease?

\begin{itemize}
\item[A)] Blue
\item[B)] Green
\item[C)] Red
\item[D)] Yellow
\end{itemize}

Answer: B) Green
\end{frame}

%%%%%%%%%%%%%%%%%%%%%%%%%%%%%%%%%%%%%%%%%%%%%%%%%%%%%%%%%%%%%%%%%%%%%%%%%%%%%%%%%%
\begin{frame}[fragile]\frametitle{}

Question: The therapy done with the help of sun rays is known as-

\begin{itemize}
\item[A)] Heliotherapy
\item[B)] Mesotherapy
\item[C)] Hemotherapy
\item[D)] Heatherapy
\end{itemize}

Answer: A) Heliotherapy
\end{frame}

%%%%%%%%%%%%%%%%%%%%%%%%%%%%%%%%%%%%%%%%%%%%%%%%%%%%%%%%%%%%%%%%%%%%%%%%%%%%%%%%%%
\begin{frame}[fragile]\frametitle{}

Question: Astringent taste-

\begin{itemize}
\item[A)] Balances pitta and kapha
\item[B)] Aggravates vata
\item[C)] Is dry, cold and heavy
\item[D)] All the above
\end{itemize}

Answer: D) All the above
\end{frame}

%%%%%%%%%%%%%%%%%%%%%%%%%%%%%%%%%%%%%%%%%%%%%%%%%%%%%%%%%%%%%%%%%%%%%%%%%%%%%%%%%%
\begin{frame}[fragile]\frametitle{}

Question: Which statement is correct regarding astringent taste?

\begin{itemize}
\item[A)] It is found in beans, apple, avocado, cabbage
\item[B)] Its elements are air and earth
\item[C)] It absorbs water and tightens tissues
\item[D)] All the above
\end{itemize}

Answer: D) All the above
\end{frame}

%%%%%%%%%%%%%%%%%%%%%%%%%%%%%%%%%%%%%%%%%%%%%%%%%%%%%%%%%%%%%%%%%%%%%%%%%%%%%%%%%%
\begin{frame}[fragile]\frametitle{}

Question: Bitter taste-

\begin{itemize}
\item[A)] Balances pitta and kapha
\item[B)] Aggravates vata
\item[C)] Is dry, cold, light
\item[D)] All the above
\end{itemize}

Answer: D) All the above
\end{frame}

%%%%%%%%%%%%%%%%%%%%%%%%%%%%%%%%%%%%%%%%%%%%%%%%%%%%%%%%%%%%%%%%%%%%%%%%%%%%%%%%%%
\begin{frame}[fragile]\frametitle{}

Question: Which statement is not correct regarding bitter taste?

\begin{itemize}
\item[A)] It is found in sesame seeds, coffee, saffron
\item[B)] Its elements are air and ether
\item[C)] Its affinity organs are stomach and heart
\item[D)] It detoxifies and lightens tissues
\end{itemize}

Answer: A) It is found in sesame seeds, coffee, saffron
\end{frame}

%%%%%%%%%%%%%%%%%%%%%%%%%%%%%%%%%%%%%%%%%%%%%%%%%%%%%%%%%%%%%%%%%%%%%%%%%%%%%%%%%%
\begin{frame}[fragile]\frametitle{}

Question: Pungent taste-

\begin{itemize}
\item[A)] Balances kapha
\item[B)] Aggravates pitta and vata
\item[C)] Is hot, dry, light and sharp
\item[D)] All the above
\end{itemize}

Answer: D) All the above
\end{frame}

%%%%%%%%%%%%%%%%%%%%%%%%%%%%%%%%%%%%%%%%%%%%%%%%%%%%%%%%%%%%%%%%%%%%%%%%%%%%%%%%%%
\begin{frame}[fragile]\frametitle{}

Question: Which of the following statements is not correct regarding pungent taste?

\begin{itemize}
\item[A)] It is antispasmodic and antipyretic
\item[B)] Its elements are fire and air
\item[C)] It's found in milk and curd
\item[D)] It stimulates digestion and metabolism
\end{itemize}

Answer: C) It's found in milk and curd
\end{frame}

%%%%%%%%%%%%%%%%%%%%%%%%%%%%%%%%%%%%%%%%%%%%%%%%%%%%%%%%%%%%%%%%%%%%%%%%%%%%%%%%%%
\begin{frame}[fragile]\frametitle{}

Question: Salty taste-

\begin{itemize}
\item[A)] Balances vata
\item[B)] Aggravates pitta and kapha
\item[C)] Is light, hot, oily and liquid
\item[D)] All the above
\end{itemize}

Answer: D) All the above
\end{frame}

%%%%%%%%%%%%%%%%%%%%%%%%%%%%%%%%%%%%%%%%%%%%%%%%%%%%%%%%%%%%%%%%%%%%%%%%%%%%%%%%%%
\begin{frame}[fragile]\frametitle{}

Question: Which is incorrect regarding salty taste?

\begin{itemize}
\item[A)] Its affinity organ is heart
\item[B)] It lubricates tissues and is an appetizer
\item[C)] Its elements are water and fire
\item[D)] It is found in seaweeds like rock salts
\end{itemize}

Answer: A) Its affinity organ is heart
\end{frame}

%%%%%%%%%%%%%%%%%%%%%%%%%%%%%%%%%%%%%%%%%%%%%%%%%%%%%%%%%%%%%%%%%%%%%%%%%%%%%%%%%%
\begin{frame}[fragile]\frametitle{}

Question: Sour taste-

\begin{itemize}
\item[A)] Balances vata
\item[B)] Aggravates pitta and kapha
\item[C)] Is light, hot, oily and liquid
\item[D)] All the above
\end{itemize}

Answer: D) All the above
\end{frame}

%%%%%%%%%%%%%%%%%%%%%%%%%%%%%%%%%%%%%%%%%%%%%%%%%%%%%%%%%%%%%%%%%%%%%%%%%%%%%%%%%%
\begin{frame}[fragile]\frametitle{}

Question: The correct statement regarding sour taste is-

\begin{itemize}
\item[A)] Its elements are earth and fire
\item[B)] It increases absorption of minerals
\item[C)] It's found in lemon, grapes, lime, tamarind, curd and fermented food
\item[D)] All the above
\end{itemize}

Answer: D) All the above
\end{frame}

%%%%%%%%%%%%%%%%%%%%%%%%%%%%%%%%%%%%%%%%%%%%%%%%%%%%%%%%%%%%%%%%%%%%%%%%%%%%%%%%%%
\begin{frame}[fragile]\frametitle{}

Question: Sweet taste-

\begin{itemize}
\item[A)] Balances vata and pitta
\item[B)] Aggravates kapha
\item[C)] Is cold, oily and heavy
\item[D)] All the above
\end{itemize}

Answer: D) All the above
\end{frame}

%%%%%%%%%%%%%%%%%%%%%%%%%%%%%%%%%%%%%%%%%%%%%%%%%%%%%%%%%%%%%%%%%%%%%%%%%%%%%%%%%%
\begin{frame}[fragile]\frametitle{}

Question: The elements related to sweet taste are-

\begin{itemize}
\item[A)] Earth and water
\item[B)] Earth and fire
\item[C)] Fire and water
\item[D)] Air and earth
\end{itemize}

Answer: A) Earth and water
\end{frame}

%%%%%%%%%%%%%%%%%%%%%%%%%%%%%%%%%%%%%%%%%%%%%%%%%%%%%%%%%%%%%%%%%%%%%%%%%%%%%%%%%%
\begin{frame}[fragile]\frametitle{}

Question: Which of the following is part of 'Trayopastambha' in Ayurveda?

\begin{itemize}
\item[A)] Brahmacharya
\item[B)] Ahara
\item[C)] Nidra
\item[D)] All the above
\end{itemize}

Answer: D) All the above
\end{frame}

%%%%%%%%%%%%%%%%%%%%%%%%%%%%%%%%%%%%%%%%%%%%%%%%%%%%%%%%%%%%%%%%%%%%%%%%%%%%%%%%%%
\begin{frame}[fragile]\frametitle{}

Question: Which of the following tastes aggravate vata in the body?

\begin{itemize}
\item[A)] Bitter, Sweet, Salty
\item[B)] Bitter, Pungent, Astringent
\item[C)] Pungent, Sour, Salty
\item[D)] Sweet, Sour, Salty
\end{itemize}

Answer: B) Bitter, Pungent, Astringent
\end{frame}

%%%%%%%%%%%%%%%%%%%%%%%%%%%%%%%%%%%%%%%%%%%%%%%%%%%%%%%%%%%%%%%%%%%%%%%%%%%%%%%%%%
\begin{frame}[fragile]\frametitle{}

Question: Which of the following tastes aggravate kapha in the body?

\begin{itemize}
\item[A)] Sweet, Sour, Salty
\item[B)] Sweet, Pungent, Bitter
\item[C)] Sweet, Astringent, Bitter
\item[D)] Sweet, Bitter, Salty
\end{itemize}

Answer: A) Sweet, Sour, Salty
\end{frame}

%%%%%%%%%%%%%%%%%%%%%%%%%%%%%%%%%%%%%%%%%%%%%%%%%%%%%%%%%%%%%%%%%%%%%%%%%%%%%%%%%%
\begin{frame}[fragile]\frametitle{}

Question: Which of the following tastes aggravate pitta in the body?

\begin{itemize}
\item[A)] Bitter, Sour, Sweet
\item[B)] Pungent, Salty, Sour
\item[C)] Pungent, Astringent, Bitter
\item[D)] Bitter, Astringent, Salty
\end{itemize}

Answer: B) Pungent, Salty, Sour
\end{frame}

%%%%%%%%%%%%%%%%%%%%%%%%%%%%%%%%%%%%%%%%%%%%%%%%%%%%%%%%%%%%%%%%%%%%%%%%%%%%%%%%%
\begin{frame}[fragile]\frametitle{Question 1}
Question: Who first introduced yoga in a classical way?
\begin{itemize}
\item A) Kapil Muni
\item B) Maharishi Patanjali
\item C) Agastya Muni
\item D) Jaimini
\end{itemize}
Answer: B) Maharishi Patanjali
\end{frame}

%%%%%%%%%%%%%%%%%%%%%%%%%%%%%%%%%%%%%%%%%%%%%%%%%%%%%%%%%%%%%%%%%%%%%%%%%%%%%%%%%
\begin{frame}[fragile]\frametitle{Question 2}
Question: Which day is celebrated as "International Day of Yoga"?
\begin{itemize}
\item A) 20th JUNE
\item B) 21st JUNE
\item C) 22nd JUNE
\item D) 23rd JUNE
\end{itemize}
Answer: B) 21st JUNE
\end{frame}

%%%%%%%%%%%%%%%%%%%%%%%%%%%%%%%%%%%%%%%%%%%%%%%%%%%%%%%%%%%%%%%%%%%%%%%%%%%%%%%%%
\begin{frame}[fragile]\frametitle{Question 3}
Question: The appropriate amount of time to wait after a meal before beginning a yoga practice
\begin{itemize}
\item A) 30 mins
\item B) 90 mins
\item C) 1 hour
\item D) 2 hours
\end{itemize}
Answer: B) 90 mins
\end{frame}

%%%%%%%%%%%%%%%%%%%%%%%%%%%%%%%%%%%%%%%%%%%%%%%%%%%%%%%%%%%%%%%%%%%%%%%%%%%%%%%%%
\begin{frame}[fragile]\frametitle{Question 4}
Question: Every Yoga teacher must start the practice session with
\begin{itemize}
\item A) Pranayam
\item B) Asana
\item C) Silence
\item D) Kriya
\end{itemize}
Answer: C) Silence
\end{frame}

%%%%%%%%%%%%%%%%%%%%%%%%%%%%%%%%%%%%%%%%%%%%%%%%%%%%%%%%%%%%%%%%%%%%%%%%%%%%%%%%%
\begin{frame}[fragile]\frametitle{Question 5}
Question: The purpose of Yoga as taught by the ancients is to attain:
\begin{itemize}
\item A) Attain Good health
\item B) Release Stress
\item C) Good Body
\item D) Self Realisation
\end{itemize}
Answer: D) Self Realisation
\end{frame}

%%%%%%%%%%%%%%%%%%%%%%%%%%%%%%%%%%%%%%%%%%%%%%%%%%%%%%%%%%%%%%%%%%%%%%%%%%%%%%%%%
\begin{frame}[fragile]\frametitle{Question 6}
Question: Who compiled Hatha Yoga Pradipika?
\begin{itemize}
\item A) Swami Swatmaram
\item B) Gherand
\item C) Matsyendranath
\item D) Gorakshanath
\end{itemize}
Answer: A) Swami Swatmaram
\end{frame}

%%%%%%%%%%%%%%%%%%%%%%%%%%%%%%%%%%%%%%%%%%%%%%%%%%%%%%%%%%%%%%%%%%%%%%%%%%%%%%%%%
\begin{frame}[fragile]\frametitle{Question 7}
Question: What is the Sanskrit root word for the etymological derivation of the word Yoga?
\begin{itemize}
\item A) Yuj
\item B) Yuge
\item C) Yuje
\item D) Yug
\end{itemize}
Answer: A) Yuj
\end{frame}

%%%%%%%%%%%%%%%%%%%%%%%%%%%%%%%%%%%%%%%%%%%%%%%%%%%%%%%%%%%%%%%%%%%%%%%%%%%%%%%%%
\begin{frame}[fragile]\frametitle{Question 8}
Question: The kaivalyadham institute of yoga was founded by
\begin{itemize}
\item A) Swami Kuvalayananda
\item B) Madhav Das
\item C) Yogendra
\item D) Sivananda
\end{itemize}
Answer: A) Swami Kuvalayananda
\end{frame}

%%%%%%%%%%%%%%%%%%%%%%%%%%%%%%%%%%%%%%%%%%%%%%%%%%%%%%%%%%%%%%%%%%%%%%%%%%%%%%%%%
\begin{frame}[fragile]\frametitle{Question 9}
Question: Purusha and Prakriti are the 2 main concepts of
\begin{itemize}
\item A) Yoga Darshan
\item B) Sankhya Darshan
\item C) Purva mimamsa
\item D) Uttar mimamsa
\end{itemize}
Answer: B) Sankhya Darshan
\end{frame}

%%%%%%%%%%%%%%%%%%%%%%%%%%%%%%%%%%%%%%%%%%%%%%%%%%%%%%%%%%%%%%%%%%%%%%%%%%%%%%%%%
\begin{frame}[fragile]\frametitle{Question 10}
Question: Founder of Sankhya Darshan is
\begin{itemize}
\item A) Kapil Muni
\item B) Jaimini
\item C) Kanada
\item D) Gautam
\end{itemize}
Answer: A) Kapil Muni
\end{frame}

%%%%%%%%%%%%%%%%%%%%%%%%%%%%%%%%%%%%%%%%%%%%%%%%%%%%%%%%%%%%%%%%%%%%%%%%%%%%%%%%%
\begin{frame}[fragile]\frametitle{Question 11}
Question: Concept of Pancha Kosha is mentioned in
\begin{itemize}
\item A) Mandukya Upanishad
\item B) Taittriya Upanishad
\item C) Chhandogya Upanishad
\item D) Patanjali yoga sutra
\end{itemize}
Answer: B) Taittriya Upanishad
\end{frame}

%%%%%%%%%%%%%%%%%%%%%%%%%%%%%%%%%%%%%%%%%%%%%%%%%%%%%%%%%%%%%%%%%%%%%%%%%%%%%%%%%
\begin{frame}[fragile]\frametitle{Question 12}
Question: \ldots belongs to nastik darshan
\begin{itemize}
\item A) Sankhya Darshan
\item B) Yoga Darshan
\item C) Buddhism
\item D) Both a) and b)
\end{itemize}
Answer: C) Buddhism
\end{frame}

%%%%%%%%%%%%%%%%%%%%%%%%%%%%%%%%%%%%%%%%%%%%%%%%%%%%%%%%%%%%%%%%%%%%%%%%%%%%%%%%%
\begin{frame}[fragile]\frametitle{Question 13}
Question: The means of jnana yoga is
\begin{itemize}
\item A) Shravana
\item B) Manana
\item C) Nidhidhyasana
\item D) All of the above
\end{itemize}
Answer: D) All of the above
\end{frame}

%%%%%%%%%%%%%%%%%%%%%%%%%%%%%%%%%%%%%%%%%%%%%%%%%%%%%%%%%%%%%%%%%%%%%%%%%%%%%%%%%
\begin{frame}[fragile]\frametitle{Question 14}
Question: Yoga karmasu kaushalam means
\begin{itemize}
\item A) Perfect action
\item B) Clevered action
\item C) Skilled action
\item D) Selfless action
\end{itemize}
Answer: C) Skilled action
\end{frame}

%%%%%%%%%%%%%%%%%%%%%%%%%%%%%%%%%%%%%%%%%%%%%%%%%%%%%%%%%%%%%%%%%%%%%%%%%%%%%%%%%
\begin{frame}[fragile]\frametitle{Question 15}
Question: Acquisition of wealth in order to progress towards higher pursuits of life is called
\begin{itemize}
\item A) Moksha
\item B) Dharma
\item C) Artha
\item D) Kama
\end{itemize}
Answer: C) Artha
\end{frame}


%%%%%%%%%%%%%%%%%%%%%%%%%%%%%%%%%%%%%%%%%%%%%%%%%%%%%%%%%%%%%%%%%%%%%%%%%%%%%%%%%
\begin{frame}[fragile]\frametitle{Question 16}
Question: Upanishads belongs to
\begin{itemize}
\item A) Karma kanda
\item B) Jnana Kanda
\item C) Vedas
\item D) None of the above
\end{itemize}
Answer: B) Jnana Kanda
\end{frame}

%%%%%%%%%%%%%%%%%%%%%%%%%%%%%%%%%%%%%%%%%%%%%%%%%%%%%%%%%%%%%%%%%%%%%%%%%%%%%%%%%
\begin{frame}[fragile]\frametitle{Question 17}
Question: \ldots belongs to Smriti Prasthan
\begin{itemize}
\item A) Bhagwad gita
\item B) Veda
\item C) Brahmasutra
\item D) All of the above
\end{itemize}
Answer: A) Bhagwad gita
\end{frame}

%%%%%%%%%%%%%%%%%%%%%%%%%%%%%%%%%%%%%%%%%%%%%%%%%%%%%%%%%%%%%%%%%%%%%%%%%%%%%%%%%
\begin{frame}[fragile]\frametitle{Question 18}
Question: Aham Brahmasi means
\begin{itemize}
\item A) I am god
\item B) God is me
\item C) God is in me
\item D) All the above
\end{itemize}
Answer: A) I am god
\end{frame}

%%%%%%%%%%%%%%%%%%%%%%%%%%%%%%%%%%%%%%%%%%%%%%%%%%%%%%%%%%%%%%%%%%%%%%%%%%%%%%%%%
\begin{frame}[fragile]\frametitle{Question 19}
Question: Gayatri mantra is written by
\begin{itemize}
\item A) Yadnyavalka
\item B) Maharishi Vishwamitra
\item C) Lord Krishna
\item D) Lord Shiva
\end{itemize}
Answer: B) Maharishi Vishwamitra
\end{frame}

%%%%%%%%%%%%%%%%%%%%%%%%%%%%%%%%%%%%%%%%%%%%%%%%%%%%%%%%%%%%%%%%%%%%%%%%%%%%%%%%%
\begin{frame}[fragile]\frametitle{Question 20}
Question: The characteristics of Rajasic personality of an individual are:
\begin{itemize}
\item A) Detached
\item B) Active
\item C) Lazy
\item D) Inertia
\end{itemize}
Answer: B) Active
\end{frame}

%%%%%%%%%%%%%%%%%%%%%%%%%%%%%%%%%%%%%%%%%%%%%%%%%%%%%%%%%%%%%%%%%%%%%%%%%%%%%%%%%
\begin{frame}[fragile]\frametitle{Question 21}
Question: Raja Yoga is the yoga of controlling our:
\begin{itemize}
\item A) Sense organs
\item B) Mind
\item C) Emotions
\item D) Organs of action
\end{itemize}
Answer: B) Mind
\end{frame}

%%%%%%%%%%%%%%%%%%%%%%%%%%%%%%%%%%%%%%%%%%%%%%%%%%%%%%%%%%%%%%%%%%%%%%%%%%%%%%%%%
\begin{frame}[fragile]\frametitle{Question 22}
Question: What is one word that occurs in every chapter of the Gita?
\begin{itemize}
\item A) Maya
\item B) Avidya
\item C) Yoga
\item D) Sanyasa
\end{itemize}
Answer: C) Yoga
\end{frame}

%%%%%%%%%%%%%%%%%%%%%%%%%%%%%%%%%%%%%%%%%%%%%%%%%%%%%%%%%%%%%%%%%%%%%%%%%%%%%%%%%
\begin{frame}[fragile]\frametitle{Question 23}
Question: A student is not able to practice as per your satisfaction. How will you deal with this student?
\begin{itemize}
\item A) Force the student by applying physical pressure on the body
\item B) Encourage the student to have patience and continue to practice
\item C) Ask him not to practice at all
\item D) Ignore the student
\end{itemize}
Answer: B) Encourage the student to have patience and continue to practice
\end{frame}

%%%%%%%%%%%%%%%%%%%%%%%%%%%%%%%%%%%%%%%%%%%%%%%%%%%%%%%%%%%%%%%%%%%%%%%%%%%%%%%%%
\begin{frame}[fragile]\frametitle{Question 24}
Question: Dress for yoga practice should be
\begin{itemize}
\item A) Tight fitting jeans
\item B) Costly and sophisticated dresses
\item C) Loose fitting and comfortable
\item D) Cotton Saree
\end{itemize}
Answer: C) Loose fitting and comfortable
\end{frame}

%%%%%%%%%%%%%%%%%%%%%%%%%%%%%%%%%%%%%%%%%%%%%%%%%%%%%%%%%%%%%%%%%%%%%%%%%%%%%%%%%
\begin{frame}[fragile]\frametitle{Question 25}
Question: Where does ajna chakra located
\begin{itemize}
\item A) Throat region
\item B) Navel region
\item C) Crown of the head
\item D) in between the eyebrows
\end{itemize}
Answer: D) in between the eyebrows
\end{frame}

%%%%%%%%%%%%%%%%%%%%%%%%%%%%%%%%%%%%%%%%%%%%%%%%%%%%%%%%%%%%%%%%%%%%%%%%%%%%%%%%%
\begin{frame}[fragile]\frametitle{Question 26}
Question: Who is not an acharya of hatha yoga
\begin{itemize}
\item A) Matsyendranath
\item B) Swatmaram
\item C) Gherands
\item D) Kapil muni
\end{itemize}
Answer: D) Kapil muni
\end{frame}

%%%%%%%%%%%%%%%%%%%%%%%%%%%%%%%%%%%%%%%%%%%%%%%%%%%%%%%%%%%%%%%%%%%%%%%%%%%%%%%%%
\begin{frame}[fragile]\frametitle{Question 27}
Question: Sankhya Darshan is
\begin{itemize}
\item A) Advaita
\item B) Dvaita
\item C) Dvaltadvaita
\item D) Vishishtadvaita
\end{itemize}
Answer: B) Dvaita
\end{frame}

%%%%%%%%%%%%%%%%%%%%%%%%%%%%%%%%%%%%%%%%%%%%%%%%%%%%%%%%%%%%%%%%%%%%%%%%%%%%%%%%%
\begin{frame}[fragile]\frametitle{Question 28}
Question: Yogabhasya is a commentary on Patanjali sutra or yoga sutra is written by
\begin{itemize}
\item A) Patanjali
\item B) Vyas
\item C) Sankara
\item D) Raja Bhoj
\end{itemize}
Answer: B) Vyas
\end{frame}

%%%%%%%%%%%%%%%%%%%%%%%%%%%%%%%%%%%%%%%%%%%%%%%%%%%%%%%%%%%%%%%%%%%%%%%%%%%%%%%%%
\begin{frame}[fragile]\frametitle{Question 29}
Question: Which of the below is not an authentic text on hatha yoga
\begin{itemize}
\item A) Hatha Yoga Pradipika
\item B) Gherand Samhita
\item C) Hatharatnavali
\item D) Patanjali Yoga Sutra
\end{itemize}
Answer: D) Patanjali Yoga Sutra
\end{frame}

%%%%%%%%%%%%%%%%%%%%%%%%%%%%%%%%%%%%%%%%%%%%%%%%%%%%%%%%%%%%%%%%%%%%%%%%%%%%%%%%%
\begin{frame}[fragile]\frametitle{Question 30}
Question: In Sanskrit Yoga Means
\begin{itemize}
\item A) To multiply
\item B) To understand
\item C) To unite
\item D) All of the above
\end{itemize}
Answer: C) To unite
\end{frame}

%%%%%%%%%%%%%%%%%%%%%%%%%%%%%%%%%%%%%%%%%%%%%%%%%%%%%%%%%%%%%%%%%%%%%%%%%%%%%%%%%
\begin{frame}[fragile]\frametitle{Question 31}
Question: According to Bhagwad Gita the main forms of Yoga are?
\begin{itemize}
\item A) Laya, Hatha, Raja
\item B) Mantra, Tantra, Yantra
\item C) Karma, Jnana, Bhakti
\item D) None of the above
\end{itemize}
Answer: C) Karma, Jnana, Bhakti
\end{frame}

%%%%%%%%%%%%%%%%%%%%%%%%%%%%%%%%%%%%%%%%%%%%%%%%%%%%%%%%%%%%%%%%%%%%%%%%%%%%%%%%%
\begin{frame}[fragile]\frametitle{Question 32}
Question: Which one of these is not a "Purushartha"?
\begin{itemize}
\item A) Dharma
\item B) Kama
\item C) Moksha
\item D) Samadhi
\end{itemize}
Answer: D) Samadhi
\end{frame}

%%%%%%%%%%%%%%%%%%%%%%%%%%%%%%%%%%%%%%%%%%%%%%%%%%%%%%%%%%%%%%%%%%%%%%%%%%%%%%%%%
\begin{frame}[fragile]\frametitle{Question 33}
Question: According to Bhagavad gita, Yoga is
\begin{itemize}
\item A) Yagaschitta vritti nirodhah
\item B) Yoga karmasu kaushalam
\item C) Manoprasamana upayah Yogan
\item D) Yogena chittasya padena vacham
\end{itemize}
Answer: B) Yoga karmasu kaushalam
\end{frame}

%%%%%%%%%%%%%%%%%%%%%%%%%%%%%%%%%%%%%%%%%%%%%%%%%%%%%%%%%%%%%%%%%%%%%%%%%%%%%%%%%
\begin{frame}[fragile]\frametitle{Question 34}
Question: What is the key feature of Yoga Practice?
\begin{itemize}
\item A) Only maintaining posture
\item B) Only breathing in and out during practice
\item C) Awareness
\item D) Normal breathing
\end{itemize}
Answer: C) Awareness
\end{frame}

%%%%%%%%%%%%%%%%%%%%%%%%%%%%%%%%%%%%%%%%%%%%%%%%%%%%%%%%%%%%%%%%%%%%%%%%%%%%%%%%%
\begin{frame}[fragile]\frametitle{Question 35}
Question: Which of these is not one of the main four vedas?
\begin{itemize}
\item A) Rig Veda
\item B) Sama Veda
\item C) Atharva Veda
\item D) Dhanurveda
\end{itemize}
Answer: D) Dhanurveda
\end{frame}

%%%%%%%%%%%%%%%%%%%%%%%%%%%%%%%%%%%%%%%%%%%%%%%%%%%%%%%%%%%%%%%%%%%%%%%%%%%%%%%%%
\begin{frame}[fragile]\frametitle{Question 36}
Question: Which of the following are important during the practice of asanas?
\begin{itemize}
\item A) Stability
\item B) Comfort
\item C) Effortlessness
\item D) All the above
\end{itemize}
Answer: D) All the above
\end{frame}

%%%%%%%%%%%%%%%%%%%%%%%%%%%%%%%%%%%%%%%%%%%%%%%%%%%%%%%%%%%%%%%%%%%%%%%%%%%%%%%%%
\begin{frame}[fragile]\frametitle{Question 37}
Question: From where has the whole knowledge of yoga believed to be originated?
\begin{itemize}
\item A) Upanishads
\item B) Vedas
\item C) Darshanas
\item D) Buddhism
\end{itemize}
Answer: B) Vedas
\end{frame}

%%%%%%%%%%%%%%%%%%%%%%%%%%%%%%%%%%%%%%%%%%%%%%%%%%%%%%%%%%%%%%%%%%%%%%%%%%%%%%%%%
\begin{frame}[fragile]\frametitle{Question 38}
Question: Karma yoga is a branch of yoga based on the teachings of which text?
\begin{itemize}
\item A) Patanjali Yoga sutras
\item B) Bhagavad Gita
\item C) Hatha pradipika
\item D) Upanishads
\end{itemize}
Answer: B) Bhagavad Gita
\end{frame}

%%%%%%%%%%%%%%%%%%%%%%%%%%%%%%%%%%%%%%%%%%%%%%%%%%%%%%%%%%%%%%%%%%%%%%%%%%%%%%%%%
\begin{frame}[fragile]\frametitle{Question 39}
Question: What is symbol of Ishwar as per Patanjal Yogasutra?
\begin{itemize}
\item A) Parmeshwar
\item B) Pranav
\item C) Purush
\item D) Samachi
\end{itemize}
Answer: B) Pranav
\end{frame}

%%%%%%%%%%%%%%%%%%%%%%%%%%%%%%%%%%%%%%%%%%%%%%%%%%%%%%%%%%%%%%%%%%%%%%%%%%%%%%%%%
\begin{frame}[fragile]\frametitle{Question 40}
Question: Samalam Yoga Uchyate definition is given in
\begin{itemize}
\item A) Shrimadbhagavatgia
\item B) Gherandsamhita
\item C) Yoga Sutra
\item D) Hathayog Pradeepika
\end{itemize}
Answer: A) Shrimadbhagavatgia
\end{frame}

%%%%%%%%%%%%%%%%%%%%%%%%%%%%%%%%%%%%%%%%%%%%%%%%%%%%%%%%%%%%%%%%%%%%%%%%%%%%%%%%%
\begin{frame}[fragile]\frametitle{Question 41}
Question: \ldots may be a best medicine for mental illnesses.
\begin{itemize}
\item A) Prayer
\item B) Exercise
\item C) Psychiatric drugs
\item D) Stimulating diet
\end{itemize}
Answer: A) Prayer
\end{frame}

%%%%%%%%%%%%%%%%%%%%%%%%%%%%%%%%%%%%%%%%%%%%%%%%%%%%%%%%%%%%%%%%%%%%%%%%%%%%%%%%%
\begin{frame}[fragile]\frametitle{Question 42}
Question: Which taste is recommended in yogic diet?
\begin{itemize}
\item A) Sweet
\item B) Sour
\item C) Bitter
\item D) Pungent
\end{itemize}
Answer: A) Sweet
\end{frame}

%%%%%%%%%%%%%%%%%%%%%%%%%%%%%%%%%%%%%%%%%%%%%%%%%%%%%%%%%%%%%%%%%%%%%%%%%%%%%%%%%
\begin{frame}[fragile]\frametitle{Question 43}
Question: Kapalabhati is named as in Gheranda Samhita
\begin{itemize}
\item A) Shudhkarma
\item B) Vatkrama
\item C) Bhalabhati
\item D) Kapalabhati
\end{itemize}
Answer: C) Bhalabhati
\end{frame}

%%%%%%%%%%%%%%%%%%%%%%%%%%%%%%%%%%%%%%%%%%%%%%%%%%%%%%%%%%%%%%%%%%%%%%%%%%%%%%%%%
\begin{frame}[fragile]\frametitle{Question 44}
Question: Which pranayama has the power of overcoming hunger and thirst?
\begin{itemize}
\item A) Bhastrika
\item B) Sheetali
\item C) Ujjayi
\item D) Plavini
\end{itemize}
Answer: D) Plavini
\end{frame}

%%%%%%%%%%%%%%%%%%%%%%%%%%%%%%%%%%%%%%%%%%%%%%%%%%%%%%%%%%%%%%%%%%%%%%%%%%%%%%%%%
\begin{frame}[fragile]\frametitle{Question 45}
Question: Which of the mentioned element corresponds to 'Sparsha"?
\begin{itemize}
\item A) Earth
\item B) Water
\item C) Air
\item D) Fire
\end{itemize}
Answer: C) Air
\end{frame}

%%%%%%%%%%%%%%%%%%%%%%%%%%%%%%%%%%%%%%%%%%%%%%%%%%%%%%%%%%%%%%%%%%%%%%%%%%%%%%%%%
\begin{frame}[fragile]\frametitle{Question 46}
Question: Which of the following Asana is good for alleviating the problems of throat, ear and nose?
\begin{itemize}
\item A) Swastikasana
\item B) Kurmasana
\item C) Simhasana
\item D) Padmasana
\end{itemize}
Answer: C) Simhasana
\end{frame}

%%%%%%%%%%%%%%%%%%%%%%%%%%%%%%%%%%%%%%%%%%%%%%%%%%%%%%%%%%%%%%%%%%%%%%%%%%%%%%%%%
\begin{frame}[fragile]\frametitle{Question 47}
Question: Which of the following is a 'Heating pranayama'?
\begin{itemize}
\item A) Sitali
\item B) Nadi shodhana
\item C) Bhastrika
\item D) None of the above
\end{itemize}
Answer: C) Bhastrika
\end{frame}

%%%%%%%%%%%%%%%%%%%%%%%%%%%%%%%%%%%%%%%%%%%%%%%%%%%%%%%%%%%%%%%%%%%%%%%%%%%%%%%%%
\begin{frame}[fragile]\frametitle{Question 48}
Question: What should be the lesson plan for the day's session?
\begin{itemize}
\item A) Rigid
\item B) Flexible
\item C) Spontaneous
\item D) Planned but Flexible
\end{itemize}
Answer: D) Planned but Flexible
\end{frame}

%%%%%%%%%%%%%%%%%%%%%%%%%%%%%%%%%%%%%%%%%%%%%%%%%%%%%%%%%%%%%%%%%%%%%%%%%%%%%%%%%
\begin{frame}[fragile]\frametitle{Question 49}
Question: Limitations of a practice should be taught when?
\begin{itemize}
\item A) Before the practice is introduced
\item B) After the practice is done
\item C) After few days of practice
\item D) Not necessary to teach the limitations
\end{itemize}
Answer: A) Before the practice is introduced
\end{frame}

%%%%%%%%%%%%%%%%%%%%%%%%%%%%%%%%%%%%%%%%%%%%%%%%%%%%%%%%%%%%%%%%%%%%%%%%%%%%%%%%%
\begin{frame}[fragile]\frametitle{Question 50}
Question: The location of Vishuddhi chakra is
\begin{itemize}
\item A) Eyebrow center
\item B) Base of throat
\item C) Heart
\item D) Navel
\end{itemize}
Answer: B) Base of throat
\end{frame}

%%%%%%%%%%%%%%%%%%%%%%%%%%%%%%%%%%%%%%%%%%%%%%%%%%%%%%%%%%%%%%%%%%%%%%%%%%%%%%%%%
\begin{frame}[fragile]\frametitle{Question 51}
Question: Hrid dhauti has types
\begin{itemize}
\item A) 3
\item B) 2
\item C) 4
\item D) 0
\end{itemize}
Answer: B) 2
\end{frame}

%%%%%%%%%%%%%%%%%%%%%%%%%%%%%%%%%%%%%%%%%%%%%%%%%%%%%%%%%%%%%%%%%%%%%%%%%%%%%%%%%
\begin{frame}[fragile]\frametitle{Question 52}
Question: In a group discussion, questions can be best entertained at what time?
\begin{itemize}
\item A) while relaxing between two practices
\item B) next day
\item C) while doing the practice
\item D) not necessary to entertain questions, keep practicing
\end{itemize}
Answer: A) while relaxing between two practices
\end{frame}

%%%%%%%%%%%%%%%%%%%%%%%%%%%%%%%%%%%%%%%%%%%%%%%%%%%%%%%%%%%%%%%%%%%%%%%%%%%%%%%%%
\begin{frame}[fragile]\frametitle{Question 53}
Question: Sukshma Vyayama are the practices to
\begin{itemize}
\item A) Warm up the body
\item B) Loosen the joints of the body
\item C) Provide flexibility
\item D) All the above
\end{itemize}
Answer: D) All the above
\end{frame}

%%%%%%%%%%%%%%%%%%%%%%%%%%%%%%%%%%%%%%%%%%%%%%%%%%%%%%%%%%%%%%%%%%%%%%%%%%%%%%%%%
\begin{frame}[fragile]\frametitle{Question 54}
Question: Worshipping God all time is
\begin{itemize}
\item A) Kirtana
\item B) Smarana
\item C) Sravana
\item D) Archana
\end{itemize}
Answer: B) Smarana
\end{frame}

%%%%%%%%%%%%%%%%%%%%%%%%%%%%%%%%%%%%%%%%%%%%%%%%%%%%%%%%%%%%%%%%%%%%%%%%%%%%%%%%%
\begin{frame}[fragile]\frametitle{Question 55}
Question: Dhanurasana is given in both Hatha Yoga pradipika and Gheranda Samhita
\begin{itemize}
\item A) yes
\item B) no
\item C) given only in Hatha Pradipika
\item D) not sure
\end{itemize}
Answer: A) yes
\end{frame}

%%%%%%%%%%%%%%%%%%%%%%%%%%%%%%%%%%%%%%%%%%%%%%%%%%%%%%%%%%%%%%%%%%%%%%%%%%%%%%%%%
\begin{frame}[fragile]\frametitle{Question 56}
Question: Concept of Triguna is very well explained in
\begin{itemize}
\item A) Patanjali Yoga sutra
\item B) Bhagwad Gita
\item C) Hatha Ratnavali
\item D) Shiv Yoga Dipika
\end{itemize}
Answer: B) Bhagwad Gita
\end{frame}

%%%%%%%%%%%%%%%%%%%%%%%%%%%%%%%%%%%%%%%%%%%%%%%%%%%%%%%%%%%%%%%%%%%%%%%%%%%%%%%%%
\begin{frame}[fragile]\frametitle{Question 57}
Question: Homeostasis is referred to
\begin{itemize}
\item A) Maintaining a constant environment in the body
\item B) Maintaining the constant environment when it is necessary
\item C) Maintaining the constant environment only during night
\item D) Maintaining the constant environment only during day
\end{itemize}
Answer: A) Maintaining a constant environment in the body
\end{frame}

%%%%%%%%%%%%%%%%%%%%%%%%%%%%%%%%%%%%%%%%%%%%%%%%%%%%%%%%%%%%%%%%%%%%%%%%%%%%%%%%%
\begin{frame}[fragile]\frametitle{Question 58}
Question: Which of the following yogic posture can be done immediately after food?
\begin{itemize}
\item A) Vajrasana
\item B) Trikonasana
\item C) Halasana
\item D) Mayurasana
\end{itemize}
Answer: A) Vajrasana
\end{frame}

%%%%%%%%%%%%%%%%%%%%%%%%%%%%%%%%%%%%%%%%%%%%%%%%%%%%%%%%%%%%%%%%%%%%%%%%%%%%%%%%%
\begin{frame}[fragile]\frametitle{Question 59}
Question: Which of the following is not a preferred teaching technique?
\begin{itemize}
\item A) Story-telling
\item B) Oral instruction
\item C) Demonstration
\item D) Correction
\end{itemize}
Answer: D) Correction
\end{frame}

%%%%%%%%%%%%%%%%%%%%%%%%%%%%%%%%%%%%%%%%%%%%%%%%%%%%%%%%%%%%%%%%%%%%%%%%%%%%%%%%%
\begin{frame}[fragile]\frametitle{Question 60}
Question: Feeling of being upset or annoyed as a result of being unable to change or achieve something is
\begin{itemize}
\item A) Conflict
\item B) Anxiety
\item C) Frustration
\item D) Depression
\end{itemize}
Answer: C) Frustration
\end{frame}

%%%%%%%%%%%%%%%%%%%%%%%%%%%%%%%%%%%%%%%%%%%%%%%%%%%%%%%%%%%%%%%%%%%%%%%%%%%%%%%%%
\begin{frame}[fragile]\frametitle{Question 61}
Question: Which Asana is good for chronic low back pain?
\begin{itemize}
\item A) Dhanurasana
\item B) Bhujangasana
\item C) Chakrasana
\item D) Halasana
\end{itemize}
Answer: B) Bhujangasana
\end{frame}
