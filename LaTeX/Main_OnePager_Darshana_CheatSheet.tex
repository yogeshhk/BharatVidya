\input{template_cheatsheet}
\usepackage{pdflscape}
\usepackage{rotating} % Provides {sideways}{sidewaysfigure}{sidewaystable} environments

\usepackage{polyglossia}
\setdefaultlanguage{sanskrit}
\setotherlanguage{english}
\usepackage{longtable}
\usepackage{fontspec}
\setmainfont{Segoe UI}

% Devanagari Fonts
\newfontfamily\devanagarifont[Script=Devanagari]{Nakula}
\newfontfamily\devanagarifontsf[Script=Devanagari]{Nakula}
\newfontfamily\devanagarifonttt[Script=Devanagari]{Nakula}
\newfontfamily\devtransl[Mapping=DevRom]{Segoe UI}

% Sharada Fonts
\newfontfamily\sharadafont[Script=Sharada]{Noto Sans Sharada}

\graphicspath{{images/}}

\usepackage{geometry}
\geometry{top=1cm, bottom=1cm, left=1cm, right=1cm}
\usepackage{multicol}
\usepackage{titlesec}

% Reduce space before and after sections
\titlespacing*{\section}{0pt}{6pt}{3pt}
\titlespacing*{\subsection}{0pt}{3pt}{2pt}


\begin{document}

\begin{center}
{\Large \textbf{भारतीय दर्शनांचे तुलनात्मक तक्ता}}\\
\end{center}

% ---------------- TOP TABLE (5 ASTIK) ----------------

\textbf{आस्तिक दर्शने (५)}

\begin{tabular}{|p{3.2cm}|p{2.4cm}|p{2.4cm}|p{2.4cm}|p{2.4cm}|p{2.4cm}|}
\hline
घटक & न्याय & वैशेषिक & मीमांसा & सांख्य & योग \\
\hline
प्रवर्तक & गौतम & कणाद & जैमिनी & कपिल & पतंजली \\
ज्ञानप्रमाण & प्र., अ., उ., श. & प्र., अ. & प्र., अ., श. & प्र., अ., श. & प्र., अ., श. \\
तत्त्वमीमांसा & तर्कवाद & अणुवाद & कर्मकांड & द्वैत & सांख्याधारित \\
अंतिम सत्य & तर्कसत्य & द्रव्य & वैदिक कर्म & पुरुष & कैवल्य \\
स्व & ज्ञाता आत्मा & नित्य आत्मा & कर्ता आत्मा & साक्षी & द्रष्टा \\
ईश्वर & मान्य & नंतर मान्य & नाही & नाही & विशेष पुरुष \\
मोक्ष & दुःखनाश & दुःखनिवृत्ती & अपवर्ग & कैवल्य & समाधी \\
पुनर्जन्म & मान्य & मान्य & मान्य & मान्य & मान्य \\
नीतिशास्त्र & विवेक & धर्म & कर्तव्य & विवेक & यम-नियम \\
सृष्टी उत्पत्ती & ईश्वर/अणू & अणू & अनादि & प्रकृती & प्रकृती \\
मानवी कर्तव्य & विवेकजीवन & धर्मपालन & यज्ञकर्म & ज्ञान & साधना \\
\hline
\end{tabular}

\vspace{0.8cm}

% ---------------- BOTTOM TABLE (1 ASTIK + 3 NASTIK) ----------------

\textbf{आस्तिक (१) व नास्तिक (३)}

\begin{tabular}{|p{3.2cm}|p{3.2cm}|p{3.2cm}|p{3.2cm}|p{3.2cm}|}
\hline
घटक & अद्वैत वेदांत & चार्वाक & बौद्ध & जैन \\
\hline
प्रकार & आस्तिक & नास्तिक & नास्तिक & नास्तिक \\
प्रवर्तक & शंकराचार्य & बृहस्पती & गौतम बुद्ध & महावीर \\
ज्ञानप्रमाण & प्र., अ., श. & फक्त प्रत्यक्ष & प्र., अ. & प्र., अ. \\
तत्त्वमीमांसा & अद्वैत & भौतिकवाद & क्षणिकवाद & अनेकांतवाद \\
अंतिम सत्य & ब्रह्म & इंद्रियसुख & निर्वाण & केवलज्ञान \\
स्व & आत्मा=ब्रह्म & देह & अनात्मा & जीव \\
ईश्वर & सगुण/निर्गुण & नाही & नाही & नाही \\
मोक्ष & ब्रह्मज्ञान & नाही & निर्वाण & मोक्ष \\
पुनर्जन्म & मान्य & नाही & मान्य & मान्य \\
नीतिशास्त्र & करुणा & भोगवाद & मध्यम मार्ग & अहिंसा \\
सृष्टी & माया & महाभूत & प्रतीत्यसमुत्पाद & अनादी \\
मानवी कर्तव्य & आत्मज्ञान & सुखभोग & दुःखनाश & संयम \\
\hline


\end{tabular}

\vspace{0.8cm}

प्रमाण: प्र (प्रत्यक्ष), अ (अनुमान), उ (उपमान), श (शब्द)
\vspace{0.8cm}

\rule{\linewidth}{0.25pt}
\scriptsize
Copyleft \textcopyleft\  Send suggestions to 
\href{http://www.yogeshkulkarni.com}{yogeshkulkarni@yahoo.com}

\end{document}
