%%%%%%%%%%%%%%%%%%%%%%%%%%%%%%%%%%%%%%%%%%%%%%%%%%%%%%%%%%%%%%%%%%%%%%%%%%%%%%%%%%
\begin{frame}[fragile]\frametitle{}
\begin{center}
{\Large Introduction to Ancient Indian Chemistry}
\end{center}
\end{frame}

%%%%%%%%%%%%%%%%%%%%%%%%%%%%%%%%%%%%%%%%%%%%%%%%%%%%%%%%%%%%%%%%%%%%%%%%%%%%%%%%%%
\begin{frame}[fragile]\frametitle{Introduction}
    
    \begin{itemize}
        \item \textbf{Early References}: Rigveda mentions medicinal uses of various plants and minerals.
        \item \textbf{Alchemical Texts}: Texts like Rasaratnakara and Rasendra Sara Sangraha discuss alchemy.
        \item \textbf{Contributions to Metallurgy}: Techniques for extraction and purification of metals were developed.
        \item \textbf{Medicinal Chemistry}: Ayurveda extensively uses herbal extracts and mineral formulations.
        \item \textbf{Chemical Processes}: Distillation, sublimation, and crystallization were known processes.
    \end{itemize}
\end{frame}

\begin{frame}
    \frametitle{Introduction (Cont'd)}
    
    \begin{itemize}
        \item \textbf{Sulfur and Mercury}: Widely used in alchemical practices.
        \item \textbf{Alchemy and Philosophy}: Alchemy was intertwined with philosophical concepts like transmutation.
        \item \textbf{Contribution to Dyes}: Indigo dye extraction techniques were developed.
        \item \textbf{Gunpowder Knowledge}: Early texts describe the preparation of gunpowder-like substances.
        \item \textbf{Astrochemistry}: Alignment of chemical practices with astrological beliefs.
    \end{itemize}
\end{frame}

\end{document}
