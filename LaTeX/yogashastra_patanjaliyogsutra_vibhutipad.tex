%%%%%%%%%%%%%%%%%%%%%%%%%%%%%%%%%%%%%%%%%%%%%%%%%%%%%%%%%%%%%%%%%%%%%%%%%%%%%%%%%%
\begin{frame}[fragile]\frametitle{}
\begin{center}
{\Large Vibhuti Pada विभूतिपाद}
\end{center}
\end{frame}


%%%%%%%%%%%%%%%%%%%%%%%%%%%%%%%%%%%%%%%%%%%%%%%%%%%%%%%%%%%
\begin{frame}[fragile]\frametitle{Introduction}


	\begin{itemize}
	\item The third chapter of the Patanjali Yoga Sutras is about the results, power, and manifestation once the union is achieved. 
		\item It is said, yogis achieve mystical powers (siddhi) due to the regular practice of yoga. However, this chapter notifies yogis that these very same powers can become a hindrance in their path to liberation. 
			\item Furthermore, it warns against the temptations of the eight siddhis or supernatural powers that a yogi can achieve in the higher levels of spiritual development.
	\item Dives deeper into the last three limbs of yoga, which are collectively known as Samyama (संयम).
	\item Patanjali explains how Samyama is used as the finer tool to remove the subtler veils of ignorance, in this chapter.
	\end{itemize}

\tiny{(Ref: Basic Introduction of Patanjali Yoga Sutras – The Best Knowledge for Yogis - Yoga Moha)}

\end{frame}


%%%%%%%%%%%%%%%%%%%%%%%%%%%%%%%%%%%%%%%%%%%%%%%%%%%%%%%%%%%
\begin{frame}[fragile]\frametitle{3-1}
\begin{sanskrit}
देशबन्धश्चित्तस्य धारणा॥१॥
\end{sanskrit}

	\begin{itemize}
	\item पदच्छेद: देश, बन्ध:, चितस्य‌, धारणा ‌॥
	\item Interpretations
		\begin{itemize}	
		\item अपने चित्त को किसी भी एक स्थान पर (नाभि, हृदय या माथे पर) बाँधना, लगाना, ठहराना, या केन्द्रित करना धारणा कहलाती है ।
		\item [HA]: Dharana Is The Mind’s (Chitta’s) Fixation On A Particular Point In Space
		\item [IT]: Concentration is the confining of the mind within a limited mental area (0bject of concentration)
		\item [VH]: Dharana-focusing is fixing of the citta- (the energy of) the field of consciousness within a focal point.
		\item [BM]: Concentration is binding the thought in one place.
		\item [SS]: Dharana is the binding of the mind to one place, object or idea.
		\item [SP]: Concentration (dharana) is holding the mind within a center of spiritual consciousness in the body, or fixing it on some divine form, either within the body or outside it.
		\item [SV]: Dharana is holding the mind on to some particular object. 
		\end{itemize}
	\end{itemize}
\end{frame}

%%%%%%%%%%%%%%%%%%%%%%%%%%%%%%%%%%%%%%%%%%%%%%%%%%%%%%%%%%%
\begin{frame}[fragile]\frametitle{3-2}
\begin{sanskrit}
तत्र प्रत्ययैकतानता ध्यानम्॥२॥
\end{sanskrit}

	\begin{itemize}
	\item पदच्छेद: तत्र, प्रत्यैक-तानता, ध्यानम् ॥
	\item Interpretations
		\begin{itemize}	
		\item जहाँ जिस स्थान पर भी धारणा का अभ्यास किया गया है, वहाँ पर उस ज्ञान या चित्त की वृत्ति की एकरूपता या उसका एक समान बने रहना ही ध्यान कहलाता है ।
		\item [HA]: In That (Dharana) The Continuous Flow Of Similar Mental Modification Is Called Dhyana Or Meditation.
		\item [IT]: Uninterrupted flow (of the mind) towards the object (chosen for meditation) is contemplation.
		\item [VH]: The single directionality of a pratyaya there (to the chosen focal point) is dhyana-mediation.
		\item [BM]: Meditation is focusing on a single conceptual flow.
		\item [SS]: Dhyana is the continuous flow of cognition toward that object.
		\item [SP]: Meditation (dhyana) is an unbroken flow of thought toward the object of concentration.
		\item [SV]: An unbroken flow of knowledge to that object is Dhyana. 
		\end{itemize}
	\end{itemize}
\end{frame}



%%%%%%%%%%%%%%%%%%%%%%%%%%%%%%%%%%%%%%%%%%%%%%%%%%%%%%%%%%%
\begin{frame}[fragile]\frametitle{3-3}
\begin{sanskrit}
तदेवार्थमात्रनिर्भासं स्वरूपशून्यमिव समाधिः॥३॥
\end{sanskrit}

	\begin{itemize}
	\item पदच्छेद: तत्, एव, अर्थ-मात्र-निर्भासम् ,स्वरुप-शून्यम् , इव, समाधि: ॥
	\item Interpretations
		\begin{itemize}	
		\item जब योगी स्वयं के स्वरूप को भूलकर केवल ध्येय (जिसका ध्यान कर रहा है) में ही लीन हो जाता है, तब योगस्थ साधक की वह अवस्था-विशेष समाधि कहलाती है ।
		\item [HA]: When The Object Of Meditation Only Shines Forth In The Mind, As Though Devoid Of The Thought Of Even The Self (Who Is Meditation) That State Is Called Samadhi Or Concentration.
		\item [IT]: The same (contemplation) when there is consciousness only of the object of meditation and not of itself (the mind) is Samadhi.
		\item [VH]: That (citta) specifically, reflecting as the object alone, as if empty of its own form, is samadhi-cognitive absorption.
		\item [BM]: Pure contemplation is meditation that illumines the object alone, as if the subject were devoid of intrinsic form.
		\item [SS]: Samadhi is the same meditation when there is the shining of the object alone, as if devoid of form.
		\item [SP]: When, in meditation, the true nature of the object shines forth, not distorted by the mind of the perceiver, that is absorption (samadhi).
		\item [SV]: When that, giving up all forms, reflects only the meaning, it is Samadhi. 
		\end{itemize}
	\end{itemize}
\end{frame}


%%%%%%%%%%%%%%%%%%%%%%%%%%%%%%%%%%%%%%%%%%%%%%%%%%%%%%%%%%%
\begin{frame}[fragile]\frametitle{3-4}
\begin{sanskrit}
त्रयमेकत्र संयमः॥४॥
\end{sanskrit}

	\begin{itemize}
	\item पदच्छेद: त्रयम् , एकत्र, संयम: ॥
	\item Interpretations
		\begin{itemize}	
		\item अष्टांग योग के अंतिम तीन अंगों (धारणा, ध्यान व समाधि) का किसी एक ही विषय में अभ्यास करना संयम कहलाता है ।
		\item [HA]: The Three Together On The Same Object Is Called Samyama.
		\item [IT]: These three taken together constitute Samyama.
		\item [VH]: The group of three (dharana, dhyana, and Samadhi) together as one is sanyama – the perfect regulation of citta.
		\item [BM]: Concentration, meditation, and pure contemplation focused on a single object constitute perfect discipline.
		\item [SS]: The practice of these three [dharana, dhyana and samadhi] upon one object is called samyama.
		\item [SP]: When these three—concentration, meditation and absorption—are brought to bear upon one subject, they are called samyama.
		\item [SV]: (These) three (when practised) in regard to one object is Samyama. 
		\end{itemize}
	\end{itemize}
\end{frame}


%%%%%%%%%%%%%%%%%%%%%%%%%%%%%%%%%%%%%%%%%%%%%%%%%%%%%%%%%%%
\begin{frame}[fragile]\frametitle{3-5}
\begin{sanskrit}
तज्जयात्प्रज्ञालोकः॥५॥
\end{sanskrit}

	\begin{itemize}
	\item पदच्छेद: तत् ,जयात् , प्रज्ञालोक: ॥
	\item Interpretations
		\begin{itemize}	
		\item संयम (धारणा-ध्यान-समाधि) पर विजय प्राप्त करने से योगी की बुद्धि समाधि परक प्रकाश से आलोकित हो जाती है।
		\item [HA]: By Mastering That, The Light Of Knowledge (Prajna) Dawns.
		\item [IT]: By mastering it (Samyama) the light of higher consciousness.
		\item [VH]: Owing to the success of that (sanyama), the brilliance of prajna-insight.
		\item [BM]: The light of wisdom comes from mastery of perfect discipline.
		\item [SS]: By mastery of samyama comes the light of knowledge.
		\item [SP]: Through mastery of samyama comes the light of knowledge.
		\item [SV]: By the conquest of that comes light of knowledge. 
		\end{itemize}
	\end{itemize}
\end{frame}



%%%%%%%%%%%%%%%%%%%%%%%%%%%%%%%%%%%%%%%%%%%%%%%%%%%%%%%%%%%
\begin{frame}[fragile]\frametitle{3-6}
\begin{sanskrit}
तस्य भूमिषु विनियोगः॥६॥
\end{sanskrit}

	\begin{itemize}
	\item पदच्छेद: तस्य, भूमिषु, विनियोग:॥
	\item Interpretations
		\begin{itemize}	
		\item संयम का विशेष अभ्यास या प्रयोग योग की अलग- अलग भूमियों या अवस्थाओं में करना चाहिए । पहले स्थूल भूमियों पर अभ्यास करके उत्तरोत्तर सूक्ष्म भूमियों पर संयम करना चाहिए ।
		\item [HA]: It (Samyama) Is To Be Applied To The Stages (Of Practice)
		\item [IT]: Its (of Samyama) use by stages.
		\item [VH]: Its (sanyama’s) application is in stages.
		\item [BM]: The practice of perfect discipline is achieved in stages.
		\item [SS]: Its practice is to be accomplished in stages.
		\item [SP]: It must be applied stage by stage.
		\item [SV]: That should be employed in stages. 
		\end{itemize}
	\end{itemize}
\end{frame}


%%%%%%%%%%%%%%%%%%%%%%%%%%%%%%%%%%%%%%%%%%%%%%%%%%%%%%%%%%%
\begin{frame}[fragile]\frametitle{3-7}
\begin{sanskrit}
त्रयमन्तरङ्गं पूर्वेभ्यः॥७॥
\end{sanskrit}

	\begin{itemize}
	\item पदच्छेद: त्रयम्, अन्तरङ्गंम् , पूर्वेभ्य: ॥
	\item Interpretations
		\begin{itemize}	
		\item साधन पाद में अष्टांग योग के प्रकरण में कहे गए यम, नियम, आसन, प्राणायाम व प्रत्याहार की अपेक्षा ये तीन धारणा, ध्यान व समाधि (जिनकी संयम परिभाषा) सम्प्रज्ञात समाधि के निकटतम साधन हैं ।
		\item [HA]: These Three Are More Intimate Practices Than The Previously Mentioned Ones.
		\item [IT]: The three are internal in relation to the preceding ones.
		\item [VH]: The group of three (dharana, dhyana, and Samadhi) is the inner limb(s) distinct from the previous (five limbs of yoga)
		\item [BM]: In contrast with the prior limbs of yoga, the final triad is internal.
		\item [SS]: These three [dharana, dhyana and Samadhi] are more internal than the preceding five limbs.
		\item [SP]: These three are more direct aids to experience than the five limbs previously described.
		\item [SV]: These three are nearer than those that precede. 
		\end{itemize}
	\end{itemize}
\end{frame}


%%%%%%%%%%%%%%%%%%%%%%%%%%%%%%%%%%%%%%%%%%%%%%%%%%%%%%%%%%%
\begin{frame}[fragile]\frametitle{3-8}
\begin{sanskrit}
तदपि बहिरङ्गं निर्बीजस्य॥८॥
\end{sanskrit}

	\begin{itemize}
	\item पदच्छेद:  तत् , अपि, बहिरङ्गं,निर्बीजस्य ॥
	\item Interpretations
		\begin{itemize}	
		\item वह धारणा, ध्यान व समाधि भी निर्बीज अथवा असम्प्रज्ञात समाधि प्राप्त करने के लिए बाहरी साधन हैं ।
		\item [HA]: That Also Is (To Be Regarded As) External In Respect Of Nirvija Or Seedless Concentration.
		\item [IT]: Even that (Sabija Samadhi) is external to the Seedless (Nirbija Samhadhi)
		\item [VH]: That (inner limb) however is an external limb of nirbija-the seedless (Samadhi-cognitive absorption)
		\item [BM]: Yet it is only an external limb of seedless contemplation.
		\item [SS]: Even these three are external to the seedless samadhi.
		\item [SP]: But even these are not direct aids to the seedless samadhi.
		\item [SV]: But even they are external to the seedless (Samadhi). 
		\end{itemize}
	\end{itemize}
\end{frame}


%%%%%%%%%%%%%%%%%%%%%%%%%%%%%%%%%%%%%%%%%%%%%%%%%%%%%%%%%%%
\begin{frame}[fragile]\frametitle{3-9}
\begin{sanskrit}
व्युत्थाननिरोधसंस्कारयोरभिभवप्रादुर्भावौ निरोधक्षणचित्तान्वयो निरोधपरिणामः॥९॥
\end{sanskrit}

	\begin{itemize}
	\item पदच्छेद: व्युत्थान, निरोध-संस्कारयो:, अभिभव-प्रादुर्भावौ, निरोध-क्षण-चित्त-अन्वय, निरोध-परिणाम: ॥
	\item व्युत्थान-चित्त की वह अवस्था जिसमें संस्कार प्रभावी होते हैं ।
	\item निरोध -वह अवस्था जिसमें संस्कार दब जाते हैं ।
	\item संस्कारयो: -विचार या घटनाओं की स्मृति या याददाश्त
	\item अभिभव -छिपना या दबना
	\item प्रादुर्भावौ-उभरना या प्रकट होना
	\item निरोध-संस्कारो के दबने की स्थिति
	\item क्षण-समय या काल
	\item चित्त -चित्त
	\item अन्वय: -मिश्रित या संयुक्त होना
	\item निरोधपरिणाम: -निरोध का परिणाम या फल है ।	
	\item Interpretations
		\begin{itemize}	
		\item चित्त निरोध की अवस्था समय में चित्त व्युत्थान और निरोध इन दोनों ही संस्कारो से संयुक्त होता है । इसे ही चित्त के निरोध का परिणाम कहा गया है ।
		\item [HA]: Suppression Of The Latencies Of Fluctuation And Appearance Of The Latencies Of Arrested State Taking Place At Every Moment Of Blankness Of The Arrested State In The Same Mind, Is The Mutation Of The Arrested State Of Mind.
		\end{itemize}
	\end{itemize}
\end{frame}

%%%%%%%%%%%%%%%%%%%%%%%%%%%%%%%%%%%%%%%%%%%%%%%%%%%%%%%%%%%
\begin{frame}[fragile]\frametitle{3-9}
\begin{sanskrit}
व्युत्थाननिरोधसंस्कारयोरभिभवप्रादुर्भावौ निरोधक्षणचित्तान्वयो निरोधपरिणामः॥९॥
\end{sanskrit}

	\begin{itemize}
	\item पदच्छेद: व्युत्थान, निरोध-संस्कारयो:, अभिभव-प्रादुर्भावौ, निरोध-क्षण-चित्त-अन्वय, निरोध-परिणाम: ॥
	\item Interpretations
		\begin{itemize}	
		\item [IT]: Nirodha Parinama is that transformation of the mind in which it becomes progressively permeated by that condition of Nirodha which intervenes momentarily between an impression which is disappearing and the impression which is taking place.
		\item [VH]: The submergence of the sanskara-subliminal activator of externalization and the emergence of the sanskara of nirodha- the act of ending (citta-vrtti) is the nirodha-parinama (nirodha-transformation) which is connected to citta- the energy field of consciousness at the moment of nirodha.
		\item [BM]: The transformation of thought leading toward its own cessation is accompanied by moments of cessation, when subliminal impression of mental distraction are overcome and those of cessation emerge in their place.
		\item [SS]: The impressions which normally arise are made to disappear by the appearance of suppressive efforts, which in turn create mental modifications. The moment of conjunction of mind and new modification is nirodha parinama.
		\item [SP]: When the vision of the lower samadhi is suppressed by an act of conscious control, so that there are no longer any thoughts or visions in the mind, that is the achievement of control of the thought-waves of the mind.
		\item [SV]: By the suppression of the disturbed modifications of the mind, and by the rise of modifications of control, the mind is said to attain the controlling modifications —following the controlling powers of the mind. 
		\end{itemize}
	\end{itemize}
\end{frame}

%%%%%%%%%%%%%%%%%%%%%%%%%%%%%%%%%%%%%%%%%%%%%%%%%%%%%%%%%%%
\begin{frame}[fragile]\frametitle{3-10}
\begin{sanskrit}
तस्य प्रशान्तवाहिता संस्कारात्॥१०॥
\end{sanskrit}

	\begin{itemize}
	\item पदच्छेद: तस्य , प्रशांत‌ , वाहिता , संस्कारात् ‌॥
	\item Interpretations
		\begin{itemize}	
		\item  निरोध संस्कारों के प्रभाव से योगी के चित्त की स्थिति पूर्णतः शान्त हो जाती है । अर्थात उस अवस्था में योगी का चित्त शान्त एवं सहज गति के साथ प्रवाहमान रहता है ।
		\item [HA]: Continuity Of The Tranquil Mind (In An Arrested State) Is Ensured By The Latent Impressions.
		\item [IT]: It’s flow becomes tranquil by repeated impression.
		\item [VH]: The calm flow of that (nirodha-parinama-tranformation) occurs due to the sanskara-subliminal activator (or nirodha)
		\item [BM]: From subliminal impression of these moments, the flow of tranquility is constant.
		\item [SS]: The flow of nirodha parinama becomes steady through habit.
		\item [SP]: When this suppression of thought-waves becomes continuous, the mind’s flow is calm.
		\item [SV]: Its flow becomes steady by habit. 
		\end{itemize}
	\end{itemize}
\end{frame}



%%%%%%%%%%%%%%%%%%%%%%%%%%%%%%%%%%%%%%%%%%%%%%%%%%%%%%%%%%%
\begin{frame}[fragile]\frametitle{3-11}
\begin{sanskrit}
सर्वार्थतैकाग्रतयोः क्षयोदयौ चित्तस्य समाधिपरिणामः॥११॥
\end{sanskrit}

	\begin{itemize}
	\item पदच्छेद:  सर्वार्थता , एकाग्रतयो: , क्षय‌ , चित्तस्य , समाधिपरिणाम: ॥
	\item सर्वार्थता-चित्त का अनेक विषयों में चलायमान होना
	\item एकाग्रतयो:- चित्त का किसी एक विषय में लगे रहना,
	\item क्षय - सर्वार्थता का नाश एवं
	\item उदयौ- एकाग्रता का प्रकट होना
	\item चित्तस्य - चित्त का
	\item समाधिपरिणाम: - समाधि लगने का परिणाम है।	
	\item Interpretations
		\begin{itemize}	
		\item चित्त की योग से अतिरिक्त जो तीन भूमियाँ है, क्षिप्त, मूढ़ और विक्षिप्त उसका नाश और एकाग्र भूमि का निरन्तर बढ़ते हुए निरुद्ध अवस्था का उदय होना, चित्त के लिए समाधि का परिणाम है ।
		\item [HA]: Diminution Of Attention To All And Sundry And Development Of One Pointedness Is Called Samadhi-Parinama Or Mutation Of The Concentrative Mind.
		\end{itemize}
	\end{itemize}
\end{frame}


%%%%%%%%%%%%%%%%%%%%%%%%%%%%%%%%%%%%%%%%%%%%%%%%%%%%%%%%%%%
\begin{frame}[fragile]\frametitle{3-11}
\begin{sanskrit}
सर्वार्थतैकाग्रतयोः क्षयोदयौ चित्तस्य समाधिपरिणामः॥११॥
\end{sanskrit}

	\begin{itemize}
	\item पदच्छेद:  सर्वार्थता , एकाग्रतयो: , क्षय‌ , चित्तस्य , समाधिपरिणाम: ॥
	\item Interpretations
		\begin{itemize}	
		\item [IT]: Samadhi transformation is the (gradual) setting of the distraction and simultaneous rising of one-pointedness.
		\item [VH]: The disappearance of the sarvatha- all-objectiveness and the arising of ekagrata-one-pointedness is Samadhi-parinama (Samadhi-tranformation) of citta-the energy friend of consciousness.
		\item [BM]: The transformation of thought towards pure contemplation occurs when concern for all external objects declines and psychic focus arises.
		\item [SS]: When there is decline in distractness and appearance of one-pointedness, then comes samadhi parinama (development in Samadhi)
		\item [SP]: When all mental distractions disappear and the mind becomes one-pointed, it enters the state called samadhi.
		\item [SV]: Taking in all sorts of objects and concentrating upon one object, these two powers being destroyed and manifested 
		\end{itemize}
	\end{itemize}
\end{frame}

%%%%%%%%%%%%%%%%%%%%%%%%%%%%%%%%%%%%%%%%%%%%%%%%%%%%%%%%%%%
\begin{frame}[fragile]\frametitle{3-12}
\begin{sanskrit}
ततः पुनः शान्तोदितौ तुल्यप्रत्ययौ चित्तस्यैकाग्रतापरिणामः॥१२॥
\end{sanskrit}

	\begin{itemize}
	\item पदच्छेद: ततः , पुनः , शान्त , उदित , तुल्य , प्रत्यय: , चितस्य , एकाग्रता , परिणाम: ॥
	\item तत: -उसके बाद (चित्त के समाधि परिणाम की स्थिति के बाद)
	\item पुनः -फिर से
	\item शान्त -शान्त एवं
	\item उदितौ-पुनः उदित होने वाले
	\item तुल्य -एक समान
	\item प्रत्ययौ-ज्ञान
	\item चित्तस्य-चित्त की
	\item एकाग्रता- एकाग्रता का
	\item परिणाम:- परिणाम है ।	
	\item Interpretations
		\begin{itemize}	
		\item चित्त के समाधि परिणाम के बाद एक वस्तु या ध्येय विषयक पूर्व ज्ञान शांत होकर पुनः उसी समान ज्ञान का उदय होना चित्त का एकाग्रता परिणाम होता है । इसमें ध्येय का ज्ञान बीच में खंडित होने पर भी अखंडित महसूस होता है ।
		\item [HA]: There (In Samadhi) Again (In The State Of Concentration) The Past And The Present Modifications Being Similar It Is Ekagrata-Parinama Or Mutation Of The Stabilised State Of Mind.
		\end{itemize}
	\end{itemize}
\end{frame}

%%%%%%%%%%%%%%%%%%%%%%%%%%%%%%%%%%%%%%%%%%%%%%%%%%%%%%%%%%%
\begin{frame}[fragile]\frametitle{3-12}
\begin{sanskrit}
ततः पुनः शान्तोदितौ तुल्यप्रत्ययौ चित्तस्यैकाग्रतापरिणामः॥१२॥
\end{sanskrit}

	\begin{itemize}
	\item पदच्छेद: ततः , पुनः , शान्त , उदित , तुल्य , प्रत्यय: , चितस्य , एकाग्रता , परिणाम: ॥
	\item Interpretations
		\begin{itemize}	
		\item [IT]: Then, again, the condition of the mind in which the ‘object’ (in the mind) which is always exactly similar to the ‘object’ which rises (in the next moment) is called Ekagrata Parinama.
		\item [VH]: Then again, when the santa-quieted and the udita-arisen are the same pratyaya, there is ekagrata-parinima-one-pointedness transformation of the citta-the field.
		\item [BM]: The transformation of thought towards psychic focus occurs when a concept is equally at rest or arising.
		\item [SS]: Then again, when subsiding past and rsing present images are identical, there is ekagrata parinam (one pointedness).
		\item [SP]: The mind becomes one-pointed when similar thoughtwaves arise in succession without any gaps between them.
		\item [SV]: The one-pointedness of the Chitta is when it grasps in one, the past and present.
		\end{itemize}
	\end{itemize}
\end{frame}

%%%%%%%%%%%%%%%%%%%%%%%%%%%%%%%%%%%%%%%%%%%%%%%%%%%%%%%%%%%
\begin{frame}[fragile]\frametitle{3-13}
\begin{sanskrit}
एतेन भूतेन्द्रियेषु धर्मलक्षणावस्थापरिणामा व्याख्याताः॥१३॥
\end{sanskrit}

	\begin{itemize}
	\item पदच्छेद: एतेन, भूत-इन्द्रियेषु , धर्म-लक्षण-अवस्था-परिणामा, व्याख्याताः ॥
	\item एतेन- चित्त के परिणामों से
	\item भूत-पांच महाभूतों एवं
	\item इन्द्रियेषु-इन्द्रियों अर्थात आँख, नाक, कान आदि में
	\item धर्मलक्षणावस्थापरिणामा-धर्म परिणाम, लक्षण परिणाम, व अवस्था परिणाम
	\item व्याख्याता-समझने चाहिए।	
	\item Interpretations
		\begin{itemize}	
		\item जिस प्रकार पूर्व के सूत्रों में चित्त के जो निरोध, समाधि व एकाग्रता परिणाम महर्षि द्वारा कहे गए हैं, उसी प्रकार सभी पंच महाभूतों व इन्द्रियों में होने वाले धर्म परिणाम, लक्षण परिणाम व अवस्था परिणाम भी समझे जाने चाहिए।
		\item [HA]: By These Are Explained The Three Changes, Viz. Of Essential Attributes Or Characteristics, Of Temporal Characters, And Of States Of The Bhutas And The Indriyas (i.e. All The Knowable Phenomena)
		\end{itemize}
	\end{itemize}
\end{frame}

%%%%%%%%%%%%%%%%%%%%%%%%%%%%%%%%%%%%%%%%%%%%%%%%%%%%%%%%%%%
\begin{frame}[fragile]\frametitle{3-13}
\begin{sanskrit}
एतेन भूतेन्द्रियेषु धर्मलक्षणावस्थापरिणामा व्याख्याताः॥१३॥
\end{sanskrit}

	\begin{itemize}
	\item पदच्छेद: एतेन, भूत-इन्द्रियेषु , धर्म-लक्षण-अवस्था-परिणामा, व्याख्याताः ॥
	\item Interpretations
		\begin{itemize}	
		\item [IT]: By this (by what has been said in the last four Sutras) the property, character and the sense-organs are also explained.
		\item [VH]: By this are explained the parinima-transformation of dharma-characteristic form, laksana-potential change and avastha-condition in regard to the bhuta-element and indriya-sense organs.
		\item [BM]: By extension, these transformation of thought explain the transformation of nature’s properties, characteristics, and conditions, which occur in material elements and sense organs.
		\item [SS]: By this [what has been said in the preceding Sutras], the transformation of the visible characteristics, time factors and conditions of elements and senses are also described.
		\item [SP]: In this state, it passes beyond the three kinds of changes which take place in subtle or gross matter, and in the organs: change of form, change of time and change of condition.
		\item [SV]: By this is explained the threefold transformations of form, time and state, in fine or gross matter, and in the organs. 
		\end{itemize}
	\end{itemize}
\end{frame}


%%%%%%%%%%%%%%%%%%%%%%%%%%%%%%%%%%%%%%%%%%%%%%%%%%%%%%%%%%%
\begin{frame}[fragile]\frametitle{3-14}
\begin{sanskrit}
शान्तोदिताव्यपदेश्यधर्मानुपाती धर्मी॥१४॥
\end{sanskrit}

	\begin{itemize}
	\item पदच्छेद:  शान्त-उदिता-अव्यपदेश्य-धर्मानुपाती, धर्मी ॥
	\item Interpretations
		\begin{itemize}	
		\item किसी भी पदार्थ के अतीत, वर्तमान व भविष्यत धर्मों में जो एक तत्त्व सदा विद्यमान रहता है वह धर्मी है ।
		\item [HA]: That Which Continues Its Existence All Through The Varying Characteristics, Namely The Quiscent, i.e. Past, The Uprisen, i.e. Present Or Unmanifest (But Remaining As Potent Force) i.e. Future, Is The Substratum (Or Object Characterised).
		\item [IT]: The substratum is that in which the properties – latent, active or unmanifest – inhere.
		\item [VH]: The form substratum (dharmi) conforms to the characteristic form, which may be Quieted, arisen, and indistinguishable (past, present, and future).
		\item [BM]: The substratum underlying the essential properties of material nature endures whether these properties are at rest, arising, or unmanifest.
		\item [SS]: It is the substratum (Prakriti) that by nature goes through latent, uprising and unmanifest phases.
		\item [SP]: A compound object has attributes and is subject to change, either past, present or yet to be manifested.
		\item [SV]: That which is acted upon by transformations, either past, present or yet to be manifested, is the qualified. 
		\end{itemize}
	\end{itemize}
\end{frame}

%%%%%%%%%%%%%%%%%%%%%%%%%%%%%%%%%%%%%%%%%%%%%%%%%%%%%%%%%%%
\begin{frame}[fragile]\frametitle{3-15}
\begin{sanskrit}
क्रमान्यत्वं परिणामान्यत्वे हेतुः॥१५॥
\end{sanskrit}

	\begin{itemize}
	\item पदच्छेद: क्रम-अन्यत्वं, परिणाम-अन्यत्वे, हेतु: ॥
	\item Interpretations
		\begin{itemize}	
		\item एक परिणाम का किसी अन्य परिणाम के पश्चात होना "क्रम" है । क्रम का भिन्न होना परिणाम के भिन्न भिन्न होने में कारण है ।
		\item [HA]: Change Of Sequence (Of Characteristics) Is Cause Of Mutative Differences.
		\item [IT]: The cause of the difference in transformation is the difference in the underlying process.
		\item [VH]: The separateness of the krama-sequential progression (of each citta-field) is the reason for the separateness of parinima-transformations.
		\item [BM]: Variation in the sequence of properties cause difference in the transformations of nature.
		\item [SS]: The succession of these different phases is the cause of the difference in stages of evolution.
		\item [SP]: The succession of these changes is the cause of manifold evolution.
		\item [SV]: The succession of changes is the cause of manifold evolution. 
		\end{itemize}
	\end{itemize}
\end{frame}


%%%%%%%%%%%%%%%%%%%%%%%%%%%%%%%%%%%%%%%%%%%%%%%%%%%%%%%%%%%
\begin{frame}[fragile]\frametitle{3-16}
\begin{sanskrit}
परिणामत्रयसंयमासदतीतानागतज्ञानम्॥१६॥
\end{sanskrit}

	\begin{itemize}
	\item पदच्छेद: परिणाम, त्रय, संयमाद् , अतीत-अनागत-ज्ञानम् ॥
	\item Interpretations
		\begin{itemize}	
		\item किसी भी पदार्थ के तीनों परिणाम (धर्म, लक्षण व अवस्था) में संयम (धारणा-ध्यान-समाधि) करने से उस पदार्थ के भूत एवं भविष्य काल का ज्ञान हो जाता है
		\item [HA]: Knowledge Of The Past And The Furute Can Be Derived Through Samyama On The Three Parinamas (Changes)
		\item [IT]: By performing Samyama on the three kinds of transformation (Nirodha, Samadhi and Ekagrata) knowledge of the past and future.
		\item [VH]: Due to the sanyama (perfect regulation of citta by dharma, dyana, samahdi) on the three transformations (dharma-characteristi, laksana-potential change, avastha-condition) there arises knowledge of the past and future.
		\item [BM]: Knowledge of the past and future comes form perfect discipline of the three transformations of thought.
		\item [SS]: By practicing samyama on the three stages of evolution comes knowledge of past and future.
		\item [SP]: By making samyama on the three kinds of changes, one obtains knowledge of past and the future.
		\item [SV]: By making Samyama on the three sorts of changes comes the knowledge of past and future. 
		\end{itemize}
	\end{itemize}
\end{frame}

%%%%%%%%%%%%%%%%%%%%%%%%%%%%%%%%%%%%%%%%%%%%%%%%%%%%%%%%%%%
\begin{frame}[fragile]\frametitle{3-17}
\begin{sanskrit}
शब्दार्थप्रत्ययानामितरेतराध्यासात् संकरस्तत्प्रविभागसंयमात् सर्वभूतरुतज्ञानम्॥१७॥
\end{sanskrit}

	\begin{itemize}
	\item पदच्छेद:  शब्द-अर्थ-प्रत्ययानाम्, इतरेतर, अध्यासात्‌, सङ्कर:, तत्, प्रविभाग-संयमात्, सर्व, भूत, रुत, ज्ञानम् ॥
	\item प्रत्ययानाम् - ज्ञान का
	\item इतरेतर-परस्पर या आपस में
	\item अध्यासात् -जुड़े होने से
	\item संकर: -उनका एक दूसरे के साथ संमिश्रण अर्थात वो एक दूसरे के साथ मिल जाते हैं।
	\item तत् -तब उनके अर्थात शब्द, अर्थ व ज्ञान में
	\item प्रविभाग-अलग-अलग विभाग में
	\item संयमात्-संयम (धारणा-ध्यान-समाधि) करने से	
	\item Interpretations
		\begin{itemize}	
		\item शब्द एवं उसके अर्थ एवं ज्ञान का परस्पर जुड़ाव होने से वे एक दुसरे के साथ मिले हुए जैसे प्रतीत होते हैं तब उनके विभाग (अलग-अलग) में संयम (धारणा-ध्यान-समाधि) करने से सभी प्राणियों के शब्दों का ज्ञान हो जाता है ।
		\item [HA]: Word, Object Implied And The Idea Thereof Overlapping, Produce One Unified Impression. If Samyama Is Practiced On Each Separately, Knowledge Of The Meaning Of The Sounds Produces By All Beings Can Be Acquired.
		\end{itemize}
	\end{itemize}
\end{frame}

%%%%%%%%%%%%%%%%%%%%%%%%%%%%%%%%%%%%%%%%%%%%%%%%%%%%%%%%%%%
\begin{frame}[fragile]\frametitle{3-17}
\begin{sanskrit}
शब्दार्थप्रत्ययानामितरेतराध्यासात् संकरस्तत्प्रविभागसंयमात् सर्वभूतरुतज्ञानम्॥१७॥
\end{sanskrit}

	\begin{itemize}
	\item पदच्छेद:  शब्द-अर्थ-प्रत्ययानाम्, इतरेतर, अध्यासात्‌, सङ्कर:, तत्, प्रविभाग-संयमात्, सर्व, भूत, रुत, ज्ञानम् ॥
	\item प्रत्ययानाम् - ज्ञान का
	\item इतरेतर-परस्पर या आपस में
	\item अध्यासात् -जुड़े होने से
	\item संकर: -उनका एक दूसरे के साथ संमिश्रण अर्थात वो एक दूसरे के साथ मिल जाते हैं।
	\item तत् -तब उनके अर्थात शब्द, अर्थ व ज्ञान में
	\item प्रविभाग-अलग-अलग विभाग में
	\item Interpretations
		\begin{itemize}	
		\item शब्द एवं उसके अर्थ एवं ज्ञान का परस्पर जुड़ाव होने से वे एक दुसरे के साथ मिले हुए जैसे प्रतीत होते हैं तब उनके विभाग (अलग-अलग) में संयम (धारणा-ध्यान-समाधि) करने से सभी प्राणियों के शब्दों का ज्ञान हो जाता है ।
		\item [IT]: The sound, the meaning (behind it) and the idea (which is present in the mind at the time) are present together in a confused state. By performing Samyama (on the sound) they are resolved and there arises comprehension of the meaning of sounds uttered by any living being.
		\end{itemize}
	\end{itemize}
\end{frame}

%%%%%%%%%%%%%%%%%%%%%%%%%%%%%%%%%%%%%%%%%%%%%%%%%%%%%%%%%%%
\begin{frame}[fragile]\frametitle{3-17}
\begin{sanskrit}
शब्दार्थप्रत्ययानामितरेतराध्यासात् संकरस्तत्प्रविभागसंयमात् सर्वभूतरुतज्ञानम्॥१७॥
\end{sanskrit}

	\begin{itemize}
	\item पदच्छेद:  शब्द-अर्थ-प्रत्ययानाम्, इतरेतर, अध्यासात्‌, सङ्कर:, तत्, प्रविभाग-संयमात्, सर्व, भूत, रुत, ज्ञानम् ॥
	\item प्रविभाग-अलग-अलग विभाग में
	\item Interpretations
		\begin{itemize}	
		\item [VH]: The confusion of words, meanings, and pratyaya is due to the super-imposition of one upon the other. By sanyama (perfect regulation of citta) on the inherent distinctness of these, there arises knowledge of the sound of all beings.
		\item [BM]: Confusion arises from erroneously identifying words, objects, and ideas with one another; knowledge of the cries of all creatures comes through perfect discipline of the distinctions between them
		\item [SS]: A word, its meaning, and the idea behind it are normally confused because of superimposition upon one and another. By samyam on the word [or sound] produced by any being, knowledge of its meaning is obtained.
		\item [SP]: By making samyama on the sound of a word, one’s perception of its meaning, and one’s reaction to it-three things which are ordinarily confused—one obtains understanding of all sounds uttered by living beings.
		\item [SV]: By making Samyama on word, meaning, and knowledge, which are ordinarily confused, comes the knowledge of all animal sounds. 
		\end{itemize}
	\end{itemize}
\end{frame}

%%%%%%%%%%%%%%%%%%%%%%%%%%%%%%%%%%%%%%%%%%%%%%%%%%%%%%%%%%%
\begin{frame}[fragile]\frametitle{3-18}
\begin{sanskrit}
संस्कारसाक्षत्करणात् पूर्वजातिज्ञानम्॥१८॥
\end{sanskrit}

	\begin{itemize}
	\item पदच्छेद:  संस्कार,साक्षात्,करणात्, पूर्व-जाति, ज्ञानम् ॥
	\item Interpretations
		\begin{itemize}	
		\item  योगी द्वारा अपने संस्कारों में संयम (धारणा-ध्यान और समाधि) करने से उसे अपने पूर्व जन्म के बारे में ज्ञान हो जाता है ।
		\item [HA]: By The Realisation Of Latent Impression, Knowledge Of Previous Births Is Acquired.
		\item [IT]: By direct perceptions of the impression a knowledge of previous births.
		\item [VH]: By direct perception of sanskara-subliminal activators, knowledge of previous births.
		\item [BM]: Through direct perception of one’s subliminal impression, one has knowledge of former births.
		\item [SS]: By direct perception, through samyama, of one’s mental impressions, knowledge of past births is obtained.
		\item [SP]: By making samyama on previous thought-waves, one obtains knowledge of one’s past lives.
		\item [SV]: By perceiving the impressions, knowledge of past life. 
		\end{itemize}
	\end{itemize}
\end{frame}

%%%%%%%%%%%%%%%%%%%%%%%%%%%%%%%%%%%%%%%%%%%%%%%%%%%%%%%%%%%
\begin{frame}[fragile]\frametitle{3-19}
\begin{sanskrit}
प्रत्ययस्य परचित्तज्ञानम्॥१९॥
\end{sanskrit}

	\begin{itemize}
	\item पदच्छेद:  प्रत्ययस्य, पर, चित्त-ज्ञानम् ॥
	\item Interpretations
		\begin{itemize}	
		\item योगी जब अपने से भिन्न व्यक्ति के चित्त में संयम करता है तो उसे दुसरे व्यक्ति के चित्त के भावों का ज्ञान हो जाता है।
		\item [HA]: (By Practicing Samyama) On Notions, Knowledge Of Other Minds Is Developed.
		\item [IT]: (By direct perception through Samyama) of the image occupying the mind, knowledge of the mind of others.
		\item [VH]: (By direct perception) of a pratyaya- knowledge of the citta of another.
		\item [BM]: Through direct perception of cognitive process, one has knowledge of the thoughts of others.
		\item [SS]: By samyama on the distinguishing signs of others’ bodies, knowledge of their mental images is obtained.
		\item [SP]: By making samyama on the distinguishing marks of another man’s body, one obtains knowledge of the nature of his mind.
		\item [SV]: By making Samyama on the signs in another’s both knowledge of that mind comes. 
		\end{itemize}
	\end{itemize}
\end{frame}


%%%%%%%%%%%%%%%%%%%%%%%%%%%%%%%%%%%%%%%%%%%%%%%%%%%%%%%%%%%
\begin{frame}[fragile]\frametitle{3-20}
\begin{sanskrit}
न च तत् सालम्बनं तस्याविषयीभूतत्वात्॥२०॥
\end{sanskrit}

	\begin{itemize}
	\item पदच्छेद:  न, च, तत्, सालम्बनम्, तस्य, विषयी-भूतत्वात् ॥
	\item Interpretations
		\begin{itemize}	
		\item लेकिन योगी को दुसरे व्यक्ति के चित्त के भावों का ज्ञान सामान्य ज्ञान की प्रक्रिया के आलंबन से रहित होता है क्योंकि योगी के संयम का विषय तो दुसरे व्यक्ति का केवल चित्त है।
		\item [HA]: The Prop (Or Basis) Of The Notion Does Not Get Known Because That Is Not The Object Of The (Yogin’s ) Observation.
		\item [IT]: But not also of other mental factors which support the mental image for that is not the object (of Samyama).
		\item [VH]: And it is not that (citta) together with its supporting object. Due to its (citta’s) nature of being that which has not object.
		\item [BM]: But this does not involve knowledge of the underlying object of thought since that is not the object of one’s perception.
		\item [SS]: But this does not include support in the person’s mind [such as the motive behind the thought, etc.], as that is not the object of samyama.
		\item [SP]: But not of its contents, because that is not the object of the samyama.
		\item [SV]: But not its contents, that not being the object of the Samyama. 
		\end{itemize}
	\end{itemize}
\end{frame}

%%%%%%%%%%%%%%%%%%%%%%%%%%%%%%%%%%%%%%%%%%%%%%%%%%%%%%%%%%%
\begin{frame}[fragile]\frametitle{3-21}
\begin{sanskrit}
कायरूपसंयमात् तद्ग्राह्यशक्तिस्तम्भे चक्षुःप्रकाशासंप्रयोगेऽन्तर्धानम्॥२१॥
\end{sanskrit}

	\begin{itemize}
	\item पदच्छेद:  काय-रूप-संयमात्, तत्-ग्राह्य-शक्ति-स्तम्भे, चक्षु:, प्रकाश-असम्प्रयोगे ॥
	\item ग्राह्यशक्ति -जिस प्रकाश रूपी ग्राह्यशक्ति से देखा जा रहा है
	\item स्तम्भे-उसके रुक जाने पर
	\item चक्षु: -तब आँखों का
	\item प्रकाश-रोशनी से
	\item असम्प्रयोगे-सम्बन्ध न होने के कारण
	\item अन्तर्धान-योगी अन्तर्धान हो जाता है, अर्थात योगी दिखाई देना बंद हो जाता है	
	\item Interpretations
		\begin{itemize}	
		\item जब योगी अपने शरीर के रूप में संयम (धारणा-ध्यान एवं समाधि ) का दृढ एवं सतत अभ्यास  कर लेता है तो उससे दूसरे व्यक्तियों की आँखों का रोशनी के साथ उसके शरीर का सम्बन्ध टूट जाता है, जिससे उनकी योगी के शरीर को देखने की शक्ति रुक जाती है । इस कारण योगी का शरीर दूसरों को दिखाई देना बंद हो जाता है या अदृश्य हो जाता है ।
		\item [HA]: When Perceptibility Of The Body Is Suppressed By Practicing Samyama On The Visual Character, Disappearance Of The Body Is Effected Through It’s Getting Beyond The Sphere Of Perception Of The Eye.
		\end{itemize}
	\end{itemize}
\end{frame}

%%%%%%%%%%%%%%%%%%%%%%%%%%%%%%%%%%%%%%%%%%%%%%%%%%%%%%%%%%%
\begin{frame}[fragile]\frametitle{3-22}
\begin{sanskrit}
सोपक्रमं निरुपक्रमं च कर्म तत्संयमादपरान्तज्ञानमरिष्टेभ्यो वा ॥२२॥
\end{sanskrit}

	\begin{itemize}
	\item पदच्छेद:  सोपक्रमम् , निरूपक्रमम् , च, कर्म, तत्,संयमात्,अपरान्त-ज्ञान-अरिष्टेभ्यो, वा ॥
	\item Interpretations
		\begin{itemize}	
		\item  शीघ्र या विलम्ब से फल देने वाले कर्माशयों में  संयम करने से या मृत्यु का ज्ञान कराने वाले अशुभ संकेतों से योगी को मृत्यु  का  पूर्व ज्ञान हो जाता है ।
		\item [IT]: By performing Samyama on Rupa (one of the five Tanmatras), on suspension of the receptive power, the contact between the eye (of the observer) and light (from the body) is broken and the whole body becomes invisible.
		\end{itemize}
	\end{itemize}
\end{frame}


%%%%%%%%%%%%%%%%%%%%%%%%%%%%%%%%%%%%%%%%%%%%%%%%%%%%%%%%%%%
\begin{frame}[fragile]\frametitle{3-22}
\begin{sanskrit}
सोपक्रमं निरुपक्रमं च कर्म तत्संयमादपरान्तज्ञानमरिष्टेभ्यो वा ॥२२॥
\end{sanskrit}

	\begin{itemize}
	\item पदच्छेद:  सोपक्रमम् , निरूपक्रमम् , च, कर्म, तत्,संयमात्,अपरान्त-ज्ञान-अरिष्टेभ्यो, वा ॥
	\item Interpretations
		\begin{itemize}	
		\item [VH]: By sanyama on form of the body, while suspending its ability to be seen, that is, the disconnecting of light to the eye – there arises invisibility (placement within).
		\item [BM]: From perfect discipline of the body’s own form, one can become invisible by paralyzing the power to perceive one’s body and blocking the contact of light from one’s eyes.
		\item [SS]: By samyama on the form of one’s body, [and by] checking the power of perception by intercepting light from the eyes of the observer, the body becomes invisible.
		\item [SP]: If one makes samyama on the form of one’s body obstructing its perceptibility and separating its power of manifestation from the eyes of the beholder, then one’s body becomes invisible.
		\item [SV]: By making Samyama on the form of the body the power of perceiving forms being obstructed, the power of manifestation in the eye being separated, the Yogi’s body becomes unseen. 
		\end{itemize}
	\end{itemize}
\end{frame}
% %%%%%%%%%%%%%%%%%%%%%%%%%%%%%%%%%%%%%%%%%%%%%%%%%%%%%%%%%%%
% \begin{frame}[fragile]\frametitle{3-22}
% \begin{sanskrit}
% एतेन शब्दाद्यन्तर्धानमुक्तम् ॥२२॥
% \end{sanskrit}

	% \begin{itemize}
	% \item पदच्छेद:  सोपक्रमम् , निरूपक्रमम् , च, कर्म, तत्,संयमात्,अपरान्त-ज्ञान-अरिष्टेभ्यो, वा ॥
	% \item सोपक्रमम् - शीघ्र ही फल देने वाले
	% \item निरूपक्रमम् - विलम्ब से फल देने वाले
	% \item कर्म- कर्माशय
	% \item संयमात् - संयम करने से
	% \item अपरान्त- मृत्यु का	
	% \item Interpretations
		% \begin{itemize}	
		% \item  शीघ्र या विलम्ब से फल देने वाले कर्माशयों में  संयम करने से या मृत्यु का ज्ञान कराने वाले अशुभ संकेतों से योगी को मृत्यु  का  पूर्व ज्ञान हो जाता है ।
		% \end{itemize}
	% \end{itemize}
% \end{frame}

% %%%%%%%%%%%%%%%%%%%%%%%%%%%%%%%%%%%%%%%%%%%%%%%%%%%%%%%%%%%
% \begin{frame}[fragile]\frametitle{3-22}
% \begin{sanskrit}
% एतेन शब्दाद्यन्तर्धानमुक्तम् ॥२२॥
% \end{sanskrit}

	% \begin{itemize}
	% \item पदच्छेद:  सोपक्रमम् , निरूपक्रमम् , च, कर्म, तत्,संयमात्,अपरान्त-ज्ञान-अरिष्टेभ्यो, वा ॥
	% \item Interpretations
		% \begin{itemize}	
		% \item [SS]: do. This changes the numbering of the following sutras; Taimini and are numbered one ahead. I’ve included their text with the rest of the translators for practicality in comparing the meanings, and kept their self-referential numbers intact as an aside.)
		% \item [IT]: From the above can be understood the disappearance of sound, etc.
		% \item [SS]: In the same way, the disappearance of sound [touch, taste, smell, etc] is explained.
		% \item [SP]: Thus, also, its sounds cease to be heard.
		% \item [SV]: By this the disappearance or concealment of words which are being spoken is also explained.
		% \end{itemize}
	% \end{itemize}
% \end{frame}

% %%%%%%%%%%%%%%%%%%%%%%%%%%%%%%%%%%%%%%%%%%%%%%%%%%%%%%%%%%%
% \begin{frame}[fragile]\frametitle{3-22}
% \begin{sanskrit}
% सोपक्रमं निरुपक्रमं च कर्म तत्संयमादपरान्तज्ञानमरिष्टेभ्यो वा॥२२॥
% \end{sanskrit}

	% \begin{itemize}
	% \item पदच्छेद:  मैत्री, आदिषु , बलानि ॥
	% \item Interpretations
		% \begin{itemize}	
		% \item मैत्री, करुणा, मुदिता इन तीन चित्त की प्रसन्नता करने वाले उपायों में संयम (धारणा-ध्यान और समाधि का अभ्यास करने से) इन तीनों भावों में अतिशय बल की प्राप्ति होती है ।
		% \item [HA]: Karma Is Either Fast Or Slow In Fructifying. By Practicing Samayama On Karma Or On Portents, Fore-Knowledge Of Death Can Be Acquired.
		% \item Karma is of two kinds: active and dormant; by performing Samyama on them (is gained) knowledge of the time of death; also by (performing Samyama) on portents.
		% \item [VH]: Karma is either sopakrama-with the advance of krama-sequential progression (fast in fruition) or nirupakrama-against the advance of krama (slow in fruition). The knowledge of time of death may be known by sanyama upon that or by signs.
		% \item [BM]: From perfect discipline of the immediate and remote effects of action, or of omens, one has foreknowledge of death.
		% \item [SS]: (III.23): Karmas are of two kinds: quickly manifesting and slowly manifesting. By samyama on them, or on the portents of death, the knowledge of the time of death is obtained.
		% \item [SP]: By making samyama on two kinds of karma-which will soon bear fruit and that which will not fruit until later—or by recognizing the portents of death, a yogi may know the exact time of his separation from the body.
		% \item [SV]: Karma is of two kinds, soon to be fructified, and late to be fructified. By making Samyama on that, or by the signs called Aristha, portents, the Yogis know the exact time of separation from their bodies. 
		% \end{itemize}
	% \end{itemize}
% \end{frame}


%%%%%%%%%%%%%%%%%%%%%%%%%%%%%%%%%%%%%%%%%%%%%%%%%%%%%%%%%%%
\begin{frame}[fragile]\frametitle{3-23}
\begin{sanskrit}
मैत्र्यादिषु बलानि॥२३॥
\end{sanskrit}

	\begin{itemize}
	\item पदच्छेद:  मैत्री, आदिषु , बलानि ॥
	\item Interpretations
		\begin{itemize}	
		\item मैत्री, करुणा, मुदिता इन तीन चित्त की प्रसन्नता करने वाले उपायों में संयम (धारणा-ध्यान और समाधि का अभ्यास करने से) इन तीनों भावों में अतिशय बल की प्राप्ति होती है ।
		\item [HA]: Through Samyama On Friendliness (Amity) And Other Similar Virtues, Strength Is Obtained Therein.
		\item [IT]: (24): (By performing Smayama) on friendliness etc (comes) strength (of that quality)
		\item [VH]: (By sanyama) on friendship, etc. – strengths (I.33)
		\item [BM]: From perfect discipline of friendship, compassion, joy, and impartiality, one has their strengths.
		\item [SS]: (24): By samayam on friendliness and other such qualities, the power to transmit them is obtained.
		\item [SP]: (24) By making samyama on friendliness, compassion, etc., one develops the powers of these qualities.
		\item [SV]: (24): By making Samyama on friendship, mercy etc., the yogi excels in the respective qualities.
		\end{itemize}
	\end{itemize}
\end{frame}


%%%%%%%%%%%%%%%%%%%%%%%%%%%%%%%%%%%%%%%%%%%%%%%%%%%%%%%%%%%
\begin{frame}[fragile]\frametitle{3-24}
\begin{sanskrit}
बलेषु हस्तिबलादीनि॥२४॥
\end{sanskrit}

	\begin{itemize}
	\item पदच्छेद:  बलेषु, हस्ति-बलादीनि‌ ‌॥
	\item Interpretations
		\begin{itemize}	
		\item अलग- अलग प्रकार के बलों में संयम करने से योगी को हाथी, गरुड़, वायु आदि के समान अलग- अलग प्रकार के बलों की प्राप्ति हो जाती है ।
		\item [HA]: By Practicing Samyama On (Physical) Strength, The Strength Of Elephants Etc. Can Be Acquired.
		\item [IT]: (25): (by performing Samyama) on the strengths (of animals) the strength of an elephant, etc.
		\item [VH]: On strengths – the strength of an elephant, etc.
		\item [BM]: From perfect discipline of the strength of an animal such as an elephant, one gains that strength.
		\item [SS]: (25): By samyama on the strength of elephants and other such animals, their strength is obtained.
		\item [SP]: (25) By making samyama on any kind of strength, such as that of the elephant, one obtains that strength.
		\item [SV]: (25): By making Samyama on the strength of the elephant, etc., that strength comes to the Yogi. 
		\end{itemize}
	\end{itemize}
\end{frame}

%%%%%%%%%%%%%%%%%%%%%%%%%%%%%%%%%%%%%%%%%%%%%%%%%%%%%%%%%%%
\begin{frame}[fragile]\frametitle{3-25}
\begin{sanskrit}
प्रवृत्त्यालोकन्यासात् सूक्ष्मव्यवहितविप्रकृष्टज्ञानम्॥२५॥
\end{sanskrit}

	\begin{itemize}
	\item पदच्छेद:   प्रवृति-आलोक-न्यासात्-सूक्ष्म-व्यवहित-विप्रकृष्ट-ज्ञानम् , ॥
	\item Interpretations
		\begin{itemize}	
		\item जब योगी द्वारा मन की प्रकाशमय प्रवृत्ति में संयम किया जाता है । तब उसे जो वस्तु या पदार्थ दिखाई नहीं देते, जो किसी आवरण या पर्दे के अन्दर हैं या किसी दूर देश में स्थित हैं । उन सभी वस्तुओं या पदार्थों का उसे भली- भाँति ज्ञान हो जाता है ।
		\item [HA]: By Applying The Effulgeant Light Of Higher Sense-Perception (Jyotismati) Knowledge Of Subtle Objects, Or Things Obstructed From View, Or Placed At A Great Distance, Can Be Acquired.
		\item [IT]: (26): Knowledge of the small, the hidden or the distant by directing the light of superphysical faculty.
		\item [VH]: By projecting the brilliance of the pravrtti-finer activity (of citta), knowledge of the subtle, concealed, and distant.
		\item [BM]: From placing light on the minds activity, one has knowledge of that which is subtle, hidden, and distant.
		\item [SS]: (26): By samyama on the Light within, the knowledge of the subtle, hidden and remotes is obtained. [Note: subtle as atoms, hidden as treasure, remote as far distanct lands]
		\item [SP]: (26) By making samyama on the Inner Light one obtains knowledge of what is subtle, hidden, or far distant.
		\item [SV]: (26): By making Samyama on that effulgent light comes the knowledge of the fine, the obstructed, and the remote. that strength comes to the Yogi. 
		\end{itemize}
	\end{itemize}
\end{frame}

%%%%%%%%%%%%%%%%%%%%%%%%%%%%%%%%%%%%%%%%%%%%%%%%%%%%%%%%%%%
\begin{frame}[fragile]\frametitle{3-26}
\begin{sanskrit}
भुवनज्ञानं सूर्ये संयमात्॥२६॥
\end{sanskrit}

	\begin{itemize}
	\item पदच्छेद:  भुवन-ज्ञानं, सूर्ये, संयमात् ॥
	\item Interpretations
		\begin{itemize}	
		\item  शरीर से बाहर स्थित सूर्य में संयम करने से योगी को सभी लोक – लोकान्तरों का ज्ञान हो जाता है ।
		\item [HA]: (By Practicing Samyama) On The Sun (The Point Of Body Known As The Solar Entrance) The Knowledge Of The Cosmic Regions Is Acquired.
		\item [IT]: (27): Knowledge of the Solar system by performing Samyama on the Sun.
		\item [VH]: By samyama on the sun- knowledge of the worlds.
		\item [BM]: From perfect discipline of the sun, one has knowledge of the worlds.
		\item [SS]: (27): By samyama on the sun, knowledge of the entire solar system is obtained.
		\item [SP]: (27) By making samyama on the sun, one gains knowledge of the cosmic spaces.
		\item [SV]: (27): By making Samyama on the sun, (comes) the knowledge of the world. 
		\end{itemize}
	\end{itemize}
\end{frame}

%%%%%%%%%%%%%%%%%%%%%%%%%%%%%%%%%%%%%%%%%%%%%%%%%%%%%%%%%%%
\begin{frame}[fragile]\frametitle{3-27}
\begin{sanskrit}
चन्द्रे ताराव्यूहज्ञानम्॥२७॥
\end{sanskrit}

	\begin{itemize}
	\item पदच्छेद:  चन्द्रे, तारा-व्यूह-ज्ञानम् ॥
	\item Interpretations
		\begin{itemize}	
		\item चन्द्रमा अथवा चाँद में संयम करने से सभी तारों की ठीक ठीक स्थिति की जानकारी हो जाती है ।
		\item [HA]: (By Practicing Samyama) On The Moon (The Lunar Entrance) Knowledge Of The Arrangements Of The Stars Is Acquired.
		\item [IT]: (28): (By performing Samyama) on the moon knowledge concerning the arrangement of the stars.
		\item [VH]: By samyama on the moon- knowledge of the organization of the stars.
		\item [BM]: From perfect discipline of the moon, one has knowledge of the arrangements of the stars.
		\item [SS]: (28): By samyama on the moon comes knowledge of the stars’ arrangement.
		\item [SP]: (28) By making samyama on the moon, one gains knowledge of the arrangement of the stars.
		\item [SV]: (28): On the moon, (comes) the knowledge of the cluster of stars. 
		\end{itemize}
	\end{itemize}
\end{frame}

%%%%%%%%%%%%%%%%%%%%%%%%%%%%%%%%%%%%%%%%%%%%%%%%%%%%%%%%%%%
\begin{frame}[fragile]\frametitle{3-28}
\begin{sanskrit}
ध्रुवे तद्गतिज्ञानम्॥२८॥
\end{sanskrit}

	\begin{itemize}
	\item पदच्छेद:   ध्रुवे, तत्, गति-ज्ञानम् ॥
	\item Interpretations
		\begin{itemize}	
		\item ध्रुव तारे में संयम करने से अन्य तारा समूह की गति का ज्ञान हो जाता है ।
		\item [HA]: (By Practicing Samyama) On The Pole Star, Motion Of The Stars Is Known.
		\item [IT]: (29): (By performing Samyama) on the pole-star knowledge of their movements.
		\item [VH]: By samyama on the pole star – knowledge of their motion.
		\item [BM]: From perfect discipline of the polestar, one has knowledge of the movements of the stars.
		\item [SS]: (29): By samyama on the pole star comes knowledge of the stars movements.
		\item [SP]: (29) By making samyama on the polestar, one gains knowledge of the motions of the stars.
		\item [SV]: (29): On the pole star (comes) the knowledge of the motions of the stars. 
		\end{itemize}
	\end{itemize}
\end{frame}


%%%%%%%%%%%%%%%%%%%%%%%%%%%%%%%%%%%%%%%%%%%%%%%%%%%%%%%%%%%
\begin{frame}[fragile]\frametitle{3-29}
\begin{sanskrit}
नाभिचक्रे कायव्यूहज्ञानम्॥२९॥
\end{sanskrit}

	\begin{itemize}
	\item पदच्छेद:  नाभिचक्रे, काय-व्यूह-ज्ञानम् ॥
	\item Interpretations
		\begin{itemize}	
		\item पेट के मध्य में स्थित नाभिप्रदेश (नाभिचक्र) में संयम करने से योगी को पूरे शरीर की संरचना या बनावट का ज्ञान हो जाता है ।
		\item [HA]: (By Practicing Samyama) On The Navel Plexus, Knowledge Of The Composition Of The Body Is Derived.
		\item [IT]: (30): (By performing Samyama) on the navel centre knowledge of the organization of the body.
		\item [VH]: By samyama on the navel cakra – knowledge of the organization of the body.
		\item [BM]: From perfect discipline of the circle of the navel, one has knowledge of the body’s arrangement.
		\item [SS]: (30): By samyama on the navel plexus, knowledge of the body’s constitution is obtained.
		\item [SP]: (30) By making samyama on the navel, one gains knowledge of the constitution of the body.
		\item [SV]: (30): On the navel circle (comes) the knowledge of the constitution of the body. 
		\end{itemize}
	\end{itemize}
\end{frame}

%%%%%%%%%%%%%%%%%%%%%%%%%%%%%%%%%%%%%%%%%%%%%%%%%%%%%%%%%%%
\begin{frame}[fragile]\frametitle{3-30}
\begin{sanskrit}
कण्ठकूपे क्षुत्पिपासानिवृत्तिः॥३०॥
\end{sanskrit}

	\begin{itemize}
	\item पदच्छेद:  कन्ठकूपे, क्षुत्, पिपासा, निवृत्ति:‌॥
	\item Interpretations
		\begin{itemize}	
		\item  कण्ठकूप अर्थात गले में  संयम (धारणा-ध्यान और सामधि ) का दृढ एवं सतत अभ्यास करने से साधक को साधनाकाल में भूख- प्यास के ऊपर सम्पूर्ण से नियंत्रण आने लग  जाता है ।
		\item [HA]: (By Practicing Samyama) On The Trachea, Hunger And Thirst Can Be Subdued.
		\item [IT]: (31): (By performing Samyama) on the gullet the cessation of hunger and thirst
		\item [VH]: By samyama on the well of the throat – the ceasing of hunger and thirst.
		\item [BM]: From perfect discipline of the cavity of the throat, hunger and thirst are subdued.
		\item [SS]: (31): By samyama on the pit of the throat, cessation of hunger and thirst is achieved.
		\item [SP]: (31) By making samyama on the hollow of the throat, one stills hunger and thirst.
		\item [SV]: (31): On the hollow of the throat (comes) cessation of hunger. 
		\end{itemize}
	\end{itemize}
\end{frame}


%%%%%%%%%%%%%%%%%%%%%%%%%%%%%%%%%%%%%%%%%%%%%%%%%%%%%%%%%%%
\begin{frame}[fragile]\frametitle{3-31}
\begin{sanskrit}
कूर्मनाड्यां स्थैर्यम्॥३१॥
\end{sanskrit}

	\begin{itemize}
	\item पदच्छेद:  कूर्म,नाड्याम् ,स्थैर्यम् ॥
	\item Interpretations
		\begin{itemize}	
		\item कूर्म (गले से नीचे उदर स्थल में कछुए की आकार की नाड़ी होती है) नामक नाड़ी में संयम करने से योगी को स्थिरता की प्राप्ति होती है ।
		\item [HA]: Calmness Is Attained By Samyama On The Bronchial Tube.
		\item [IT]: (32): (By performing Samyama) on the Kurma-nadi steadiness.
		\item [VH]: On the tortoise duct (tortoise), steadiness.
		\item [BM]: From perfect discipline of the “tortoise vein,” one’s being becomes steady.
		\item [SS]: (32): By samyama on the kurma nadi (a subtle tortoise-shaped tube located below the throat), motionless in the meditative posture is achieved.
		\item [SP]: (32) By making samyama on the tube within the chest, one acquires absolute motionlessness.
		\item [SV]: (32): On the nerve called Kurma (comes) fixity of the body. 
		\end{itemize}
	\end{itemize}
\end{frame}


%%%%%%%%%%%%%%%%%%%%%%%%%%%%%%%%%%%%%%%%%%%%%%%%%%%%%%%%%%%
\begin{frame}[fragile]\frametitle{3-32}
\begin{sanskrit}
मूर्धज्योतिषि सिद्धदर्शनम्॥३२॥
\end{sanskrit}

	\begin{itemize}
	\item पदच्छेद:  मूर्ध-ज्योतिषि , सिद्ध-दर्शनम् ॥
	\item Interpretations
		\begin{itemize}	
		\item कपाल प्रदेश में स्थित मूर्ध ज्योति के प्रकाश में संयम करने से सिद्ध योगियों को देखने या जानने की योग्यता आती है ।
		\item [HA]: (By Practicing Samyama) On The Coronal Light, Siddhas Can Be Seen.
		\item [IT]: (33): (By performing Samyama) on the light under the crown of the head vision of perfected Beings.
		\item [VH]: On the light on the top of the head – vision of the perfected ones.
		\item [BM]: From perfect discipline of the light in the head, one gets a vision of the perfected beings.
		\item [SS]: (33): By samyama on the light at the crown of the head (sahasrara chakra), visions of masters and adepts are obtained.
		\item [SP]: (33) By making samyama on the radiance within the back of the head, one becomes able to see the celestial beings.
		\item [SV]: (33): On the light emanating from the top of the head sight of the Siddhas. 
		\end{itemize}
	\end{itemize}
\end{frame}

%%%%%%%%%%%%%%%%%%%%%%%%%%%%%%%%%%%%%%%%%%%%%%%%%%%%%%%%%%%
\begin{frame}[fragile]\frametitle{3-33}
\begin{sanskrit}
प्रातिभाद्वा सर्वम्॥३३॥
\end{sanskrit}

	\begin{itemize}
	\item पदच्छेद:  ‌प्रातिभाद् , वा, सर्वम् ॥
	\item Interpretations
		\begin{itemize}	
		\item योग के निरंतर अभ्यास से प्रातिभ नामक ज्ञान के उत्पन्न होने से योगी के भीतर अन्तःप्रज्ञा निःसृत होने से उसे सब प्रकार का ज्ञान हो जाता है ।
		\item [HA]: From Knowledge Kown As Pratibha (Intuition) Everything Becomes Known.
		\item [IT]: (34): (Knowledge of) everything from intuition.
		\item [VH]: From pratibha- the flash of illumination, all knowledge.
		\item [BM]: From intuition, one knows everything.
		\item [SS]: (34): Or, in the knowledge that dawns spontaneous enlightenment [through a life of purity], all the powers comes by themselves.
		\item [SP]: (34) All these powers of knowledge may also come to one whose mind is spontaneously enlightened through purity.
		\item [SV]: (34): Or by the power of Pratibha all knowledge. 
		\end{itemize}
	\end{itemize}
\end{frame}

%%%%%%%%%%%%%%%%%%%%%%%%%%%%%%%%%%%%%%%%%%%%%%%%%%%%%%%%%%%
\begin{frame}[fragile]\frametitle{3-34}
\begin{sanskrit}
हृदये चित्तसंवित्॥३४॥
\end{sanskrit}

	\begin{itemize}
	\item पदच्छेद:  हृदये, चित्त-संवित् ॥
	\item Interpretations
		\begin{itemize}	
		\item हृदय चक्र अर्थात हृदय प्रदेश में संयम करने से योगी अपने चित्त के स्वरूप को ठीक ठीक जान लेता है ।
		\item [HA]: (By Practicing Samyama) On The Heart, Knowledge Of The Mind Is Acquired.
		\item [IT]: (35): (By performing Samyama) on the heart, awareness of the nature of the mind.
		\item [VH]: On the heart – full knowledge of the citta- the field.
		\item [BM]: From perfect discipline of the heart, one has full consciousness of one’s thought.
		\item [SS]: (35): By samyama on the heart, the knowledge of the mind-stuff is obtained.
		\item [SP]: (35) By making samayama on the heart, one gains knowledge of the contents of the mind.
		\item [SV]: (35): In the heart, knowledge of minds. 
		\end{itemize}
	\end{itemize}
\end{frame}


%%%%%%%%%%%%%%%%%%%%%%%%%%%%%%%%%%%%%%%%%%%%%%%%%%%%%%%%%%%
\begin{frame}[fragile]\frametitle{3-35}
\begin{sanskrit}
सत्त्वपुरुषयोरत्यन्तासंकीर्णयोः प्रत्ययाविशेषो भोगः परार्थान्यस्वार्थसंयमात् पुरुषज्ञानम्॥३५॥
\end{sanskrit}

	\begin{itemize}
	\item पदच्छेद:  सत्त्व-पुरुषयो:, अत्यंत-असंकीर्णयो:, प्रत्यय-अविशेषो, भोगः, परार्थत्वात् , संयमात् , पुरुषज्ञानम् ॥
	\item Interpretations
		\begin{itemize}	
		\item बुद्धि और पुरुष दोनों स्वरुप से ही एक दुसरे से पुर्णतः भिन्न या अलग- अलग हैं । इन दोनों को अभिन्न अर्थात एक ही मानना ‘भोग’ कहलाता है । इसमें बुद्धि का विपरीत ज्ञान कारण है । इससे भिन्न जो स्वार्थ अर्थात केवल पुरुष के विषय में ही संयम करने से आत्मा का ज्ञान होता है ।
		\item [HA]: Experience (Of Pleasure Or Pain) Arises From A Conception Which Does Not Distinguish Between The Two Extremely Different Entities, Viz. Buddhisattva And Parusa. Such Experience Exists For Another (i.e. Parusa). That Is Why Through Samyama On Parusa (Who Oversees All Experience And Also Their Complete Cessation), A Knowledge Regarding Parusa Is Acquired.
		\item [IT]: (36): Experience is the result of inability to distinguish between the Parusa and the Sattva though they are absolutely distinct. Knowledge of the Parusa results from Damyama on theSelf-interest (of the Parusa) apart from another’s interest.
		\item [VH]: Experience is a pratyaya which does not distinguish sattva (guna of brightness, a primary constituent of matter) and parusa – the self as absolutely unmixed. By sanyama on what exists for its own sake (parusa) distinct from that (sattva) which exists for the other – the knowledge of parusa.		
		\end{itemize}
	\end{itemize}
\end{frame}


%%%%%%%%%%%%%%%%%%%%%%%%%%%%%%%%%%%%%%%%%%%%%%%%%%%%%%%%%%%
\begin{frame}[fragile]\frametitle{3-35}
\begin{sanskrit}
सत्त्वपुरुषयोरत्यन्तासंकीर्णयोः प्रत्ययाविशेषो भोगः परार्थान्यस्वार्थसंयमात् पुरुषज्ञानम्॥३५॥
\end{sanskrit}

	\begin{itemize}
	% \item पदच्छेद:  सत्त्व-पुरुषयो:, अत्यंत-असंकीर्णयो:, प्रत्यय-अविशेषो, भोगः, परार्थत्वात् , संयमात् , पुरुषज्ञानम् ॥
	\item Interpretations
		\begin{itemize}	

		\item [BM]: Worldly experience is caused by a failure to differentiate between the lucid quality of nature and the spirit. From perfect discipline of the distinction between spirit as the subject of itself and the lucid quality of nature as a dependent object, one gains knowledge of the spirit.
		\item [SS]: (36): The intellect and the Parusha (or Atman) are totally different, the intellect existing for the sake of Parusha, while the Parusha exists for its own sake. Not distinguishing this is the cause of all experience; and by samyama on the distinction, knowledge of the Parusha is gained.
		\item [SP]: (36) The power of enjoyment arises from a failure to discriminate between the Atman and the sattwa guna, which are totally different. The sattwa guna is merely the agent of the Atman, which is independent, existing only for its own sake. By making samyama on the independence of the Atman, one gains knowledge of the Atman.
		\item [SV]: (36): Enjoyment comes by the non-discrimination of the very distant soul and Sattva. Its actions are for another; Samyama on this gives knowledge of the Puruca. 
		\end{itemize}
	\end{itemize}
\end{frame}


%%%%%%%%%%%%%%%%%%%%%%%%%%%%%%%%%%%%%%%%%%%%%%%%%%%%%%%%%%%
\begin{frame}[fragile]\frametitle{3-36}
\begin{sanskrit}
ततः प्रातिभश्रावणवेदनादर्शास्वादवार्ता जायन्ते॥३६॥
\end{sanskrit}

	\begin{itemize}
	\item पदच्छेद:  ततः, प्रातिभ, श्रावण, वेदन, आदर्श, आस्वाद, वार्ता, जायन्ते ॥
	\item Interpretations
		\begin{itemize}	
		\item उस पुरुष में संयम करने से प्रातिभ, श्रावण, वेदन, आदर्श, आस्वाद व वार्ता जैसी सिद्धियाँ उत्पन्न होती हैं ।
		\item [HA]: Thence (From Knowledge Of Parusa) Arises Pratibha (Prescience), Sravna (Supernormal Power Of Hearing), Vedana (Supernormal Power of Touch), Adarsha (Supernormal Power Of Sight), Asvada (Supernormal Power of Taste) And Varta (Supernormal Power of Smell).
		\item [IT]: (37): Thence are produced intuitional hearing, touch, taste, and smell.
		\item [VH]: From that arises pratibha- the flash of illumination, suprasensory hearing, feeling, seeing, tasting smelling and intelligence.
		\item [BM]: This knowledge engenders intuitive forms of hearing, touch, sight, taste, and smell.
		\item [SS]: (37): From this knowledge arises superphysical hearing, touching, seeing, tasting, and smelling through spontaneous intuition.
		\item [SP]: (37) Hence one gains the knowledge due to spontaneous enlightenment, and obtains supernatural powers of hearing, touch, sight, taste and smell.
		\item [SV]: (37): From that arises the knowledge of hearing, touching, seeing, tasting, and smelling, belonging to Pratibha. 
		\end{itemize}
	\end{itemize}
\end{frame}

%%%%%%%%%%%%%%%%%%%%%%%%%%%%%%%%%%%%%%%%%%%%%%%%%%%%%%%%%%%
\begin{frame}[fragile]\frametitle{3-37}
\begin{sanskrit}
ते समाधावुपसर्गा व्युत्थाने सिद्धयः॥३७॥
\end{sanskrit}

	\begin{itemize}
	\item पदच्छेद:  ते, समाधौ, उपसर्गा:, व्युत्थाने, सिद्धय:॥
	\item Interpretations
		\begin{itemize}	
		\item वे सभी प्रातिभ, श्रावण, वेदन, आदर्श, आस्वाद और वार्ता नाम की छ: सिद्धियाँ समाधि की प्राप्ति में विध्न उत्पन्न करती हैं, लेकिन व्युत्थान दशा अर्थात जिसमें चित्त योग की स्थिति में नहीं रहता, में यह सिद्धियाँ हैं ।
		\item [HA]: They (These Powers) Are Impediments To Samadhi, But Are Acquitions IN a Normal Fluxuating State Of Mind.
		\item [IT]: (38): They are the obstacles in the way of Samadhi and powers when the mind is outward-turned.
		\item [VH]: These siddhi-attainments in the externalized state are obstacles in reference to Samadhi-cognitive absorption (nirbija-seedless).
		\item [BM]: If they become a distraction these powers of perfection are impediments to pure contemplation.
		\item [SS]: (38): These [superphysical sense] are obstacles to [nirbija] Samadhi but are siddhis (powers or accomplishments) in the worldly pursuits.
		\item [SP]: (38) They are powers in the wordly state, but they are obstacles to samadhi.
		\item [SV]: (38): These are obstacles to Samadhi; but they are powers in the worldly state. 
		\end{itemize}
	\end{itemize}
\end{frame}


%%%%%%%%%%%%%%%%%%%%%%%%%%%%%%%%%%%%%%%%%%%%%%%%%%%%%%%%%%%
\begin{frame}[fragile]\frametitle{3-38}
\begin{sanskrit}
बन्धकारणशैथिल्यात्प्रचारसंवेदनाच्च चित्तस्य परशरीरावेशः॥३८॥
\end{sanskrit}

	\begin{itemize}
	\item पदच्छेद:  निबन्ध-कारण-शैथिल्यात, प्रचार-संवेदनात् , च, चित्तस्य, पर-शरीर-आवेश:॥
	\item Interpretations
		\begin{itemize}	
		\item  कर्म के बन्धनों के शिथिल या रुकने से व मन के संस्कारों की गति का अच्छी प्रकार ज्ञान होने पर मन दूसरे व्यक्ति के शरीर में प्रविष्ट हो जाता है । यह दुसरे के शरीर में प्परवेश करने की सिद्धि है जिसे पर शरीर गमन या परकाया प्रवेश भी कहते हैं | 
		\item [HA]: When the Cause Of Bondage Gets Weakend And The Movements Of The Mind Are Known, The Mind Can Get Into Another Body.
		\item [IT]: (39): The mind can enter another’s body on relaxation of the cause of bondage and from knowledge of passages.
		\item [VH]: By relatxation of the cause of the relationship (to the body), and the experience of going forth, citta’s entry into another body.
		\item [BM]: From loosening the fetters of bondage to the body and from awareness of the body’s fluidity, one’s thought can enter into the body of another.
		\item [SS]: (39): By loosening the cause [of the bondage of the mind to the body] and by knowledge of the procedure of the mind-stuff’s functioning, entering another’s body is accomplished.
		\item [SP]: When the bonds of the mind caused by karma have been loosened, the yogi can enter into the body of another by knowledge of the operation of its nervecurrents.
		\item [SV]: (39): When the cause of bondage has become loosened, the Yogi, by his knowledge of manifestation through the organs, enters another’s body. 
		\end{itemize}
	\end{itemize}
\end{frame}


%%%%%%%%%%%%%%%%%%%%%%%%%%%%%%%%%%%%%%%%%%%%%%%%%%%%%%%%%%%
\begin{frame}[fragile]\frametitle{3-39}
\begin{sanskrit}
उदानजयाज्जलपङ्ककण्टकादिष्वसङ्ग उत्क्रान्तिश्च॥३९॥
\end{sanskrit}

	\begin{itemize}
	\item पदच्छेद:  उदान-जयात्-जल-पंक-कण्टक-आदिषु-असंग, उत्क्रान्ति: च ॥
	\item Interpretations
		\begin{itemize}	
		\item पञ्च प्राणों में से उदान नामक प्राण पर विजय प्राप्त करने के बाद साधक पर  जल, कीचड़ व काँटो का किसी भी प्रकार से प्रभाव नहीं पड़ता है ।
		\item [HA]: By Conquering The Vital Force Called Udana The Chance Of Immersion In Water Or Mud, Or Entanglement In The Thorns, Is Avoided And Exit From The Body At Will Is Assured.
		\item [IT]: (40): By mastery over Udana levitation and non-contact with water, mire, thorns etc.
		\item [VH]: Through mastery of udana- non-contact, and rising above with regard to water, mud, thorns, etc.
		\item [BM]: From mastery of the vital breath rising in the body, one does not sink into water, mud, or thorns, but rather rises above them.
		\item [SS]: (40): By mastery over the udana nerve current (the upward vital air), one accomplishes levitation over water, swamps, thorns, etc. and can leave the body at will.
		\item [SP]: (40) By controlling the nerve-currents that govern the lungs and the upper part of the body, the yogi can walk on water and swamps, or on thorns and similar objects, and he can die at will.
		\item [SV]: (40): By conquering the current called Udana the Yogi does not sink in water, or in swamps, and he can walk on thorns. 
		\end{itemize}
	\end{itemize}
\end{frame}

%%%%%%%%%%%%%%%%%%%%%%%%%%%%%%%%%%%%%%%%%%%%%%%%%%%%%%%%%%%
\begin{frame}[fragile]\frametitle{3-40}
\begin{sanskrit}
समानजयाज्ज्वलनम्॥४०॥
\end{sanskrit}

	\begin{itemize}
	\item पदच्छेद:  समान-जयात्-ज्वलनम् ॥
	\item Interpretations
		\begin{itemize}	
		\item समान नामक प्राण पर विजय प्राप्त करने से योगी के शरीर में विलक्षण तेजस्विता दिखाई देने लग जाती है।
		\item [HA]: By Conquering The Vital Force Called Samana, Effulgence Is Acquired
		\item [IT]: (41): By mastery over Samana blazing of gastric fire.
		\item [VH]: Through mastery of samana – radiance.
		\item [BM]: From mastery of the breath of fire in the belly, one acquires fiery radiance.
		\item [SS]: (41): By mastery over the samana nerve current (the equalizing vital air) comes radiance to surround the body.
		\item [SP]: (41) By controlling the force which governs the prana, he can surround himself with a blaze of light.
		\item [SV]: (41): By the conquest of the current Samana he is surrounded by blaze. 
		\end{itemize}
	\end{itemize}
\end{frame}

%%%%%%%%%%%%%%%%%%%%%%%%%%%%%%%%%%%%%%%%%%%%%%%%%%%%%%%%%%%
\begin{frame}[fragile]\frametitle{3-41}
\begin{sanskrit}
श्रोत्राकाशयोः संबन्धसंयमाद्दिव्यं श्रोत्रम्॥४१॥
\end{sanskrit}

	\begin{itemize}
	\item पदच्छेद:   श्रोत-आकाशयो:, सम्बन्ध-संयमात् -दिव्यं, श्रोतम् ॥
	\item Interpretations
		\begin{itemize}	
		\item कानों एवं आकाश के आपसी सम्बन्ध में संयम करने से योगी के दिव्य श्रोत अथवा सुनने की दिव्य क्षमता की प्राप्ति होती है । जिस शक्ति के द्वारा वह सूक्ष्म से सूक्ष्म शब्दों को भी सहजता से सुनने में सक्षम हो जाता है ।
		\item [HA]: (By Practicing Samyama) On The Relationship Between Akasa And The Power Of Hearing, Divine Sense Of Hearing Is Gained.
		\item [IT]: (42): By performing Samyama on the relation between Akasa and the ear superphysical hearing.
		\item [VH]: By samyama on the interrelation between hearing and akasa- space/ether – divine hearing.
		\item [BM]: From perfect discipline of the relation between the ear and space, one has divine hearing.
		\item [SS]: (42): By samyama on the relationship between the ear and ether, supernormal hearing becomes possible.
		\item [SP]: (42) By making samyama on the relation between the ear and the ether, one obtains supernatural powers of hearing.
		\item [SV]: (42): By making Samyama on the relation between the ear and the Akaca comes divine hearing.
		\end{itemize}
	\end{itemize}
\end{frame}


%%%%%%%%%%%%%%%%%%%%%%%%%%%%%%%%%%%%%%%%%%%%%%%%%%%%%%%%%%%
\begin{frame}[fragile]\frametitle{3-42}
\begin{sanskrit}
कायाकाशयोः संबन्धसंयमाल्लघुतूलसमापत्तेश्चाकाशगमनम्॥४२॥
\end{sanskrit}

	\begin{itemize}
	\item पदच्छेद:  काय-आकाशयो:, सम्बन्ध-‌संयमात् -लघु-तूल-समापत्ते:, च, आकाश-गमनम् ॥
	\item Interpretations
		\begin{itemize}	
		\item शरीर एवं आकाश तत्त्व के आपसी सम्बन्ध में संयम करने और रुई जैसे बहुत ही हल्के पदार्थों में संयम करने से योगी का शरीर और चित्त उन्हीं पदार्थों की तरह हलके और सूक्ष्म हो जाते हैं , जिससे उसे आकाश में गमन करने  रूप शक्ति की प्राप्त हो जाती है ।
		\item [HA]: (By Practicing Samyama) On The Relationship Between Body and Akasa And By Concentrating On The Lightness Of Cotton Wool, Passage Through The Sky Can Be Secured.
		\item [IT]: (43): By performing Samyama on the relation between the body and Akas and at the same time bringing about coalescence of the mind with light (things like cotton down) there comes the power of) passage through space.
		\end{itemize}
	\end{itemize}
\end{frame}

%%%%%%%%%%%%%%%%%%%%%%%%%%%%%%%%%%%%%%%%%%%%%%%%%%%%%%%%%%%
\begin{frame}[fragile]\frametitle{3-42}
\begin{sanskrit}
कायाकाशयोः संबन्धसंयमाल्लघुतूलसमापत्तेश्चाकाशगमनम्॥४२॥
\end{sanskrit}

	\begin{itemize}
	\item पदच्छेद:  काय-आकाशयो:, सम्बन्ध-‌संयमात् -लघु-तूल-समापत्ते:, च, आकाश-गमनम् ॥
	\item Interpretations
		\begin{itemize}	
		\item [VH]: By sanyama on the relation between the body and akasa – space/ether, and samaoatti-cognitive blending with light objects like cotton – traversing akasa.
		\item [BM]: From perfect discipline of the relation between the body and space and from contemplative poise in which the body is as light as cotton, one can mover through space.
		\item [SS]: (43): By samyama on the relationship between the body and ether, lightness of cotton fiber is attained, and thus traveling through the ether is possible.
		\item [SP]: (43) By making samyama on the relation between the body and the ether, or by acquiring through meditation the lightness of cotton fibre, the yogi can fly through the air.
		\item [SV]: (43): By making Samyama on the relation between the Akaca and the body the Yogi becoming light as cotton wool goes through the skies.
		\end{itemize}
	\end{itemize}
\end{frame}


%%%%%%%%%%%%%%%%%%%%%%%%%%%%%%%%%%%%%%%%%%%%%%%%%%%%%%%%%%%
\begin{frame}[fragile]\frametitle{3-43}
\begin{sanskrit}
बहिरकल्पिता वृत्तिर्महाविदेहा ततः प्रकाशावरणक्षयः॥४३॥
\end{sanskrit}

	\begin{itemize}
	\item पदच्छेद: बहि:-कल्पिता, वृत्ति: महाविदेहा , ततः , प्रकाश-आवरण-क्षय:
	\item Interpretations
		\begin{itemize}	
		\item शरीर के साथ सम्पर्क होते हुए भी मन का शरीर से बाहर के पदार्थों के साथ व्यापार होता है । वह महाविदेहा नामक धारणा कहलाती है । इस धारणा से ज्ञान के प्रकाश पर पड़ने वाले आवरण का नाश हो जाता है ।
		\item [HA]: When The Unimagined Conception Can Be Held Outside, i.e. Unconnected With The Body, It Is Called Mahavideha Or The Great Discarnate. By Samyam On That The Veil Over The Illumination (Of Buddhisattva) Is Removed.
		\end{itemize}
	\end{itemize}
\end{frame}

%%%%%%%%%%%%%%%%%%%%%%%%%%%%%%%%%%%%%%%%%%%%%%%%%%%%%%%%%%%
\begin{frame}[fragile]\frametitle{3-43}
\begin{sanskrit}
बहिरकल्पिता वृत्तिर्महाविदेहा ततः प्रकाशावरणक्षयः॥४३॥
\end{sanskrit}

	\begin{itemize}
	\item पदच्छेद: बहि:-कल्पिता, वृत्ति: महाविदेहा , ततः , प्रकाश-आवरण-क्षय:
	\item Interpretations
		\begin{itemize}	
		\item [VH]: The non-imaginary (actual) external vrtti activity (defining citta) is the great out-of-body state. From that, the dispersing of the covering of light.
		\item [BM]: The turning of thought without reference to the external world is called “the great disembodied thought”; from which this veil that obscures the light is destroyed.
		\item [SS]: (44): By samyama on thought waves unidentified by and external to the body [maha-videha, or the great bodilessness], the veil over the light of the Self is destroyed.
		\item [SP]: (44) By making samyama on the thought-waves of the mind when it is separated from the body—the state known as the Great Disincarnation–all coverings can be removed from the light of knowledge.
		\item [SV]: (44): By making Samyama on the real modifications of the mind, which are outside, called great disembodiness, comes disappearance of the covering to light. 
		\end{itemize}
	\end{itemize}
\end{frame}



%%%%%%%%%%%%%%%%%%%%%%%%%%%%%%%%%%%%%%%%%%%%%%%%%%%%%%%%%%%
\begin{frame}[fragile]\frametitle{3-44}
\begin{sanskrit}
स्थूलस्वरूपसूक्ष्मान्वयार्थवत्त्वसंयमाद भूतजयः॥४४॥
\end{sanskrit}

	\begin{itemize}
	\item पदच्छेद:  स्थूल-स्वरुप-सूक्ष्म-अन्वय-अर्थवत्त्व-‌संयमाद्-भूतजय:॥
	\item Interpretations
		\begin{itemize}	
		\item पृथ्वी, जल, अग्नि, वायु व आकाश ये पांच महाभूत होते हैं । पञ्च महाभूतों की स्थूल, स्वरूप, सूक्ष्म, अन्वय और अर्थवत्त्व रूप से पांच विभाग हैं । इन सभी विभागों में एक साथ या क्रम से संयम करने पर योगी सभी पञ्च महाभूतों पर विजय प्राप्त कर लेता है ।
		\item [HA]: (By Practicing Samyama) On The Grossness, The Essential Character, The Subtlety, The Inherence And The Objectiveness Which Are The Five Forms Of The Bhutas Or Elements, Mastery Over Bhutas Is Obtained.
		\end{itemize}
	\end{itemize}
\end{frame}

%%%%%%%%%%%%%%%%%%%%%%%%%%%%%%%%%%%%%%%%%%%%%%%%%%%%%%%%%%%
\begin{frame}[fragile]\frametitle{3-44}
\begin{sanskrit}
स्थूलस्वरूपसूक्ष्मान्वयार्थवत्त्वसंयमाद भूतजयः॥४४॥
\end{sanskrit}

	\begin{itemize}
	\item पदच्छेद:  स्थूल-स्वरुप-सूक्ष्म-अन्वय-अर्थवत्त्व-‌संयमाद्-भूतजय:॥
	\item Interpretations
		\begin{itemize}	
		\item [IT]: (45): Mastery over the Panca-Bhutas by performing Samyama on their gross, constant, subtle, all–pervading and functional states.
		\item [VH]: By sanyama on (their) gross state, svarupa-essential nature, subtle state, interconnectedness and nature of serving a purpose – mastery of the bhuta-elements.
		\item [BM]: From perfect discipline of the gross, intrinsic, subtle, relational, and purposive aspects of the elements of matterm one attains mastery over them.
		\item [SS]: (45): By samyama on the gross and subtle elements and on their essential nature, correlations and purpose, mastery over them is gained.
		\item [SP]: (45) By making samyama on the gross and subtle forms of the elements, on their essential characteristics and the inherence of the gunas in them, and on the experiences they provide for the individual, one gains mastery of the elements.
		\item [SV]: (45): By making Samyama on the elements, beginning with the gross, and ending with the superfine, comes mastery of the elements.
		\end{itemize}
	\end{itemize}
\end{frame}


%%%%%%%%%%%%%%%%%%%%%%%%%%%%%%%%%%%%%%%%%%%%%%%%%%%%%%%%%%%
\begin{frame}[fragile]\frametitle{3-45}
\begin{sanskrit}
ततोऽणिमादिप्रादुर्भावः कायसंपत्तद्धर्मानभिघातश्च॥४५॥
\end{sanskrit}

	\begin{itemize}
	\item पदच्छेद:  तत:,अणिमादि-प्रादुर्भाव:, काय-सम्पत्,तद् -धर्म ,अनभिघात:, च ॥
	\item Interpretations
		\begin{itemize}	
		\item पञ्च भूतों पर विजय प्राप्त करने के बाद योगी को अणिमा आदि अष्ट सिद्धियों की प्राप्ति हो जाती है ।और शरीर का विशेष सामर्थ्य बढ़ता है और साथ ही पृथ्वी आदि पञ्च भूतों के धर्म योगी को किसी भी प्रकार से बाधा नहीं पहुंचाते हैं ।
		\item [HA]: Thence Develop The Power Of Minification And Other Boduly Acquitions. There Is No Resistance By Its Characteristics.
		\item [IT]: (46): Thence, the attainment of the Animan etc., perfection of the body and the non-obstruction of its functions (of the body) by the powers (of the elements).
		\item [VH]: From that, the emergence of such powers as becoming minute, etc. perfection of the body, and non-affliction by its constituents.		
		\end{itemize}
	\end{itemize}
\end{frame}

%%%%%%%%%%%%%%%%%%%%%%%%%%%%%%%%%%%%%%%%%%%%%%%%%%%%%%%%%%%
\begin{frame}[fragile]\frametitle{3-45}
\begin{sanskrit}
ततोऽणिमादिप्रादुर्भावः कायसंपत्तद्धर्मानभिघातश्च॥४५॥
\end{sanskrit}

	\begin{itemize}
	\item पदच्छेद:  तत:,अणिमादि-प्रादुर्भाव:, काय-सम्पत्,तद् -धर्म ,अनभिघात:, च ॥
	\item Interpretations
		\begin{itemize}	

		\item [BM]: Then extraordinary powers appears, such as the power to be small as an atom, as well as bodily perfection and immunity from the constraints of matter.
		\item [SS]: (46): From that comes attainment of anima and other siddis, bodily perfection and the non-obstruction of bodily functions by the influence of the elements. [Note: the eight major siddhis alluded to here are: anima (to become very small); mahima (to become very big); laghima (very light); garima (heavy); prapti (to reach anywhere); prakamya (to achieve all one’s desires); isatva (ability to creat anything); vasitva (ability to command and control everything).]
		\item [SP]: (46) Hence one gains the power of becoming as tiny as an atom and all similar powers: also perfection of the body, which is no longer subject to the obstruction of the elements.
		\item [SV]: (46): From that comes minuteness, and the rest of the powers, “glorification of the body,” and indestructibleness of the bodily qualities. 
		\end{itemize}
	\end{itemize}
\end{frame}



%%%%%%%%%%%%%%%%%%%%%%%%%%%%%%%%%%%%%%%%%%%%%%%%%%%%%%%%%%%
\begin{frame}[fragile]\frametitle{3-46}
\begin{sanskrit}
रूपलावण्यबलवज्रसंहननत्वानि कायसंपत्॥४६॥
\end{sanskrit}

	\begin{itemize}
	\item पदच्छेद:  रूप-लावण्य-बल-वज्र-संहननत्वानि,कायसम्पत् ॥
	\item Interpretations
		\begin{itemize}	
		\item पञ्च महाभूतों में संयम करने के बाद योगी को ऐश्वर्य स्वरुप रूप, लावण्य, बल एवं सुदृढ़ शरीर रूपी शरीर की संपत्ति प्राप्त होती है । इस सिद्धि को कायसम्पत् नाम से कहा गया है
		\item [HA]: Perfection Of Body Consists In Beauty, Grace, Strength And Adamantine Hardness.
		\item [IT]: (47): Beauty, fine complexion, strength and adamantine hardness constitute the perfection of the body.
		\item [VH]: Beauty, gracefulness, strength, and adamantine firmness is the perfection of the body.
		\item [BM]: Bodily foerfection includes beauty, grace, strength, and a diamond’s hard glow.
		\item [SS]: (47): Beauty, grace, strength, and adamantine hardness constitute bodily perfection.
		\item [SP]: (47) Perfection of the body includes beauty, grace, strength and the hardness of a thunderbolt.
		\item [SV]: (47): The glorifications of the body are beauty, complexion, strength, adamantine hardness. 
		\end{itemize}
	\end{itemize}
\end{frame}



%%%%%%%%%%%%%%%%%%%%%%%%%%%%%%%%%%%%%%%%%%%%%%%%%%%%%%%%%%%
\begin{frame}[fragile]\frametitle{3-47}
\begin{sanskrit}
ग्रहणस्वरूपास्मितान्वयार्थवत्त्वसंयमादिन्द्रियजयः॥४७॥
\end{sanskrit}

	\begin{itemize}
	\item पदच्छेद:  ग्रहण-स्वरुप-अस्मिता-अन्वय-अर्थवत्त्व-संयमाद्-इन्द्रियजय:॥
	\item Interpretations
		\begin{itemize}	
		\item  इन्द्रियों की ग्रहण, स्वरूप, अस्मिता, अन्वय व अर्थवत्त्व इन पांच विभागों में संयम करने से योगी सभी इन्द्रियों पर पूर्ण विजय प्राप्त कर लेता है ।
		\item [HA]: By Samyama On The Receptivity, Essential Character, I-sense, Inherant Quality And Objectiveness Of The Five Organs, Mastery Over Them Can Be Acquired.
		\item [IT]: (48): Mastery of the sense-organs by performing Samyama on their power of cognition, realy nature, egoism, all pervasiveness and functions.
		\end{itemize}
	\end{itemize}
\end{frame}

%%%%%%%%%%%%%%%%%%%%%%%%%%%%%%%%%%%%%%%%%%%%%%%%%%%%%%%%%%%
\begin{frame}[fragile]\frametitle{3-47}
\begin{sanskrit}
ग्रहणस्वरूपास्मितान्वयार्थवत्त्वसंयमादिन्द्रियजयः॥४७॥
\end{sanskrit}

	\begin{itemize}
	\item पदच्छेद:  ग्रहण-स्वरुप-अस्मिता-अन्वय-अर्थवत्त्व-संयमाद्-इन्द्रियजय:॥
	\item Interpretations
		\begin{itemize}	
		\item [VH]: By sanyama on (their) process of perception, svarupa – essential nature, the sense of “I am”, (their) interconnectedness and nature of serving a purpose, mastery of the indriya-sense organs.
		\item [BM]: From perfect discipline of the receptive, intrinsic, egotistic, relational, and purposive functions of the sense organs, one attains mastery over them,
		\item [SS]: (48): By samyama on the power of perception and on essential nature, correlation with the ego sense and purpose of the sense organs, mastery over them is gained.
		\item [SP]: (48) By making samyama on the transformation that the sense-organs undergo when they contact objects, on the power of illumination of the sense-organs, on the ego-sense, on the gunas which constitute the organs, and on the experiences they provide for the individual, one gains mastery of the organs.
		\item [SV]: (48): By making Samyama on the objectivity, knowledge and egoism of the organs, by gradation comes the conquest of the organs. 
		\end{itemize}
	\end{itemize}
\end{frame}


%%%%%%%%%%%%%%%%%%%%%%%%%%%%%%%%%%%%%%%%%%%%%%%%%%%%%%%%%%%
\begin{frame}[fragile]\frametitle{3-48}
\begin{sanskrit}
ततो मनोजवित्वं विकरणभावः प्रधानजयश्च॥४८॥
\end{sanskrit}

	\begin{itemize}
	\item पदच्छेद:  तत:, मनोजवित्वं, विकरणभाव:, प्रधानजय, च ॥
	\item Interpretations
		\begin{itemize}	
		\item  उन इन्द्रियों पर पूर्ण रूप से विजय प्राप्त करने से योगी का शरीर मन की तरह ही तेज गति वाला, बिना शरीर के इन्द्रियों से काम करने वाला और प्रकृति से बने सभी अंगों पर अधिकार प्राप्त कर लेता है ।
		\item [HA]: Thence Come Powers Of Rapid Movement As Of The Mind, Action Of Organs Independent Of The Body And Mastery Over Pradhana, The Primordial Cause.
		\item [IT]: (49): Thence, instantaneous cognition without the use of any vehicle and complete mastery over Pradhana.
		\end{itemize}
	\end{itemize}
\end{frame}

%%%%%%%%%%%%%%%%%%%%%%%%%%%%%%%%%%%%%%%%%%%%%%%%%%%%%%%%%%%
\begin{frame}[fragile]\frametitle{3-48}
\begin{sanskrit}
ततो मनोजवित्वं विकरणभावः प्रधानजयश्च॥४८॥
\end{sanskrit}

	\begin{itemize}
	\item पदच्छेद:  तत:, मनोजवित्वं, विकरणभाव:, प्रधानजय, च ॥
	\item Interpretations
		\begin{itemize}	
		\item [VH]: From that, the swiftness of the mind, the state of transcending the sense organs, and mastery over pradhana-the primary matrix of matter (prakrti).
		\item [BM]: From this one acquires quickness of mind, perception without the aide of the senses, and mastery over primordial matter.
		\item [SS]: (49): From that, the body gains the power to move as fast as the mind, ability to function without the aid of the sense organs, and complete mastery over the primary cause (Prakriti).
		\item [SP]: (49) Hence the body gains the power of movement as rapid as that of the mind, the power of using the sense-organs outside the confines of the body, and the mastery of Prakriti.
		\item [SV]: (49): From that comes glorified mind, power of the organs independently of the body, and conquest of nature. 
		\end{itemize}
	\end{itemize}
\end{frame}


%%%%%%%%%%%%%%%%%%%%%%%%%%%%%%%%%%%%%%%%%%%%%%%%%%%%%%%%%%%
\begin{frame}[fragile]\frametitle{3-49}
\begin{sanskrit}
सत्त्वपुरुषान्यताख्यातिमात्रस्य सर्वभावाधिष्ठातृत्वं सर्वज्ञातृत्वं च॥४९॥
\end{sanskrit}

	\begin{itemize}
	\item पदच्छेद:  सत्त्व-पुरुष-अन्यता-ख्यातिमात्रस्य, सर्व-भाव-अधिष्ठातृत्वम् , सर्व-ज्ञातृत्वम् ॥
	\item Interpretations
		\begin{itemize}	
		\item बुद्धि आदि जड़ पदार्थो और आत्मा-जीवात्मा दोनों को पृथक-पृथक मानने वाले योगी का सभी वर्तमान पदार्थों पर अधिकार हो जाता है और उसे सभी पदार्थों का ज्ञान हो जाता है ।
		\item [HA]: To One Established IN Th4e Discernment Between Buddhi And Parusa Come Supremeacy Over All Beings And Omniscience.
		\item [IT]: (50): Only from awareness of the distinction between Sattva and Parusa arise supremacy over all states and forms of existence (omnipotence) and knowledge of everything (omniscience).

		\end{itemize}
	\end{itemize}
\end{frame}

%%%%%%%%%%%%%%%%%%%%%%%%%%%%%%%%%%%%%%%%%%%%%%%%%%%%%%%%%%%
\begin{frame}[fragile]\frametitle{3-49}
\begin{sanskrit}
सत्त्वपुरुषान्यताख्यातिमात्रस्य सर्वभावाधिष्ठातृत्वं सर्वज्ञातृत्वं च॥४९॥
\end{sanskrit}

	\begin{itemize}
	\item पदच्छेद:  सत्त्व-पुरुष-अन्यता-ख्यातिमात्रस्य, सर्व-भाव-अधिष्ठातृत्वम् , सर्व-ज्ञातृत्वम् ॥
	\item Interpretations
		\begin{itemize}	
		\item [VH]: Of (that citta-field), having only the identification of the separateness of the sattva guna, and parusa-self, there is supremacy over all states, and omniscience.
		\item [BM]: For one who is attentive to the distinction between the lucid perception of nature and spirit, omniscience and power over all states of existence result.
		\item [SS]: (50): By recognition of the distinction between the sattva (pure reflective nature) and the Self, supremacy over all states and forms of existence [omnipotence] is gained as is omniscience.
		\item [SP]: (50) By making samyama on the discrimination between the sattwa guna and the Atman, one gains omnipotence and omniscience.
		\item [SV]: (50): By making Samyama on the Sattva, to him who has discriminated between the intellect and the Purusa comes omnipresence and omniscience. 
		\end{itemize}
	\end{itemize}
\end{frame}

%%%%%%%%%%%%%%%%%%%%%%%%%%%%%%%%%%%%%%%%%%%%%%%%%%%%%%%%%%%
\begin{frame}[fragile]\frametitle{3-50}
\begin{sanskrit}
तद्वैराग्यादपि दोषबीजक्षये कैवल्यम्॥५०॥
\end{sanskrit}

	\begin{itemize}
	\item पदच्छेद:  तद् , अपि-वैराग्यात्-अपि , दोष-बीजक्षये,कैवल्यम् ॥
	\item Interpretations
		\begin{itemize}	
		\item योगी को विशोका नामक सिद्धि से भी वैराग्य हो जाने पर अविद्या आदि पञ्च क्लेशों का नाश हो जाता है । जिससे योगी जीवन मुक्ति हो जाती है ।
		\item [HA]: By Renunciation Of That (Visoka Attainment) Even, Comes Liberation On Account Of The Destruction Of The Seeds Of Evil.
		\item [IT]: (51): By non-attachment even to that, on the very seed of bondage being destroyed, follows Kaivalya
		\item [VH]: Through vairagya – non-attachment even to this (supremacy and omniscience), upon the disappearance of the seeds of impediments (klesa), there is kaivalya – the aloneness (of the power of seeing-the seer).
		\item [BM]: From dispassion even toward these powers, freedom of the spirit occurs with the destruction of the seeds of sin.
		\item [SS]: (51): By non-attachment even to that [all these siddhis]. The seed of bondage is destroyed and thus follows Kailvalya (Independence).
		\item [SP]: (51) By giving up even these powers, the seed of evil is destroyed and liberation follows.
		\item [SV]: (51): By giving up even these comes the destruction of the very seed of evil; he attains Kaivalya. 
		\end{itemize}
	\end{itemize}
\end{frame}


%%%%%%%%%%%%%%%%%%%%%%%%%%%%%%%%%%%%%%%%%%%%%%%%%%%%%%%%%%%
\begin{frame}[fragile]\frametitle{3-51}
\begin{sanskrit}
स्थान्युपनिमन्त्रणे सङ्गस्मयाकरणं पुनरनिष्टप्रसङ्गात्॥५१॥
\end{sanskrit}

	\begin{itemize}
	\item पदच्छेद:  स्थानि-उपनिमन्त्रणे,सङ्ग-स्मय,अकरणम् , पुनः अनिष्ट-प्रसङ्गात् ॥
	\item Interpretations
		\begin{itemize}	
		\item  उच्च स्थान प्राप्त किसी व्यक्ति या कुल के द्वारा निमन्त्रण मिलने पर योगी को राग, अभिमान या अहंकार नहीं करना चाहिए । ऐसा करने से योग मार्ग से पुनः नीचे गिरने का अवसर उपस्थित हो जाता है ।
		\item [HA]: When Invited By The Celestial Beings That Invitation Should Not Be Accepted Nor Should It Cause Vanity Because It Involves Possibility Of Undesirable Consequences.
		\end{itemize}
	\end{itemize}
\end{frame}

%%%%%%%%%%%%%%%%%%%%%%%%%%%%%%%%%%%%%%%%%%%%%%%%%%%%%%%%%%%
\begin{frame}[fragile]\frametitle{3-51}
\begin{sanskrit}
स्थान्युपनिमन्त्रणे सङ्गस्मयाकरणं पुनरनिष्टप्रसङ्गात्॥५१॥
\end{sanskrit}

	\begin{itemize}
	\item पदच्छेद:  स्थानि-उपनिमन्त्रणे,सङ्ग-स्मय,अकरणम् , पुनः अनिष्ट-प्रसङ्गात् ॥
	\item Interpretations
		\begin{itemize}	
		\item [IT]: (52): (There should be) avoidance of pleasure of pride on being invited by the super-physical entities in charge of various planes because there is the possibility of revival of evil.
		\item [VH]: Upon the invitation of the high placed, (there should be) no cause for contact or pride because of the renewed, undesired inclination.
		\item [BM]: One should avoid enthusiasm or pride in the enticements of the gods, lest harmful attachments recur.
		\item [SS]: (52): The Yogi should neither accept nor smile with pride at the admiration of even the celestial beings, as there is the possibility of his getting caught again in the undesirable.
		\item [SP]: (52) When tempted by the invisible beings in high places, let the yogi feel neither allured nor flattered; for he is in danger of being caught once more by ignorance.
		\item [SV]: (52): The Yogi should not feel allured or flattered by the overtures of celestial beings, for fear of evil again. 
		\end{itemize}
	\end{itemize}
\end{frame}

%%%%%%%%%%%%%%%%%%%%%%%%%%%%%%%%%%%%%%%%%%%%%%%%%%%%%%%%%%%
\begin{frame}[fragile]\frametitle{3-52}
\begin{sanskrit}
क्षणतत्क्रमयोः संयमाद्विवेकजं ज्ञानम्॥५२॥
\end{sanskrit}

	\begin{itemize}
	\item पदच्छेद:  क्षण-तत्-क्रमयो:, संयमात् , विवेकजं, ज्ञानम् ‌॥
	\item Interpretations
		\begin{itemize}	
		\item क्षण एवं क्षण कि निरंतरता में संयम करने से योगी को विवेकज ज्ञान की प्राप्ति होती है ।
		\item [HA]: Differentiating Knowledge Of The Self And The Non-Self Comes From Practicing Samyama On Moment and It Sequence.
		\item [IT]: (53): Knowledge born of awareness of Reality by performing Samyama on moment and (the process of) its succession.
		\item [VH]: By sanyama on the ksana-moment and its krama-sequential progression – the knowledge born of viveka-discernment (between sattva guna and parusa- the self).
		\item [BM]: From perfect discipline of moments and their sequence in time, one has the knowledge born of discrimination.
		\item [SS]: (53): By samyama on single moments in sequence comes discriminative knowledge.
		\item [SP]: (53) By making samyama on single moments and on their sequence in time, one gains discriminative knowledge.
		\item [SV]: (53): By making Samyama on a particle of time and its multiples comes discrimination. 
		\end{itemize}
	\end{itemize}
\end{frame}


%%%%%%%%%%%%%%%%%%%%%%%%%%%%%%%%%%%%%%%%%%%%%%%%%%%%%%%%%%%
\begin{frame}[fragile]\frametitle{3-53}
\begin{sanskrit}
जातिलक्षणदेशैरन्यतानवच्छेदात् तुल्ययोस्ततः प्रतिपत्तिः॥५३॥
\end{sanskrit}

	\begin{itemize}
	\item पदच्छेद:  जाति-लक्षण-देशै:,अन्यताऽनवच्छेदात् ,तुल्ययो:,तत:,प्रतिपत्ति: ॥
	\item Interpretations
		\begin{itemize}	
		\item समान जाति, लक्षणों व स्थान से एक ही जैसी दिखने वाली वस्तुओं में विवेकज ज्ञान के द्वारा योगी को भे पृथकता का ज्ञान हो जाता है ।
		\item [HA]: When Species, Temporal Character And Position Of Two Different Things Being Indiscernable They Look Alike, They Can Be Differentiated Thereby (By This Knowledge)
		\item [IT]: (54): From it (Vivekajam’Jnanam) knowledge of distinction between similars which cannot be distinguished by class, characteristic or postion.
		\item [VH]: From that, the ascertainment of the differences between similars (not normally distinguished), due to not being restricted in their separateness by means of category, potential, and position.
		\item [BM]: Through discrimination one comprehends differences of origin, characteristic, or position that distinguish two seemingly similar things.
		\item [SS]: (54): Thus, the indistinguishable differences between objects that are alike in species, characteristics marks and positions become indistinguishable.
		\item [SP]: (54) Thus one is able to distinguish between two exactly similar objects, which cannot be distinguished by their species, characteristic marks, or positions in space.
		\item [SV]: (54): Those which cannot be differentiated by species, sign and place, even they will be discriminated by the above Samyama. 
		\end{itemize}
	\end{itemize}
\end{frame}

%%%%%%%%%%%%%%%%%%%%%%%%%%%%%%%%%%%%%%%%%%%%%%%%%%%%%%%%%%%
\begin{frame}[fragile]\frametitle{3-54}
\begin{sanskrit}
तारकं सर्वविषयं सर्वथाविषयमक्रमं चेति विवेकजं ज्ञानम्॥५४॥
\end{sanskrit}

	\begin{itemize}
	\item पदच्छेद:  तारकं, सर्व-विषयं, सर्वथा-विषयम् , अक्रमं, च-इति,विवेकजं, ज्ञानम् ॥
	\item Interpretations
		\begin{itemize}	
		\item योगी के स्वयं के योग साधना से प्राप्त सामर्थ्य से उत्पन्न होने वाला, जो सभी पदार्थों के विषयों को उनके सभी कालों भूत, वर्तमान व भविष्य को जानने वाला होता है और यह सब बिना किसी क्रम अर्थात विभाग के ही पैदा होने वाला ज्ञान होता है । ऐसे ज्ञान को विवेकज या विवेक से उत्पन्न ज्ञान कहा जाता है ।
		\item [HA]: Knowledge Of Discernment Is Taraka Or Intuitional, Is Comprehension Of All Things And Of All Times And Has No Sequence.
		\item [IT]: (55): The highest knowledge born of awareness of Reality is transcendent, includes the cognition of all objects simultaneously, pertains to all objects and processes whatsoever in the past, present and future and also transcends World Process.
		\end{itemize}
	\end{itemize}
\end{frame}

%%%%%%%%%%%%%%%%%%%%%%%%%%%%%%%%%%%%%%%%%%%%%%%%%%%%%%%%%%%
\begin{frame}[fragile]\frametitle{3-54}
\begin{sanskrit}
तारकं सर्वविषयं सर्वथाविषयमक्रमं चेति विवेकजं ज्ञानम्॥५४॥
\end{sanskrit}

	\begin{itemize}
	\item पदच्छेद:  तारकं, सर्व-विषयं, सर्वथा-विषयम् , अक्रमं, च-इति,विवेकजं, ज्ञानम् ॥
	\item Interpretations
		\begin{itemize}	
		\item [VH]: (This knowledge born of viveka is) taraka-causing to cross beyond (klesa/karma etc.), sarva-visaya – omni-temporal (encompassing objects of perception), and is sarvatha-visaya –omni-temporal (encompassing objects in all conditions), and is akrama- non-sequential. Thus concludes the knowledge born of viveka-discernment.
		\item [BM]: Knowledge born of discrimination is transcendent, comprehensive, concerned with all states of things, and instantaneous.
		\item [SS]: (55): The discriminative knowledge that simultaneously comprehends all objects in all conditions is the intuitive knowledge which brings liberation. The gnosis born of discernment is the ‘deliverer’, and is omni-objective, omni-temporal and non-sequential.
		\item [SP]: (55) This discriminative knowledge delivers a man from the bondage of ignorance.
		\item [SV]: (55): The saving knowledge is that knowledge of discrimination which covers all objects, all means. 
		\end{itemize}
	\end{itemize}
\end{frame}


%%%%%%%%%%%%%%%%%%%%%%%%%%%%%%%%%%%%%%%%%%%%%%%%%%%%%%%%%%%
\begin{frame}[fragile]\frametitle{3-55}
\begin{sanskrit}
सत्त्वपुरुषयोः शुद्धिसाम्ये कैवल्यमिति॥५५॥
\end{sanskrit}

	\begin{itemize}
	\item पदच्छेद:  सत्त्व-पुरुषयो: , शुद्धि-साम्ये, कैवल्यम्, इति ॥
	\item Interpretations
		\begin{itemize}	
		\item जब बुद्धि और जीवात्मा की एक समान रूप से शुद्धि हो जाती है । तब वह कैवल्य या मोक्ष की अवस्था कहलाती है ।
		\item [HA]: When Equality Is Established Between Buddhi-Sattva And Purusa In Their Purity, Liberation Takes Place.
		\item [IT]: (56): Kaivalya is attained when there is equality of purity between the Parusa and Sattva.
		\item [VH]: Upon the equal purity of sattva and parusa–kaivalya-the aloness (of seeing).
		\item [BM]: Absolute freedom occurs when the lucidity of material nature and spirit are in pure equlibrium.
		\item [SS]: (56): When the tranquil mind attains purity equal to that of the Self, there is Absoluteness.
		\item [SP]: (56) Perfection is attained when the mind becomes as pure as the Atman itself.
		\item [SV]: (56): By the similarity of purity between the Sattva and the Purusa comes Kaivalya. 
		\end{itemize}
	\end{itemize}
\end{frame}

