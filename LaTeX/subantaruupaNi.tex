% !TeX program = XeLaTeX
% !TeX root = subantaruupaNi.tex
\documentclass[leqno,fleqn,12pt]{article}%{book}twoside,
\usepackage{etoolbox}
\usepackage{multicol}
\usepackage{supertabular}
\usepackage[table]{xcolor}% http://ctan.org/pkg/xcolor
\usepackage{refcount}% http://ctan.org/pkg/refcount
\usepackage{enumitem}
%\usepackage{hyperref} %http://en.wikibooks.org/wiki/LaTeX/Hyperlinks
%\usepackage{setspace}\onehalfspacing
%\AtBeginDocument{%
  %\addtolength\abovedisplayskip{-0.9\baselineskip}%
  %\addtolength\belowdisplayskip{-0.9\baselineskip}%
%%  \addtolength\abovedisplayshortskip{-0.5\baselineskip}%
%%  \addtolength\belowdisplayshortskip{-0.5\baselineskip}%
%}
%\usepackage{amsmath}% http://ctan.org/pkg/amsmath
%\usepackage{xcolor,colortbl}
%\definecolor{green}{rgb}{0.1,0.1,0.1}
%\newcommand{\done}{\cellcolor{teal}done}  %{0.9}
%\newcommand{\hcyan}[1]{{\color{teal} #1}}
%\usepackage{caption}
\newbool{kindle}
\newbool{print}
%
\newcommand{\txcb}[1]{\textcolor{blue}{#1}}
\newcommand{\txcr}[1]{\textcolor{red}{#1}}
\newcommand{\txcg}[1]{\textcolor{green}{#1}}
\newcommand{\txcy}[1]{\textcolor{yellow}{#1}}
\newcommand{\sutra}[2]{\href{http://avg-sanskrit.org/avgdocs/doku.php?id=sutras:#1}{\begin{equation} \mbox{#2 ॥#1॥} \label{#1} \end{equation}}}
\newcommand{\sref}[1]{(\hyperref[#1]{\ref{#1})}}
\newcommand{\hsref}[1]{Sutra \hyperref[#1]{\sref{#1}} is used to derive:}

\setbool{kindle}{false}
\setbool{print}{true}%{false}
%Override print boolean if kindle is true
\ifbool{kindle}{\setbool{print}{false}}{}

%\ifbool{kindle}{\usepackage[paperwidth=126mm,paperheight=168mm,left=5mm,right=5mm,top=15mm,bottom=20mm]{geometry}}{\usepackage[top=1.5cm, bottom=1.8cm, left=1.5cm, %right=1.5cm,paperwidth=148mm,paperheight=210mm]{geometry}}
\ifbool{kindle}{\usepackage[paperwidth=126mm,paperheight=168mm,left=5mm,right=5mm,top=15mm,bottom=20mm]{geometry}}{\usepackage[top=1.5cm, bottom=1.7cm, left=1.5cm, right=1.5cm,paperwidth=210mm,paperheight=280mm]{geometry}}

\newcount\n
\n=0
\def\tablebody{}
\makeatletter
\loop\ifnum\n<100
        \advance\n by 1
        \protected@edef\tablebody{\tablebody
                \textbf{\number\n.}& shortText
                \tabularnewline
        }
\repeat

\makeatletter
\let\mcnewpage=\newpage
\newcommand{\TrickSupertabularIntoMulticols}{%
  \renewcommand\newpage{%
    \if@firstcolumn
      \hrule width\linewidth height0pt
      \columnbreak
    \else
      \mcnewpage
    \fi
  }%
}
%\renewcommand{\theequation}{\devanumber\c@equation}
\makeatother

\input{preamble}

\ifbool{print}{%Margin changes for print
\addtolength{\evensidemargin}{-0.5cm}
\addtolength{\oddsidemargin}{0.5cm}}{}
\title{पाणिनीय-पद्धत्या शब्दरूपाणि \\ (पञ्चविंशति-प्रकार-शब्दानां रूपावलिः अन्यानि सुबन्तरूपाणि च)}
\author{संशोधनाय सम्पर्कः - himanshu.pota@gmail.com}
\newenvironment{ewosp}{\begin{enumerate} \setlength{\itemsep}{0mm}
    \setlength{\parskip}{0mm}}{\end{enumerate}}
\newenvironment{dwosp}{\begin{description} \setlength{\itemsep}{0mm} \setlength{\parskip}{0mm}}{\end{description}}
%%Karthik e-mail 13 Dec 2014
%Another trick to clean overhangs etc (I remember seeing them in earlier versions) --- happens when you write long Sanskrit paras is something like this:
%\fontsize{18pt}{22pt}\selectfont %This selects some other font size if you want
%%The following alters word spacing to ensure that words aren't too close to one another!
\spaceskip=1\fontdimen2\font plus 1\fontdimen3\font minus 1.25\fontdimen4\font
\setlength{\parindent}{0pt}
\setlength{\emergencystretch}{3em} %I think this avoids overhangs...

\newcounter{vtable}
\newcommand{\nvtable}{%number vibhakti-table
        \stepcounter{vtable}%
        (\thevtable)}
%http://tex.stackexchange.com/questions/21300/custom-counter-and-cross-referencing
%http://en.wikibooks.org/wiki/LaTeX/Macros
\makeatletter
\newcounter{tablecount}
\newcommand{\nextTable}{\stepcounter{tablecount}(\devanumber\c@tablecount) \relax}
\newcommand{\resetTable}{\setcounter{tablecount}{0}}
\resetTable
\newcounter{wordcount}
\newcommand{\nextWord}[1]{\refstepcounter{wordcount}(\devanumber\c@wordcount) \label{#1} \relax}
\newcommand{\resetWord}{\setcounter{wordcount}{0}}
\resetWord
\newcommand{\wref}[1]{(\devanumber{\getrefnumber{#1}}) \relax}

%Prasad Bhat sent this in Dec 2014.
%\newcounter{wordcount}
%\newcommand{\nextWord}[1]{\refstepcounter{wordcount}(\devanumber\c@wordcount) \label{#1}\relax}
%\newcommand{\resetWord}{\setcounter{wordcount}{0}}
%\resetWord
%\renewcommand{\thewordcount}{\devanumber\c@wordcount}

%\newcounter{wordcount}
%\newcommand{\nextWord}{\refstepcounter{wordcount}(\devanumber\c@wordcount)\relax}
%\newcommand{\resetWord}{\setcounter{wordcount}{0}}
%\resetWord
%\newcommand{\wref}[1]{\devanumber\c@\ref{#1}}
\makeatother
%\newcommand{\rtask}[1]{\refstepcounter{rtaskno}\label{#1}}	
\hypersetup{
    bookmarks=true,         % show bookmarks bar?
    unicode=true,           % non-Latin characters in Acrobat’s bookmarks
    pdftoolbar=true,        % show Acrobat’s toolbar?
    pdfmenubar=true,        % show Acrobat’s menu?
		pdfborder={0 0 0.5},    %{RadiusH RadiusV Width [Dash-Pattern]} 
    pdffitwindow=false,     % window fit to page when opened
    pdfstartview={FitH},    % fits the width of the page to the window
    pdftitle={My title},    % title
    pdfauthor={Himanshu Pota},     % author
    pdfsubject={Subant Forms},   % subject of the document
    pdfcreator={XeLaTex},   % creator of the document
    pdfproducer={Producer}, % producer of the document
    pdfkeywords={keyword1} {key2} {key3}, % list of keywords
    pdfnewwindow=true,      % links in new PDF window
    colorlinks=true,       % false: boxed links; true: colored links
    linkcolor=blue,          % color of internal links (change box color with linkbordercolor)
    citecolor=green,        % color of links to bibliography
    filecolor=magenta,      % color of file links
    urlcolor=cyan           % color of external links
}		
%\usepackage[all]{hypcap}	
\newcommand{\csarva}{\cellcolor{red!20}}
\newcommand{\cbham}{\cellcolor{green!30}}
\newcommand{\cpada}{\cellcolor{yellow!50}}
\newcommand{\cnorm}{\cellcolor{white}}
\setmainfont[Script=Devanagari,Mapping=tex-text]{Sanskrit 2003}
%\Large\setmainfont[Script=Gujarati,Mapping=tex-text]{Shruti}
%\begin{multicols}{2}
%\twocolumn
%रूपाणि 
\setlength{\columnsep}{0.5cm}
\setlength{\columnseprule}{0.25pt}
\begin{document}
%Pota Bhai,
%
%Please check the following:
%
%Page No. 6.  Chart (6) and Page No. 20, Chart (59). Why repetition?
%Page No. 16.  Chart (29) and Page No. 17, Chart (36). Why both?
%
%Mistakes:  Page No. 23, Chart No. 89.
%Please check the  third and fourth cases in singular number.
%
%With regards,
%Narendra
%
%Pota Bhai,
%
%Here are some more to be corrected:
%
%Page No. 22, Chart No. (84), third case singular.
%Page No. 23, Chart No. (92), third case singular.
%Page No. 26, Chart No. (116), third case singular.
%
%Narendra
%\setlength{\belowdisplayskip}{0pt} \setlength{\belowdisplayshortskip}{0pt}
%\setlength{\abovedisplayskip}{0pt} \setlength{\abovedisplayshortskip}{0pt}

%\input{frontmatter}
%\Large
%\setmainfont[Script=Devanagari,Mapping=tex-text]{Sanskrit 2003}
%\Large\setmainfont[Script=Gujarati,Mapping=tex-text]{Shruti}
\maketitle
%\begin{multicols}{2}
%\twocolumn
%रूपाणि 
%\setlength{\columnsep}{0.5cm}
%\setlength{\columnseprule}{0.25pt}

\begin{abstract}
With a little help from Panini, a method is proposed that makes it easier to remember a couple of hundred विभक्ति-tables than one simple विभक्ति-table. The process is entirely mechanical and is based on going through a few simple steps repeatedly. To speak or write in a new language, one needs a small but sufficiently large subset of the language at one's command. Doing one विभक्ति-table at a time, forgetting it, and then doing it again, is a never ending process that all the Sanskrit learners know well. This limitation can be overcome by mastering a large subset of the language rapidly. Sanskrit learners are very fortunate that Panini has made it possible to bite, chew, swallow, and digest a huge chunk of Sanskrit language in one go. Please try this method and if you don't succeed, then in the words of श्री पण्डित ब्रह्मदत्त जिज्ञासुजी, रटने का कुछ काम पड़ा हो, तो मुझे साक्षी सहित पत्र लिखिये।
\end{abstract}

\section{Introduction}
The motivation to write this note has been provided by the small booklet:
पाणिनीय-पद्धत्या शब्दरूपाणि (पञ्चविंशति-प्रकार-शब्दानां रूपावलिः), डॉ॰ नरेन्द्रः, संस्कृतकार्यालयः, श्रीअरविन्दाश्रमः, पुदुच्चेरी - ६०५००२ भारतः १९९९. Pundits who know Panini's Ashtadhyayi understand what सिद्धिः means and the सिद्धिः method of learning Ashtadhayayi. For someone trying to enter the Panini system on their own, unless the motivation is clear, the entry becomes difficult. This note is to help students who know a little Sanskrit and want to understand the Panini system. This note is also helpful if one wants to memorise the विभक्ति-tables quickly with a near perfect recall.

My understanding of Panini is through the following three excellent books from Sri Aurobindo Ashrama.
\begin{ewosp}
\item व्यावहारिकं पाणिनीयम्, डॉ॰ नरेन्द्रः, संस्कृतकार्यालयः, श्रीअरविन्दाश्रमः, पुदुच्चेरी - ६०५००२ भारतः १९९९.
\item पाणिनीय-पद्धत्या शब्दरूपाणि (पञ्चविंशति-प्रकार-शब्दानां रूपावलिः), डॉ॰ नरेन्द्रः, संस्कृतकार्यालयः, श्रीअरविन्दाश्रमः, पुदुच्चेरी - ६०५००२ भारतः १९९९.
\item पाणिनीयप्रवेशः, डॉ॰ नरेन्द्रः, संस्कृतकार्यालयः, श्रीअरविन्दाश्रमः, पुदुच्चेरी - ६०५००२ भारतः, २००९ .
\end{ewosp}
I am a beginner and this note is to share with other beginners the idea that even with a little effort one can go very far, and having tasted a bit of Panini one can continue to enjoy the beauty of the Sanskrit language.

This note tries to give a simple description of what Panini was up to. Panini first collected {\em all} the Sanskrit words in use, prepared three lists of (raw) words---गणपाठः, उणादिकोषः, and धातुपाठः (with a few thousand words in each of the lists)---and made about 4000 rules to derive all the Sanskrit words (numbering in hundred of thousands), from the three lists of raw words. 

To understand or benefit from पाणिनि-अष्टाध्यायी it is crucial that the process followed by Panini is understood well. First were the millions of Sanskrit words, Panini observed them and saw some patterns and used those patterns to formulate the 4000 rules. To learn and appreciate Panini one must {\em observe} Sanskrit words first, try to identify patterns, and make rules to derive the words. These rules can then be compared with the rules made by Panini to do the same derivation. It is very likely that one would come up with a few rules that are identical to the rules made by Panini and this will open up the mind to soak in the rest of Panini. This simple exercise will give an insight into the overall motivation and the philosophy behind पाणिनि-अष्टाध्यायी and from then on the learning journey will be a joy. Let us start on that journey.

Fortunate for us beginners, we don't have to observe millions of Sanskrit words to understand the अष्टाध्यायी philosophy, there are smaller groups of words that have sufficient variety to enable us to develop our observation powers and need only a small subset of the 4000 rules to complete the derivation process. We begin with a group of words which are divided into 25 subgroups.  The collection of the विभक्ति-tables of these 25 subgroups is the starting point. We observe the character of these tables and then explore the making of the rules to derive these विभक्ति-tables starting from raw words. Let the beginner observe the tables, as Panini would have done, make rules on their own, and then compare it with how Panini has done it. Who knows some beginners might go on to better Panini!

The विभक्ति-tables in this document are given for one entry in each of the 25 subgroups in the list below. All the words in a subgroup follow the same set of rules, i.e., to obtain tables for नामन्, धामन्, व्योमन्, and लोमन् only one set of rules is necessary. 

Abbreviations पु॰, स्त्री॰, नपु॰, stand for पुँल्लिङ्गः शब्दः, स्त्रीलिङ्गः शब्दः, नपुंसकलिङ्गः शब्दः, respectively. Some links in this document link to the tables in the document itself (in most pdf browsers, Alt + left-arrow can be used to go back); there are some links to very helpful external websites as well. 

\begin{multicols}{2}
\begin{ewosp}
\item[]\hspace{-6ex} \nextWord{नामन् (नपु॰) Name} \hyperlink{नामन् (नपु॰) Name}{नामन् (नपु॰) Name}, धामन्, व्योमन्, लोमन्
\item[]\hspace{-6ex}  \nextWord{जन्मन् (नपु॰) Birth} \hyperlink{जन्मन् (नपु॰) Birth}{जन्मन् (नपु॰) Birth}, कर्मन्, चर्मन्, पर्वन्, भस्मन्, ब्रह्मन्
\item[]\hspace{-6ex}  \nextWord{वारि (नपु॰) Water} \hyperlink{वारि (नपु॰) Water}{वारि (नपु॰) Water}, शुचि
\item[]\hspace{-6ex}  \nextWord{दधि (नपु॰) Curd} \hyperlink{दधि (नपु॰) Curd}{दधि (नपु॰) Curd}, अस्थि, अक्षि
\item[]\hspace{-6ex}  \nextWord{मधु (नपु॰) Honey} \hyperlink{मधु (नपु॰) Honey}{मधु (नपु॰) Honey}, वस्तु, अश्रु, जानु, श्मश्रु, गुरु, बहु
\item[]\hspace{-6ex}  \nextWord{जगत् (नपु॰) World} \hyperlink{जगत् (नपु॰) World}{जगत् (नपु॰) World}, यकृत्, कियत्, पठत्, बलवत्, महत्
\item[]\hspace{-6ex}  \nextWord{मनस् (नपु॰) Mind} \hyperlink{मनस् (नपु॰) Mind}{मनस् (नपु॰) Mind}, सरस्, छन्दस्, तपस्, तेजस्, वयस्
\item[]\hspace{-6ex}  \nextWord{ज्योतिस् (नपु॰) Light} \hyperlink{ज्योतिस् (नपु॰) Light}{ज्योतिस् (नपु॰) Light}, हविस्, धनुस्, आयुस्, चक्षुस्
\item[]\hspace{-6ex}  \nextWord{फल (नपु॰) Fruit} \hyperlink{फल (नपु॰) Fruit}{फल (नपु॰) Fruit}, पत्र, पुष्प, उद्यान, वन, पुस्तक
\item[]\hspace{-6ex}  \nextWord{अभिजित् (पु॰) Victorious} \hyperlink{अभिजित् (पु॰) Victorious}{अभिजित् (पु॰) Victorious}, इन्द्रजित्, परिक्षित्, सुकृत्
\item[]\hspace{-6ex}  \nextWord{धीमत् (पु॰) Intelligent} \hyperlink{धीमत् (पु॰) Intelligent}{धीमत् (पु॰) Wise}, बलवत्, बुद्धिमत्, भवत्, भगवत्, कियत्
\item[]\hspace{-6ex}  \nextWord{सरित् (स्त्री॰) River} \hyperlink{सरित् (स्त्री॰) River}{सरित् (स्त्री॰) River}, विद्युत्, त्रिंशत्, चत्वारिंशत्, पञ्चशत्
\item[]\hspace{-6ex}  \nextWord{खादत् (पु॰) The Eating One} \hyperlink{खादत् (पु॰) The Eating One}{खादत् (पु॰) Eating One}, पठत्, गच्छत्, खेलत्, हसत्, पश्यत्
\item[]\hspace{-6ex}  \nextWord{ज्ञानिन् (पु॰) Wise} \hyperlink{ज्ञानिन् (पु॰) Wise}{ज्ञानिन् (पु॰) Wise}, धनिन्, योगिन्, सन्न्यासिन्, हस्तिन्
\item[]\hspace{-6ex}  \nextWord{राम (पु॰) Ram} \hyperlink{राम (पु॰) Ram}{राम (पु॰) Ram}, गोविन्द, बालक, देव, विद्यालय
\item[]\hspace{-6ex}  \nextWord{लता (स्त्री॰) Creeper} \hyperlink{लता (स्त्री॰) Creeper}{लता (स्त्री॰) Creeper}, माला, बालिका, टोपिका, कविता
\item[]\hspace{-6ex}  \nextWord{रवि (पु॰) Sun} \hyperlink{रवि (पु॰) Sun}{रवि (पु॰) Sun}, हरि, मुनि, अतिथि, कवि, कपि
\item[]\hspace{-6ex}  \nextWord{मति (स्त्री॰) Idea} \hyperlink{मति (स्त्री॰) Idea}{मति (स्त्री॰) Idea}, स्वाति, अङ्गुलि, युक्ति, विंशति, कोटि
\item[]\hspace{-6ex}  \nextWord{नदी (स्त्री॰) River} \hyperlink{नदी (स्त्री॰) River}{नदी (स्त्री॰) River}, रेवती, भगवती, कियती, पठन्ती, महती
\item[]\hspace{-6ex}  \nextWord{साधु (पु॰) Saint} \hyperlink{साधु (पु॰) Saint}{साधु (पु॰) Saint}, शान्तनु, गुरु, बन्धु, शत्रु, शिशु
\item[]\hspace{-6ex}  \nextWord{धेनु (स्त्री॰) Cow} \hyperlink{धेनु (स्त्री॰) Cow}{धेनु (स्त्री॰) Cow}, रज्जु, स्नायु, चञ्चु
\item[]\hspace{-6ex}  \nextWord{वधू (स्त्री॰) Bride} \hyperlink{वधू (स्त्री॰) Bride}{वधू (स्त्री॰) Bride}, श्वश्रू,  चमू
\item[]\hspace{-6ex}  \nextWord{पितृ (पु॰) Father} \hyperlink{पितृ (पु॰) Father}{पितृ (पु॰) Father},  मातृ,  भ्रातृ,  जामातृ
\item[]\hspace{-6ex}  \nextWord{कर्तृ (पु॰) Doer} \hyperlink{कर्तृ (पु॰) Doer}{कर्तृ (पु॰) Doer},  दातृ,  अभिनेतृ,  द्रष्टृ,  श्रोतृ
\item[]\hspace{-6ex}  \nextWord{आत्मन् (पु॰) Self} \hyperlink{आत्मन् (पु॰) Self}{आत्मन् (पु॰) Self},  ब्रह्मन्,  कृत्कर्मन्,  धृतजन्मन्
\end{ewosp}
\end{multicols}
%\clearpage

\section{The Beginning}
Now we will briefly have a look at how to observe विभक्ति-tables and what are the normal patterns. We will first consider a simple example to concentrate on the elementary process of putting a raw word and suffixes to form new words. सुगण् holds the same place in learning पाणिनि-अष्टाध्यायी as held by `hello world' in learning software programming languages. So let us start with सुगण्।
%

\clearpage

\begin{center}
 %\captionof{table}{सुगण् पुँल्लिङ्गः शब्दः} \label{tab:sugan}
%\begin{table}
	%\centering
	%सुगण् पुँल्लिङ्गः शब्दः\\[1.5ex]
		\begin{supertabular}{|c|c|c|c|}\hline
		 \multicolumn{4}{|c|}{\cellcolor{blue!10} \hypertarget{सुगण् (पु॰) Good Calculator}{सुगण् (पु॰) Good Calculator}} \\ \hline
		& एक॰ & द्वि॰ & बहु॰ \\ \hline
प्रथमा & \cnorm सुगण् & \cnorm सुगणौ & \cnorm सुगणः\\ \hline
द्वितीया & \cnorm सुगणम् & \cnorm सुगणौ & \cnorm सुगणः\\ \hline
तृतीया & \cnorm सुगणा & \cnorm सुगण्भ्याम् & \cnorm सुगण्भिः\\ \hline
चतुर्थी & \cnorm सुगणे & \cnorm सुगण्भ्याम् & \cnorm सुगण्भ्यः\\ \hline
पञ्चमी & \cnorm सुगणः & \cnorm सुगण्भ्याम् & \cnorm सुगण्भ्यः\\ \hline
षष्ठी & \cnorm सुगणः & \cnorm सुगणोः & \cnorm सुगणाम्\\ \hline
सप्तमी & \cnorm सुगणि & \cnorm सुगणोः & \cnorm सुगण्सु\\ \hline			
		\end{supertabular}
		\end{center}
		%\caption{चकारान्तः पुंलिङ्गः प्राञ्च् शब्दः}
%\end{table}
%
%\ref{tab:sugan}
सुगण् means one who can count well. When we use सुगण् in a sentence, we have to use an appropriate form from the \hyperlink{सुगण् (पु॰) Good Calculator}{सुगण्-table}. For example, 
सुगण् गच्छति (One who counts well goes)। सः सुगणं पश्यति (He is looking at the one who counts well)। सा सुगणा सह वार्तालापं करोति (She is talking with the one who counts well)। सुगणे नमः (Salutations to the one who counts well)। सुगणः पत्रं आनय (Bring a paper from the one who counts well)। इदं पत्रं सुगणः अस्ति (This is the paper of the one who counts well)। सुगणि नैकाः गुणाः सन्ति (There are many qualities in the one who counts well)। The seven cases used here are called the nominative, accusative, instrumental, dative, ablative, possessive, and locative respectively. These cases are used in most languages but because the words don't change their form as they change in Sanskrit, these cases go unobserved. Also prepositions are used in modern languages to indicate different cases instead of modifying the word itself as is seen in the 21 forms in the सुगण्-table.

\section{The First Task}
Let us start as Panini would have started. He had the सुगण्-table (and other tables as we will see as we go) and his first task was to find the minimum set of rules that will generate the सुगण्-table starting from the raw word सुगण्.  To get started Panini would have taken सुगण् out of the सुगण्-table and made a table of what remains and given a rule such as: take the raw word, join each of the 21 suffixes shown in the \hyperlink{सुप् प्रत्ययाः}{सुप् प्रत्ययाः-table} to the raw word, and get the विभक्ति-table for that raw word. From the \hyperlink{सुप् प्रत्ययाः}{सुप् प्रत्ययाः-table} use the suffixes on the right side of $\rightarrow$. Please note that स् at the end of a word changes to a visarga and no Sanskrit word can have two consonants at the end, i.e., सुगण् + अस् becomes सुगणः and सुगण् + स् becomes सुगण्. Panini might have as well dropped सुँ in प्रथमा-एकवचानम्, but as we will see, this form of the suffix has many uses.

The terms on the left side of the $\rightarrow$ are the ``names'' of the suffixes and the terms of the right side are the actual suffixes as they are applied. The reason why the name is different from the final form is one of the interesting contributions of Panini. The purpose of this short write-up is to encourage you to discover this interesting contribution for yourself. 

%\clearpage
\begin{center}
%\begin{table}
	%\centering
	%सुप् प्रत्ययाः\\[1.5ex]
		\begin{supertabular}{|c|c|c|c|}\hline
\multicolumn{4}{|c|}{\cellcolor{blue!10} \hypertarget{सुप् प्रत्ययाः}{सुप् प्रत्ययाः}} \\ \hline
		& एक॰ & द्वि॰ & बहु॰ \\ \hline
प्रथमा & \csarva सुँ $\rightarrow$ स् & \csarva औ & \csarva जस् $\rightarrow$ अस्\\ \hline
द्वितीया & \csarva अम् & \csarva औट् $\rightarrow$ औ & \cbham शस् $\rightarrow$ अस्\\ \hline
तृतीया & \cbham टा $\rightarrow$ आ & \cpada भ्याम् & \cpada भिस्\\ \hline
चतुर्थी & \cbham ङे $\rightarrow$ ए & \cpada भ्याम् & \cpada भ्यस्\\ \hline
पञ्चमी & \cbham ङसिँ $\rightarrow$ अस् & \cpada भ्याम् & \cpada भ्यस्\\ \hline
षष्ठी & \cbham ङस् $\rightarrow$ अस् & \cbham ओस् & \cbham आम्\\ \hline
सप्तमी & \cbham ङि $\rightarrow$ इ & \cbham ओस् & \cpada सुप् $\rightarrow$ सु\\ \hline			
		\end{supertabular}
		\end{center}		
		%\caption{चकारान्तः पुंलिङ्गः प्राञ्च् शब्दः}
%\end{table}
%\clearpage
%\par
Three terms are used in describing the application of the 21 सुप् प्रत्ययाः in the above table: सर्वनामस्थान, भ, and, पद| The thing (अङ्गम्) before the प्रत्ययाः --- भ्याम् भिस् भ्यस् सुप् --- is called a पद; the word before the following प्रत्ययाः, which start with a vowel, --- शस् $\rightarrow$ अस्, टा $\rightarrow$ आ, ङे $\rightarrow$ ए, ङसिँ $\rightarrow$ अस्, ङस् $\rightarrow$ अस्, ओस् $\rightarrow$ आम्, ङि $\rightarrow$ इ --- are called भ; for पुँल्लिङ्गः and स्त्रीलिङ्गः the five प्रत्ययाः --- सुँ औ जस् अम् औट् --- are called सर्वनामस्थान and for नपुंसकलिङ्गः जस् and शस् are called सर्वनामस्थान. 

The terms सर्वनामस्थान, भ, and, पद are created by Panini because it is easy to identify patterns based on these groupings of the 21 सुप् प्रत्ययाः| Please keep these groupings in mind as you work through remembering the विभक्ति-tables. These three groups are colour coded.


\section{The Second Task}	
After the first elegant rule was formed, Panini would have looked at another table, such as the नामन्-table and wondered how the \hyperlink{नामन्first}{first-cut नामन् -table}, as shown below, using the rule proposed above, can be modified to get the right \hyperlink{नामन् (नपु॰) Name}{नामन्-table}.

%\clearpage

\begin{center}
%\begin{table}
	%\centering
	%नामन् नकारान्तः नपुंसकलिङ्गः शब्दः (First Cut) \\[1.5ex]
		\begin{supertabular}{|c|c|c|c|}\hline
\multicolumn{4}{|c|}{\cellcolor{blue!10} \hypertarget{नामन्first}{नामन्  (नपु॰) (First Cut)}} \\ \hline
		& एक॰ & द्वि॰ & बहु॰ \\ \hline
प्रथमा & \cnorm नामन्स् & \cnorm नामनौ & \cnorm नामनः\\ \hline
द्वितीया & \cnorm नामनम् & \cnorm नामनौ & \cnorm  नामनः\\ \hline
तृतीया & \cnorm नामना & \cnorm नामन्भ्याम् & \cnorm नामन्भिः\\ \hline
चतुर्थी & \cnorm नामने & \cnorm  नामन्भ्याम् & \cnorm नामन्भ्याम्\\ \hline
पञ्चमी & \cnorm नामनः & \cnorm नामन्भ्याम् & \cnorm नामन्भ्याम्\\ \hline
षष्ठी & \cnorm नामनः & \cnorm नामनोः & \cnorm नमनाम्\\ \hline
सप्तमी & \cnorm नामनि & \cnorm नामनोः & \cnorm नामन्सु\\ \hline			
		\end{supertabular}
		\end{center}
		%\caption{चकारान्तः पुंलिङ्गः प्राञ्च् शब्दः}
%\end{table}

%\clearpage

Before we see Panini's solution let us observe the difference between the table one would obtain by using the first rule, as shown above, and the actual नामन्-table. There are two forms in each of the five cells (1.2, 2.2. 7.1, 8.1, 8.1) and for our initial discussion we will concentrate only on the first form. The notation used for each cell is $n.m$, where $n \in 1,\ldots, 7$ (corresponding to प्रथमा ... सप्तमी), and $m \in 1,2,3$ (corresponding to singular, dual, and plural).

A few differences between the first cut नामन्-table and the correct नामन्-table are:
\begin{ewosp}
\item Starting from the 1.1 entry नाम (instead of नामन्स्), the entries in {1.2, 1.3, 2.1, 2.2, 2.3} are very different.
\item The अ in the अन् ending of नामन् is missing in {1.2, 2.2, 3.1, 4.1, 5.1, 6.1, 7.1, 6.2, 6.3, 7.2}.
\item The last न् of नामन् is missing in {1.1, 2.1, 3.2, 3.3, 4.2, 4.3, 5.2, 5.3, 7.3}.
\end{ewosp}

With these observations we form three sets of cells called $A_{n}, B_{n}$, and $C_{n}$, where, \\
$A_{n} = \{1.1, 1.2, 1.3, 2.1, 2.2, 2.3\}$, \\%\\ 
$B_{n} = \{1.2, 2.2, 3.1, 4.1, 5.1, 6.1, 7.1, 6.2, 6.3, 7.2\}$,  and \\ 
$C_{n} = \{1.1, 2.1, 3.2, 3.3, 4.2, 4.3, 5.2, 5.3, 7.3\}$. 

Based on our observations we can now make the following rules:
\begin{ewosp}
\item Modify cells in $A_{n}$ of the सुप् प्रत्ययाः-table with the corresponding cells in the \hyperlink{सुप् प्रत्ययाः (नपु॰)}{सुप् प्रत्ययाः (नपुंसकलिङ्गः)-table}.
\item Drop अ in the अन्  of नामन् for cells in $B_{n}$ and then add the entries in the सुप् प्रत्ययाः-table.
\item Drop न् of नामन् for cells in $C_{n}$ and then add the entries in the सुप् प्रत्ययाः-table.
\end{ewosp}

Apply all the three rules above and see if you can get the correct \hyperlink{नामन् (नपु॰) Name}{नामन्-table} below.

%%\begin{minipage}{\textwidth}
\begin{multicols}{2}
\TrickSupertabularIntoMulticols

 \begin{center} 
 \begin{supertabular}{|c|c|c|c|}\hline 
 \multicolumn {4}{|c|}{\cellcolor{blue!10}  \nextTable \hypertarget{नामन् (नपु॰) Name}{नामन् (नपु॰) Name}}  \\ \hline  
 & एक॰ & द्वि॰ & बहु॰  \\ \hline 
 प्रथमा & \cpada नाम & \cbham नाम्नी-नामनी & \csarva नामानि \\ \hline 
 द्वितीया & \cbham नाम & \cbham नाम्नी-नामनी & \csarva नामानि \\ \hline 
 तृतीया & \cbham नाम्ना & \cpada नामभ्याम् & \cpada नामभिः \\ \hline 
 चतुर्थी & \cbham नाम्ने & \cpada नामभ्याम् & \cpada नामभ्यः \\ \hline 
 पञ्चमी & \cbham नाम्नः & \cpada नामभ्याम् & \cpada नामभ्यः \\ \hline 
 षष्ठी & \cbham नाम्नः & \cbham नाम्नोः & \cbham नम्नाम् \\ \hline 
सप्तमी & \cbham नाम्नि-नामनि & \cbham नाम्नोः & \cpada नामसु \\ \hline 
सम्बोधन & हे नाम-नामन् & हे नाम्नि-नामनी &  हे नामानि \\ \hline 
\end{supertabular} 
\end{center} 

\columnbreak
\small
\begin{center}
%\begin{table}
	%\centering	
	%सुप् प्रत्ययाः (नपुंसकलिङ्गः)\\[1.5ex]
		\begin{supertabular}{|c|c|c|c|}\hline
\multicolumn{4}{|c|}{\cellcolor{blue!10} \hypertarget{सुप् प्रत्ययाः (नपु॰)}{सुप् प्रत्ययाः (नपुंसकलिङ्गः)}} \\ \hline
		& एक॰ & द्वि॰ & बहु॰ \\ \hline
प्रथमा & \cpada सुँ $\rightarrow$ (लोपः) & \cbham औ $\rightarrow$ शी $\rightarrow$ ई & \csarva जस् $\rightarrow$ शि $\rightarrow$ इ \\ \hline
द्वितीया & \cbham अम् $\rightarrow$ (लोपः) & \cbham औट् $\rightarrow$ शी $\rightarrow$ ई & \csarva शस् $\rightarrow$ शि $\rightarrow$ इ\\ \hline
तृतीया & \cbham टा $\rightarrow$ आ & \cpada भ्याम् & \cpada भिस्\\ \hline
चतुर्थी & \cbham ङे $\rightarrow$ ए & \cpada भ्याम् & \cpada भ्यस्\\ \hline
पञ्चमी & \cbham ङसिँ $\rightarrow$ अस् & \cpada भ्याम् & \cpada भ्यस्\\ \hline
षष्ठी & \cbham ङस् $\rightarrow$ अस् & \cbham ओस् & \cbham आम्\\ \hline
सप्तमी & \cbham ङि $\rightarrow$ इ & \cbham ओस् & \cpada सुप् $\rightarrow$ सु\\ \hline		
सम्बोधन &  &  & \\ \hline			
		\end{supertabular}
		\end{center}
\end{multicols}

%\end{minipage}
\normalsize


%\clearpage
\section{The Third Task}
Let us look at the \hyperlink{जन्मन् (नपु॰) Birth}{जन्मन्-table} and observe that the only difference between the \hyperlink{नामन् (नपु॰) Name}{नामन्-table} and the \hyperlink{जन्मन् (नपु॰) Birth}{जन्मन्-table} is that the अ in the अन् ending of जन्मन् is never dropped. This can be brought in as a simple rule: Do not drop अ in the अन् ending of जन्मन्, and apply all the other rules that were used to form the नामन्-table. 

Following on, let us look at the \hyperlink{वारि (नपु॰) Water}{वारि-table}, and we observe that a ण् (न्) appears in various cells. Is there something common to the suffixes in those cells? What rule can we make from there?

%\vspace*{1cm}

%\par
%\clearpage

\begin{multicols}{2}
\TrickSupertabularIntoMulticols
 \begin{center} 
 \begin{supertabular}{|c|c|c|c|}\hline 
 \multicolumn {4}{|c|}{\cellcolor{blue!10}  \nextTable \hypertarget{जन्मन् (नपु॰) Birth}{जन्मन् (नपु॰) Birth}}  \\ \hline  
 & एक॰ & द्वि॰ & बहु॰  \\ \hline 
 प्रथमा & \cpada जन्म & \cbham जन्मनी & \csarva जन्मानि \\ \hline 
 द्वितीया & \cbham जन्म & \cbham जन्मनी & \csarva जन्मानि \\ \hline 
 तृतीया & \cbham जन्मनः & \cpada जन्मभ्याम् & \cpada जन्मभिः \\ \hline 
 चतुर्थी & \cbham जन्मने & \cpada जन्मभ्याम् & \cpada जन्मभ्यः \\ \hline 
 पञ्चमी & \cbham जन्मनः & \cpada जन्मभ्याम् & \cpada जन्मभ्यः \\ \hline 
 षष्ठी & \cbham जन्मनः & \cbham जन्मनोः & \cbham जन्मनाम् \\ \hline 
सप्तमी & \cbham जन्मनि & \cbham जन्मनोः & \cpada जन्मसु \\ \hline 
सम्बोधन & हे जन्म-जन्मन् & हे जन्मनी & हे जन्मानि \\ \hline 
\end{supertabular} 
\end{center} 

\columnbreak

\begin{center} 
 \begin{supertabular}{|c|c|c|c|}\hline 
 \multicolumn {4}{|c|}{\cellcolor{blue!10}  \nextTable \hypertarget{वारि (नपु॰) Water}{वारि (नपु॰) Water}}  \\ \hline  
 & एक॰ & द्वि॰ & बहु॰  \\ \hline 
 प्रथमा & \cpada वारि & \cbham वारिणी & \csarva वरीणि \\ \hline 
 द्वितीया & \cbham वारि & \cbham वारिणी & \csarva वरीणि \\ \hline 
 तृतीया & \cbham वारिणा & \cpada वारिभ्याम् & \cpada वारिभिः \\ \hline 
 चतुर्थी & \cbham वारिणे & \cpada वारिभ्याम् & \cpada वारिभ्यः \\ \hline 
 पञ्चमी & \cbham वारिणः & \cpada वारिभ्याम् & \cpada वारिभ्यः \\ \hline 
 षष्ठी & \cbham वारिणः & \cbham वारिणोः & \cbham वारीणाम् \\ \hline 
सप्तमी & \cbham वारिणि & \cbham वारिणोः & \cpada वारिषु \\ \hline 
सम्बोधन & हे वारि-वारे & हे वारिणी & हे वरीणि \\ \hline 
\end{supertabular} 
\end{center}		
\end{multicols}	
		%\caption{चकारान्तः पुंलिङ्गः प्राञ्च् शब्दः}
%\end{table}

\section{Your Task}

\vspace{-0.5cm}

\begin{ewosp}
\item Memorise the सुप् प्रत्ययाः-table. Recite the correct table 2-3 times loudly, if possible in a rhythm.
\item Get a new 64-page exercise book. Reserve two pages for each of the following tables (one from each item in the following list).
\item Using the raw word and the सुप् प्रत्ययाः-table, write down the ``First Cut'' table.
\item Copy the correct table on that page below the ``First Cut'' table.
\item Highlight the cell entries that are different between the two tables and identify if there is a pattern among the entries that differ. 
\item Suggest rules to obtain the tables using the raw word + सुप् प्रत्ययाः-table.
\item By observing the difference in tables try to memorise all the 25 tables.
\end{ewosp}
%\vspace{5cm}

\clearpage

\begin{multicols}{2}
\TrickSupertabularIntoMulticols
\begin{center} 
 \begin{supertabular}{|c|c|c|c|}\hline 
 \multicolumn {4}{|c|}{\cellcolor{blue!10}  \nextTable \hypertarget{दधि (नपु॰) Curd}{दधि (नपु॰) Curd}}  \\ \hline  
 & एक॰ & द्वि॰ & बहु॰  \\ \hline 
 प्रथमा & \cpada दधि & \cbham दधिनी & \csarva दधीनि \\ \hline 
 द्वितीया & \cbham दधि & \cbham दधिनी & \csarva दधीनि \\ \hline 
 तृतीया & \cbham दध्ना & \cpada दधिभ्याम् & \cpada दधिभिः \\ \hline 
 चतुर्थी & \cbham दध्ने & \cpada दधिभ्याम् & \cpada दधिभ्यः \\ \hline 
 पञ्चमी & \cbham दध्नः & \cpada दधिभ्याम् & \cpada दधिभ्यः \\ \hline 
 षष्ठी & \cbham दध्नः & \cbham दध्नोः & \cbham दध्नाम् \\ \hline 
सप्तमी & \cbham दध्नि-दधिनि & \cbham दध्नोः & \cpada दधिषु \\ \hline 
सम्बोधन & हे दधि-दधे & हे दधिनी & हे दधीनि \\ \hline 
\end{supertabular} 
\end{center}

 \begin{center} 
 \begin{supertabular}{|c|c|c|c|}\hline 
 \multicolumn {4}{|c|}{\cellcolor{blue!10}  \nextTable \hypertarget{मधु (नपु॰) Honey}{मधु (नपु॰) Honey}}  \\ \hline  
 & एक॰ & द्वि॰ & बहु॰  \\ \hline 
 प्रथमा & \cpada मधु & \cbham मधुनी & \csarva मधूनि \\ \hline 
 द्वितीया & \cbham मधु & \cbham मधुनी & \csarva मधूनि \\ \hline 
 तृतीया & \cbham मधुना & \cpada मधुभ्याम् & \cpada मधुभिः \\ \hline 
 चतुर्थी & \cbham मधुने & \cpada मधुभ्याम् & \cpada मधुभ्यः \\ \hline 
 पञ्चमी & \cbham मधुनः & \cpada मधुभ्याम् & \cpada मधुभ्यः \\ \hline 
 षष्ठी & \cbham मधुनः & \cbham मधुनोः & \cbham मधूनाम् \\ \hline 
सप्तमी & \cbham मधुनि & \cbham मधुनोः & \cpada मधुषु \\ \hline 
सम्बोधन & हे मधु & हे मधुनी & हे मधुनि \\ \hline 
\end{supertabular} 
\end{center} 
	
 \begin{center} 
 \begin{supertabular}{|c|c|c|c|}\hline 
 \multicolumn {4}{|c|}{\cellcolor{blue!10}  \nextTable \hypertarget{जगत् (नपु॰) World}{जगत् (नपु॰) World}}  \\ \hline  
 & एक॰ & द्वि॰ & बहु॰  \\ \hline 
 प्रथमा & \cpada जगत्-द् & \cbham जगती & \csarva जगन्ति \\ \hline 
 द्वितीया & \cbham जगत्-द् & \cbham जगती & \csarva जगन्ति \\ \hline 
 तृतीया & \cbham जगता & \cpada जगद्भ्याम् & \cpada जगद्भिः \\ \hline 
 चतुर्थी & \cbham जगते & \cpada जगद्भ्याम् & \cpada जगद्भ्यः \\ \hline 
 पञ्चमी & \cbham जगतः & \cpada जगद्भ्याम् & \cpada जगद्भ्यः \\ \hline 
 षष्ठी & \cbham जगतः & \cbham जगतोः & \cbham जगताम् \\ \hline 
सप्तमी & \cbham जगति & \cbham जगतोः & \cpada जगत्सु \\ \hline 
सम्बोधन & हे जगत्-द् & हे जगती & हे जगन्ति \\ \hline 
\end{supertabular} 
\end{center} 

 \begin{center} 
 \begin{supertabular}{|c|c|c|c|}\hline 
 \multicolumn {4}{|c|}{\cellcolor{blue!10}  \nextTable \hypertarget{मनस् (नपु॰) Mind}{मनस् (नपु॰) Mind}}  \\ \hline  
 & एक॰ & द्वि॰ & बहु॰  \\ \hline 
 प्रथमा & \cpada मनः \sref{7-1-23} & \cbham मनसी & \csarva मनांसि \\ \hline 
 द्वितीया & \cbham मनः & \cbham मनसी & \csarva मनांसि \\ \hline 
 तृतीया & \cbham मनसा & \cpada मनोभ्याम् & \cpada मनोभिः \\ \hline 
 चतुर्थी & \cbham मनसे & \cpada मनोभ्याम् & \cpada मनोभिः \\ \hline 
 पञ्चमी & \cbham मनसः & \cpada मनोभ्याम् & \cpada मनोभ्यः \\ \hline 
 षष्ठी & \cbham मनसः & \cbham मनसोः & \cbham मनसाम् \\ \hline 
सप्तमी & \cbham मनसि & \cbham मनसोः & \cpada मनस्सु \\ \hline 
सम्बोधन & हे मनः & हे मनसी & हे मनांसि \\ \hline 
\end{supertabular} 
\end{center} 

\medskip

 \begin{center} 
 \begin{supertabular}{|c|c|c|c|}\hline 
 \multicolumn {4}{|c|}{\cellcolor{blue!10}  \nextTable \hypertarget{ज्योतिस् (नपु॰) Light}{ज्योतिस् (नपु॰) Light}}  \\ \hline  
 & एक॰ & द्वि॰ & बहु॰  \\ \hline 
 प्रथमा & \cpada ज्योतिः & \cbham ज्योतिषी & \csarva ज्योतिंषि \\ \hline 
 द्वितीया & \cbham ज्योतिः & \cbham ज्योतिषी & \csarva ज्योतिंषि \\ \hline 
 तृतीया & \cbham ज्योतिषा & \cpada ज्योतिर्भ्याम् & \cpada ज्योतिर्भिः \\ \hline 
 चतुर्थी & \cbham ज्योतिषे & \cpada ज्योतिर्भ्याम् & \cpada ज्योतिर्भ्यः \\ \hline 
 पञ्चमी & \cbham ज्योतिषः & \cpada ज्योतिर्भ्याम् & \cpada ज्योतिर्भ्यः \\ \hline 
 षष्ठी & \cbham ज्योतिषः & \cbham ज्योतिषोः & \cbham ज्योतिषाम् \\ \hline 
सप्तमी & \cbham ज्योतिषि & \cbham ज्योतिषोः & \cpada ज्योतिष्षु \\ \hline 
सम्बोधन & हे ज्योतिः & हे ज्योतिषी & हे ज्योतिंषि \\ \hline 
\end{supertabular} 
\end{center} 

 \begin{center} 
 \begin{supertabular}{|c|c|c|c|}\hline 
 \multicolumn {4}{|c|}{\cellcolor{blue!10}  \nextTable \hypertarget{फल (नपु॰) Fruit}{फल (नपु॰) Fruit}}  \\ \hline  
 & एक॰ & द्वि॰ & बहु॰  \\ \hline 
 प्रथमा & \cpada फलम् & \cbham फले & \csarva फलानि \\ \hline 
 द्वितीया & \cbham फलम् & \cbham फले & \csarva फलानि \\ \hline 
 तृतीया & \cbham फलेन & \cpada फलाभ्याम् & \cpada फलैः \\ \hline 
 चतुर्थी & \cbham फलाय & \cpada फलाभ्याम् & \cpada फलेभ्यः \\ \hline 
 पञ्चमी & \cbham फलात् & \cpada फलाभ्याम् & \cpada फलेभ्यः \\ \hline 
 षष्ठी & \cbham फलस्य & \cbham फलयोः & \cbham फलानाम् \\ \hline 
सप्तमी & \cbham फले & \cbham फलयोः & \cpada फलेषु \\ \hline 
सम्बोधन & हे फल & हे फले & हे फलानि \\ \hline 
\end{supertabular} 
\end{center} 

 \begin{center} 
 \begin{supertabular}{|c|c|c|c|}\hline 
 \multicolumn {4}{|c|}{\cellcolor{blue!10}  \nextTable \hypertarget{अभिजित् (पु॰) Victorious}{अभिजित् (पु॰) Victorious}}  \\ \hline  
 & एक॰ & द्वि॰ & बहु॰  \\ \hline 
 प्रथमा & \csarva अभिजित् & \csarva अभिजितौ & \csarva अभिजितः \\ \hline 
 द्वितीया & \csarva अभिजितम् & \csarva अभिजितौ & \cbham अभिजितः \\ \hline 
 तृतीया & \cbham अभिजिता & \cpada अभिजिद्ब्ध्याम् & \cpada अभिजित्द्भिः \\ \hline 
 चतुर्थी & \cbham अभिजिते & \cpada अभिजिद्भ्याम् & \cpada अभिजिद्भ्यः \\ \hline 
 पञ्चमी & \cbham अभिजितः & \cpada अभिजिताद्भ्याम् & \cpada अभिजितभ्यः \\ \hline 
 षष्ठी & \cbham अभिजितः & \cbham अभिजितोः & \cbham अभिजिताम् \\ \hline 
सप्तमी & \cbham अभिजिति & \cbham अभिजितोः & \cpada अभिजित्सु \\ \hline 
सम्बोधन & हे अभिजित् & हे अभिजितौ & हे अभिजितः \\ \hline 
\end{supertabular} 
\end{center}

 \begin{center} 
 \begin{supertabular}{|c|c|c|c|}\hline 
 \multicolumn {4}{|c|}{\cellcolor{blue!10}  \nextTable \hypertarget{धीमत् (पु॰) Intelligent}{धीमत् (पु॰) Intelligent}}  \\ \hline  
 & एक॰ & द्वि॰ & बहु॰  \\ \hline 
 प्रथमा & \csarva धीमान् & \csarva धीमन्तौ & \csarva धीमन्तः \\ \hline 
 द्वितीया & \csarva धीमन्तम् & \csarva धीमन्तौ & \cbham धीमतः \\ \hline 
 तृतीया & \cbham धीमता & \cpada धीमद्भ्याम् & \cpada धीमद्भिः \\ \hline 
 चतुर्थी & \cbham धीमते & \cpada धीमद्भ्याम् & \cpada धीमद्भ्यः \\ \hline 
 पञ्चमी & \cbham धीमतः & \cpada धीमद्भ्याम् & \cpada धीमद्भ्यः \\ \hline 
 षष्ठी & \cbham धीमतः & \cbham धीमतोः & \cbham धीमताम् \\ \hline 
सप्तमी & \cbham धीमति & \cbham धीमतोः & \cpada धीमत्सु \\ \hline 
सम्बोधन & हे धीमन् & हे धीमन्तौ & हे धीमन्तः \\ \hline 
\end{supertabular} 
\end{center} 

 \begin{center} 
 \begin{supertabular}{|c|c|c|c|}\hline 
 \multicolumn {4}{|c|}{\cellcolor{blue!10}  \nextTable \hypertarget{सरित् (स्त्री॰) River}{सरित् (स्त्री॰) River}}  \\ \hline  
 & एक॰ & द्वि॰ & बहु॰  \\ \hline 
 प्रथमा & \csarva सरित् & \csarva सरितौ & \csarva सरितः \\ \hline 
 द्वितीया & \csarva सरितम् & \csarva सरितौ & \cbham सरितः \\ \hline 
 तृतीया & \cbham सरिता & \cpada सरिद्भ्याम् & \cpada सरिद्भिः \\ \hline 
 चतुर्थी & \cbham सरिते & \cpada सरिद्भ्याम् & \cpada सरिद्भ्यः \\ \hline 
 पञ्चमी & \cbham सरितः & \cpada सरिद्भ्याम् & \cpada सरिद्भ्यः \\ \hline 
 षष्ठी & \cbham सरितः & \cbham सरितोः & \cbham सरिताम् \\ \hline 
सप्तमी & \cbham सरिति & \cbham सरितोः & \cpada सरित्सु \\ \hline 
सम्बोधन & हे सरित् & हे सरितौ & हे सरितः \\ \hline 
\end{supertabular} 
\end{center} 

 \begin{center} 
 \begin{supertabular}{|c|c|c|c|}\hline 
 \multicolumn {4}{|c|}{\cellcolor{blue!10}  \nextTable \hypertarget{खादत् (पु॰) The Eating One}{खादत् (पु॰) The Eating One}}  \\ \hline  
 & एक॰ & द्वि॰ & बहु॰  \\ \hline 
 प्रथमा & \csarva खादन् & \csarva खादन्तौ & \csarva खादन्तः \\ \hline 
 द्वितीया & \csarva खादन्तम् & \csarva खादन्तौ & \cbham खादतः \\ \hline 
 तृतीया & \cbham खादता & \cpada खादद्भ्याम् & \cpada खादद्भिः \\ \hline 
 चतुर्थी & \cbham खादते & \cpada खादद्भ्याम् & \cpada खदद्भ्यः \\ \hline 
 पञ्चमी & \cbham खादतः & \cpada खादद्भ्याम् & \cpada खादद्भ्यः \\ \hline 
 षष्ठी & \cbham खादतः & \cbham खादतोः & \cbham खादताम् \\ \hline 
सप्तमी & \cbham खादति & \cbham खादतोः & \cpada खादत्सु \\ \hline 
सम्बोधन & हे खादन् & हे खादन्तौ & हे खादन्तः \\ \hline 
\end{supertabular} 
\end{center} 

 \begin{center} 
 \begin{supertabular}{|c|c|c|c|}\hline 
 \multicolumn {4}{|c|}{\cellcolor{blue!10}  \nextTable \hypertarget{ज्ञानिन् (पु॰) Wise}{ज्ञानिन् (पु॰) Wise}}  \\ \hline  
 & एक॰ & द्वि॰ & बहु॰  \\ \hline 
 प्रथमा & \csarva ज्ञानी & \csarva ज्ञानिनौ & \csarva ज्ञानिनः \\ \hline 
 द्वितीया & \csarva ज्ञानिनम् & \csarva ज्ञानिनौ & \cbham ज्ञानिनः \\ \hline 
 तृतीया & \cbham ज्ञानिना & \cpada ज्ञानिभ्याम् & \cpada ज्ञानिभिः \\ \hline 
 चतुर्थी & \cbham ज्ञानिने & \cpada ज्ञानिभ्याम् & \cpada ज्ञानिभ्यः \\ \hline 
 पञ्चमी & \cbham ज्ञानिनः & \cpada ज्ञानिभ्याम् & \cpada ज्ञानिभ्यः \\ \hline 
 षष्ठी & \cbham ज्ञानिनः & \cbham ज्ञानिनोः & \cbham ज्ञानिनाम् \\ \hline 
सप्तमी & \cbham ज्ञानिनि & \cbham ज्ञानिनोः & \cpada ज्ञानिषु \\ \hline 
सम्बोधन & हे ज्ञानिन् & हे ज्ञानिनौ & हे ज्ञानिनः \\ \hline 
\end{supertabular} 
\end{center}
 
 \begin{center} 
 \begin{supertabular}{|c|c|c|c|}\hline 
 \multicolumn {4}{|c|}{\cellcolor{blue!10}  \nextTable \hypertarget{राम (पु॰) Ram}{राम (पु॰) Ram}}  \\ \hline  
 & एक॰ & द्वि॰ & बहु॰  \\ \hline 
 प्रथमा & \csarva रामः & \csarva रामौ & \csarva रामाः \\ \hline 
 द्वितीया & \csarva रामम् & \csarva रामौ & \cbham रामान् \\ \hline 
 तृतीया & \cbham रामेण & \cpada रामाभ्याम् & \cpada रामैः \\ \hline 
 चतुर्थी & \cbham रामाय & \cpada रामाभ्याम् & \cpada रामेभ्यः \\ \hline 
 पञ्चमी & \cbham रामात् & \cpada रामाभ्याम् & \cpada रामेभ्यः \\ \hline 
 षष्ठी & \cbham रामस्य & \cbham रामयोः & \cbham रामाणाम् \\ \hline 
सप्तमी & \cbham रामे & \cbham रामयोः & \cpada रामेषु \\ \hline 
सम्बोधन & हे राम & हे रामौ & हे रामाः \\ \hline 
\end{supertabular} 
\end{center} 

\begin{center} 
 \begin{supertabular}{|c|c|c|c|}\hline 
 \multicolumn {4}{|c|}{\cellcolor{blue!10}  \nextTable \hypertarget{लता (स्त्री॰) Creeper}{लता (स्त्री॰) Creeper}}  \\ \hline  
 & एक॰ & द्वि॰ & बहु॰  \\ \hline 
 प्रथमा & \csarva लता & \csarva लते & \csarva लताः \\ \hline 
 द्वितीया & \csarva लताम् & \csarva लते & \cbham लताः \\ \hline 
 तृतीया & \cbham लतया & \cpada लताभ्याम् & \cpada लताभिः \\ \hline 
 चतुर्थी & \cbham लतायै & \cpada लताभ्याम् & \cpada लताभ्यः \\ \hline 
 पञ्चमी & \cbham लतायाः & \cpada लताभ्याम् & \cpada लथाभ्यः \\ \hline 
 षष्ठी & \cbham लतायाः & \cbham लतयोः & \cbham लतानाम् \\ \hline 
सप्तमी & \cbham लतायाम् & \cbham लतयोः & \cpada लतासु \\ \hline 
सम्बोधन & हे लते & हे लते & हे लताः \\ \hline 
\end{supertabular} 
\end{center} 

 \begin{center} 
 \begin{supertabular}{|c|c|c|c|}\hline 
 \multicolumn {4}{|c|}{\cellcolor{blue!10}  \nextTable \hypertarget{रवि (पु॰) Sun}{रवि (पु॰) Sun}}  \\ \hline  
 & एक॰ & द्वि॰ & बहु॰  \\ \hline 
 प्रथमा & \csarva रविः & \csarva रवी & \csarva रवयः \\ \hline 
 द्वितीया & \csarva रविम् & \csarva रवी & \cbham रवीन् \\ \hline 
 तृतीया & \cbham रविणा & \cpada रविभ्याम् & \cpada रविभिः \\ \hline 
 चतुर्थी & \cbham रवये & \cpada रविभ्याम् & \cpada रविभ्यः \\ \hline 
 पञ्चमी & \cbham रवेः & \cpada रविभ्याम् & \cpada रविभ्यः \\ \hline 
 षष्ठी & \cbham रवेः & \cbham रव्योः & \cbham रवीणाम् \\ \hline 
सप्तमी & \cbham रवौ & \cbham रव्योः & \cpada रविषु \\ \hline 
सम्बोधन & हे रवे & हे रवी & हे रवयः \\ \hline 
\end{supertabular} 
\end{center} 

 \begin{center} 
 \begin{supertabular}{|c|c|c|c|}\hline 
 \multicolumn {4}{|c|}{\cellcolor{blue!10}  \nextTable \hypertarget{मति (स्त्री॰) Idea}{मति (स्त्री॰) Idea}}  \\ \hline  
 & एक॰ & द्वि॰ & बहु॰  \\ \hline 
 प्रथमा & \csarva मतिः & \csarva मती & \csarva मतयः \\ \hline 
 द्वितीया & \csarva मतिम् & \csarva मती & \cbham मतीः \\ \hline 
 तृतीया & \cbham मत्या & \cpada मतिभ्याम् & \cpada मतिभिः \\ \hline 
 चतुर्थी & \cbham मतये-मत्यै & \cpada मतिभ्याम् & \cpada मतिभ्यः \\ \hline 
 पञ्चमी & \cbham मतेः-मत्याः & \cpada मतिभ्याम् & \cpada मतिभ्यः \\ \hline 
 षष्ठी & \cbham मतेः-मत्याः & \cbham मत्योः & \cbham मतीनाम् \\ \hline 
सप्तमी & \cbham मतौ-मत्याम् & \cbham मत्योः & \cpada मतिषु \\ \hline 
सम्बोधन & हे मते & हे मती & हे मतयः \\ \hline 
\end{supertabular} 
\end{center} 

 \begin{center} 
 \begin{supertabular}{|c|c|c|c|}\hline 
 \multicolumn {4}{|c|}{\cellcolor{blue!10}  \nextTable \hypertarget{नदी (स्त्री॰) River}{नदी (स्त्री॰) River}}  \\ \hline  
 & एक॰ & द्वि॰ & बहु॰  \\ \hline 
 प्रथमा & \csarva नदी & \csarva नद्यौ & \csarva नद्यः \\ \hline 
 द्वितीया & \csarva नदीम् & \csarva नद्यौ & \cbham नदीः \\ \hline 
 तृतीया & \cbham नद्या & \cpada नदीभ्याम् & \cpada नदिभिः \\ \hline 
 चतुर्थी & \cbham नद्यै & \cpada नदीभ्याम् & \cpada नदीभ्यः \\ \hline 
 पञ्चमी & \cbham नद्याः & \cpada नदीभ्याम् & \cpada नदीभ्यः \\ \hline 
 षष्ठी & \cbham नद्याः & \cbham नद्योः & \cbham नदीनाम् \\ \hline 
सप्तमी & \cbham नद्याम् & \cbham नद्योः & \cpada नदीषु \\ \hline 
सम्बोधन & हे नदि & हे नद्यौ &  हे नदी \\ \hline 
\end{supertabular} 
\end{center} 

 \begin{center} 
 \begin{supertabular}{|c|c|c|c|}\hline 
 \multicolumn {4}{|c|}{\cellcolor{blue!10}  \nextTable \hypertarget{साधु (पु॰) Saint}{साधु (पु॰) Saint}}  \\ \hline  
 & एक॰ & द्वि॰ & बहु॰  \\ \hline 
 प्रथमा & \csarva साधुः & \csarva साधू & \csarva साधवः \\ \hline 
 द्वितीया & \csarva साधुम् & \csarva साधू & \cbham साधून् \\ \hline 
 तृतीया & \cbham साधुना & \cpada साधुभ्याम् & \cpada साधुभिः \\ \hline 
 चतुर्थी & \cbham साधवे & \cpada साधुभ्याम् & \cpada साधुभ्यः \\ \hline 
 पञ्चमी & \cbham साधोः & \cpada साधुभ्याम् & \cpada साधुभ्यः \\ \hline 
 षष्ठी & \cbham साधोः & \cbham साध्वोः & \cbham साधूनाम् \\ \hline 
सप्तमी & \cbham साधौ & \cbham साध्वोः & \cpada साधुषु \\ \hline 
सम्बोधन & हे साधो & हे साधू & हे साधवः \\ \hline 
\end{supertabular} 
\end{center} 

 \begin{center} 
 \begin{supertabular}{|c|c|c|c|}\hline 
 \multicolumn {4}{|c|}{\cellcolor{blue!10}  \nextTable \hypertarget{धेनु (स्त्री॰) Cow}{धेनु (स्त्री॰) Cow}}  \\ \hline  
 & एक॰ & द्वि॰ & बहु॰  \\ \hline 
 प्रथमा & \csarva धेनुः & \csarva धेनू & \csarva धेनवः \\ \hline 
 द्वितीया & \csarva धेनुम् & \csarva धेनू & \cbham धेनूः \\ \hline 
 तृतीया & \cbham धेन्वा & \cpada धेनुभ्याम् & \cpada धेनुभिः \\ \hline 
 चतुर्थी & \cbham धेन्वे-धेन्वै & \cpada धनुभ्याम् & \cpada धेनुभ्यः \\ \hline 
 पञ्चमी & \cbham धेनोः-धेन्वाः & \cpada धेनुभ्याम् & \cpada धेनुभ्यः \\ \hline 
 षष्ठी & \cbham धेनोः-धेन्वाः & \cbham धेन्वोः & \cbham धेनूनाम् \\ \hline 
सप्तमी & \cbham धेनौ-धेन्वाम् & \cbham धेन्वोः & \cpada धेनुषु \\ \hline 
सम्बोधन & हे धेनो & हे धेनू & हे धेनवः \\ \hline 
\end{supertabular} 
\end{center} 

 \begin{center} 
 \begin{supertabular}{|c|c|c|c|}\hline 
 \multicolumn {4}{|c|}{\cellcolor{blue!10}  \nextTable \hypertarget{वधू (स्त्री॰) Bride}{वधू (स्त्री॰) Bride}}  \\ \hline  
 & एक॰ & द्वि॰ & बहु॰  \\ \hline 
 प्रथमा & \csarva वधूः & \csarva वध्वौ & \csarva वध्वः \\ \hline 
 द्वितीया & \csarva वधूम् & \csarva वध्वौ & \cbham वधूः \\ \hline 
 तृतीया & \cbham वध्वा & \cpada वधूभ्याम् & \cpada वधूभिः \\ \hline 
 चतुर्थी & \cbham वध्वै & \cpada वधूभ्याम् & \cpada वधूभ्यः \\ \hline 
 पञ्चमी & \cbham वध्वाः & \cpada वधूभ्याम् & \cpada वधूभ्यः \\ \hline 
 षष्ठी & \cbham वध्वाः & \cbham वध्वोः & \cbham वधूनाम् \\ \hline 
सप्तमी & \cbham वध्वाम् & \cbham वध्वोः & \cpada वधूषु \\ \hline 
सम्बोधन & हे वधु & हे वध्वौ & हे वध्वः \\ \hline 
\end{supertabular} 
\end{center} 

\columnbreak

 \begin{center} 
 \begin{supertabular}{|c|c|c|c|}\hline 
 \multicolumn {4}{|c|}{\cellcolor{blue!10}  \nextTable \hypertarget{पितृ (पु॰) Father}{पितृ (पु॰) Father}}  \\ \hline  
 & एक॰ & द्वि॰ & बहु॰  \\ \hline 
 प्रथमा & \csarva पिता & \csarva पितरौ & \csarva पितरः \\ \hline 
 द्वितीया & \csarva पितरम् & \csarva पितरौ & \cbham पितॄन् \\ \hline 
 तृतीया & \cbham पित्रा & \cpada पितृभ्याम् & \cpada पितृभिः \\ \hline 
 चतुर्थी & \cbham पित्रे & \cpada पितृभ्याम् & \cpada पितृभ्यः \\ \hline 
 पञ्चमी & \cbham पितुः & \cpada पितृभ्याम् & \cpada पितृभ्यः \\ \hline 
 षष्ठी & \cbham पितुः & \cbham पित्रोः & \cbham पितॄणाम् \\ \hline 
सप्तमी & \cbham पितरि & \cbham पित्रोः & \cpada पितृषु \\ \hline 
सम्बोधन & हे पितः & हे पितरौ & हे पितरः \\ \hline 
\end{supertabular} 
\end{center} 

 \begin{center} 
 \begin{supertabular}{|c|c|c|c|}\hline 
 \multicolumn {4}{|c|}{\cellcolor{blue!10}  \nextTable \hypertarget{कर्तृ (पु॰) Doer}{कर्तृ (पु॰) Doer}}  \\ \hline  
 & एक॰ & द्वि॰ & बहु॰  \\ \hline 
 प्रथमा & \csarva कर्ता & \csarva कर्तारौ & \csarva कर्तारः \\ \hline 
 द्वितीया & \csarva कर्तारम् & \csarva कर्तारौ & \cbham कर्तॄन् \\ \hline 
 तृतीया & \cbham कर्त्रा & \cpada कर्तृभ्याम् & \cpada कर्तृभिः \\ \hline 
 चतुर्थी & \cbham कर्त्रे & \cpada कर्तृभ्याम् & \cpada कर्तृभ्यः \\ \hline 
 पञ्चमी & \cbham कर्तुः & \cpada कर्तृभ्याम् & \cpada कर्तृभ्यः \\ \hline 
 षष्ठी & \cbham कर्तुः & \cbham कर्त्रोः & \cbham कर्तॄणाम् \\ \hline 
सप्तमी & \cbham कर्तरि & \cbham कर्त्रोः & \cpada कर्तृषु \\ \hline 
सम्बोधन & हे कर्तः & हे कर्तारौ & हे कर्तारः \\ \hline 
\end{supertabular} 
\end{center} 

 \begin{center} 
 \begin{supertabular}{|c|c|c|c|}\hline 
 \multicolumn {4}{|c|}{\cellcolor{blue!10}  \nextTable \hypertarget{आत्मन् (पु॰) Self}{आत्मन् (पु॰) Self}}  \\ \hline  
 & एक॰ & द्वि॰ & बहु॰  \\ \hline 
 प्रथमा & \csarva आत्मा & \csarva आत्मानौ & \csarva आत्मानः \\ \hline 
 द्वितीया & \csarva आत्मानम् & \csarva आत्मानौ & \cbham आत्मनः \\ \hline 
 तृतीया & \cbham आत्मना & \cpada आत्मभ्याम् & \cpada आत्मभिः \\ \hline 
 चतुर्थी & \cbham आत्मने & \cpada आत्मभ्याम् & \cpada आत्मभ्यः \\ \hline 
 पञ्चमी & \cbham आत्मनः & \cpada आत्मभ्याम् & \cpada आत्मभ्यः \\ \hline 
 षष्ठी & \cbham आत्मनः & \cbham आत्मनोः & \cbham आत्मनान् \\ \hline 
सप्तमी & \cbham आत्मनि & \cbham आत्मनोः & \cpada आत्मसु \\ \hline 
सम्बोधन & हे आत्मन् & हे आत्मानौ & हे आत्मानः \\ \hline 
\end{supertabular} 
\end{center} 
\end{multicols}

\clearpage


%\belowdisplayskip=2pt%12pt %plus 3pt minus 9pt
%\belowdisplayshortskip=2pt%7pt %plus 3pt minus 4pt
\section{सूत्राणि}
The sutras used to derive the above tables are given below. The red part in the sutras is the अनुवृत्तिः। The description with the sutras below provide only hints on their applicability. After you memorise the above 25 tables use the following sutras to consolidate the memorised material. 

A full derivation normally uses multiple sutras; to see this process please see books like व्यावहारिकं पाणिनीयम्, डॉ॰ नरेन्द्रः, संस्कृतकार्यालयः, श्रीअरविन्दाश्रमः, पुदुच्चेरी - ६०५००२ भारतः १९९९, or visit \url{http://lanover.com/lan/sanskrit/subanta.html} that can be used to obtain a complete derivation of the entire विभक्ति-table. The site \url{http://avg-sanskrit.org/documents/} has many documents which have complete derivations of many words.

%\medskip

\begingroup
%\setlength\abovedisplayskip{0pt}
%\setlength\belowdisplayskip{0pt}
%\setlength\parskip{-5pt}
%\begin{equation}
%\mbox{अत् एङ् गुणः वृद्धिः}
%\end{equation}

%\belowdisplayskip=2pt%12pt %plus 3pt minus 9pt
%\belowdisplayshortskip=2pt%7pt %plus 3pt minus 4pt

%\input{sutrani.tex}
%\sutra{7-1-23}{स्वमोः नपुंसकात् \txcr{लुक्}}
%\sutra{6-1-68}{हल्ङ्याब्भ्यः दीर्घात् सुतिसि अपृक्तं हल्}
%\sutra{1-1-3}{इकः गुणवृद्धी \txcr{वृद्धिः} \txcg{गुणः}}
%\belowdisplayskip=2pt%12pt %plus 3pt minus 9pt
%\belowdisplayshortskip=2pt%7pt %plus 3pt minus 4pt
\sutra{7-1-23}{स्वमोः नपुंसकात् \txcr{लुक् अङ्गस्य}}
\hsref{7-1-23} मनः, वारि, नाम, ज्योतिः, धनुः and other नपुंसकलिङ्गः words for the \{1.1\} and \{2.1\} forms.
\sutra{6-1-68}{हल्ङ्याब्भ्यः दीर्घात् सुतिसि अपृक्तं हल्}
\hsref{6-1-68} अभिजित्, नदी, माला, धनी, आत्मा, पिता, दाता, बलवान, पठन् and other words for the \{1.1\} (सु प्रत्ययः) form.
\sutra{1-2-41}{अपृक्तः एकाल् प्रत्ययः}
Sutra \sref{1-2-41} says that a प्रत्ययः with only one letter like सुँ $\rightarrow$ स् is called अपृक्तः।\\
\sutra{6-4-8}{सर्वनामस्थाने च असम्बुद्धौ \txcr{न उपधायाः अङ्गस्य दीर्घः}}
\hsref{6-4-8} आत्मा, आत्मानौ, आत्मानः, आत्मानम्, पिता।
Once a न् is inserted for नपुंसकलिङ् words by sutra \sref{7-1-72} in \{1.3\} and \{2.3\} positions, sutra \sref{6-4-8} instructs to make the उपधा of the resulting word long and so we get फलानि, वारीणि, मधूनि, जन्मानि, and नामानि।
\sutra{6-4-12}{इन्हन्पूषार्यम्णां शौ}
Sutra \sref{6-4-12} prevents दीर्घः उपधा (due to \sref{6-4-8}) for words ending in इन् and words हन् and पूषन् unless the following प्रत्ययः is the शि सर्वनामस्थानं प्रत्ययः for नपुंसकलिङ्गः।
\sutra{6-4-13}{सौ च \txcr{उपधायाः असम्बुद्धौ इन्हन्पूषार्यम्णाम् अङ्गस्य दीर्घः}}
Sutra \sref{6-4-13} instructs to have दीर्घः उपधा for words ending in इन् and words हन् and पूषन् when the following प्रत्ययः is सुँ| Thus \hsref{6-4-13} धनी from धनिन् + सुँ।
\sutra{7-1-94}{ऋदुशनस्पुरुदंसोनेहसां च \txcr{सर्वनामस्थाने असम्बुद्धौ अनङ् सौ अङ्गस्य}}
\hsref{7-1-94} पिता दाता; sutra \sref{7-1-94} brings in अनङ् as an आदेश for ऋ and then sutra \sref{6-4-8} make the उपधा दीर्घः thus getting the form पिता and दाता।
\sutra{6-4-11}{अप्तृन्तृच्स्वसृनप्तृनेष्टृत्वष्टृक्षतृहोतृपोतृप्रशास्तॄप्रशास्तॄणाम् \txcr{न उपधायाः सर्वनामस्थाने असम्बुद्धौ अङ्गस्य दीर्घः}}
\hsref{6-4-11} दाता दातारौ दातारः दातारम्| Sutra \sref{6-4-11} is to make उपधा दीर्घः for all the words listed in the sutra for all the सर्वनामस्थान positions while sutra \sref{7-1-94} works only for \{1.1\}.
\sutra{6-4-14}{अतु असन्तस्य च अधातोः \txcr{न उपधायाः असम्बुद्धौ सौ अङ्गस्य दीर्घः}}
\hsref{6-4-14} बलवान् (बलवत् - अत् अन्तः), विद्वान् (विद्वस् - अस् अन्तः), it applied to \{1.1\} only. Remember that बलवत् is both उगित् and अतु; पठत् is उगित् but not अतु so sutra \sref{6-4-14} does not apply to पठत् thus पठन् but बलवान्।
\sutra{7-1-70}{उगित् अचां सर्वनामस्थाने अधातोः \txcr{नुम् अङ्गस्य}}
\hsref{7-1-70} बलवान्, बलवन्तौ, बलवन्तः, बलवन्तम्, पठन्, पठन्तौ, पठन्तः, पठन्तम्।
\sutra{8-2-23}{संयोगान्तस्य लोपः \txcr{पदस्य पूर्वत्र असिद्धम्}}
\hsref{8-2-23} बलवान् त् $\rightarrow$ बलवान् 
\sutra{7-1-24}{अतः अम् \txcr{स्वमोः नपुंसकात् अङ्गस्य}}
\hsref{7-1-24} फलम्
\sutra{6-1-107}{अमि पूर्वः संहितायाम् \txcr{एकः पूर्वपरयोः अकः}}
\hsref{6-1-107} फलम् रामम् मालाम् मुनिम् नदीम् साधुम् वधूम्।
\sutra{1-3-4}{न विभक्तौ तुस्माः \txcr{धातवः इत् हल् अन्त्यम्}}\vspace*{-24pt}
\sutra{7-3-110}{ऋतः ङिसर्वनामस्थानयोः \txcr{गुणः अङ्गस्य}}
\hsref{7-3-110} पितरौ पितरः पितरम् दातारौ दातारः दातारम्।
\sutra{8-3-59}{आदेशप्रत्यययोः \txcr{पूर्वत्र असिद्धम् संहितायाम् अपदान्तस्य मूर्धन्यः सः इण्कोः नुम्विसर्जनीयशर्व्यवाये अपि}}
\hsref{8-3-59} ज्योतिषा धनुषा। 
\sutra{6-4-134}{अत् लोपः अनः \href{भस्य अङ्गस्य}}
\hsref{6-4-134} नाम्नी नाम्नी नाम्ने नाम्नः नाम्नोः नाम्नाम् नाम्न।
\sutra{6-4-137}{न संयोगात् वमन्तात् \txcr{अल् लोपः अनः}}
Sutra \sref{6-4-137} stops the dropping of अ in, जन्मनी जन्मना जन्मने जन्मनः जन्मनोः जन्मनाम् जन्मनि, because of म् before अन्।
\sutra{7-3-120}{आङो ना अस्त्रियाम् \txcr{घेः अङ्गस्य}}
\hsref{7-3-120} वारिणा मधुना रविणा साधुना।
\sutra{1-4-7}{शेषः घि असखि \txcr{आ कडारात् एका सञ्ज्ञा ह्रस्वः च}} \vspace*{-24pt}
\sutra{7-1-75}{अस्थिदधिसक्थ्यक्ष्णाम् अनङुदात्तः \txcr{अचि विभक्तौ अङ्गस्य नपुंसकस्य तृतीयादिषु}}
Sutra \sref{7-1-75} tells that the final vowel in the words अस्थि, दधि, सक्थि, and अक्षि is replaced by अनङ् $\rightarrow$ अन् from \{3.1\} onwards when the प्रत्ययः starts with a vowel. This means that these इकारान्त-words take forms like the हलन्त् words from \{3.1\} onwards; remember that the final न् disappears for \{3.2, 3.3, 4.2, 4.3, 5.2, 5.3, 7.3\} प्रत्ययाः।
\sutra{7-1-12}{टाङसिङसाम् इनात्स्याः \txcr{अतः अङ्गस्य}}
\hsref{7-1-12} रामेण फलेन रामाय फलाय रामस्य फलस्य।
\sutra{7-3-105}{आङि च आपः \txcr{एत् ओसि अङ्गस्य}}
\hsref{7-3-105} लतयाः।
\sutra{7-1-73}{इकः अचि विभक्तौ \txcr{नुम् नपुंसकस्य अङ्गस्य}}
\hsref{7-1-73} मधुनी मधुने मधुनः मधुनोः मधूनाम् मधुनि वरिणी वरिणे वारिणः वारिणोः वारीणाम् वारिणे। 
\sutra{7-3-111}{घेः ङिति \txcr{गुणः अङ्गस्य सुपि}}
\hsref{7-3-111} रवये रवेः साधवे साधोः।
\sutra{7-3-113}{याट् आपः \txcr{ङिति अङ्गस्य}}
\hsref{7-3-113}लतायै लतायाः लतायाम्। 
\sutra{7-3-112}{आण् नद्याः \txcr{ङिति अङ्गस्य}}
\hsref{7-3-112} नद्यै नद्याः नद्याम्। 
\sutra{1-4-3}{यू स्त्र्याख्यौ नदी \txcr{आ कडारात् एका सञ्ज्ञा}}
Sutra \sref{1-4-3} is a definition (सञ्ज्ञा) sutra and says that the स्त्रीलिङ्ग words that end in either ई or ऊ or यू are given नदीसंज्ञा, some examples: नदी लेखनी वधू स्वश्रू। 
\sutra{1-4-6}{ङिति ह्रस्वः च \txcr{आ कडारात् एका सञ्ज्ञा यू स्त्र्याख्यौ नदी न इयङुवङ्स्थानौ अस्त्री वा}}\vspace*{-24pt}
\sutra{6-1-90}{आटः च \txcr{संहितायाम् अचि एकः पूर्वपरयोः वृद्धिः}}
\hsref{6-1-90} नद्यै नद्याः नद्याम्। 
\sutra{7-1-13}{ङेः यः \txcr{अतः अङ्गस्य}}
\hsref{7-1-13} फलाय रामाय।
\sutra{7-3-102}{सुपि च \txcr{अतः दीर्घः यञि अङ्गस्य}}
\hsref{7-3-102} फलाय रामाय।
\sutra{6-1-110}{ङसिङसोः च \txcr{संहितायाम् एकः पूर्वपरयोः पूर्वः एङः अति }}
\hsref{6-1-110} रवेः साधोः।
\sutra{6-1-111}{ऋतः उत् \txcr{संहितायाम्  एकः पूर्वपरयोः अति ङसिङसोः}}
\hsref{6-1-111} पितुः दातुः।
\sutra{8-2-24}{रात् सस्य \txcr{पदस्य पूर्वत्र असिद्धम् संयोगान्तस्य लोपः}}
\hsref{8-2-24} पितुः दातुः।
\sutra{7-3-116}{ङेः आम् नद्याम्नीभ्यः \txcr{अङ्गस्य}}
\hsref{7-3-116} नद्याम् वध्वाम् लतायाम्।
\sutra{7-3-119}{अत् च घेः \txcr{ङेः औत् इदुद्भ्याम् अङ्गस्य}}
\hsref{7-3-119} रवौ साधौ।
\sutra{2-3-47}{सम्बोधने च प्रथमा}
Sutra \sref{2-3-47} that in the sense of सम्बोधन, प्रथमा विभक्ति is used.
\sutra{8-2-8}{न ङिसम्बुद्ध्योः \txcr{पदस्य पूर्वत्र असिद्धम् नलोपः प्रातिपदिकान्तस्य}}
\hsref{8-2-8} हे नामन्, हे जन्मन्।
\sutra{2-3-49}{एकवचनं सम्बुद्धिः \txcr{सम्बोधने च}}
Sutra \sref{2-3-49} is the definition of सम्बुद्धि and it says in सम्बोधन, एकवचन is called सम्बुद्धि।
\sutra{6-1-69}{एङ्ह्रस्वात् सम्बुद्धेः \txcr{हल् लोपः}}
\hsref{6-1-69} राम फल।
\sutra{7-3-108}{ह्रस्वस्य गुणः \txcr{सम्बुद्धौ अङ्गस्य}}
\hsref{7-3-108} रवे स्वाते साधो धेनो।
\sutra{7-3-106}{सम्बुद्धौ च \txcr{एत् च आपः अङ्गस्य}}
\hsref{7-3-106} लते।
\sutra{7-3-107}{अम्बार्थनद्योः ह्रस्वः \txcr{सम्बुद्धौ अङ्गस्य}}
\hsref{7-3-107} हे अम्ब, हे अक्क, हे नदि, हे केतकि, हे वधु।
\sutra{7-1-19}{नपुंसकात् च \txcr{शी औङः अङ्गस्य}}
\hsref{7-1-19} नाम्नी जन्मनी वारिणी मधुनी।
\sutra{7-1-18}{औङः आपः \txcr{शी अङ्गस्य}}
\hsref{7-1-18} लते।  
\sutra{6-1-102}{प्रथमयोः पूर्वसवर्णः \txcr{संहितायाम् अचि एकः पूर्वपरयोः अकः दीर्घः}}
\hsref{6-1-102} रवी साधू स्वाती धेनू मतिः नदीः।
\sutra{7-3-104}{ओसि च \txcr{अतः एत् अङ्गस्य}}
\hsref{7-3-104} रामयोः फलयोः लतयोः।
\sutra{6-1-105}{दीर्घात् जसि च \txcr{संहितायाम् अचि एकः पूर्वपरयोः अकः दीर्घः पूर्वसवर्णः न आत् इचि}}
Sutra \sref{6-1-105} stops the application of sutra \sref{6-1-102} and results in लताः। भैमीव्याख्या says that this is not significant for लताः as sutra \sref{6-1-102} would have given the same final result but there are situations where this sutra \sref{6-1-105} is needed. In the derivation of लताः it is normally included to ensure that the right process is being followed.
\sutra{7-3-109}{जसि च \txcr{ह्रस्वस्य गुणः अङ्गस्य}}
\hsref{7-3-109} कवयः मतयः साधवः धेनवः।
\sutra{7-1-20}{जश्शसोः शिः \txcr{नपुंसकात् अङ्गस्य}}
Sutra \sref{7-1-20} tells that for नपुंसकलिङ्गः, जस् शस् are replaced by शि।
\sutra{1-1-42}{शि सर्वनामस्थानम्}
Sutra \sref{1-1-42} that शि is also called सर्वनामस्थानम्।
\sutra{7-1-72}{नपुंसकस्य झल् अचः \txcr{नुम् सर्वनामस्थाने अङ्गस्य}}
It is due to sutra \sref{7-1-72} that नि is seen for all नपुंसकलिङ्गः words (that end in झल् or अच् ) in \{1.3\} and \{2.3\}; झल् includes the 1st, 2nd, 3rd, and 4th consonants of each वर्ग (कवर्ग, चवर्ग, टवर्ग, तवर्ग, and पवर्ग), and अच् includes all the vowels.
\sutra{6-4-10}{सान्त महतः संयोगस्य \txcr{न उपधायाः सर्वनामस्थाने असम्बुद्धौ अङ्गस्य दीर्घः}}
\hsref{6-4-10} धनूंषि ज्योतींषि मनांसि। 
\sutra{8-3-58}{नुम्विसर्जनीयशर्व्यवाये अपि \txcr{पूर्वत्र असिद्धम् संहितायाम् अपदान्तस्य मूर्धन्यः सः इण्कोः}}
Sutra \sref{8-3-58} is a part of the षत्वविधानम् where स् changes to ष्।
\sutra{6-1-103}{तस्मात् शसः नः पुंसि \txcr{संहितायाम् पूर्वसवर्णः}}
\hsref{6-1-103} रामान् साधून् रवीन्।
\sutra{8-2-7}{नलोपः प्रातिपदिकान्तस्य \txcr{पदस्य पूर्वत्र असिद्धम्}}
\hsref{8-2-7} नामभिः जन्मभिः नामभ्याम् जन्माभ्याम् ज्ञानिभिः ज्ञानिभ्याम् ज्ञानिभ्यः नामसु ज्ञानिषु इत्यादयः।
\sutra{7-1-9}{अतः भिस् ऐस् \txcr{अङ्गस्य}}
\hsref{7-1-9} रामैः।
\sutra{7-3-103}{बहुवचने झलि एत् \txcr{अतः सुपि अङ्गस्य}}
\hsref{7-3-103} रामेभ्यः रामेषु।
\sutra{7-1-54}{ह्रस्वनद्यापः नुट् \txcr{आमि अङ्गस्य}}
Sutra \sref{7-1-54} applies to all the words with a vowel ending and makes the effective \{6.3\} प्रत्ययः for the vowel ending words as नाम् instead आम्।
\hsref{7-1-54} नदीनाम्, वधूनाम्, लतानाम्, मुनीनाम्, रामाणाम्, साधूनाम्, मधूनाम्, पितॄणाम्, कर्तॄणाम्| After sutra \sref{7-1-54}, sutra \sref{6-4-3} is applied to the words with the ह्रस्वः endings to get the final form.
\sutra{6-4-3}{नामि \txcr{अङ्गस्य दीर्घः}}
\hsref{6-4-3} मुनीनाम्, रामाणाम्, साधूनाम्, मधूनाम्, पितॄणाम्, कर्तॄणाम्।
\sutra{8-3-16}{रोः सुपि रः \txcr{पूर्वत्र असिद्धम् संहितायाम् विसर्जनीयः}}
\hsref{8-3-16} मनःसु ज्योतिःषु।
\sutra{7-1-17}{जसः शी \txcr{अतः सर्वनाम्नः}}
\hsref{7-1-17} सर्वे।
\sutra{7-1-14}{सर्वनाम्नः स्मै \txcr{अतः ङेः}}
\hsref{7-1-14} यस्मै कस्मै अस्मै तस्मै एतस्मै अन्यस्मै एकस्मै।
\sutra{7-1-15}{ङसिङ्योः स्मात्स्मिनौ \txcr{अतः सर्वनाम्नः}}
\hsref{7-1-15} सर्वस्मात् यस्मात् एकस्मात् सर्वस्मिन् यस्मिन् एकस्मिन् इत्यादयः।
\sutra{7-1-52}{आमि सर्वनाम्नः सुट् \txcr{आत्}}
\hsref{7-1-52} सर्वेषाम् तेषाम् इत्यादयः।
\sutra{7-3-114}{सर्वनाम्नः स्याट् ह्रस्वः च \txcr{ङिति आपः}}
\hsref{7-3-114} सर्वस्यै यस्यै कस्यै अस्यै तस्यै एतस्यै अन्यस्यै एकस्यै।
\sutra{7-2-102}{त्यदादीनाम् अः \txcr{विभक्तौ}}
Sutra \sref{7-2-102} changes the last letter of त्यद्, तद्, यद्, एतद, इदम्, अदस्, and द्वि to अ।
\sutra{7-2-106}{तदोः सः सौ अनन्त्ययोः \txcr{विभक्तौ}}
\hsref{7-2-106} सः एषः।
\sutra{1-1-63}{न लुमता अङ्गस्य \txcr{प्रत्ययलोपे प्रत्ययलक्षणम्}} 
\hsref{1-1-63} तद्।
\sutra{7-2-103}{किमः कः \txcr{विभक्तौ}} 
Sutra \sref{7-2-103} says that किम् becomes क for the purposes of obtaining the विभक्ति-tables.
\sutra{7-1-53}{त्रेः त्रयः \txcr{आमि}}
\hsref{7-1-53} त्रयाणाम्।
\sutra{7-2-99}{त्रिचतुरोः स्त्रियां तिसृचतसृ \txcr{विभक्तौ}} 
\hsref{7-2-99} तिस्रः तिसृभिः तिसृभ्यः तिसृणाम् तिसृषु चतस्रः चतसृभिः चतसृभ्यः चतसृणाम् चतसृषु।
\sutra{7-2-100}{अचि रः ऋतः \txcr{तिसृचतसृ विभक्तौ}}
\hsref{7-2-100} तिस्रः तिसृभिः तिसृभ्यः तिसृणाम् तिसृषु चतस्रः चतसृभिः चतसृभ्यः चतसृणाम् चतसृषु।
\sutra{6-4-4}{न तिसृचतसृ \txcr{नामि}} 
\hsref{6-4-4} तिसृणाम्, चतसृणाम्।
\sutra{7-1-98}{चतुरनडुहोः आम् उदात्तः \txcr{सर्वनामस्थाने}}
 \hsref{7-1-98} चत्वारः।
\sutra{7-1-55}{षट्चतुर्भ्यः च \txcr{आमि नुट्}}
\hsref{7-1-55} चतुर्णाम् पञ्चानाम् षण्णाम् सप्तानाम् अष्टानाम् नवानाम् दशानाम्।
\sutra{7-1-22}{षड्भ्यः लुक् \txcr{जश्शसोः}}
\hsref{7-1-22} पञ्च षड् सप्त अष्ट नव दश।
\sutra{6-4-7}{न उपधायाः \txcr{नामि दीर्घः}}
\hsref{6-4-7} पञ्चानाम् सप्तानाम् अष्टानाम् नवानाम् दशानाम्।
\sutra{6-4-79}{स्त्रियाः \txcr{अचि इयङ् अङ्गस्य}}
\hsref{6-4-79} स्त्रियौ
\sutra{6-4-82}{एरनेकाचोऽसंयोगपूर्वस्य \txcr{अचि यण् अङ्गस्य धातोः}}
\hsref{6-4-82} प्रध्यौ। Please note that ``गतिकारकाभ्यामन्यपूर्वस्य नेष्यते'' इति वार्तिकं काशिकायामेव वर्तते, परमनियौ, परमनियः इति।
%\the\abovedisplayskip \the\abovedisplayskip
\endgroup


%\sref{1-1-3} is the third sutra in Ashtadhyayi.
%\hyperref[1-1-3]{third \sref{1-1-3}}
%\par
%\par
%सौ च ६-४-१३ इन्हन्पूषार्यम्णाम् असम्बुद्धौ उपधाया अङ्गस्य दीर्घः 
%\par
%उगिदचां सर्वनामस्थानेऽधातोः 7-1-70
%अधातोरुगितो नलोपिनोऽञ्चतेश्च नुम् स्यात्सर्वनामस्थाने
%\par
%सर्वनामस्थाने चाऽसम्बुद्धौ 6-4-8 
%वृत्ति: नान्तस्योपधाया दीर्घोऽसम्बुद्धौ सर्वनामस्थाने ।
%\par
%हल्ङ्याब्भ्यो दीर्घात् सुँतिस्यपृक्तं हल् 6-1-68 
%वृत्ति: हलन्तात् परम्, दीर्घौ यौ ङ्यापौ तदन्ताच्च परम्, ’सुँ-ति-सि’ इत्येतद् अपृक्तं हल् लुप्यते । A single letter affix “सुँ”, “ति” or “सि” is dropped following a base ending in a consonant or in the long feminine affix “ङी” or “आप्”। नदी, लता, अभिजित्
%\par
%सान्तमहतः संयोगस्य 6-4-10
%सान्तसंयोगस्य महतश्च यो नकारस्तस्योपधाया दीर्घोऽसम्बुद्धौ सौ परे
%\par
%अत्वसन्तस्य चाधातोः 6-4-14
%अत्वन्तस्योपधाया दीर्घो धातुभिन्नासन्तस्य चासम्बुद्धौ सौ
%\par
%जक्षित्यादयः षट् 6-1-6 
%षड्धातवोऽन्ये जक्षितिश्च सप्तम एते अभ्स्तसंज्ञा स्युः
%\par
%नाभ्यस्ताच्छतुः 7-1-78
%अभ्यस्तात्परस्य शतुः नुम् न स्यात्
%
%The story appears like this for नकारान्त like आत्मन् दीर्घः happens for all सर्वनामस्थानं but if the नकारान्त is from इन्हन्पूषार्यम्णाम् then it happens only for सौ.
%अतन्त और असन्त में प्रथमा एकवचन में दीर्घ होगा. 

\clearpage
\section{188 More सुबन्तरूपाणि}

%\the\abovedisplayskip \the\abovedisplayskip
\begin{multicols}{2}
\TrickSupertabularIntoMulticols
\small
\begin{center} 
 \begin{supertabular}{|c|c|c|c|}\hline 
 \multicolumn {4}{|c|}{\cellcolor{blue!10}  \nextTable \hypertarget{भवादृश (पु॰) Like You}{भवादृश (पु॰) Like You}}  \\ \hline  
 & एक॰ & द्वि॰ & बहु॰  \\ \hline 
 प्रथमा & \csarva भवादृशः & \csarva भवादृशौ & \csarva भवादृशाः \\ \hline 
 द्वितीया & \csarva भवादृशम् & \csarva भवादृशौ & \cbham भवादृशान् \\ \hline 
 तृतीया & \cbham भवादृशेन & \cpada भवादृशाभ्याम् & \cpada भवादृशैः \\ \hline 
 चतुर्थी & \cbham भवादृशाय & \cpada भवादृशाभ्याम् & \cpada भवादृशेभ्यः \\ \hline 
 पञ्चमी & \cbham भवादृशात् & \cpada भवादृशाभ्याम् & \cpada भवादृशेभ्यः \\ \hline 
 षष्ठी & \cbham भवादृशस्य & \cbham भवादृशयोः & \cbham भवादृशानाम् \\ \hline 
सप्तमी & \cbham भवादृशे & \cbham भवादृशयोः & \cpada भवादृशेषु \\ \hline 
सम्बोधन & हे भवादृश & हे भवादृशौ & हे भवादृशाः \\ \hline 
\end{supertabular} 
\end{center} 
%
 \begin{center} 
 \begin{supertabular}{|c|c|c|c|}\hline 
 \multicolumn {4}{|c|}{\cellcolor{blue!10}  \nextTable \hypertarget{विश्वपा (पु॰) Protector of the Universe}{विश्वपा (पु॰) Protector of the Universe}}  \\ \hline  
 & एक॰ & द्वि॰ & बहु॰  \\ \hline 
 प्रथमा & \csarva विश्वपाः & \csarva विश्वपा & \csarva विश्वपाः \\ \hline 
 द्वितीया & \csarva विश्वपाम् & \csarva विश्वपौ & \cbham विश्वपः \\ \hline 
 तृतीया & \cbham विश्वपा & \cpada विश्वपाभ्याम् & \cpada विश्वपाभिः \\ \hline 
 चतुर्थी & \cbham विश्वपे & \cpada विश्वपाभ्याम् & \cpada विश्वपाभ्यः \\ \hline 
 पञ्चमी & \cbham विश्वपः & \cpada विश्वपाभ्याम् & \cpada विश्वपाभ्यः \\ \hline 
 षष्ठी & \cbham विश्वपः & \cbham विश्वपोः & \cbham विश्वपाम् \\ \hline 
सप्तमी & \cbham विश्वपि & \cbham विश्वपोः & \cpada विश्वपासु \\ \hline 
सम्बोधन & हे विश्वपाः & हे विश्वपौ & हे विश्वपाः \\ \hline 
\end{supertabular} 
\end{center} 
%
 \begin{center} 
 \begin{supertabular}{|c|c|c|c|}\hline 
 \multicolumn {4}{|c|}{\cellcolor{blue!10}  \nextTable \hypertarget{पति (पु॰) Husband}{पति (पु॰) Husband}}  \\ \hline  
 & एक॰ & द्वि॰ & बहु॰  \\ \hline 
 प्रथमा & \csarva पतिः & \csarva पती & \csarva पतयः \\ \hline 
 द्वितीया & \csarva पतिम् & \csarva पती & \cbham पतीन् \\ \hline 
 तृतीया & \cbham पत्या & \cpada पतिभ्याम् & \cpada पतिभिः \\ \hline 
 चतुर्थी & \cbham पत्ये & \cpada पतिभ्याम् & \cpada पतिभ्यः \\ \hline 
 पञ्चमी & \cbham पत्युः & \cpada पतिभ्याम् & \cpada पतिभ्यः \\ \hline 
 षष्ठी & \cbham पत्युः & \cbham पत्योः & \cbham पतीनाम् \\ \hline 
सप्तमी & \cbham पत्यौ & \cbham पत्योः & \cpada पतिषु \\ \hline 
सम्बोधन & हे पते & हे पती & हे पतयः \\ \hline 
\end{supertabular} 
\end{center} 
%
  \begin{center} 
 \begin{supertabular}{|c|c|c|c|}\hline 
 \multicolumn {4}{|c|}{\cellcolor{blue!10}  \nextTable \hypertarget{सखि (पु॰) Friend}{सखि (पु॰) Friend}}  \\ \hline  
 & एक॰ & द्वि॰ & बहु॰  \\ \hline 
 प्रथमा & \csarva सखा & \csarva सखायौ & \csarva सखायः \\ \hline 
 द्वितीया & \csarva सखायम् & \csarva सखायौ & \cbham सखीन् \\ \hline 
 तृतीया & \cbham सख्या & \cpada सखिभ्याम् & \cpada सखिभिः \\ \hline 
 चतुर्थी & \cbham सख्ये & \cpada सखिभ्याम् & \cpada सखिभ्यः \\ \hline 
 पञ्चमी & \cbham सख्युः & \cpada सखिभ्याम् & \cpada सखिभ्यः \\ \hline 
 षष्ठी & \cbham सख्युः & \cbham सख्योः & \cbham सखीनाम् \\ \hline 
सप्तमी & \cbham सख्यौ & \cbham सख्योः & \cpada सखिषु \\ \hline 
सम्बोधन & हे सखे & हे सखायौ & हे सखायः \\ \hline 
\end{supertabular} 
\end{center}

\medskip

\begin{center} 
 \begin{supertabular}{|c|c|c|c|}\hline 
 \multicolumn {4}{|c|}{\cellcolor{blue!10}  \nextTable \hypertarget{अक्षि (नपु॰) Eye}{अक्षि (नपु॰) Eye}}  \\ \hline  
 & एक॰ & द्वि॰ & बहु॰  \\ \hline 
 प्रथमा & \cpada अक्षि & \cbham अक्षिणी & \csarva अक्षीणि \\ \hline 
 द्वितीया & \cbham अक्षि & \cbham अक्षिणी & \csarva अक्षीणि \\ \hline 
 तृतीया & \cbham अक्ष्णा & \cpada अक्षिभ्याम् & \cpada अक्षिभिः \\ \hline 
 चतुर्थी & \cbham अक्ष्णे & \cpada अक्षिभ्याम् & \cpada अक्षिभ्यः \\ \hline 
 पञ्चमी & \cbham अक्ष्णः & \cpada अक्षिभ्याम् & \cpada अक्षिभ्यः \\ \hline 
 षष्ठी & \cbham अक्ष्णः & \cbham अक्ष्णोः & \cbham अक्ष्णाम् \\ \hline 
सप्तमी & \cbham अक्षिण-अक्षणि & \cbham अक्ष्णोः & \cpada अक्षिषु \\ \hline 
सम्बोधन & हे अक्षे-अक्षि & हे अक्षिणी & हे अक्षीणि \\ \hline 
\end{supertabular} 
\end{center}

\begin{center} 
 \begin{supertabular}{|c|c|c|c|}\hline 
 \multicolumn {4}{|c|}{\cellcolor{blue!10}  \nextTable \hypertarget{शुचि (नपु॰) Clean}{शुचि (नपु॰) Clean}}  \\ \hline  
 & एक॰ & द्वि॰ & बहु॰  \\ \hline 
 प्रथमा & \cpada शुचि & \cbham शुचिनी & \csarva शुचीनि \\ \hline 
 द्वितीया & \cbham शुचि & \cbham शुचिनी & \csarva शुचीनि \\ \hline 
 तृतीया & \cbham शुचिना & \cpada शुचिभ्याम् & \cpada शुचिभिः \\ \hline 
 चतुर्थी & \cbham शुचये-शुचिने & \cpada शुचिभ्याम् & \cpada शुचिभ्यः \\ \hline 
 पञ्चमी & \cbham शुचेः-शुचिनः & \cpada शुचिभ्याम् & \cpada शुचिभ्यः \\ \hline 
 षष्ठी & \cbham शुचेः-शुचिनः & \cbham शुच्योः-शुचिनोः & \cbham शुचीनाम् \\ \hline 
सप्तमी & \cbham शुचौ-शुचिनि & \cbham शुच्योः-शुचिनोः & \cpada शुचिषु \\ \hline 
सम्बोधन & हे शुचि-शुचे & हे शुचिनी & हे शुचीनि \\ \hline 
\end{supertabular} 
\end{center} 
 
 \begin{center} 
 \begin{supertabular}{|c|c|c|c|}\hline 
 \multicolumn {4}{|c|}{\cellcolor{blue!10}  \nextTable \hypertarget{प्रधी (पु॰) Great Intelligence}{प्रधी (पु॰) Great Intelligence}}  \\ \hline  
 & एक॰ & द्वि॰ & बहु॰  \\ \hline 
 प्रथमा & \csarva प्रधीः & \csarva प्रध्यौ & \csarva प्रध्यः \\ \hline 
 द्वितीया & \csarva प्रध्यम् & \csarva प्रध्यौ & \cbham प्रध्यः \\ \hline 
 तृतीया & \cbham प्रध्या & \cpada प्रधीभ्याम् & \cpada प्रदीभिः \\ \hline 
 चतुर्थी & \cbham प्रध्ये & \cpada प्रधीभ्याम् & \cpada प्रधीभ्यः \\ \hline 
 पञ्चमी & \cbham प्रध्यः & \cpada प्रधीभ्याम् & \cpada प्रधिभ्यः \\ \hline 
 षष्ठी & \cbham प्रध्यः & \cbham प्रध्योः & \cbham प्रध्याम् \\ \hline 
सप्तमी & \cbham प्रध्यि & \cbham प्रध्योः & \cpada प्रधीषु \\ \hline 
सम्बोधन & हे प्रधीः & हे प्रध्यौ & हे प्रध्यः \\ \hline 
\end{supertabular} 
\end{center} 
%
 \begin{center} 
 \begin{supertabular}{|c|c|c|c|}\hline 
 \multicolumn {4}{|c|}{\cellcolor{blue!10}  \nextTable \hypertarget{सुधी (पु॰) Intelligence}{सुधी (पु॰) Intelligence}}  \\ \hline  
 & एक॰ & द्वि॰ & बहु॰  \\ \hline 
 प्रथमा & \csarva सुधीः & \csarva सुधियौ & \csarva सुधियः \\ \hline 
 द्वितीया & \csarva सुधियम् & \csarva सुधियौ & \cbham सुधियः \\ \hline 
 तृतीया & \cbham सुधिया & \cpada सुधीभ्याम् & \cpada सुधीभिः \\ \hline 
 चतुर्थी & \cbham सुधिये & \cpada सुधीभ्याम् & \cpada सुधीभ्यः \\ \hline 
 पञ्चमी & \cbham सुधियः & \cpada सुधीभ्याम् & \cpada सुधीभ्यः \\ \hline 
 षष्ठी & \cbham सुधियः & \cbham सुधियोः & \cbham सुधियाम् \\ \hline 
सप्तमी & \cbham सुधियि & \cbham सुधियोः & \cpada सुधीषु \\ \hline 
सम्बोधन & हे सुधीः & हे सुधियौ & हे सुधियः \\ \hline 
\end{supertabular} 
\end{center} 
%
\clearpage

\normalsize
\begin{center} 
 \begin{supertabular}{|c|c|c|c|}\hline 
 \multicolumn {4}{|c|}{\cellcolor{blue!10}  \nextTable \hypertarget{श्री (स्त्री॰) Prosperity}{श्री (स्त्री॰) Prosperity}}  \\ \hline  
 & एक॰ & द्वि॰ & बहु॰  \\ \hline 
 प्रथमा & \csarva श्रीः & \csarva श्रियौ & \csarva श्रियः \\ \hline 
 द्वितीया & \csarva श्रियम् & \csarva श्रियौ & \cbham श्रियः \\ \hline 
 तृतीया & \cbham श्रिया & \cpada श्रीभ्याम् & \cpada श्रीभिः \\ \hline 
 चतुर्थी & \cbham श्रियै-श्रिये & \cpada श्रीभ्याम् & \cpada श्रीभ्यः \\ \hline 
 पञ्चमी & \cbham श्रियाः-श्रियः & \cpada श्रीभ्याम् & \cpada श्रीभ्यः \\ \hline 
 षष्ठी & \cbham श्रीयाः-श्रियः & \cbham श्रियोः & \cbham श्रीणाम्-श्रियाम् \\ \hline 
सप्तमी & \cbham श्रियाम्-श्रियि & \cbham श्रियोः & \cpada श्रीषु \\ \hline 
सम्बोधन & हे श्रीः & हे श्रियौ & हे श्रियः \\ \hline 
\end{supertabular} 
\end{center} 

 \begin{center} 
 \begin{supertabular}{|c|c|c|c|}\hline 
 \multicolumn {4}{|c|}{\cellcolor{blue!10}  \nextTable \hypertarget{स्त्री (स्त्री॰) Woman}{स्त्री (स्त्री॰) Woman}}  \\ \hline  
 & एक॰ & द्वि॰ & बहु॰  \\ \hline 
 प्रथमा & \csarva स्त्री & \csarva स्त्रियौ & \csarva स्त्रियः \\ \hline 
 द्वितीया & \csarva स्त्रियम्-स्त्रीम् & \csarva स्त्रियौ & \cbham स्त्रियः-स्त्रीः \\ \hline 
 तृतीया & \cbham स्त्रिया & \cpada स्त्रीभ्याम् & \cpada स्त्रीभिः \\ \hline 
 चतुर्थी & \cbham स्त्रियै & \cpada स्त्रीभ्याम् & \cpada स्त्रीभ्यः \\ \hline 
 पञ्चमी & \cbham स्त्रियाः & \cpada स्त्रीभ्याम् & \cpada स्त्रीभ्यः \\ \hline 
 षष्ठी & \cbham स्त्रियाः & \cbham स्त्रियोः & \cbham स्त्रीणाम् \\ \hline 
सप्तमी & \cbham स्त्रियाम् & \cbham स्त्रियोः & \cpada स्त्रीषु \\ \hline 
सम्बोधन & हे स्त्रि & हे स्त्रियौ & हे स्त्रियः \\ \hline 
\end{supertabular} 
\end{center} 

\begin{center} 
 \begin{supertabular}{|c|c|c|c|}\hline 
 \multicolumn {4}{|c|}{\cellcolor{blue!10}  \nextTable \hypertarget{सखी (पु॰) Friend}{सखी (पु॰) Friend}}  \\ \hline  
 & एक॰ & द्वि॰ & बहु॰  \\ \hline 
 प्रथमा & \csarva सखा & \csarva सखायौ & \csarva सखायः \\ \hline 
 द्वितीया & \csarva सखायम् & \csarva सखायौ & \cbham सख्यः \\ \hline 
 तृतीया & \cbham सख्या & \cpada सखीभ्याम् & \cpada सखीभिः \\ \hline 
 चतुर्थी & \cbham सख्ये & \cpada सखीभ्याम् & \cpada सखीभ्यः \\ \hline 
 पञ्चमी & \cbham सख्युः & \cpada सखीभ्याम् & \cpada सखीभ्यः \\ \hline 
 षष्ठी & \cbham सख्युः & \cbham सख्योः & \cbham सख्याम् \\ \hline 
सप्तमी & \cbham सख्यि & \cbham सख्योः & \cpada सखीषु \\ \hline 
सम्बोधन & हे सखा & हे सखायौ & हे सखायः \\ \hline 
\end{supertabular} 
\end{center} 
%
 
\begin{center} 
 \begin{supertabular}{|c|c|c|c|}\hline 
 \multicolumn {4}{|c|}{\cellcolor{blue!10}  \nextTable \hypertarget{बहु (नपु॰) Many}{बहु (नपु॰) Many}}  \\ \hline  
 & एक॰ & द्वि॰ & बहु॰  \\ \hline 
 प्रथमा & \cpada बहु & \cbham बहुनी & \csarva बहूनि \\ \hline 
 द्वितीया & \cbham बहु & \cbham बहुनी & \csarva बहूनि \\ \hline 
 तृतीया & \cbham बहुना & \cpada बहुभ्याम् & \cpada बहुभिः \\ \hline 
 चतुर्थी & \cbham बहुने-बहवे & \cpada बहुभ्याम् & \cpada बहुभ्यः \\ \hline 
 पञ्चमी & \cbham बहोः-बहुनः & \cpada बहुभ्याम् & \cpada बहुभ्यः \\ \hline 
 षष्ठी & \cbham बहोः-बहुनः & \cbham बह्वोः-बहुनोः & \cbham बहूनाम् \\ \hline 
सप्तमी & \cbham बहौ-बहुनि & \cbham बह्वोः-बहुनोः & \cpada बहुषु \\ \hline 
सम्बोधन & हे बहु-बहो & हे बहुनी & हे बहूनि \\ \hline 
\end{supertabular} 
\end{center} 

\begin{center} 
 \begin{supertabular}{|c|c|c|c|}\hline 
 \multicolumn {4}{|c|}{\cellcolor{blue!10}  \nextTable \hypertarget{स्वयम्भू (पु॰) Brahma}{स्वयम्भू (पु॰) Brahma}}  \\ \hline  
 & एक॰ & द्वि॰ & बहु॰  \\ \hline 
 प्रथमा & \csarva स्वयम्भूः & \csarva स्वयम्भुवौ & \csarva स्वयम्भुवः \\ \hline 
 द्वितीया & \csarva स्वयम्भुवम् & \csarva स्वयम्भुवौ & \cbham स्वयम्भुवः \\ \hline 
 तृतीया & \cbham स्वयम्भुवा & \cpada स्वयम्भूभ्याम् & \cpada स्वयम्भूभिः \\ \hline 
 चतुर्थी & \cbham स्वयम्भुवे & \cpada स्वयम्भूभ्याम् & \cpada स्वयम्भूभ्यः \\ \hline 
 पञ्चमी & \cbham स्वयम्भुवः & \cpada स्वयम्भूभ्याम् & \cpada स्वयम्भूभ्यः \\ \hline 
 षष्ठी & \cbham स्वयम्भुवः & \cbham स्वयम्भुवोः & \cbham स्वयम्भुवाम् \\ \hline 
सप्तमी & \cbham स्वयम्भुवि & \cbham स्वयम्भुवोः & \cpada स्वयम्भूषु \\ \hline 
सम्बोधन & हे स्वयम्भूः & हे स्वयम्भुवौ & हे स्वयम्भुवः \\ \hline 
\end{supertabular} 
\end{center}
%
 
 \begin{center} 
 \begin{supertabular}{|c|c|c|c|}\hline 
 \multicolumn {4}{|c|}{\cellcolor{blue!10}  \nextTable \hypertarget{भू (स्त्री॰) Earth}{भू (स्त्री॰) Earth}}  \\ \hline  
 & एक॰ & द्वि॰ & बहु॰  \\ \hline 
 प्रथमा & \csarva भूः & \csarva भुवौ & \csarva भुवः \\ \hline 
 द्वितीया & \csarva भुवम् & \csarva भुवौ & \cbham भुवः \\ \hline 
 तृतीया & \cbham भुवा & \cpada भूभ्याम् & \cpada भूभिः \\ \hline 
 चतुर्थी & \cbham भुवै-भुवे & \cpada भूभ्याम् & \cpada भूभ्यः \\ \hline 
 पञ्चमी & \cbham भुवाः-भुवः & \cpada भूभ्याम् & \cpada भूभ्यः \\ \hline 
 षष्ठी & \cbham भुवाः-भुवः & \cbham भुवोः & \cbham भूनाम्-भुवाम् \\ \hline 
सप्तमी & \cbham भुवाम्-भुवि & \cbham भुवोः & \cpada भूषु \\ \hline 
सम्बोधन & हे भूः & हे भुवौ & हे भुवः \\ \hline 
\end{supertabular} 
\end{center} 

 \begin{center} 
 \begin{supertabular}{|c|c|c|c|}\hline 
 \multicolumn {4}{|c|}{\cellcolor{blue!10}  \nextTable \hypertarget{नृ (पु॰) Man}{नृ (पु॰) Man}}  \\ \hline  
 & एक॰ & द्वि॰ & बहु॰  \\ \hline 
 प्रथमा & \csarva ना & \csarva नरौ & \csarva नरः \\ \hline 
 द्वितीया & \csarva नरम् & \csarva नरौ & \cbham नॄन् \\ \hline 
 तृतीया & \cbham न्रा & \cpada नृभ्याम् & \cpada नृभिः \\ \hline 
 चतुर्थी & \cbham न्रे & \cpada नृभ्याम् & \cpada नृभ्यः \\ \hline 
 पञ्चमी & \cbham नुः & \cpada नृभ्याम् & \cpada नृभ्यः \\ \hline 
 षष्ठी & \cbham नुः & \cbham न्रोः & \cbham नृणाम्-नॄणाम् \\ \hline 
सप्तमी & \cbham नरि & \cbham न्रोः & \cpada नृषु \\ \hline 
सम्बोधन & हे नः & हे नरौ &  हे नरः \\ \hline 
\end{supertabular} 
\end{center} 
 
 \begin{center} 
 \begin{supertabular}{|c|c|c|c|}\hline 
 \multicolumn {4}{|c|}{\cellcolor{blue!10}  \nextTable \hypertarget{स्वसृ (स्त्री॰) Sister}{स्वसृ (स्त्री॰) Sister}}  \\ \hline  
 & एक॰ & द्वि॰ & बहु॰  \\ \hline 
 प्रथमा & \csarva स्वसा & \csarva स्वसारौ & \csarva स्वसारः \\ \hline 
 द्वितीया & \csarva स्वसारम् & \csarva स्वसारौ & \cbham स्वसॄः \\ \hline 
 तृतीया & \cbham स्वस्रा & \cpada स्वसृभ्याम् & \cpada स्वसृभिः \\ \hline 
 चतुर्थी & \cbham स्वस्रे & \cpada स्वसृभ्याम् & \cpada स्वसृभ्यः \\ \hline 
 पञ्चमी & \cbham स्वसुः & \cpada स्वसृभ्याम् & \cpada स्वसृभ्यः \\ \hline 
 षष्ठी & \cbham स्वसुः & \cbham स्वस्रोः & \cbham स्वसॄणाम् \\ \hline 
सप्तमी & \cbham स्वसरि & \cbham स्वस्रोः & \cpada स्वसृषु \\ \hline 
सम्बोधन & हे स्वसः & हे स्वसारौ & हे स्वसारः \\ \hline 
\end{supertabular} 
\end{center} 

 \begin{center} 
 \begin{supertabular}{|c|c|c|c|}\hline 
 \multicolumn {4}{|c|}{\cellcolor{blue!10}  \nextTable \hypertarget{मातृ (स्त्री॰) Mother}{मातृ (स्त्री॰) Mother}}  \\ \hline  
 & एक॰ & द्वि॰ & बहु॰  \\ \hline 
 प्रथमा & \csarva माता & \csarva मातरौ & \csarva मातरः \\ \hline 
 द्वितीया & \csarva मातरम् & \csarva मातरौ & \cbham मातॄः \\ \hline 
 तृतीया & \cbham मात्रा & \cpada मातृभ्याम् & \cpada मातृभिः \\ \hline 
 चतुर्थी & \cbham मात्रे & \cpada मातृभ्याम् & \cpada मातृभ्यः \\ \hline 
 पञ्चमी & \cbham मातुः & \cpada मातृभ्याम् & \cpada मातृभ्यः \\ \hline 
 षष्ठी & \cbham मातुः & \cbham मात्रोः & \cbham मातॄणाम् \\ \hline 
सप्तमी & \cbham मातरि & \cbham मात्रोः & \cpada मातृषु \\ \hline 
सम्बोधन & हे मातः & हे मातरौ & हे मातरः \\ \hline 
\end{supertabular} 
\end{center} 

 \begin{center} 
 \begin{supertabular}{|c|c|c|c|}\hline 
 \multicolumn {4}{|c|}{\cellcolor{blue!10}  \nextTable \hypertarget{दातृ (नपु॰) Giver}{दातृ (नपु॰) Giver}}  \\ \hline  
 & एक॰ & द्वि॰ & बहु॰  \\ \hline 
 प्रथमा & \cpada दातृ & \cbham दातृणी & \csarva दातॄणी \\ \hline 
 द्वितीया & \cbham दातृ & \cbham दातृणी & \csarva दातॄणी \\ \hline 
 तृतीया & \cbham दातृणा-दात्रा & \cpada दातृभ्याम् & \cpada दातृभिः \\ \hline 
 चतुर्थी & \cbham दातृणे-दात्रे & \cpada दातृभ्याम् & \cpada दातृभ्यः \\ \hline 
 पञ्चमी & \cbham दातृणः-दातुः & \cpada दातृभ्याम् & \cpada दातृभ्यः \\ \hline 
 षष्ठी & \cbham दातृणः-दातुः & \cbham दातृणोः-दात्रोः & \cbham दातॄणाम् \\ \hline 
सप्तमी & \cbham दातृणि-दातरि & \cbham दातृणोः-दात्रोः & \cpada दातृषु \\ \hline 
सम्बोधन & हे दातः-दातृ & हे दातृणी & हे दातॄणी \\ \hline 
\end{supertabular} 
\end{center} 

\begin{center} 
 \begin{supertabular}{|c|c|c|c|}\hline 
 \multicolumn {4}{|c|}{\cellcolor{blue!10}  \nextTable \hypertarget{रै (पु॰) Wealth}{रै (पु॰) Wealth}}  \\ \hline  
 & एक॰ & द्वि॰ & बहु॰  \\ \hline 
 प्रथमा & \csarva राः & \csarva रायौ & \csarva रायः \\ \hline 
 द्वितीया & \csarva रायम् & \csarva रायौ & \cbham रायः \\ \hline 
 तृतीया & \cbham राया & \cpada राभ्याम् & \cpada राभिः \\ \hline 
 चतुर्थी & \cbham राये & \cpada राभ्याम् & \cpada राभ्यः \\ \hline 
 पञ्चमी & \cbham रायः & \cpada राभ्याम् & \cpada राभ्यः \\ \hline 
 षष्ठी & \cbham रायः & \cbham रायोः & \cbham रायाम् \\ \hline 
सप्तमी & \cbham रायि & \cbham रायोः & \cpada रासु \\ \hline 
सम्बोधन & हे राः & हे रायौ & हे रायः \\ \hline 
\end{supertabular} 
\end{center} 

\begin{center} 
 \begin{supertabular}{|c|c|c|c|}\hline 
 \multicolumn {4}{|c|}{\cellcolor{blue!10}  \nextTable \hypertarget{गो (पु॰) Cow}{गो (पु॰) Cow}}  \\ \hline  
 & एक॰ & द्वि॰ & बहु॰  \\ \hline 
 प्रथमा & \csarva गौः & \csarva गावौ & \csarva गावः \\ \hline 
 द्वितीया & \csarva गाम् & \csarva गावौ & \cbham गाः \\ \hline 
 तृतीया & \cbham गवा & \cpada गोभ्याम् & \cpada गोभिः \\ \hline 
 चतुर्थी & \cbham गवे & \cpada गोभ्याम् & \cpada गोभ्यः \\ \hline 
 पञ्चमी & \cbham गोः & \cpada गोभ्याम् & \cpada गोभ्यः \\ \hline 
 षष्ठी & \cbham गोः & \cbham गवोः & \cbham गवाम् \\ \hline 
सप्तमी & \cbham गवि & \cbham गवोः & \cpada गोषु \\ \hline 
सम्बोधन & हे गौः & हे गावौ & हे गावः \\ \hline 
\end{supertabular} 
\end{center} 

 \begin{center} 
 \begin{supertabular}{|c|c|c|c|}\hline 
 \multicolumn {4}{|c|}{\cellcolor{blue!10}  \nextTable \hypertarget{ग्लौ (पु॰) Moon}{ग्लौ (पु॰) Moon}}  \\ \hline  
 & एक॰ & द्वि॰ & बहु॰  \\ \hline 
 प्रथमा & \csarva ग्लौः & \csarva ग्लावौ & \csarva ग्लावः \\ \hline 
 द्वितीया & \csarva ग्लावम् & \csarva ग्लावौ & \cbham ग्लावः \\ \hline 
 तृतीया & \cbham ग्लावा & \cpada ग्लौभ्याम् & \cpada ग्लौभिः \\ \hline 
 चतुर्थी & \cbham ग्लावे & \cpada ग्लौभ्याम् & \cpada ग्लौभ्यः \\ \hline 
 पञ्चमी & \cbham ग्लावः & \cpada ग्लौभ्याम् & \cpada ग्लौभ्यः \\ \hline 
 षष्ठी & \cbham ग्लावः & \cbham ग्लावोः & \cbham ग्लावाम् \\ \hline 
सप्तमी & \cbham ग्लावि & \cbham ग्लावोः & \cpada ग्लौषु \\ \hline 
सम्बोधन & हे ग्लौः & हे ग्लावौ & हे ग्लावः \\ \hline 
\end{supertabular} 
\end{center} 

 \begin{center} 
 \begin{supertabular}{|c|c|c|c|}\hline 
 \multicolumn {4}{|c|}{\cellcolor{blue!10}  \nextTable \hypertarget{नौ (स्त्री॰) Boat}{नौ (स्त्री॰) Boat}}  \\ \hline  
 & एक॰ & द्वि॰ & बहु॰  \\ \hline 
 प्रथमा & \csarva नौः & \csarva नावौ & \csarva नावः \\ \hline 
 द्वितीया & \csarva नावम् & \csarva नावौ & \cbham नावः \\ \hline 
 तृतीया & \cbham नावे & \cpada नौभ्याम् & \cpada नौभिः \\ \hline 
 चतुर्थी & \cbham नावे & \cpada नौभ्याम् & \cpada नौभ्यः \\ \hline 
 पञ्चमी & \cbham नावः & \cpada नौभ्याम् & \cpada नौभ्यः \\ \hline 
 षष्ठी & \cbham नावः & \cbham नावोः & \cbham नावाम् \\ \hline 
सप्तमी & \cbham नावि & \cbham नावोः & \cpada नौषु \\ \hline 
सम्बोधन & हे नौः & हे नवौ &  हे नावः \\ \hline 
\end{supertabular} 
\end{center} 
 
\begin{center}
%\begin{table}
	%\centering
	%चकारान्तः पुंलिङ्गः प्राञ्च् शब्दः\\[1.5ex]
		\begin{supertabular}{|c|c|c|c|}\hline
\multicolumn {4}{|c|}{\cellcolor{blue!10}  \nextTable \hypertarget{प्राञ्च् (पु॰) East}{प्राञ्च् (पु॰) East}}  \\ \hline 
		& एक॰ & द्वि॰ & बहु॰ \\ \hline
प्रथमा & \csarva प्राङ् & \csarva प्राञ्चौ & \csarva प्राञ्चः \\ \hline
द्वितीया & \csarva \csarva प्राञ्चम् & प्राञ्चौ & \cbham प्राचः \\ \hline
तृतीया & \cbham प्राचा & \cpada प्राग्भ्याम् & \cpada प्राग्भिः \\ \hline
चतुर्थी & \cbham प्राचे & \cpada प्राग्भ्याम् & \cpada प्राग्भ्यः \\ \hline
पञ्चमी & \cbham प्राचः & प\cpada ्राग्भ्याम् & \cpada प्राग्भ्यः \\ \hline
षष्ठी & \cbham प्राचः & \cbham प्राचोः & \cbham प्राचाम् \\ \hline
सप्तमी & \cbham प्राचि & \cbham प्राचोः & \cpada प्राक्षु \\ \hline
सम्बोधन & हे प्राङ् & हे प्राञ्चौ & हे प्राञ्चः \\ \hline			
		\end{supertabular}
		\end{center}
		%\caption{चकारान्तः पुंलिङ्गः प्राञ्च् शब्दः}
%\end{table}
%
%\par

 \begin{center} 
 \begin{supertabular}{|c|c|c|c|}\hline 
 \multicolumn {4}{|c|}{\cellcolor{blue!10}  \nextTable \hypertarget{जलमुच् (पु॰) Cloud}{जलमुच् (पु॰) Cloud}}  \\ \hline  
 & एक॰ & द्वि॰ & बहु॰  \\ \hline 
 प्रथमा & \csarva जलमुक्-ग् & \csarva जलमुचौ & \csarva जलमुचः \\ \hline 
 द्वितीया & \csarva जलमुचम् & \csarva जलमुचौ & \cbham जलमुचः \\ \hline 
 तृतीया & \cbham जलमुचा & \cpada जलमुग्भ्याम् & \cpada जलमुग्भिः \\ \hline 
 चतुर्थी & \cbham जलमुचे & \cpada जलमुग्भ्याम् & \cpada जलमुग्भ्यः \\ \hline 
 पञ्चमी & \cbham जलमुचः & \cpada जलमुग्भ्याम् & \cpada जलमुग्भ्यः \\ \hline 
 षष्ठी & \cbham जलमुचः & \cbham जलमुचोः & \cbham जलमुचाम् \\ \hline 
सप्तमी & \cbham जलमुचि & \cbham जलमुचोः & \cpada जलमुक्षु \\ \hline 
सम्बोधन & हे जलमुक् & हे जलमुचौ & हे जलमुचः \\ \hline 
\end{supertabular} 
\end{center} 

\begin{center} 
 \begin{supertabular}{|c|c|c|c|}\hline 
 \multicolumn {4}{|c|}{\cellcolor{blue!10}  \nextTable \hypertarget{वाच् (स्त्री॰) Speech}{वाच् (स्त्री॰) Speech}}  \\ \hline  
 & एक॰ & द्वि॰ & बहु॰  \\ \hline 
 प्रथमा & \csarva वाक्-वाग् & \csarva वाचौ & \csarva वाचः \\ \hline 
 द्वितीया & \csarva वाचम् & \csarva वाचौ & \cbham वाचः \\ \hline 
 तृतीया & \cbham वाचा & \cpada वाग्भ्याम् & \cpada वाग्भिः \\ \hline 
 चतुर्थी & \cbham वाचे & \cpada वाग्भ्याम् & \cpada वाग्भ्यः \\ \hline 
 पञ्चमी & \cbham वाचः & \cpada वाग्भ्याम् & \cpada वाग्भ्यः \\ \hline 
 षष्ठी & \cbham वाचः & \cbham वाचोः & \cbham वाचाम् \\ \hline 
सप्तमी & \cbham वाचि & \cbham वाचोः & \cpada वाक्षु \\ \hline 
सम्बोधन & हे वाक्-वाग् & हे वाचौ & हे वाचः \\ \hline 
\end{supertabular} 
\end{center} 

 \begin{center} 
 \begin{supertabular}{|c|c|c|c|}\hline 
 \multicolumn {4}{|c|}{\cellcolor{blue!10}  \nextTable \hypertarget{सुवाच् (नपु॰) Praiseworthy}{सुवाच् (नपु॰) Praiseworthy}}  \\ \hline  
 & एक॰ & द्वि॰ & बहु॰  \\ \hline 
 प्रथमा & \cpada सुवाक् & \cbham सुवाची & \csarva सुवाञ्चि \\ \hline 
 द्वितीया & \cbham सुवाक् & \cbham सुवाची & \csarva सुवाञ्चि \\ \hline 
सम्बोधन & हे सुवाक् & हे सुवाची & हे सुवाञ्चि \\ \hline 
 \multicolumn{4}{|c|}{ शेषं जलमुच् शब्दवत् } \\ \hline 
\end{supertabular} 
\end{center} 


 \begin{center} 
 \begin{supertabular}{|c|c|c|c|}\hline 
 \multicolumn {4}{|c|}{\cellcolor{blue!10}  \nextTable \hypertarget{वणिज् (पु॰) Trader}{वणिज् (पु॰) Trader}}  \\ \hline  
 & एक॰ & द्वि॰ & बहु॰  \\ \hline 
 प्रथमा & \csarva वणिक्-ग् & \csarva वणिजौ & \csarva वणिजः \\ \hline 
 द्वितीया & \csarva वणिजम् & \csarva वणिजौ & \cbham वणिजः \\ \hline 
 तृतीया & \cbham वणिजा & \cpada वणिग्भ्याम् & \cpada वणिग्भिः \\ \hline 
 चतुर्थी & \cbham वणिजे & \cpada वणिग्भ्याम् & \cpada वणिग्भ्यः \\ \hline 
 पञ्चमी & \cbham वणिजः & \cpada वणिग्भ्याम् & \cpada वणिग्भ्यः \\ \hline 
 षष्ठी & \cbham वणिजः & \cbham वणिजोः & \cbham वणिजाम् \\ \hline 
सप्तमी & \cbham वणिजि & \cbham वणिजोः & \cpada वणिक्षु \\ \hline 
सम्बोधन & हे वणिक्-ग् & हे वणिजौ & हे वणिजः \\ \hline 
\end{supertabular} 
\end{center} 

\small
\begin{center} 
 \begin{supertabular}{|c|c|c|c|}\hline 
 \multicolumn {4}{|c|}{\cellcolor{blue!10}  \nextTable \hypertarget{सम्राज् (पु॰) Sovereign}{सम्राज् (पु॰) Sovereign}}  \\ \hline  
 & एक॰ & द्वि॰ & बहु॰  \\ \hline 
 प्रथमा & \csarva सम्राट्-ड् & \csarva सम्राजौ & \csarva सम्राजः \\ \hline 
 द्वितीया & \csarva सम्राजम् & \csarva सम्राजौ & \cbham सम्राजः \\ \hline 
 तृतीया & \cbham सम्राजा & \cpada सम्राड्भ्याम् & \cpada सम्राड्भिः \\ \hline 
 चतुर्थी & \cbham सम्राजे & \cpada सम्राड्भ्याम् & \cpada सम्राड्भ्यः \\ \hline 
 पञ्चमी & \cbham सम्राजः & \cpada सम्राड्भ्याम् & \cpada सम्राड्भ्यः \\ \hline 
 षष्ठी & \cbham सम्राजः & \cbham सम्राजोः & \cbham सम्राजाम् \\ \hline 
सप्तमी & \cbham सम्राजि & \cbham सम्राजोः & \cpada सम्राट्सु \\ \hline 
सम्बोधन & हे सम्राट्-ड् & हे सम्राजौ & हे सम्राजः \\ \hline 
\end{supertabular} 
\end{center} 


\begin{center} 
 \begin{supertabular}{|c|c|c|c|}\hline 
 \multicolumn {4}{|c|}{\cellcolor{blue!10}  \nextTable \hypertarget{ऋत्विज् (पु॰) Priest}{ऋत्विज् (पु॰) Priest}}  \\ \hline  
 & एक॰ & द्वि॰ & बहु॰  \\ \hline 
 प्रथमा & \csarva ऋत्विक् & \csarva ऋत्विजौ & \csarva ऋत्विजः \\ \hline 
 द्वितीया & \csarva ऋत्विजम् & \csarva ऋत्विजौ & \cbham ऋत्विजः \\ \hline 
 तृतीया & \cbham ऋत्विजा & \cpada ऋत्विग्भ्याम् & \cpada ऋत्विग्भिः \\ \hline 
 चतुर्थी & \cbham ऋत्विजे & \cpada ऋत्विग्भ्याम् & \cpada ऋत्विग्भ्यः \\ \hline 
 पञ्चमी & \cbham ऋत्विजः & \cpada ऋत्विग्भ्याम् & \cpada ऋत्विग्भ्यः \\ \hline 
 षष्ठी & \cbham ऋत्विजः & \cbham ऋत्विजोः & \cbham ऋत्विजाम् \\ \hline 
सप्तमी & \cbham ऋत्विजि & \cbham ऋत्विजोः & \cpada ऋत्विक्षु \\ \hline 
सम्बोधन & हे ऋत्विक् & हे ऋत्विजौ & हे ऋत्विजः \\ \hline 
\end{supertabular} 
\end{center} 
 
\begin{center} 
 \begin{supertabular}{|c|c|c|c|}\hline 
 \multicolumn {4}{|c|}{\cellcolor{blue!10}  \nextTable \hypertarget{स्रज् (स्त्री॰) Garland}{स्रज् (स्त्री॰) Garland}}  \\ \hline  
 & एक॰ & द्वि॰ & बहु॰  \\ \hline 
 प्रथमा & \csarva स्रक्-ग् & \csarva स्रजौ & \csarva स्रजः \\ \hline 
 द्वितीया & \csarva स्रजम् & \csarva स्रजौ & \cbham स्रजः \\ \hline 
 तृतीया & \cbham स्रजा & \cpada स्रग्भ्याम् & \cpada स्रग्भिः \\ \hline 
 चतुर्थी & \cbham स्रजे & \cpada स्रग्भ्याम् & \cpada स्रग्भ्यः \\ \hline 
 पञ्चमी & \cbham स्रजः & \cpada स्रग्भ्याम् & \cpada स्रग्भ्यः \\ \hline 
 षष्ठी & \cbham स्रजः & \cbham स्रजोः & \cbham स्रजाम् \\ \hline 
सप्तमी & \cbham स्रजि & \cbham स्रजोः & \cpada स्रक्षु \\ \hline 
सम्बोधन & हे स्रक् & हे स्रजौ & हे स्रजः \\ \hline 
\end{supertabular} 
\end{center} 
 
\normalsize
 \begin{center} 
 \begin{supertabular}{|c|c|c|c|}\hline 
 \multicolumn {4}{|c|}{\cellcolor{blue!10}  \nextTable \hypertarget{असृज् (नपु॰) Saffron}{असृज् (नपु॰) Saffron}}  \\ \hline  
 & एक॰ & द्वि॰ & बहु॰  \\ \hline 
 प्रथमा & \cpada असृक् & \cbham असृजी & \csarva असृञ्जि \\ \hline 
 द्वितीया & \cbham असृक् & \cbham असृजी & \csarva असृञ्जि \\ \hline 
सम्बोधन & हे असृक् & हे असृजी & हे असृञ्जि \\ \hline 
 \multicolumn{4}{|c|}{ शेषं वणिज् शब्दवत् } \\ \hline 
\end{supertabular} 
\end{center} 

\begin{center}
%सरट् स्त्रीलिङ्गः शब्दः\\[1.5ex]	
		\begin{supertabular}{|c|c|c|c|}\hline
\multicolumn {4}{|c|}{\cellcolor{blue!10}  \nextTable \hypertarget{सरट् (स्त्री॰) Lizard}{सरट् (स्त्री॰) Lizard}}  \\ \hline 
		& एक॰ & द्वि॰ & बहु॰ \\ \hline
प्रथमा  & \csarva सरट्-सरड् & \csarva सरटौ & \csarva सरटः  \\ \hline
द्वितीया & \csarva सरटम् & \csarva सरटौ & \cbham सरटः   \\ \hline
तृतीया & \cbham सरटा & \cpada सरड्भ्याम् & \cpada सरड्भिः  \\ \hline
चतुर्थी &  \cbham सरटे & \cpada सरड्भ्याम् & \cpada सरड्भ्यः \\ \hline
पञ्चमी & \cbham सरटः & \cpada सरड्भ्याम् & \cpada सरड्भ्यः \\ \hline
षष्ठी & \cbham सरटः & \cbham सरटोः & \cbham सरटाम् \\ \hline
सप्तमी & \cbham सरटि & \cbham सरटोः & \cpada सरट्सु \\ \hline
सम्बोधन & हे सरट्-सरड् & हे सरटौ & हे सरटः \\ \hline		
		\end{supertabular}
		\end{center}
		%\caption{चकारान्तः पुंलिङ्गः प्राञ्च् शब्दः}
%\end{table}
%\par

\begin{center} 
 \begin{supertabular}{|c|c|c|c|}\hline 
 \multicolumn {4}{|c|}{\cellcolor{blue!10}  \nextTable \hypertarget{मरुत् (पु॰) Wind}{मरुत् (पु॰) Wind}}  \\ \hline  
 & एक॰ & द्वि॰ & बहु॰  \\ \hline 
 प्रथमा & \csarva मरुत् & \csarva मरुतौ & \csarva मरुतः \\ \hline 
 द्वितीया & \csarva मरुतम् & \csarva मरुतौ & \cbham मरुतः \\ \hline 
 तृतीया & \cbham मरुता & \cpada मरुद्भ्याम् & \cpada मरुद्भिः \\ \hline 
 चतुर्थी & \cbham मरुते & \cpada मरुद्भ्याम् & \cpada मरुद्भ्यः \\ \hline 
 पञ्चमी & \cbham मरुतः & \cpada मरुद्भ्याम् & \cpada मरुद्भ्यः \\ \hline 
 षष्ठी & \cbham मरुतः & \cbham मरुतोः & \cbham मरुताम् \\ \hline 
सप्तमी & \cbham मरुति & \cbham मरुतोः & \cpada मरुत्सु \\ \hline 
सम्बोधन & हे मरुत् & हे मरुतौ & हे मरुतः \\ \hline 
\end{supertabular} 
\end{center} 

 \begin{center} 
 \begin{supertabular}{|c|c|c|c|}\hline 
 \multicolumn {4}{|c|}{\cellcolor{blue!10}  \nextTable \hypertarget{जगत् (नपु॰) World}{जगत् (नपु॰) World}}  \\ \hline  
 & एक॰ & द्वि॰ & बहु॰  \\ \hline 
 प्रथमा & \cpada जगत् & \cbham जगती & \csarva जगन्ति \\ \hline 
 द्वितीया & \cbham जगत् & \cbham जगती & \csarva जगन्ति \\ \hline 
सम्बोधन & हे जगत् & हे जगती & हे जगन्ति \\ \hline 
 \multicolumn{4}{|c|}{ शेषं मरुत् शब्दवत् } \\ \hline 
\end{supertabular} 
\end{center} 

 \begin{center} 
 \begin{supertabular}{|c|c|c|c|}\hline 
 \multicolumn {4}{|c|}{\cellcolor{blue!10}  \nextTable \hypertarget{ददत् (नपु॰) The Giving One}{ददत् (नपु॰) The Giving One}}  \\ \hline  
 & एक॰ & द्वि॰ & बहु॰  \\ \hline 
 प्रथमा & \cpada ददत् & \cbham ददती & \csarva ददति-ददन्ति \\ \hline 
 द्वितीया & \cbham ददत् & \cbham ददती & \csarva ददति-ददन्ति \\ \hline 
सम्बोधन & हे ददत् & हे ददती & हे ददति-ददन्ति \\ \hline 
 \multicolumn{4}{|c|}{ शेषं मरुत् शब्दवत् } \\ \hline 
\end{supertabular} 
\end{center} 

 \begin{center} 
 \begin{supertabular}{|c|c|c|c|}\hline 
 \multicolumn {4}{|c|}{\cellcolor{blue!10}  \nextTable \hypertarget{तुदत् (नपु॰) The Troubling One}{तुदत् (नपु॰) The Troubling One}}  \\ \hline  
 & एक॰ & द्वि॰ & बहु॰  \\ \hline 
 प्रथमा & \cpada तुदत् & \cbham तुदन्ती & \csarva तुदती-तुदन्ति \\ \hline 
 द्वितीया & \cbham तुदत् & \cbham तुदन्ती & \csarva तुदती-तुदन्ति \\ \hline 
सम्बोधन & हे तुदत् & हे तुदन्ती & हे तुदती-तुदन्ति \\ \hline 
 \multicolumn{4}{|c|}{ शेषं मरुत् शब्दवत् } \\ \hline 
\end{supertabular} 
\end{center} 

 \begin{center} 
 \begin{supertabular}{|c|c|c|c|}\hline 
 \multicolumn {4}{|c|}{\cellcolor{blue!10}  \nextTable \hypertarget{पचत् (नपु॰) The Cooking One}{पचत् (नपु॰) The Cooking One}}  \\ \hline  
 & एक॰ & द्वि॰ & बहु॰  \\ \hline 
 प्रथमा & \cpada पचत् & \cbham पचन्ती & \csarva पचन्ति \\ \hline 
 द्वितीया & \cbham पचत् & \cbham पचन्ती & \csarva पचन्ति \\ \hline 
सम्बोधन & हे पचत् & हे पचन्ती & हे पचन्ति \\ \hline 
 \multicolumn{4}{|c|}{ शेषं मरुत् शब्दवत् } \\ \hline 
\end{supertabular} 
\end{center} 

 \begin{center} 
 \begin{supertabular}{|c|c|c|c|}\hline 
 \multicolumn {4}{|c|}{\cellcolor{blue!10}  \nextTable \hypertarget{महत् (नपु॰) Great}{महत् (नपु॰) Great}}  \\ \hline  
 & एक॰ & द्वि॰ & बहु॰  \\ \hline 
 प्रथमा & \cpada महत् & \cbham महती & \csarva महान्ति \\ \hline 
 द्वितीया & \cbham महत् & \cbham महती & \csarva महान्ति \\ \hline 
सम्बोधन & हे महत् & हे महती & हे महान्ति \\ \hline 
 \multicolumn{4}{|c|}{शेषं महत्-पुंवत् } \\ \hline 
\end{supertabular} 
\end{center} 


%\begin{table}
	%\centering

%\begin{table}
	%\centering

 \begin{center} 
 \begin{supertabular}{|c|c|c|c|}\hline 
 \multicolumn {4}{|c|}{\cellcolor{blue!10}  \nextTable \hypertarget{दत् (पु॰) Tooth}{दत् (पु॰) Tooth}}  \\ \hline  
 & एक॰ & द्वि॰ & बहु॰  \\ \hline 
 प्रथमा & \csarva  & \csarva  & \csarva  \\ \hline 
 द्वितीया & \csarva  & \csarva  & \cbham दतः \\ \hline 
 तृतीया & \cbham दता & \cpada दद्भ्याम् & \cpada दद्भिः \\ \hline 
 चतुर्थी & \cbham दते & \cpada दद्भ्याम् & \cpada दद्भ्यः \\ \hline 
 पञ्चमी & \cbham दतः & \cpada दद्भ्याम् & \cpada दद्भ्यः \\ \hline 
 षष्ठी & \cbham दतः & \cbham दतोः & \cbham दताम् \\ \hline 
सप्तमी & \cbham दति & \cbham दतोः & \cpada दत्सु \\ \hline 
%सम्बोधन &  &  &  \\ \hline 
\end{supertabular} 
\end{center} 

\begin{center} 
 \begin{supertabular}{|c|c|c|c|}\hline 
 \multicolumn {4}{|c|}{\cellcolor{blue!10}  \nextTable \hypertarget{महत् (पु॰) Great}{महत् (पु॰) Great}}  \\ \hline  
 & एक॰ & द्वि॰ & बहु॰  \\ \hline 
 प्रथमा & \csarva महान् & \csarva महान्तौ & \csarva महान्तः \\ \hline 
 द्वितीया & \csarva महान्तम् & \csarva महान्तौ & \cbham महतः \\ \hline 
 तृतीया & \cbham महता & \cpada महद्भ्याम् & \cpada महद्भिः \\ \hline 
 चतुर्थी & \cbham महते & \cpada महद्भ्याम् & \cpada महद्भ्यः \\ \hline 
 पञ्चमी & \cbham महतः & \cpada महद्भ्याम् & \cpada महद्भ्यः \\ \hline 
 षष्ठी & \cbham महतः & \cbham महतोः & \cbham महताम् \\ \hline 
सप्तमी & \cbham महति & \cbham महतोः & \cpada महत्सु \\ \hline 
सम्बोधन & हे महन् & हे महान्तौ & हे महान्तः \\ \hline 
\end{supertabular} 
\end{center} 
 
\begin{center}
%शरद् स्त्रीलिङ्गः शब्दः\\[1.5ex]
		\begin{supertabular}{|c|c|c|c|}\hline
\multicolumn {4}{|c|}{\cellcolor{blue!10}  \nextTable \hypertarget{शरद् (स्त्री॰) Winter}{शरद् (स्त्री॰) Winter}}  \\ \hline 
		& एक॰ & द्वि॰ & बहु॰ \\ \hline
प्रथमा  & \csarva शरत्-शरद् & \csarva शरदौ & \csarva शरदः  \\ \hline
द्वितीया & \csarva शरदम् & \csarva शरदौ & \cbham शरदः   \\ \hline
तृतीया & \cbham शरदा & \cpada शरद्भ्याम् & \cpada शरद्भिः  \\ \hline
चतुर्थी & \cbham शरदे & \cpada शरद्भ्याम् & \cpada शरद्भ्यः \\ \hline
पञ्चमी & \cbham शरदः & \cpada शरद्भ्याम् & \cpada शरद्भ्यः \\ \hline
षष्ठी & \cbham शरदः & \cbham शरदोः & शरदाम् \\ \hline
सप्तमी & \cbham शरदि & \cbham शरदोः & \cpada शरत्सु \\ \hline
सम्बोधन & हे शरत्-शरद् & हे शरदौ & हे शरदः \\ \hline		
		\end{supertabular}
		\end{center}
		%\caption{चकारान्तः पुंलिङ्गः प्राञ्च् शब्दः}
%\end{table}
%\onecolumn


\begin{center} 
 \begin{supertabular}{|c|c|c|c|}\hline 
 \multicolumn {4}{|c|}{\cellcolor{blue!10}  \nextTable \hypertarget{पद् (पु॰) Foot}{पद् (पु॰) Foot}}  \\ \hline  
 & एक॰ & द्वि॰ & बहु॰  \\ \hline 
 प्रथमा & \csarva  & \csarva  & \csarva  \\ \hline 
 द्वितीया & \csarva  & \csarva  & \cbham पदः \\ \hline 
 तृतीया & \cbham पदा & \cpada पद्भ्याम् & \cpada पद्भिः \\ \hline 
 चतुर्थी & \cbham पदे & \cpada पद्भ्याम् & \cpada पद्भ्यः \\ \hline 
 पञ्चमी & \cbham पदः & \cpada पद्भ्याम् & \cpada पद्भ्यः \\ \hline 
 षष्ठी & \cbham पदः & \cbham प्दोः & \cbham पदाम् \\ \hline 
सप्तमी & \cbham पदि & \cbham पदोः & \cpada पत्सु \\ \hline 
सम्बोधन &  &  &   \\ \hline 
\end{supertabular} 
\end{center} 
 
 \begin{center} 
 \begin{supertabular}{|c|c|c|c|}\hline 
 \multicolumn {4}{|c|}{\cellcolor{blue!10}  \nextTable \hypertarget{हृद् (नपु॰) Heart}{हृद् (नपु॰) Heart}}  \\ \hline  
 & एक॰ & द्वि॰ & बहु॰  \\ \hline 
 प्रथमा & \cpada हृत् & \cbham हृदी & \csarva हृन्दि \\ \hline 
 द्वितीया & \cbham हृत् & \cbham हृदी & \csarva हृन्दि \\ \hline 
सम्बोधन & हे हृत् & हे हृदी & हे हृन्दि \\ \hline 
 \multicolumn{4}{|c|}{ शेषं सुहृद् शब्दवत् } \\ \hline 
\end{supertabular} 
\end{center} 

\begin{center} 
 \begin{supertabular}{|c|c|c|c|}\hline 
 \multicolumn {4}{|c|}{\cellcolor{blue!10}  \nextTable \hypertarget{सुहृद् (पु॰) Kind-hearted}{सुहृद् (पु॰) Kind-hearted}}  \\ \hline  
 & एक॰ & द्वि॰ & बहु॰  \\ \hline 
 प्रथमा & \csarva सुहृत्-द् & \csarva सुहृदौ & \csarva सुहृदः \\ \hline 
 द्वितीया & \csarva सुहृदम् & \csarva सुहृदौ & \cbham सुहृदः \\ \hline 
 तृतीया & \cbham सुहृदा & \cpada सुहृद्भ्याम् & \cpada सुहृद्भिः \\ \hline 
 चतुर्थी & \cbham सुहृदे & \cpada सुहृद्भ्याम् & \cpada सुहृद्भ्यः \\ \hline 
 पञ्चमी & \cbham सुहृदः & \cpada सुहृद्भ्याम् & \cpada सुहृद्भिः \\ \hline 
 षष्ठी & \cbham सुहृदः & \cbham सुहृदोः & \cbham सुहृदाम् \\ \hline 
सप्तमी & \cbham सुहृदि & \cbham सुहृदोः & \cpada सुहृद्सु \\ \hline 
सम्बोधन & हे सुहृत्-द् & हे सुहृदौ & हे सुहृदः \\ \hline 
\end{supertabular} 
\end{center}
 
 \begin{center} 
 \begin{supertabular}{|c|c|c|c|}\hline 
 \multicolumn {4}{|c|}{\cellcolor{blue!10}  \nextTable \hypertarget{समिध् (स्त्री॰) Firewood}{समिध् (स्त्री॰) Firewood}}  \\ \hline  
 & एक॰ & द्वि॰ & बहु॰  \\ \hline 
 प्रथमा & \csarva समित्-द् & \csarva समिधौ & \csarva समिधः \\ \hline 
 द्वितीया & \csarva समिधम् & \csarva समिधौ & \cbham समिधः \\ \hline 
 तृतीया & \cbham समिधा & \cpada समिद्भ्याम् & \cpada समिद्भिः \\ \hline 
 चतुर्थी & \cbham समिधे & \cpada समिद्भ्याम् & \cpada समिद्भ्यः \\ \hline 
 पञ्चमी & \cbham समिधः & \cpada समिध्भ्याम् & \cpada समिद्भ्यः \\ \hline 
 षष्ठी & \cbham समिधः & \cbham समिधोः & \cbham समिधाम् \\ \hline 
सप्तमी & \cbham समिधि & \cbham समिधोः & \cpada समित्सु \\ \hline 
सम्बोधन & हे समित्-द् & हे समिधौ & हे समिधः \\ \hline 
\end{supertabular} 
\end{center} 


\begin{center} 
 \begin{supertabular}{|c|c|c|c|}\hline 
 \multicolumn {4}{|c|}{\cellcolor{blue!10}  \nextTable \hypertarget{क्षुध् (स्त्री॰) Hunger}{क्षुध् (स्त्री॰) Hunger}}  \\ \hline  
 & एक॰ & द्वि॰ & बहु॰  \\ \hline 
 प्रथमा & \csarva क्षुत् & \csarva क्षुधौ & \csarva क्षुधः \\ \hline 
 द्वितीया & \csarva क्षुधम् & \csarva क्षुधौ & \cbham क्षुधः \\ \hline 
 तृतीया & \cbham क्षुधा & \cpada क्षुद्भ्याम् & \cpada क्षुद्भिः \\ \hline 
 चतुर्थी & \cbham क्षुधे & \cpada क्षुद्भ्याम् & \cpada क्षुद्भ्यः \\ \hline 
 पञ्चमी & \cbham क्षुधः & \cpada क्षुद्भ्याम् & \cpada क्षुद्भ्यः \\ \hline 
 षष्ठी & \cbham क्षुधः & \cbham क्षुधोः & \cbham क्षुधाम् \\ \hline 
सप्तमी & \cbham क्षुधि & \cbham क्षुधोः & \cpada क्षुत्सु \\ \hline 
सम्बोधन & हे क्षुत् & हे क्षुधौ & हे क्षुधः \\ \hline 
\end{supertabular} 
\end{center} 

 \begin{center} 
 \begin{supertabular}{|c|c|c|c|}\hline 
 \multicolumn {4}{|c|}{\cellcolor{blue!10}  \nextTable \hypertarget{राजन् (पु॰) King}{राजन् (पु॰) King}}  \\ \hline  
 & एक॰ & द्वि॰ & बहु॰  \\ \hline 
 प्रथमा & \csarva राजा & \csarva रजानौ & \csarva राजानः \\ \hline 
 द्वितीया & \csarva राजानम् & \csarva राजानौ & \cbham राज्ञः \\ \hline 
 तृतीया & \cbham राज्ञा & \cpada राजभ्याम् & \cpada राजभिः \\ \hline 
 चतुर्थी & \cbham राज्ञे & \cpada राजभ्याम् & \cpada राजभ्यः \\ \hline 
 पञ्चमी & \cbham राज्ञः & \cpada राजभ्याम् & \cpada राजभ्यः \\ \hline 
 षष्ठी & \cbham राज्ञः & \cbham राज्ञोः & \cbham राज्ञाम् \\ \hline 
सप्तमी & \cbham राज्ञि & \cbham राज्ञोः & \cpada राजसु \\ \hline 
सम्बोधन & हे राजन् & हे राजानौ & हे राजानः \\ \hline 
\end{supertabular} 
\end{center} 
%
 
%\end{multicols}
%
%\clearpage
%
%\begin{multicols}{2}
%\TrickSupertabularIntoMulticols
 

 \begin{center} 
 \begin{supertabular}{|c|c|c|c|}\hline 
 \multicolumn {4}{|c|}{\cellcolor{blue!10}  \nextTable \hypertarget{मघवन् (पु॰) Indra}{मघवन् (पु॰) Indra}}  \\ \hline  
 & एक॰ & द्वि॰ & बहु॰  \\ \hline 
 प्रथमा & \csarva मघवा & \csarva मघवानौ & \csarva मघवानः \\ \hline 
 द्वितीया & \csarva मघवानम् & \csarva मघवानौ & \cbham मघोनः \\ \hline 
 तृतीया & \cbham मघोना & \cpada मघवभ्याम् & \cpada मघवभिः \\ \hline 
 चतुर्थी & \cbham मघोने & \cpada मघवभ्याम् & \cpada मघवभ्यः \\ \hline 
 पञ्चमी & \cbham मघोनः & \cpada मघवभ्याम् & \cpada मघवभ्यः \\ \hline 
 षष्ठी & \cbham मघोनः & \cbham मघोनोः & \cbham मघोनाम् \\ \hline 
सप्तमी & \cbham मघोनि & \cbham मघोनोः & \cpada मघवसु \\ \hline 
सम्बोधन & हे मघवन् & हे मघवानौ & हे मघवानः\\ \hline 
\end{supertabular} 
\end{center} 

\begin{center} 
 \begin{supertabular}{|c|c|c|c|}\hline 
 \multicolumn {4}{|c|}{\cellcolor{blue!10}  \nextTable \hypertarget{मघवन् (पु॰) Indra}{मघवन् (पु॰) Indra}}  \\ \hline  
 & एक॰ & द्वि॰ & बहु॰  \\ \hline 
 प्रथमा & \csarva मघवान् & \csarva मघवन्तौ & \csarva मघवन्तः \\ \hline 
 द्वितीया & \csarva मघवन्तम् & \csarva मघवन्तौ & \cbham मघवतः \\ \hline 
सम्बोधन & हे मघवन् & हे मघवन्तौ & हे मघवन्तः \\ \hline 
 \multicolumn{4}{|c|}{ शेषं मरुत् शब्दवत् } \\ \hline 
\end{supertabular} 
\end{center} 
 
 \begin{center} 
 \begin{supertabular}{|c|c|c|c|}\hline 
 \multicolumn {4}{|c|}{\cellcolor{blue!10}  \nextTable \hypertarget{युवन् (पु॰) Youth}{युवन् (पु॰) Youth}}  \\ \hline  
 & एक॰ & द्वि॰ & बहु॰  \\ \hline 
 प्रथमा & \csarva युवा & \csarva युवानौ & \csarva युवानः \\ \hline 
 द्वितीया & \csarva युवानम् & \csarva युवानौ & \cbham यूनः \\ \hline 
 तृतीया & \cbham यूना & \cpada युवभ्याम् & \cpada युवभिः \\ \hline 
 चतुर्थी & \cbham यूने & \cpada युवभ्याम् & \cpada युवभ्यः \\ \hline 
 पञ्चमी & \cbham यूनः & \cpada युवभ्याम् & \cpada युवभ्यः \\ \hline 
 षष्ठी & \cbham यूनः & \cbham यूनोः & \cbham यूनाम् \\ \hline 
सप्तमी & \cbham युनि & \cbham युनोः & \cpada युवसु \\ \hline 
सम्बोधन & हे युवन् & हे युवानौ & हे युवानः\\ \hline 
\end{supertabular} 
\end{center} 

\small
 \begin{center} 
 \begin{supertabular}{|c|c|c|c|}\hline 
 \multicolumn {4}{|c|}{\cellcolor{blue!10}  \nextTable \hypertarget{श्वन् (पु॰) Dog}{श्वन् (पु॰) Dog}}  \\ \hline  
 & एक॰ & द्वि॰ & बहु॰  \\ \hline 
 प्रथमा & \csarva श्वा & \csarva श्वानौ & \csarva श्वानः \\ \hline 
 द्वितीया & \csarva श्वानम् & \csarva श्वानौ & \cbham शुनः \\ \hline 
 तृतीया & \cbham शुना & \cpada श्वभ्याम् & \cpada श्वभिः \\ \hline 
 चतुर्थी & \cbham शुने & \cpada श्वभ्याम् & \cpada श्वभ्यः \\ \hline 
 पञ्चमी & \cbham शुनः & \cpada श्वभ्याम् & \cpada श्वभ्यः \\ \hline 
 षष्ठी & \cbham शुनः & \cbham शुनोः & \cbham शुनाम् \\ \hline 
सप्तमी & \cbham शुनि & \cbham शुनोः & \cpada श्वसु \\ \hline 
सम्बोधन & हे श्वन् & हे श्वानौ & हे श्वानः \\ \hline 
\end{supertabular} 
\end{center} 

\begin{center} 
 \begin{supertabular}{|c|c|c|c|}\hline 
 \multicolumn {4}{|c|}{\cellcolor{blue!10}  \nextTable \hypertarget{अर्वन् (पु॰) Horse}{अर्वन् (पु॰) Horse}}  \\ \hline  
 & एक॰ & द्वि॰ & बहु॰  \\ \hline 
 प्रथमा & \csarva अर्वा & \csarva अर्वन्तौ & \csarva अर्वन्तः \\ \hline 
 द्वितीया & \csarva अर्वन्तम् & \csarva अर्वन्तौ & \cbham अर्वतः \\ \hline 
 तृतीया & \cbham अर्वता & \cpada अर्वद्भ्याम् & \cpada अर्वद्भिः \\ \hline 
 चतुर्थी & \cbham अर्वते & \cpada अर्वद्भ्याम् & \cpada अर्वद्भ्यः \\ \hline 
 पञ्चमी & \cbham अर्वतः & \cpada अर्वद्भ्याम् & \cpada अर्वद्भ्यः \\ \hline 
 षष्ठी & \cbham अर्वतः & \cbham अर्वतोः & \cbham अर्वताम् \\ \hline 
सप्तमी & \cbham अर्वति & \cbham अर्वतोः & \cpada अर्वत्सु \\ \hline 
सम्बोधन & हे अर्वन् & हे अर्वन्तौ & हे अर्वन्तः \\ \hline 
\end{supertabular} 
\end{center} 

 \normalsize
\begin{center} 
 \begin{supertabular}{|c|c|c|c|}\hline 
 \multicolumn {4}{|c|}{\cellcolor{blue!10}  \nextTable \hypertarget{मूर्धन् (पु॰) Head}{मूर्धन् (पु॰) Head}}  \\ \hline  
 & एक॰ & द्वि॰ & बहु॰  \\ \hline 
 प्रथमा & \csarva मूर्धा & \csarva मूर्धानौ & \csarva मूर्धानः \\ \hline 
 द्वितीया & \csarva मूर्धानम् & \csarva मूर्धानौ & \cbham मूर्ध्नः \\ \hline 
 तृतीया & \cbham मूर्ध्ना & \cpada मूर्धभ्याम् & \cpada मूर्धभिः \\ \hline 
 चतुर्थी & \cbham मूर्ध्ने & \cpada मूर्धभ्यम् & \cpada मूर्धभ्यः \\ \hline 
 पञ्चमी & \cbham मूर्धनः & \cpada मूर्धभ्यम् & \cpada मूर्धभ्यः \\ \hline 
 षष्ठी & \cbham मूर्धनः & \cbham मूर्ध्नोः & \cbham मूर्ध्नाम् \\ \hline 
सप्तमी & \cbham मूर्ध्नि-मूर्धनि & \cbham मूर्ध्नोः & \cpada मूर्धसु \\ \hline 
सम्बोधन & हे मूर्धन् & हे मूर्धानौ & हे मूर्धानः \\ \hline 
\end{supertabular} 
\end{center} 

 \begin{center} 
 \begin{supertabular}{|c|c|c|c|}\hline 
 \multicolumn {4}{|c|}{\cellcolor{blue!10}  \nextTable \hypertarget{लघिमन् (पु॰) Smaller}{लघिमन् (पु॰) Smaller}}  \\ \hline  
 & एक॰ & द्वि॰ & बहु॰  \\ \hline 
 प्रथमा & \csarva लघिमा & \csarva लघिमानौ & \csarva लघिमानः \\ \hline 
 द्वितीया & \csarva लघिमानम् & \csarva लघिमानौ & \cbham लघिम्नः \\ \hline 
 तृतीया & \cbham लघिम्ना & \cpada लघिमभ्याम् & \cpada लघिमभिः \\ \hline 
 चतुर्थी & \cbham लघिम्ने & \cpada लघिमभ्याम् & \cpada लघिमभ्यः \\ \hline 
 पञ्चमी & \cbham लघिम्नः & \cpada लघिमभ्याम् & \cpada लघिमभ्यः \\ \hline 
 षष्ठी & \cbham लघिम्नः & \cbham लघिम्नोः & \cbham लघिम्नाम् \\ \hline 
सप्तमी & \cbham लघिम्निलघिमनि & \cbham लघिम्नोः & \cpada लघिमसु \\ \hline 
सम्बोधन & हे लघिमन् & हे लघिमानौ & हे लघिमानः \\ \hline 
\end{supertabular} 
\end{center} 

\begin{center} 
 \begin{supertabular}{|c|c|c|c|}\hline 
 \multicolumn {4}{|c|}{\cellcolor{blue!10}  \nextTable \hypertarget{महिमन् (पु॰) Greatness}{महिमन् (पु॰) Greatness}}  \\ \hline  
 & एक॰ & द्वि॰ & बहु॰  \\ \hline 
 प्रथमा & \csarva महिमा & \csarva महिमानौ & \csarva महिमानः \\ \hline 
 द्वितीया & \csarva महिमानम् & \csarva महिमानौ & \cbham महिम्नः \\ \hline 
 तृतीया & \cbham महिम्ना & \cpada महिमभ्याम् & \cpada महिमभिः \\ \hline 
 चतुर्थी & \cbham महिम्ने & \cpada महिमभ्याम् & \cpada महिमभ्यः \\ \hline 
 पञ्चमी & \cbham महिम्नः & \cpada महिमभ्याम् & \cpada महिमभ्यः \\ \hline 
 षष्ठी & \cbham महिम्नः & \cbham महिम्नोः & \cbham महिम्नाम् \\ \hline 
सप्तमी & \cbham महिम्नि-महिमनि & \cbham महिम्नोः & \cpada महिमसु \\ \hline 
सम्बोधन & हे महिमन् & हे महिमानौ & हे महिमानः \\ \hline 
\end{supertabular} 
\end{center} 
 
%\small
 \begin{center} 
 \begin{supertabular}{|c|c|c|c|}\hline 
 \multicolumn {4}{|c|}{\cellcolor{blue!10}  \nextTable \hypertarget{अस्थायिन् (नपु॰) Temporary}{अस्थायिन् (नपु॰) Temporary}}  \\ \hline  
 & एक॰ & द्वि॰ & बहु॰  \\ \hline 
 प्रथमा & \cpada अस्थायि & \cbham अस्थायिनी & \csarva अस्थायीनि \\ \hline 
 द्वितीया & \cbham अस्थायि & \cbham अस्थायिनी & \csarva अस्थायीनि \\ \hline 
 तृतीया & \cbham अस्थायिना & \cpada अस्थायिभ्याम् & \cpada अस्थायिभिः \\ \hline 
 चतुर्थी & \cbham अस्थायिने & \cpada अस्थायिभ्याम् & \cpada अस्थायिभ्यः \\ \hline 
 पञ्चमी & \cbham अस्थायिनः & \cpada अस्थायिभ्याम् & \cpada अस्थयिभ्यः \\ \hline 
 षष्ठी & \cbham अस्थायिनः & \cbham अस्थायिनोः & \cbham अस्थ्यायिनाम् \\ \hline 
सप्तमी & \cbham अस्थायिनि & \cbham अस्थायिनोः & \cpada अस्थायिषु \\ \hline 
सम्बोधन & हे अस्थायि-अस्थायिन् & हे अस्थायिनी & हे अस्थायीनि \\ \hline 
\end{supertabular} 
\end{center} 

 %\normalsize
\begin{center} 
 \begin{supertabular}{|c|c|c|c|}\hline 
 \multicolumn {4}{|c|}{\cellcolor{blue!10}  \nextTable \hypertarget{गुणिन् (नपु॰) With Good Qualities}{गुणिन् (नपु॰) With Good Qualities}}  \\ \hline  
 & एक॰ & द्वि॰ & बहु॰  \\ \hline 
 प्रथमा & \cpada गुणि & \cbham गुणिनी & \csarva गुणीनि \\ \hline 
 द्वितीया & \cbham गुणि & \cbham गुणिनी & \csarva गुणीनि \\ \hline 
सम्बोधन & हे गुणिन्-गुणि & हे गुणिनी & हे गुणीनि \\ \hline 
 \multicolumn{4}{|c|}{ शेषं ज्ञानिन्-पुंवत् } \\ \hline 
\end{supertabular} 
\end{center}

\medskip

 \begin{center} 
 \begin{supertabular}{|c|c|c|c|}\hline 
 \multicolumn {4}{|c|}{\cellcolor{blue!10}  \nextTable \hypertarget{पथिन् (पु॰) Traveller}{पथिन् (पु॰) Traveller}}  \\ \hline  
 & एक॰ & द्वि॰ & बहु॰  \\ \hline 
 प्रथमा & \csarva पन्थाः & \csarva पन्थानौ & \csarva पन्थानः \\ \hline 
 द्वितीया & \csarva पन्थानम् & \csarva पन्थानौ & \cbham पथः \\ \hline 
 तृतीया & \cbham पथाः & \cpada पथिभ्याम् & \cpada प्थिभिः \\ \hline 
 चतुर्थी & \cbham पथे & \cpada पथिभ्याम् & \cpada पथिभ्यः \\ \hline 
 पञ्चमी & \cbham पथः & \cpada पथिभ्याम् & \cpada पथिभ्यः \\ \hline 
 षष्ठी & \cbham पथः & \cbham पथोः & \cbham पथाम् \\ \hline 
सप्तमी & \cbham पथि & \cbham पथोः & \cpada पथिषु \\ \hline 
सम्बोधन & हे पन्थाः & हे पन्थानौ & हे पन्थानः \\ \hline 
\end{supertabular} 
\end{center} 

 
%\end{multicols}
%
%\clearpage
%
%\begin{multicols}{2}
%\TrickSupertabularIntoMulticols
 \begin{center} 
 \begin{supertabular}{|c|c|c|c|}\hline 
 \multicolumn {4}{|c|}{\cellcolor{blue!10}  \nextTable \hypertarget{अहन् (नपु॰) Day}{अहन् (नपु॰) Day}}  \\ \hline  
 & एक॰ & द्वि॰ & बहु॰  \\ \hline 
 प्रथमा & \cpada अहः & \cbham अह्नी-अहनी & \csarva अहानि \\ \hline 
 द्वितीया & \cbham अहः & \cbham अह्नी-अहनी & \csarva अहानि \\ \hline 
 तृतीया & \cbham अह्ना & \cpada अहोभ्याम् & \cpada अहोभिः \\ \hline 
 चतुर्थी & \cbham अह्ने & \cpada अहोभ्याम् & \cpada अहोभ्यः \\ \hline 
 पञ्चमी & \cbham अह्नः & \cpada अहोभ्याम् & \cpada अहोभ्यः \\ \hline 
 षष्ठी & \cbham अह्नः & \cbham अह्नोः & \cbham अह्नाम् \\ \hline 
सप्तमी & \cbham अह्नि-अहनि & \cbham अह्नोः & \cpada अहःसु-अहस्सु \\ \hline 
सम्बोधन & हे अहः & हे अह्नी-अहनी & हे अहानि \\ \hline 
\end{supertabular} 
\end{center} 

 \begin{center} 
 \begin{supertabular}{|c|c|c|c|}\hline 
 \multicolumn {4}{|c|}{\cellcolor{blue!10}  \nextTable \hypertarget{भाविन् (नपु॰) Imminent}{भाविन् (नपु॰) Imminent}}  \\ \hline  
 & एक॰ & द्वि॰ & बहु॰  \\ \hline 
 प्रथमा & \cpada भावि & \cbham भाविनी & \csarva भावीनि \\ \hline 
 द्वितीया & \cbham भावि & \cbham भाविनी & \csarva भवीनि \\ \hline 
 तृतीया & \cbham भाविना & \cpada भाविभ्याम् & \cpada भाविभ्यः \\ \hline 
 चतुर्थी & \cbham भाविने & \cpada भाविभ्याम् & \cpada भाविभ्यः \\ \hline 
 पञ्चमी & \cbham भाविनः & \cpada भाविभ्याम् & \cpada भाविभ्यः \\ \hline 
 षष्ठी & \cbham भाविनः & \cbham भाविनोः & \cbham भाविनाम् \\ \hline 
सप्तमी & \cbham भाविनि & \cbham भाविनोः & \cpada भाविषु \\ \hline 
सम्बोधन & हे भावि & हे भाविनी & हे भावीनि \\ \hline 
\end{supertabular} 
\end{center} 

\vspace{2cm}

 \begin{center} 
 \begin{supertabular}{|c|c|c|c|}\hline 
 \multicolumn {4}{|c|}{\cellcolor{blue!10}  \nextTable \hypertarget{अप् (स्त्री॰) Water}{अप् (स्त्री॰) Water}}  \\ \hline  
 & एक॰ & द्वि॰ & बहु॰  \\ \hline 
 प्रथमा & \csarva & \csarva   & \csarva आपः \\ \hline 
 द्वितीया & \csarva   & \csarva   & \cbham अपः \\ \hline 
 तृतीया & \cbham   & \cpada   & \cpada अद्भिः \\ \hline 
 चतुर्थी & \cbham   & \cpada   & \cpada अद्भ्यः \\ \hline 
 पञ्चमी & \cbham   & \cpada   & \cpada अद्भ्यः \\ \hline 
 षष्ठी & \cbham   & \cbham   & \cbham अपाम् \\ \hline 
सप्तमी & \cbham   & \cbham   & \cpada अप्सु \\ \hline 
सम्बोधन &   &   & हे आपः \\ \hline 
\end{supertabular} 
\end{center} 

 \begin{center} 
 \begin{supertabular}{|c|c|c|c|}\hline 
 \multicolumn {4}{|c|}{\cellcolor{blue!10}  \nextTable \hypertarget{ककुभ् (स्त्री॰) Splendour}{ककुभ् (स्त्री॰) Splendour}}  \\ \hline  
 & एक॰ & द्वि॰ & बहु॰  \\ \hline 
 प्रथमा & \csarva ककुप् & \csarva ककुभौ & \csarva ककुभः \\ \hline 
 द्वितीया & \csarva ककुभम् & \csarva ककुभौ & \cbham ककुभः \\ \hline 
 तृतीया & \cbham ककुभा & \cpada ककुब्भ्याम् & \cpada ककुब्भिः \\ \hline 
 चतुर्थी & \cbham ककुभे & \cpada ककुब्भ्याम् & \cpada ककुब्भ्यः \\ \hline 
 पञ्चमी & \cbham ककुभः & \cpada ककुब्भ्याम् & \cpada ककुब्भ्यः \\ \hline 
 षष्ठी & \cbham ककुभः & \cbham ककुभोः & \cbham ककुभाम् \\ \hline 
सप्तमी & \cbham ककुभि & \cbham ककुभोः & \cpada ककुप्सु \\ \hline 
सम्बोधन & हे ककुभ् & हे ककुभौ & हे ककुभः \\ \hline 
\end{supertabular} 
\end{center} 

  \begin{center} 
 \begin{supertabular}{|c|c|c|c|}\hline 
 \multicolumn {4}{|c|}{\cellcolor{blue!10}  \nextTable \hypertarget{वार् (नपु॰) Pond}{वार् (नपु॰) Pond}}  \\ \hline  
 & एक॰ & द्वि॰ & बहु॰  \\ \hline 
 प्रथमा & \cpada वाः & \cbham वारी & \csarva वारि \\ \hline 
 द्वितीया & \cbham वाः & \cbham वारी & \csarva वारि \\ \hline 
 तृतीया & \cbham वारा & \cpada वार्भ्याम् & \cpada वार्भिः \\ \hline 
 चतुर्थी & \cbham वारे & \cpada वार्भ्याम् & \cpada वार्भ्यः \\ \hline 
 पञ्चमी & \cbham वारः & \cpada वार्भ्याम् & \cpada वार्भ्यः \\ \hline 
 षष्ठी & \cbham वारः & \cbham वारोः & \cbham वाराम् \\ \hline 
सप्तमी & \cbham वारि & \cbham वारोः & \cpada वार्षु \\ \hline 
सम्बोधन & हे वाः & हे वारी & हे वारि \\ \hline 
\end{supertabular} 
\end{center} 

 \begin{center} 
 \begin{supertabular}{|c|c|c|c|}\hline 
 \multicolumn {4}{|c|}{\cellcolor{blue!10}  \nextTable \hypertarget{गिर् (स्त्री॰) Voice}{गिर् (स्त्री॰) Voice}}  \\ \hline  
 & एक॰ & द्वि॰ & बहु॰  \\ \hline 
 प्रथमा & \csarva गीः & \csarva गिरौ & \csarva गिरः \\ \hline 
 द्वितीया & \csarva गिरम् & \csarva गिरौ & \cbham गिरः \\ \hline 
 तृतीया & \cbham गिरा & \cpada गीर्भ्याम् & \cpada गीर्भिः \\ \hline 
 चतुर्थी & \cbham गिरे & \cpada गीर्भ्याम् & \cpada गीर्भ्यः \\ \hline 
 पञ्चमी & \cbham गिरः & \cpada गीर्भ्याम् & \cpada गीर्भ्यः \\ \hline 
 षष्ठी & \cbham गिरः & \cbham गिरोः & \cbham गिराम् \\ \hline 
सप्तमी & \cbham गिरि & \cbham गिरोः & \cpada गीर्षु \\ \hline 
सम्बोधन & हे गीः & हे गिरौ & हे गिरः \\ \hline 
\end{supertabular} 
\end{center} 

 \begin{center} 
 \begin{supertabular}{|c|c|c|c|}\hline 
 \multicolumn {4}{|c|}{\cellcolor{blue!10}  \nextTable \hypertarget{पुर् (स्त्री॰) City}{पुर् (स्त्री॰) City}}  \\ \hline  
 & एक॰ & द्वि॰ & बहु॰  \\ \hline 
 प्रथमा & \csarva पूः & \csarva पुरौ & \csarva पुरः \\ \hline 
 द्वितीया & \csarva पुरम् & \csarva पुरौ & \cbham पुरः \\ \hline 
 तृतीया & \cbham पुरा & \cpada पूर्भ्याम् & \cpada पूर्भिः \\ \hline 
 चतुर्थी & \cbham पुरे & \cpada पूर्भ्याम् & \cpada पूर्भ्यः \\ \hline 
 पञ्चमी & \cbham पुरः & \cpada पूर्भ्याम् & \cpada पूर्भ्यः \\ \hline 
 षष्ठी & \cbham पुरः & \cbham पुरोः & \cbham पुराम् \\ \hline 
सप्तमी & \cbham पुरि & \cbham पुरोः & \cpada पूर्षु \\ \hline 
सम्बोधन & हे पूः & हे पुरौ & हे पुरः \\ \hline 
\end{supertabular} 
\end{center} 

 \begin{center} 
 \begin{supertabular}{|c|c|c|c|}\hline 
 \multicolumn {4}{|c|}{\cellcolor{blue!10}  \nextTable \hypertarget{दिव् (स्त्री॰) Sky}{दिव् (स्त्री॰) Sky}}  \\ \hline  
 & एक॰ & द्वि॰ & बहु॰  \\ \hline 
 प्रथमा & \csarva द्यौः & \csarva दिवौ & \csarva दिवः \\ \hline 
 द्वितीया & \csarva दिवम् & \csarva दिवौ & \cbham दिवः \\ \hline 
 तृतीया & \cbham दिवा & \cpada द्युभ्याम् & \cpada द्युभिः \\ \hline 
 चतुर्थी & \cbham दिवे & \cpada द्युभ्याम् & \cpada द्युभ्यः \\ \hline 
 पञ्चमी & \cbham दिवः & \cpada द्युभ्याम् & \cpada द्युभ्यः \\ \hline 
 षष्ठी & \cbham दिवः & \cbham दिवोः & \cbham दिवाम् \\ \hline 
सप्तमी & \cbham दिवि & \cbham दिवोः & \cpada द्युषु \\ \hline 
सम्बोधन & हे द्यौः & हे दिवौ & हे दिवः \\ \hline 
\end{supertabular} 
\end{center} 

 \begin{center} 
 \begin{supertabular}{|c|c|c|c|}\hline 
 \multicolumn {4}{|c|}{\cellcolor{blue!10}  \nextTable \hypertarget{दिश् (स्त्री॰) Direction}{दिश् (स्त्री॰) Direction}}  \\ \hline  
 & एक॰ & द्वि॰ & बहु॰  \\ \hline 
 प्रथमा & \csarva दिक्-ग् & \csarva दिशौ & \csarva दिशः \\ \hline 
 द्वितीया & \csarva दिशम् & \csarva दिशौ & \cbham दिशः \\ \hline 
 तृतीया & \cbham दिशा & \cpada दिग्भ्याम् & \cpada दिग्भिः \\ \hline 
 चतुर्थी & \cbham दिशे & \cpada दिग्भ्याम् & \cpada दिग्भ्यः \\ \hline 
 पञ्चमी & \cbham दिशः & \cpada दिग्भ्याम् & \cpada दिग्भ्यः \\ \hline 
 षष्ठी & \cbham दिशः & \cbham दिशोः & \cbham दिशाम् \\ \hline 
सप्तमी & \cbham दिशि & \cbham दिशोः & \cpada दिक्षु \\ \hline 
सम्बोधन & हे दिक्-ग् & हे दिशौ & हे दिशः \\ \hline 
\end{supertabular} 
\end{center} 

\begin{center} 
 \begin{supertabular}{|c|c|c|c|}\hline 
 \multicolumn {4}{|c|}{\cellcolor{blue!10}  \nextTable \hypertarget{विश् (पु॰) House}{विश् (पु॰) House}}  \\ \hline  
 & एक॰ & द्वि॰ & बहु॰  \\ \hline 
 प्रथमा & \csarva विट् & \csarva विशौ & \csarva विशः \\ \hline 
 द्वितीया & \csarva विशम् & \csarva विशौ & \cbham विशः \\ \hline 
 तृतीया & \cbham विशा & \cpada विड्भ्याम् & \cpada विड्भिः \\ \hline 
 चतुर्थी & \cbham विशे & \cpada विड्भ्याम् & \cpada विड्भ्यः \\ \hline 
 पञ्चमी & \cbham विशः & \cpada विड्भ्याम् & \cpada विड्भ्यः \\ \hline 
 षष्ठी & \cbham विशः & \cbham विशोः & \cbham विशाम् \\ \hline 
सप्तमी & \cbham विशि & \cbham विशोः & \cpada विट्सु \\ \hline 
सम्बोधन & हे विट् & हे विशौ & हे विशः \\ \hline 
\end{supertabular} 
\end{center} 

\small
 \begin{center} 
 \begin{supertabular}{|c|c|c|c|}\hline 
 \multicolumn {4}{|c|}{\cellcolor{blue!10}  \nextTable \hypertarget{भवादृश् (पु॰) Like You}{भवादृश् (पु॰) Like You}}  \\ \hline  
 & एक॰ & द्वि॰ & बहु॰  \\ \hline 
 प्रथमा & \csarva भवादृक् & \csarva भवादृशौ & \csarva भवादृशः \\ \hline 
 द्वितीया & \csarva भवादृशम् & \csarva भवादृशौ & \cbham भवादृशः \\ \hline 
 तृतीया & \cbham भवादृशा & \cpada भवादृग्भ्याम् & \cpada भवादृग्भिः \\ \hline 
 चतुर्थी & \cbham भवादृशे & \cpada भवादृग्भ्याम् & \cpada भवादृग्भ्यः \\ \hline 
 पञ्चमी & \cbham भवादृशः & \cpada भवादृग्भ्याम् & \cpada भवादृग्भ्यः \\ \hline 
 षष्ठी & \cbham भवादृशः & \cbham भवादृशोः & \cbham भवादृशाम् \\ \hline 
सप्तमी & \cbham भवादृशि & \cbham भवादृशोः & \cpada भवादृक्षु \\ \hline 
सम्बोधन & हे भवादृक् & हे भवादृशौ & हे भवादृशः \\ \hline 
\end{supertabular} 
\end{center} 

 \begin{center} 
 \begin{supertabular}{|c|c|c|c|}\hline 
 \multicolumn {4}{|c|}{\cellcolor{blue!10}  \nextTable \hypertarget{भवादृश् (नपु॰) Like You}{भवादृश् (नपु॰) Like You}}  \\ \hline  
 & एक॰ & द्वि॰ & बहु॰  \\ \hline 
 प्रथमा & \cpada भवादृक् & \cbham भवादृशी & \csarva भवादृंशि \\ \hline 
 द्वितीया & \cbham भवादृक् & \cbham भवादृशी & \csarva भवादृंशि \\ \hline 
 तृतीया & \cbham भवादृशा & \cpada भवादृग्भ्याम् & \cpada भवादृग्भिः \\ \hline 
 चतुर्थी & \cbham भवादृशे & \cpada भवादृग्भ्याम् & \cpada भवादृग्भ्यः \\ \hline 
 पञ्चमी & \cbham भवादृशः & \cpada भवादृग्भ्याम् & \cpada भवादृग्भ्यः \\ \hline 
 षष्ठी & \cbham भवादृशः & \cbham भवादृशोः & \cbham भवादृशाम् \\ \hline 
सप्तमी & \cbham भवादृशि & \cbham भवादृशोः & \cpada भवादृक्षु \\ \hline 
सम्बोधन & हे भवादृक् & हे भवादृशौ & हे भवादृशः \\ \hline 
\end{supertabular} 
\end{center} 


 \begin{center} 
 \begin{supertabular}{|c|c|c|c|}\hline 
 \multicolumn {4}{|c|}{\cellcolor{blue!10}  \nextTable \hypertarget{निश् (स्त्री॰) Night}{निश् (स्त्री॰) Night}}  \\ \hline  
 & एक॰ & द्वि॰ & बहु॰  \\ \hline 
 प्रथमा & \csarva & \csarva  & \csarva  \\ \hline 
 द्वितीया & \csarva  & \csarva  & \cbham निशः \\ \hline 
 तृतीया & \cbham निशा & \cpada निज्-निड्-भ्याम् & \cpada निज्-निड्-भिः \\ \hline 
 चतुर्थी & \cbham निशे & \cpada निज्-निड्-भ्याम् & \cpada निज्-निड्-भ्यः \\ \hline 
 पञ्चमी & \cbham निशः & \cpada निज्-निड्-भ्याम् & \cpada निज्-निड्-भ्यः \\ \hline 
 षष्ठी & \cbham निशः & \cbham निशोः & \cbham निशाम् \\ \hline 
सप्तमी & \cbham निशि & \cbham निशोः & \cpada निच्-निट्-निट्त्-सु \\ \hline 
सम्बोधन &  &  &  \\ \hline 
\end{supertabular} 
\end{center} 

\normalsize
\begin{center} 
 \begin{supertabular}{|c|c|c|c|}\hline 
 \multicolumn {4}{|c|}{\cellcolor{blue!10}  \nextTable \hypertarget{तादृश् (नपु॰) Like That}{तादृश् (नपु॰) Like That}}  \\ \hline  
 & एक॰ & द्वि॰ & बहु॰  \\ \hline 
 प्रथमा & \cpada तादृक् & \cbham तादृशी & \csarva तादृंशि \\ \hline 
 द्वितीया & \cbham तादृक् & \cbham तादृशी & \csarva तादृंशि \\ \hline 
सम्बोधन & हे तादृक् & हे तादृशी & हे तादृंशि \\ \hline 
 \multicolumn{4}{|c|}{शेषं तादृश्-पुंवत्} \\ \hline 
\end{supertabular} 
\end{center} 

 \begin{center} 
 \begin{supertabular}{|c|c|c|c|}\hline 
 \multicolumn {4}{|c|}{\cellcolor{blue!10}  \nextTable \hypertarget{सुत्विष् (नपु॰) Very Lustrous}{सुत्विष् (नपु॰) Very Lustrous}}  \\ \hline  
 & एक॰ & द्वि॰ & बहु॰  \\ \hline 
 प्रथमा & \cpada सुत्विट् & \cbham सुत्विषी & \csarva सुत्वींषि \\ \hline 
 द्वितीया & \cbham सुत्विट् & \cbham सुत्विषी & \csarva सुत्वींषि \\ \hline 
सम्बोधन & हे सुत्विट् & हे सुत्विषी & हे सुत्वींषि \\ \hline 
 \multicolumn{4}{|c|}{शेषं द्विष् शब्दवत् } \\ \hline 
\end{supertabular} 
\end{center} 

 \begin{center} 
 \begin{supertabular}{|c|c|c|c|}\hline 
 \multicolumn {4}{|c|}{\cellcolor{blue!10}  \nextTable \hypertarget{द्विष् (पु॰) Dislike}{द्विष् (पु॰) Dislike}}  \\ \hline  
 & एक॰ & द्वि॰ & बहु॰  \\ \hline 
 प्रथमा & \csarva द्विट् & \csarva द्विषौ & \csarva द्विषः \\ \hline 
 द्वितीया & \csarva द्विषम् & \csarva द्विषौ & \cbham द्विषः \\ \hline 
 तृतीया & \cbham द्विषा & \cpada द्विड्भ्याम् & \cpada द्विड्भिः \\ \hline 
 चतुर्थी & \cbham द्विषे & \cpada द्विड्भ्याम् & \cpada द्विड्भ्यः \\ \hline 
 पञ्चमी & \cbham द्विषः & \cpada द्विड्भ्याम् & \cpada द्विड्भ्यः \\ \hline 
 षष्ठी & \cbham द्विषः & \cbham द्विषोः & \cbham द्विषाम् \\ \hline 
सप्तमी & \cbham द्विषि & \cbham द्विषोः & \cpada द्विट्सु \\ \hline 
सम्बोधन & हे द्विट् & हे द्विषौ & हे द्विषः \\ \hline 
\end{supertabular} 
\end{center} 

%\medskip

 \begin{center} 
 \begin{supertabular}{|c|c|c|c|}\hline 
 \multicolumn {4}{|c|}{\cellcolor{blue!10}  \nextTable \hypertarget{प्रावृष् (स्त्री॰) Rain}{प्रावृष् (स्त्री॰) Rain}}  \\ \hline  
 & एक॰ & द्वि॰ & बहु॰  \\ \hline 
 प्रथमा & \csarva प्रावृट्-ड् & \csarva प्रावृषौ & \csarva प्रावृषः \\ \hline 
 द्वितीया & \csarva प्रावृषम् & \csarva प्रावृषौ & \cbham प्रावृषः \\ \hline 
 तृतीया & \cbham प्रावृषा & \cpada प्रावृड्भ्याम् & \cpada प्रावृड्भिः \\ \hline 
 चतुर्थी & \cbham प्रावृषे & \cpada प्रावृड्भ्याम् & \cpada प्रावृड्भ्यः \\ \hline 
 पञ्चमी & \cbham प्रावृषः & \cpada प्रावृड्भ्याम् & \cpada प्रावृड्भ्यः \\ \hline 
 षष्ठी & \cbham प्रावृषः & \cbham प्रावृषोः & \cbham प्रावृषाम् \\ \hline 
सप्तमी & \cbham प्रावृषि & \cbham प्रावृषोः & \cpada प्रावृट्सु \\ \hline 
सम्बोधन & हे प्रावृट्-ड् & हे प्रावृषौ & हे प्रावृषः \\ \hline 
\end{supertabular} 
\end{center} 

 \begin{center} 
 \begin{supertabular}{|c|c|c|c|}\hline 
 \multicolumn {4}{|c|}{\cellcolor{blue!10}  \nextTable \hypertarget{विद्वस् (पु॰) Learned}{विद्वस् (पु॰) Learned}}  \\ \hline  
 & एक॰ & द्वि॰ & बहु॰  \\ \hline 
 प्रथमा & \csarva विद्वान् & \csarva विद्वांसौ & \csarva विद्वांसः \\ \hline 
 द्वितीया & \csarva विद्वांसम् & \csarva विद्वांसौ & \cbham विदुषः \\ \hline 
 तृतीया & \cbham विदुषा & \cpada विद्वद्भ्याम् & \cpada विद्वद्भिः \\ \hline 
 चतुर्थी & \cbham विदुषे & \cpada विद्वद्भ्याम् & \cpada विद्वद्भ्यः \\ \hline 
 पञ्चमी & \cbham विदुषः & \cpada विद्वद्भ्याम् & \cpada विद्वद्भ्यः \\ \hline 
 षष्ठी & \cbham विदुषः & \cbham विदुषोः & \cbham विदुषाम् \\ \hline 
सप्तमी & \cbham विदुषि & \cbham विदुषोः & \cpada विद्वत्सु \\ \hline 
सम्बोधन & हे विद्वन् & हे विद्वांसौ & हे विद्वांसः \\ \hline 
\end{supertabular} 
\end{center} 
%

 \begin{center} 
 \begin{supertabular}{|c|c|c|c|}\hline 
 \multicolumn {4}{|c|}{\cellcolor{blue!10}  \nextTable \hypertarget{कनीयस् (पु॰) Younger}{कनीयस् (पु॰) Younger}}  \\ \hline  
 & एक॰ & द्वि॰ & बहु॰  \\ \hline 
 प्रथमा & \csarva कनीयान् & \csarva कनीयांसौ & \csarva कनीयांसः \\ \hline 
 द्वितीया & \csarva कनीयांसम् & \csarva कनीयांसौ & \cbham कनीयसः \\ \hline 
 तृतीया & \cbham कनीयसा & \cpada कनीयोभ्याम् & \cpada कनीयोभिः \\ \hline 
 चतुर्थी & \cbham कनीयसे & \cpada कनीयोभ्याम् & \cpada कनीयोभ्यः \\ \hline 
 पञ्चमी & \cbham कनीयसः & \cpada कनीयोभ्याम् & \cpada कनीयोभ्यः \\ \hline 
 षष्ठी & \cbham कनीयसः & \cbham कनीयसोः & \cbham कनीयसाम् \\ \hline 
सप्तमी & \cbham कनीयसि & \cbham कनीयसोः & \cpada कनीयःसु \\ \hline 
सम्बोधन & हे कनीयन् & हे कनीयांसौ & हे कनीयांसः \\ \hline 
\end{supertabular} 
\end{center} 

%\clearpage

\begin{center} 
 \begin{supertabular}{|c|c|c|c|}\hline 
 \multicolumn {4}{|c|}{\cellcolor{blue!10}  \nextTable \hypertarget{अप्सरस् (स्त्री॰) Fairy}{अप्सरस् (स्त्री॰) Fairy}}  \\ \hline  
 & एक॰ & द्वि॰ & बहु॰  \\ \hline 
 प्रथमा & \csarva अप्सराः & \csarva अप्सरसौ & \csarva अप्सरसः \\ \hline 
 द्वितीया & \csarva अप्सरसम् & \csarva अप्सरसौ & \cbham अप्सरसः \\ \hline 
 तृतीया & \cbham अप्सरसा & \cpada अप्सरोभ्याम् & \cpada अप्सरोभिः \\ \hline 
 चतुर्थी & \cbham अप्सरसे & \cpada अप्सरोभ्याम् & \cpada अप्सरोभ्यः \\ \hline 
 पञ्चमी & \cbham अप्सरसः & \cpada अप्सरोभ्याम् & \cpada अप्सरोभ्यः \\ \hline 
 षष्ठी & \cbham अप्सरसः & \cbham अप्सरसोः & \cbham अप्सरसाम् \\ \hline 
सप्तमी & \cbham अप्सरसि & \cbham अप्सरसोः & \cpada अप्सरःसु \\ \hline 
सम्बोधन & हे अप्सरः & हे अप्सरसौ & हे अप्सरसः \\ \hline 
\end{supertabular} 
\end{center} 

 \begin{center} 
 \begin{supertabular}{|c|c|c|c|}\hline 
 \multicolumn {4}{|c|}{\cellcolor{blue!10}  \nextTable \hypertarget{वेधस् (पु॰) Brahma}{वेधस् (पु॰) Brahma}}  \\ \hline  
 & एक॰ & द्वि॰ & बहु॰  \\ \hline 
 प्रथमा & \csarva वेधाः & \csarva वेधसौ & \csarva वेधसः \\ \hline 
 द्वितीया & \csarva वेधसम् & \csarva वेधसौ & \cbham वेधसः \\ \hline 
 तृतीया & \cbham वेधसा & \cpada वेधोभ्याम् & \cpada वेधोभिः \\ \hline 
 चतुर्थी & \cbham वेधसे & \cpada वेधोभ्याम् & \cpada वेधोभ्यः \\ \hline 
 पञ्चमी & \cbham वेधसः & \cpada वेधोभ्याम् & \cpada वेधोभ्यः \\ \hline 
 षष्ठी & \cbham वेधसः & \cbham वेधसोः & \cbham वेधसाम् \\ \hline 
सप्तमी & \cbham वेधसि & \cbham वेधसोः & \cpada वेधस्सु \\ \hline 
सम्बोधन & हे वेधः & हे वेधसौ & हे वेधसः \\ \hline 
\end{supertabular} 
\end{center} 

\begin{center} 
 \begin{supertabular}{|c|c|c|c|}\hline 
 \multicolumn {4}{|c|}{\cellcolor{blue!10}  \nextTable \hypertarget{चन्द्रमस् (पु॰) Moon}{चन्द्रमस् (पु॰) Moon}}  \\ \hline  
 & एक॰ & द्वि॰ & बहु॰  \\ \hline 
 प्रथमा & \csarva चन्द्रमाः & \csarva चन्द्रमसौ & \csarva चन्द्रमसः \\ \hline 
 द्वितीया & \csarva चन्द्रमसम् & \csarva चन्द्रमसौ & \cbham चन्द्रमसः \\ \hline 
 तृतीया & \cbham चन्द्रमसा & \cpada चन्द्रमोभ्याम् & \cpada चन्द्रमोभिः \\ \hline 
 चतुर्थी & \cbham चन्द्रमसे & \cpada चन्द्रमोभ्याम् & \cpada चन्द्रमोभ्यः \\ \hline 
 पञ्चमी & \cbham चन्द्रमसः & \cpada चन्द्रमोभ्याम् & \cpada चन्द्रमोभ्यः \\ \hline 
 षष्ठी & \cbham चन्द्रमसः & \cbham चन्द्रमसोः & \cbham चन्द्रमसाम् \\ \hline 
सप्तमी & \cbham चन्द्रमसि & \cbham चन्द्रमसोः & \cpada चन्द्रमःसु-स्सु \\ \hline 
सम्बोधन & हे चन्द्रमः & हे चन्द्रमसौ & हे चन्द्रमसः \\ \hline 
\end{supertabular} 
\end{center} 
 
 \begin{center} 
 \begin{supertabular}{|c|c|c|c|}\hline 
 \multicolumn {4}{|c|}{\cellcolor{blue!10}  \nextTable \hypertarget{मास् (पु॰) Month}{मास् (पु॰) Month}}  \\ \hline  
 & एक॰ & द्वि॰ & बहु॰  \\ \hline 
 प्रथमा & \csarva & \csarva &  \csarva \\ \hline 
 द्वितीया & \csarva & \csarva & \cbham मासः \\ \hline 
 तृतीया & \cbham मासा & \cpada माभ्याम् & \cpada माभिः \\ \hline 
 चतुर्थी & \cbham मासे & \cpada माभ्याम् & \cpada माभ्यः \\ \hline 
 पञ्चमी & \cbham मासः & \cpada माभ्याम् & \cpada माभ्यः \\ \hline 
 षष्ठी & \cbham मासः & \cbham मासोः & \cbham मासाम् \\ \hline 
सप्तमी & \cbham मासि & \cbham मासोः & \cpada माःसु-मास्सु \\ \hline 
सम्बोधन &  &  &  \\ \hline 
\end{supertabular} 
\end{center} 

\medskip

 \begin{center} 
 \begin{supertabular}{|c|c|c|c|}\hline 
 \multicolumn {4}{|c|}{\cellcolor{blue!10}  \nextTable \hypertarget{श्रेयस् (पु॰) Better}{श्रेयस् (पु॰) Better}}  \\ \hline  
 & एक॰ & द्वि॰ & बहु॰  \\ \hline 
 प्रथमा & \csarva श्रेयान् & \csarva श्रेयांसौ & \csarva श्रेयांसः \\ \hline 
 द्वितीया & \csarva श्रेयांसम् & \csarva श्रेयांसौ & \cbham श्रेयसः \\ \hline 
 तृतीया & \cbham श्रेयसा & \cpada श्रेयोभ्याम् & \cpada श्रेयोभिः \\ \hline 
 चतुर्थी & \cbham श्रेयसे & \cpada श्रेयोभ्याम् & \cpada श्रेयोभ्यः \\ \hline 
 पञ्चमी & \cbham श्रेयसः & \cpada श्रेयोभ्याम् & \cpada श्रेयोभ्यः \\ \hline 
 षष्ठी & \cbham श्रेयसः & \cbham श्रेयसोः & \cbham श्रेयसाम् \\ \hline 
सप्तमी & \cbham श्रेयसि & \cbham श्रेयसोः & \cpada श्रेयस्सु \\ \hline 
सम्बोधन & हे श्रेयन् & हे श्रेयांसौ & हे श्रेयांसः \\ \hline 
\end{supertabular} 
\end{center} 

 \begin{center} 
 \begin{supertabular}{|c|c|c|c|}\hline 
 \multicolumn {4}{|c|}{\cellcolor{blue!10}  \nextTable \hypertarget{पुंस् (पु॰) Person}{पुंस् (पु॰) Person}}  \\ \hline  
 & एक॰ & द्वि॰ & बहु॰  \\ \hline 
 प्रथमा & \csarva पुमान् & \csarva पुमांसौ & \csarva पुमांसः \\ \hline 
 द्वितीया & \csarva पुमांसम् & \csarva पुमांसौ & \cbham पुंसः \\ \hline 
 तृतीया & \cbham पुंसा & \cpada पुंभ्याम् & \cpada पुंभिः \\ \hline 
 चतुर्थी & \cbham पुंसे & \cpada पुंभ्याम् & \cpada पुंभ्यः \\ \hline 
 पञ्चमी & \cbham पुंसः & \cpada पुंभ्याम् & \cpada पुंभ्यः \\ \hline 
 षष्ठी & \cbham पुंसः & \cbham पुंसोः & \cbham पुंसाम् \\ \hline 
सप्तमी & \cbham पुंसि & \cbham पुंसोः & \cpada पुंसु \\ \hline 
सम्बोधन & हे पुमन् & हे पुमांसौ & हे पुमांसः \\ \hline 
\end{supertabular} 
\end{center} 

 \begin{center} 
 \begin{supertabular}{|c|c|c|c|}\hline 
 \multicolumn {4}{|c|}{\cellcolor{blue!10}  \nextTable \hypertarget{दोस् (पु॰) Arm}{दोस् (पु॰) Arm}}  \\ \hline  
 & एक॰ & द्वि॰ & बहु॰  \\ \hline 
 प्रथमा & \csarva दोः & \csarva दोषौ & \csarva दोषः \\ \hline 
 द्वितीया & \csarva दोषम् & \csarva दोषौ & \cbham दोषः \\ \hline 
 तृतीया & \cbham दोषा & \cpada दोर्भ्याम् & \cpada दोर्भिः \\ \hline 
 चतुर्थी & \cbham दोषे & \cpada दोर्भ्याम् & \cpada दोर्भ्यः \\ \hline 
 पञ्चमी & \cbham दोषः & \cpada दोर्भ्याम् & \cpada दोर्भ्यः \\ \hline 
 षष्ठी & \cbham दोषः & \cbham दोषोः & \cbham दोषाम् \\ \hline 
सप्तमी & \cbham दोषि & \cbham दोषोः & \cpada दोष्षु \\ \hline 
सम्बोधन & हे दोः & हे दोषौ & हे दोषः \\ \hline 
\end{supertabular} 
\end{center} 

 \begin{center} 
 \begin{supertabular}{|c|c|c|c|}\hline 
 \multicolumn {4}{|c|}{\cellcolor{blue!10}  \nextTable \hypertarget{भास् (स्त्री॰) Light}{भास् (स्त्री॰) Light}}  \\ \hline  
 & एक॰ & द्वि॰ & बहु॰  \\ \hline 
 प्रथमा & \csarva भाः & \csarva भासौ & \csarva भासः \\ \hline 
 द्वितीया & \csarva भासम् & \csarva भासौ & \cbham भासः \\ \hline 
 तृतीया & \cbham भासा & \cpada भाभ्याम् & \cpada भाभिः \\ \hline 
 चतुर्थी & \cbham भासे & \cpada भाभ्याम् & \cpada भाभ्यः \\ \hline 
 पञ्चमी & \cbham भासः & \cpada भाभ्याम् & \cpada भाभ्यः \\ \hline 
 षष्ठी & \cbham भासः & \cbham भासोः & \cbham भासाम् \\ \hline 
सप्तमी & \cbham भासि & \cbham भासोः & \cpada भास्सु \\ \hline 
सम्बोधन & हे भाः & हे भासौ & हे भासः \\ \hline 
\end{supertabular} 
\end{center} 

 \begin{center} 
 \begin{supertabular}{|c|c|c|c|}\hline 
 \multicolumn {4}{|c|}{\cellcolor{blue!10}  \nextTable \hypertarget{आशिस् (स्त्री॰) Wish}{आशिस् (स्त्री॰) Wish}}  \\ \hline  
 & एक॰ & द्वि॰ & बहु॰  \\ \hline 
 प्रथमा & \csarva आशीः & \csarva आशिषौ & \csarva आशिषः \\ \hline 
 द्वितीया & \csarva आशिषम् & \csarva आशिषौ & \cbham आशिषः \\ \hline 
 तृतीया & \cbham आशिषा & \cpada आशिर्भ्याम् & \cpada आशिर्भिः \\ \hline 
 चतुर्थी & \cbham आशिषे & \cpada आशिर्भ्याम् & \cpada आशिर्भ्यः \\ \hline 
 पञ्चमी & \cbham आशिषः & \cpada आशिर्भ्याम् & \cpada आशिर्भ्यः \\ \hline 
 षष्ठी & \cbham आशिषः & \cbham आशिषोः & \cbham आशिषाम् \\ \hline 
सप्तमी & \cbham आशिषि & \cbham आशिषोः & \cpada आशिष्षु \\ \hline 
सम्बोधन & हे आशीः & हे आशिषौ & हे आशिषः \\ \hline 
\end{supertabular} 
\end{center} 

 \begin{center} 
 \begin{supertabular}{|c|c|c|c|}\hline 
 \multicolumn {4}{|c|}{\cellcolor{blue!10}  \nextTable \hypertarget{हविस् (नपु॰) Oblation}{हविस् (नपु॰) Oblation}}  \\ \hline  
 & एक॰ & द्वि॰ & बहु॰  \\ \hline 
 प्रथमा & \cpada हविः & \cbham हविषी & \csarva हवींषि \\ \hline 
 द्वितीया & \cbham हविः & \cbham हविषी & \csarva हवींषि \\ \hline 
 तृतीया & \cbham हविषा & \cpada हविर्भ्याम् & \cpada हविर्भिः \\ \hline 
 चतुर्थी & \cbham हरिषे & \cpada हविर्भ्याम् & \cpada हविर्भ्यः \\ \hline 
 पञ्चमी & \cbham हविषः & \cpada हविर्भ्याम् & \cpada हविर्भ्यः \\ \hline 
 षष्ठी & \cbham हविषः & \cbham हविषोः & \cbham हविषाम् \\ \hline 
सप्तमी & \cbham हविषि & \cbham हविषोः & \cpada हविष्षु \\ \hline 
सम्बोधन & हे हविः & हे हविषी & हे हवींषि \\ \hline 
\end{supertabular} 
\end{center} 

 \begin{center} 
 \begin{supertabular}{|c|c|c|c|}\hline 
 \multicolumn {4}{|c|}{\cellcolor{blue!10}  \nextTable \hypertarget{वपुस् (नपु॰) Body}{वपुस् (नपु॰) Body}}  \\ \hline  
 & एक॰ & द्वि॰ & बहु॰  \\ \hline 
 प्रथमा & \cpada वपुः & \cbham वपुषी & \csarva वपूंषि \\ \hline 
 द्वितीया & \cbham वपुः & \cbham वपुषी & \csarva वपूंषि \\ \hline 
 तृतीया & \cbham वपुषा & \cpada वपुर्भ्याम् & \cpada वपुर्भिः \\ \hline 
 चतुर्थी & \cbham वपुषे & \cpada वपुर्भ्याम् & \cpada वपुर्भ्यः \\ \hline 
 पञ्चमी & \cbham वपुषः & \cpada वपुर्भ्याम् & \cpada वपुर्भ्यः \\ \hline 
 षष्ठी & \cbham वपुषः & \cbham वपुषोः & \cbham वपुषाम् \\ \hline 
सप्तमी & \cbham वपुषि & \cbham वपुषोः & \cpada वपुष्षु \\ \hline 
सम्बोधन & हे वपुः & हे वपुषी & हे वपूंषि \\ \hline 
\end{supertabular} 
\end{center} 

 \begin{center} 
 \begin{supertabular}{|c|c|c|c|}\hline 
 \multicolumn {4}{|c|}{\cellcolor{blue!10}  \nextTable \hypertarget{तस्थिवस् (नपु॰) Standing}{तस्थिवस् (नपु॰) Standing}}  \\ \hline  
 & एक॰ & द्वि॰ & बहु॰  \\ \hline 
 प्रथमा & \cpada तस्थिवस् & \cbham तस्थुषी & \csarva तस्थिवांसि \\ \hline 
 द्वितीया & \cbham तस्थिवस् & \cbham तस्थुषी & \csarva तस्थिवांसि \\ \hline 
सम्बोधन & हे तस्थिवस् & हे तस्थुषी & हे तस्थिवांसि \\ \hline 
 \multicolumn{4}{|c|}{ शेषं विद्वस् शब्दवत् } \\ \hline 
\end{supertabular} 
\end{center} 

\begin{center} 
 \begin{supertabular}{|c|c|c|c|}\hline 
 \multicolumn {4}{|c|}{\cellcolor{blue!10}  \nextTable \hypertarget{लिह् (पु॰) Licking}{लिह् (पु॰) Licking}}  \\ \hline  
 & एक॰ & द्वि॰ & बहु॰  \\ \hline 
 प्रथमा & \csarva लिट् & \csarva लिहौ & \csarva लिहः \\ \hline 
 द्वितीया & \csarva लिहम् & \csarva लिहौ & \cbham लिहः \\ \hline 
 तृतीया & \cbham लिहा & \cpada लिड्भ्याम् & \cpada लिड्भिः \\ \hline 
 चतुर्थी & \cbham लिहे & \cpada लिड्भ्याम् & \cpada लिड्भ्यः \\ \hline 
 पञ्चमी & \cbham लिहः & \cpada लिड्भ्याम् & \cpada लिड्भ्यः \\ \hline 
 षष्ठी & \cbham लिहः & \cbham लिहोः & \cbham लिहाम् \\ \hline 
सप्तमी & \cbham लिहि & \cbham लिहोः & \cpada लिट्सु \\ \hline 
सम्बोधन & हे लित् & हे लिहौ & हे लिहः \\ \hline 
\end{supertabular} 
\end{center} 
%\clearpage

 \begin{center} 
 \begin{supertabular}{|c|c|c|c|}\hline 
 \multicolumn {4}{|c|}{\cellcolor{blue!10}  \nextTable \hypertarget{उपानह् (स्त्री॰) Shoe}{उपानह् (स्त्री॰) Shoe}}  \\ \hline  
 & एक॰ & द्वि॰ & बहु॰  \\ \hline 
 प्रथमा & \csarva उपानत् & \csarva उपानहौ & \csarva उपानहः \\ \hline 
 द्वितीया & \csarva उपानहम् & \csarva उपानहौ & \cbham उपानहः \\ \hline 
 तृतीया & \cbham उपानहा & \cpada उपानद्भ्याम् & \cpada उपानद्भिः \\ \hline 
 चतुर्थी & \cbham उपानहे & \cpada उपानद्भ्याम् & \cpada उपानद्भ्यः \\ \hline 
 पञ्चमी & \cbham उपानहः & \cpada उपानद्भ्याम् & \cpada उपानद्भ्यः \\ \hline 
 षष्ठी & \cbham उपानहः & \cbham उपानहोः & \cbham उपानहाम् \\ \hline 
सप्तमी & \cbham उपानहि & \cbham उपानहोः & \cpada उपानत्सु \\ \hline 
सम्बोधन & हे उपानत् & हे उपानहौ & हे उपानहः \\ \hline 
\end{supertabular} 
\end{center} 

\begin{center} 
 \begin{supertabular}{|c|c|c|c|}\hline 
 \multicolumn {4}{|c|}{\cellcolor{blue!10}  \nextTable \hypertarget{अम्भोरुह् (नपु॰) Lotus}{अम्भोरुह् (नपु॰) Lotus}}  \\ \hline  
 & एक॰ & द्वि॰ & बहु॰  \\ \hline 
 प्रथमा & \cpada अम्भोरुट् & \cbham अम्भोरुही & \csarva अम्भोरुंहि \\ \hline 
 द्वितीया & \cbham अम्भोरुट् & \cbham अम्भोरुही & \csarva अम्भोरुंहि \\ \hline 
सम्बोधन & हे अम्भोरुट् & हे अम्भोरुही & हे अम्भोरुंहि \\ \hline 
 \multicolumn{4}{|c|}{ शेषं लिह् शब्दवत् } \\ \hline 
\end{supertabular} 
\end{center} 

\begin{center} 
 \begin{supertabular}{|c|c|c|c|}\hline 
 \multicolumn {4}{|c|}{\cellcolor{blue!10}  \nextTable \hypertarget{अस्मद्}{अस्मद्}}  \\ \hline  
 & एक॰ & द्वि॰ & बहु॰  \\ \hline 
 प्रथमा & \csarva अहम् & \csarva आवाम् & \csarva वयम् \\ \hline 
 द्वितीया & \csarva माम्-मा & \csarva आवाम्-नौ & \cbham अस्मान्-नः \\ \hline 
 तृतीया & \cbham मया & \cpada आवाभ्याम् & \cpada अस्माभिः \\ \hline 
 चतुर्थी & \cbham मह्यम्-मे & \cpada आवाभ्याम्-नौ & \cpada अस्मभ्यम्-नः \\ \hline 
 पञ्चमी & \cbham मत् & \cpada आवाभ्याम् & \cpada अस्मत् \\ \hline 
 षष्ठी & \cbham मम-मे & \cbham आवयोः-नौ & \cbham अस्माकम्-नः \\ \hline 
सप्तमी & \cbham मयि & \cbham आवयोः & \cpada अस्मासु \\ \hline 
%सम्बोधन &  &  &  \\ \hline 
\end{supertabular} 
\end{center} 

 \begin{center} 
 \begin{supertabular}{|c|c|c|c|}\hline 
 \multicolumn {4}{|c|}{\cellcolor{blue!10}  \nextTable \hypertarget{युष्मद्}{युष्मद्}}  \\ \hline  
 & एक॰ & द्वि॰ & बहु॰  \\ \hline 
 प्रथमा & \csarva त्वम् & \csarva युवाम् & \csarva यूयम् \\ \hline 
 द्वितीया & \csarva त्वाम्-त्वा & \csarva युवाम्-वाम् & \cbham युष्मान्-वः \\ \hline 
 तृतीया & \cbham त्वया & \cpada युवाभ्याम् & \cpada युष्माभिः \\ \hline 
 चतुर्थी & \cbham तुभ्यम्-ते & \cpada युवाभ्याम् & \cpada युष्मभ्यम्-वः \\ \hline 
 पञ्चमी & \cbham त्वत् & \cpada युवाभ्याम् & \cpada युष्मत् \\ \hline 
 षष्ठी & \cbham तव-ते & \cbham युवयोः-वाम् & \cbham युष्माकम्-वः \\ \hline 
सप्तमी & \cbham त्वयि & \cbham युवयोः & \cpada युष्मासु \\ \hline 
%सम्बोधन &  &  &  \\ \hline 
\end{supertabular} 
\end{center} 

\begin{center}
%\begin{table}
	%\centering
	%सुप् प्रत्ययाः\\[1.5ex]
		\begin{supertabular}{|c|c|c|c|}\hline
\multicolumn {4}{|c|}{\cellcolor{blue!10}  \nextTable \hypertarget{तद् (पु॰) That}{तत् (पु॰) That}}  \\ \hline 
		& एक॰ & द्वि॰ & बहु॰ \\ \hline
प्रथमा & \csarva सः  & \csarva तौ & \csarva ते\\ \hline
द्वितीया & \csarva तम्   & \csarva तौ & \cbham तान् \\ \hline
तृतीया & \cbham तेन  & \cpada ताभ्याम् & \cpada तैः\\ \hline
चतुर्थी & \cbham तस्मै  & \cpada ताभ्याम् & \cpada तेभ्यः\\ \hline
पञ्चमी & \cbham तस्मात् & \cpada ताभ्याम् & \cpada तेभ्यः\\ \hline
षष्ठी & \cbham तस्य  & \cbham तयोः  & \cbham तेषाम्\\ \hline
सप्तमी & \cbham तस्मिन्   & \cbham तयोः  & \cpada तेषु \\ \hline	
%सम्बोधन &  &  & \\ \hline				
		\end{supertabular}
		\end{center}		
		%\caption{चकारान्तः पुंलिङ्गः प्राञ्च् शब्दः}
%\end{table}
%

\begin{center}
%\begin{table}
	%\centering
	%सुप् प्रत्ययाः\\[1.5ex]
		\begin{supertabular}{|c|c|c|c|}\hline
\multicolumn {4}{|c|}{\cellcolor{blue!10}  \nextTable \hypertarget{तद् (स्त्री॰) That}{तत् (स्त्री॰) That}}  \\ \hline 
		& एक॰ & द्वि॰ & बहु॰ \\ \hline
प्रथमा & \csarva सा  & \csarva ते & \csarva ताः\\ \hline
द्वितीया & \csarva ताम्  & \csarva ते & \cbham ताः \\ \hline
तृतीया & \cbham तया  & \cpada ताभ्याम् & \cpada ताभिः \\ \hline
चतुर्थी & \cbham तस्यै  & \cpada ताभ्याम् & \cpada ताभ्यः\\ \hline
पञ्चमी & \cbham तस्याः & \cpada ताभ्याम् & \cpada ताभ्यः\\ \hline
षष्ठी & \cbham तस्याः & \cbham तयोः  & \cbham तासाम्\\ \hline
सप्तमी & \cbham तस्याम्   & \cbham तयोः  & \cpada तासु \\ \hline			
%सम्बोधन &  &  & \\ \hline				
		\end{supertabular}
		\end{center}		
		%\caption{चकारान्तः पुंलिङ्गः प्राञ्च् शब्दः}
%\end{table}

\begin{center} 
 \begin{supertabular}{|c|c|c|c|}\hline 
 \multicolumn {4}{|c|}{\cellcolor{blue!10}  \nextTable \hypertarget{तद् (नपु॰) That}{तत् (नपु॰) That}}  \\ \hline  
 & एक॰ & द्वि॰ & बहु॰  \\ \hline 
 प्रथमा & \csarva तत् & \csarva ते & \csarva तानि \\ \hline 
 द्वितीया & \csarva तत् & \csarva ते & \cbham तानि \\ \hline 
 तृतीया & \cbham तेन & \cpada ताभ्याम् & \cpada तैः \\ \hline 
 चतुर्थी & \cbham तस्मै & \cpada ताभ्याम् & \cpada तेभ्यः \\ \hline 
 पञ्चमी & \cbham तस्मात् & \cpada ताभ्याम् & \cpada तेभ्यः \\ \hline 
 षष्ठी & \cbham तस्य & \cbham तयोः & \cbham तेषाम् \\ \hline 
सप्तमी & \cbham तस्मिन् & \cbham तयोः & \cpada तेषु \\ \hline 
%सम्बोधन &  &  &  \\ \hline 
\end{supertabular} 
\end{center} 
%

 \begin{center} 
 \begin{supertabular}{|c|c|c|c|}\hline 
 \multicolumn {4}{|c|}{\cellcolor{blue!10}  \nextTable \hypertarget{इदम् (पु॰) This}{इदम् (पु॰) This}}  \\ \hline  
 & एक॰ & द्वि॰ & बहु॰  \\ \hline 
 प्रथमा & \csarva अयम् & \csarva इमौ & \csarva इमे \\ \hline 
 द्वितीया & \csarva इमम्-एनम् & \csarva इमौ & \cbham इमान् \\ \hline 
 तृतीया & \cbham अनेन-एनेन & \cpada आभ्याम् & \cpada एभिः \\ \hline 
 चतुर्थी & \cbham अस्मै & \cpada आभ्यायम् & \cpada एभ्यः \\ \hline 
 पञ्चमी & \cbham अस्मात् & \cpada आभ्याम् & \cpada एभ्यः \\ \hline 
 षष्ठी & \cbham अस्य & \cbham अनयोः-एनयोः & \cbham एषाम् \\ \hline 
सप्तमी & \cbham अस्मिन् & \cbham अनयोः-एनयोः & \cpada एषु \\ \hline 
%सम्बोधन &  & &   \\ \hline 
\end{supertabular} 
\end{center} 

\begin{center} 
 \begin{supertabular}{|c|c|c|c|}\hline 
 \multicolumn {4}{|c|}{\cellcolor{blue!10}  \nextTable \hypertarget{इदम् (स्त्री॰) This}{इदम् (स्त्री॰) This}}  \\ \hline  
 & एक॰ & द्वि॰ & बहु॰  \\ \hline 
 प्रथमा & \csarva इयम् & \csarva इमे & \csarva इमाः \\ \hline 
 द्वितीया & \csarva इमाम्-एनाम् & \csarva इमे-एने & \cbham इमाः-एनाः \\ \hline 
 तृतीया & \cbham अनया-एनया & \cpada आभ्याम् & \cpada आभिः \\ \hline 
 चतुर्थी & \cbham अस्यै & \cpada आभ्याम् & \cpada आभ्यः \\ \hline 
 पञ्चमी & \cbham अस्याः & \cpada आभ्याम् & \cpada आभ्यः \\ \hline 
 षष्ठी & \cbham अस्याः & \cbham अनयोः-एनयोः & \cbham आसाम् \\ \hline 
सप्तमी & \cbham अस्याम् & \cbham अनयोः-एनयोः & \cpada आसु \\ \hline 
%सम्बोधन &  &  &   \\ \hline 
\end{supertabular} 
\end{center} 

 \begin{center} 
 \begin{supertabular}{|c|c|c|c|}\hline 
 \multicolumn {4}{|c|}{\cellcolor{blue!10}  \nextTable \hypertarget{इदम् (नपु॰) This}{इदम् (नपु॰) This}}  \\ \hline  
 & एक॰ & द्वि॰ & बहु॰  \\ \hline 
 प्रथमा & \cpada इदम् & \cbham इमे & \csarva इमानि \\ \hline 
 द्वितीया & \cbham इदम्-एनत् & \cbham इमे-एने & \csarva इमानि-एनानि \\ \hline 
%सम्बोधन &  & &  \\ \hline 
 \multicolumn{4}{|c|}{ शेषं \hyperlink{इदम् (पु॰) This}{इदम् (पु॰)} शब्दवत् } \\ \hline 
\end{supertabular} 

\begin{center} 
 \begin{supertabular}{|c|c|c|c|}\hline 
 \multicolumn {4}{|c|}{\cellcolor{blue!10}  \nextTable \hypertarget{एतद् (पु॰) This}{एतत् (पु॰) This}}  \\ \hline  
 & एक॰ & द्वि॰ & बहु॰  \\ \hline 
 प्रथमा & \csarva एषः & \csarva एतौ & \csarva एते \\ \hline 
 द्वितीया & \csarva एतम्-एनम् & \csarva एतौ-एनौ & \cbham एतान्-एनान् \\ \hline 
 तृतीया & \cbham एतेन & \cpada एताभ्याम् & \cpada एतैः \\ \hline 
 चतुर्थी & \cbham एतस्मै & \cpada एताभ्याम् & \cpada एतेभ्यः \\ \hline 
 पञ्चमी & \cbham एतस्मात् & \cpada एताभ्याम् & \cpada एतेभ्यः \\ \hline 
 षष्ठी & \cbham एतस्य & \cbham एतयोः-एनयोः & \cbham एतेषाम् \\ \hline 
सप्तमी & \cbham एतस्मिन् & \cbham एतयोः-एनयोः & \cpada एतेषु \\ \hline 
%सम्बोधन & & &   \\ \hline 
\end{supertabular} 
\end{center} 

\begin{center} 
 \begin{supertabular}{|c|c|c|c|}\hline 
 \multicolumn {4}{|c|}{\cellcolor{blue!10}  \nextTable \hypertarget{एतद् (स्त्री॰) This}{एतद् (स्त्री॰) This}}  \\ \hline  
 & एक॰ & द्वि॰ & बहु॰  \\ \hline 
 प्रथमा & \csarva एषा & \csarva एते & \csarva एताः \\ \hline 
 द्वितीया & \csarva एताम्-एनाम् & \csarva एते-एने & \cbham एताः-एनाः \\ \hline 
 तृतीया & \cbham एतया-एनया & \cpada एताभ्याम् & \cpada एताभिः \\ \hline 
 चतुर्थी & \cbham एतस्यै & \cpada एताभ्याम् & \cpada एताभ्यः \\ \hline 
 पञ्चमी & \cbham एतस्याः & \cpada एताभ्याम् & \cpada एताभ्यः \\ \hline 
 षष्ठी & \cbham एतस्याः & \cbham एतयोः-एनयोः & \cbham एतासाम् \\ \hline 
सप्तमी & \cbham एतस्याम् & \cbham एतयोः-एनयोः & \cpada एतासु \\ \hline 
%सम्बोधन &  &  &  \\ \hline 
\end{supertabular} 
\end{center} 

 \begin{center} 
 \begin{supertabular}{|c|c|c|c|}\hline 
 \multicolumn {4}{|c|}{\cellcolor{blue!10}  \nextTable \hypertarget{एतद् (नपु॰) This}{एतद् (नपु॰) This}}  \\ \hline  
 & एक॰ & द्वि॰ & बहु॰  \\ \hline 
 प्रथमा & \cpada एतत् & \cbham एते & \csarva एतानि \\ \hline 
 द्वितीया & \cbham एतत्-एनत् & \cbham एते-एने & \csarva एतानि-एनानि \\ \hline 
%सम्बोधन & &  & \\ \hline 
 \multicolumn{4}{|c|}{ शेषं एतद्-पुंवत् शब्दवत् } \\ \hline 
\end{supertabular} 
\end{center} 

\begin{center} 
 \begin{supertabular}{|c|c|c|c|}\hline 
 \multicolumn {4}{|c|}{\cellcolor{blue!10}  \nextTable \hypertarget{अदस् (पु॰) That}{अदस् (पु॰) That}}  \\ \hline  
 & एक॰ & द्वि॰ & बहु॰  \\ \hline 
 प्रथमा & \csarva असौ & \csarva अमू & \csarva अमी \\ \hline 
 द्वितीया & \csarva अमुम् & \csarva अमू & \cbham अमून् \\ \hline 
 तृतीया & \cbham अमुना & \cpada अमूभ्याम् & \cpada अमीभिः \\ \hline 
 चतुर्थी & \cbham अमुष्मै & \cpada अमूभ्याम् & \cpada अमीभ्यः \\ \hline 
 पञ्चमी & \cbham अमुष्मात् & \cpada अमूभ्याम् & \cpada अमीभ्यः \\ \hline 
 षष्ठी & \cbham अमुष्य & \cbham अमुयोः & \cbham अमीषाम् \\ \hline 
सप्तमी & \cbham अमुष्मिन् & \cbham अमुयोः & \cpada अमीषु \\ \hline 
%सम्बोधन &  & &   \\ \hline 
\end{supertabular} 
\end{center} 

\end{center} 
 \begin{center} 
 \begin{supertabular}{|c|c|c|c|}\hline 
 \multicolumn {4}{|c|}{\cellcolor{blue!10}  \nextTable \hypertarget{अदस् (स्त्री॰) This}{अदस् (स्त्री॰) This}}  \\ \hline  
 & एक॰ & द्वि॰ & बहु॰  \\ \hline 
 प्रथमा & \csarva असौ & \csarva अमू & \csarva अमूः \\ \hline 
 द्वितीया & \csarva अमूम् & \csarva अमू & \cbham अमूः \\ \hline 
 तृतीया & \cbham अमुया & \cpada अमूभ्याम् & \cpada अमूभिः \\ \hline 
 चतुर्थी & \cbham अमुष्यै & \cpada अमूभ्याम् & \cpada अमूभ्यः \\ \hline 
 पञ्चमी & \cbham अमुष्याः & \cpada अमूभ्याम् & \cpada अमूभ्यः \\ \hline 
 षष्ठी & \cbham अमुष्याः & \cbham अमुयोः & \cbham अमूषाम् \\ \hline 
सप्तमी & \cbham अमुष्याम् & \cbham अमुयोः & \cpada अमूषु \\ \hline 
%सम्बोधन &  &  &   \\ \hline 
\end{supertabular} 
\end{center} 

 \begin{center} 
 \begin{supertabular}{|c|c|c|c|}\hline 
 \multicolumn {4}{|c|}{\cellcolor{blue!10}  \nextTable \hypertarget{अदस् (नपु॰) This}{अदस् (नपु॰) This}}  \\ \hline  
 & एक॰ & द्वि॰ & बहु॰  \\ \hline 
 प्रथमा & \cpada अदः & \cbham अमू & \csarva अमूनि \\ \hline 
 द्वितीया & \cbham अदः & \cbham अमू & \csarva अमूनि \\ \hline 
%सम्बोधन &  &  &   \\ \hline 
 \multicolumn{4}{|c|}{ शेषं अदस्-पुंवत्  } \\ \hline 
\end{supertabular} 
\end{center} 
 
\begin{center} 
 \begin{supertabular}{|c|c|c|c|}\hline 
 \multicolumn {4}{|c|}{\cellcolor{blue!10}  \nextTable \hypertarget{यद् (पु॰) Who}{यत् (पु॰) Who}}  \\ \hline  
 & एक॰ & द्वि॰ & बहु॰  \\ \hline 
 प्रथमा & \csarva यः & \csarva यौ & \csarva ये \\ \hline 
 द्वितीया & \csarva यम् & \csarva यौ & \cbham यान् \\ \hline 
 तृतीया & \cbham येन & \cpada याभ्याम् & \cpada यैः \\ \hline 
 चतुर्थी & \cbham यस्मै & \cpada याभ्याम् & \cpada येभ्यः \\ \hline 
 पञ्चमी & \cbham यस्मात् & \cpada याभ्याम् & \cpada येभ्यः \\ \hline 
 षष्ठी & \cbham यस्य & \cbham ययोः & \cbham येषाम् \\ \hline 
सप्तमी & \cbham यस्मिन् & \cbham ययोः & \cpada येषु \\ \hline 
%सम्बोधन &  & &   \\ \hline 
\end{supertabular} 
\end{center} 

\begin{center} 
 \begin{supertabular}{|c|c|c|c|}\hline 
 \multicolumn {4}{|c|}{\cellcolor{blue!10}  \nextTable \hypertarget{यद् (स्त्री॰) Who}{यद् (स्त्री॰) Who}}  \\ \hline  
 & एक॰ & द्वि॰ & बहु॰  \\ \hline 
 प्रथमा & \csarva या & \csarva ये & \csarva याः \\ \hline 
 द्वितीया & \csarva याम् & \csarva ये & \cbham याः \\ \hline 
 तृतीया & \cbham यया & \cpada याभ्याम् & \cpada याभिः \\ \hline 
 चतुर्थी & \cbham यस्यै & \cpada याभ्याम् & \cpada याभ्यः \\ \hline 
 पञ्चमी & \cbham यस्याः & \cpada याभ्याम् & \cpada याभ्यः \\ \hline 
 षष्ठी & \cbham यस्याः & \cbham ययोः & \cbham यासाम् \\ \hline 
सप्तमी & \cbham यस्याम् & \cbham ययोः & \cpada यासु \\ \hline 
%सम्बोधन &  &  &   \\ \hline 
\end{supertabular} 
\end{center} 

 \begin{center} 
 \begin{supertabular}{|c|c|c|c|}\hline 
 \multicolumn {4}{|c|}{\cellcolor{blue!10}  \nextTable \hypertarget{यद् (नपु॰) Who}{यद् (नपु॰) Who}}  \\ \hline  
 & एक॰ & द्वि॰ & बहु॰  \\ \hline 
 प्रथमा & \cpada यत् & \cbham ये & \csarva यानि \\ \hline 
 द्वितीया & \cbham यत् & \cbham ये & \csarva यानि \\ \hline 
%सम्बोधन & &  &  \\ \hline 
 \multicolumn{4}{|c|}{ शेषं यद्-पुंवत् } \\ \hline 
\end{supertabular} 
\end{center} 

 \begin{center} 
 \begin{supertabular}{|c|c|c|c|}\hline 
 \multicolumn {4}{|c|}{\cellcolor{blue!10}  \nextTable \hypertarget{किम् (पु॰) What}{किम् (पु॰) What}}  \\ \hline  
 & एक॰ & द्वि॰ & बहु॰  \\ \hline 
 प्रथमा & \csarva कः & \csarva कौ & \csarva के \\ \hline 
 द्वितीया & \csarva कम् & \csarva कौ & \cbham कान् \\ \hline 
 तृतीया & \cbham केन & \cpada काभ्काम् & \cpada कैः \\ \hline 
 चतुर्थी & \cbham कस्मै & \cpada काभ्याम् & \cpada केभ्यः \\ \hline 
 पञ्चमी & \cbham कस्मात् & \cpada काभ्याम् & \cpada केभ्यः \\ \hline 
 षष्ठी & \cbham कस्य & \cbham कयोः & \cbham केषाम् \\ \hline 
सप्तमी & \cbham कस्मिन् & \cbham कयोः & \cpada केषु \\ \hline 
%सम्बोधन &  &  &  \\ \hline 
\end{supertabular} 
\end{center} 

 \begin{center} 
 \begin{supertabular}{|c|c|c|c|}\hline 
 \multicolumn {4}{|c|}{\cellcolor{blue!10}  \nextTable \hypertarget{किम् (स्त्री॰) What}{किम् (स्त्री॰) What}}  \\ \hline  
 & एक॰ & द्वि॰ & बहु॰  \\ \hline 
 प्रथमा & \csarva का & \csarva के & \csarva काः \\ \hline 
 द्वितीया & \csarva काम् & \csarva के & \cbham काः \\ \hline 
 तृतीया & \cbham कया & \cpada काभ्याम् & \cpada काभिः \\ \hline 
 चतुर्थी & \cbham कस्यै & \cpada काभ्याम् & \cpada काभ्यः \\ \hline 
 पञ्चमी & \cbham कस्याः & \cpada काभ्याम् & \cpada काभ्यः \\ \hline 
 षष्ठी & \cbham कस्याः & \cbham कयोः & \cbham कासाम् \\ \hline 
सप्तमी & \cbham कस्याम् & \cbham कयोः & \cpada कासु \\ \hline 
%सम्बोधन & & &  \\ \hline 
\end{supertabular} 
\end{center} 
 
\begin{center} 
 \begin{supertabular}{|c|c|c|c|}\hline 
 \multicolumn {4}{|c|}{\cellcolor{blue!10}  \nextTable \hypertarget{किम् (नपु॰) What}{किम् (नपु॰) What}}  \\ \hline  
 & एक॰ & द्वि॰ & बहु॰  \\ \hline 
 प्रथमा & \cpada किम् & \cbham के & \csarva कानि \\ \hline 
 द्वितीया & \cbham किम् & \cbham के & \csarva कानि \\ \hline 
%सम्बोधन & &  &   \\ \hline 
 \multicolumn{4}{|c|}{ शेषं किम्-पुंवत् शब्दवत् } \\ \hline 
\end{supertabular} 
\end{center} 

 \begin{center} 
 \begin{supertabular}{|c|c|c|c|}\hline 
 \multicolumn {4}{|c|}{\cellcolor{blue!10}  \nextTable \hypertarget{सर्व (पु॰) All}{सर्व (पु॰) All}}  \\ \hline  
 & एक॰ & द्वि॰ & बहु॰  \\ \hline 
 प्रथमा & \csarva सर्वः & \csarva सर्वौ & \csarva सर्वे \\ \hline 
 द्वितीया & \csarva सर्वम् & \csarva सर्वौ & \cbham सर्वान् \\ \hline 
 तृतीया & \cbham सर्वेण & \cpada सर्वाभ्याम् & \cpada सर्वैः \\ \hline 
 चतुर्थी & \cbham सर्वस्मै & \cpada सर्वाभ्याम् & \cpada सर्वेभ्यः \\ \hline 
 पञ्चमी & \cbham सर्वस्मात् & \cpada सर्वाभ्याम् & \cpada सर्वेभ्यः \\ \hline 
 षष्ठी & \cbham सर्वस्य & \cbham सर्वयोः & \cbham सर्वेषाम् \\ \hline 
सप्तमी & \cbham सर्वस्मिन् & \cbham सर्वयोः & \cpada सर्वेषु \\ \hline 
सम्बोधन & हे सर्व & हे सर्वौ & हे सर्वे \\ \hline 
\end{supertabular} 
\end{center} 

\begin{center} 
 \begin{supertabular}{|c|c|c|c|}\hline 
 \multicolumn {4}{|c|}{\cellcolor{blue!10}  \nextTable \hypertarget{सर्वा (स्त्री॰) All}{सर्वा (स्त्री॰) All}}  \\ \hline  
 & एक॰ & द्वि॰ & बहु॰  \\ \hline 
 प्रथमा & \csarva सर्वा & \csarva सर्वे & \csarva सर्वाः \\ \hline 
 द्वितीया & \csarva सर्वाम् & \csarva सर्वे & \cbham सर्वाः \\ \hline 
 तृतीया & \cbham सर्वया & \cpada सर्वाभ्याम् & \cpada सर्वाभिः \\ \hline 
 चतुर्थी & \cbham सर्वस्यै & \cpada सर्वाभ्याम् & \cpada सर्वाभ्यः \\ \hline 
 पञ्चमी & \cbham सर्वस्याः & \cpada सर्वाभ्याम् & \cpada सर्वाभ्यः \\ \hline 
 षष्ठी & \cbham सर्वस्याः & \cbham सर्वयोः & \cbham सर्वासाम् \\ \hline 
सप्तमी & \cbham सर्वस्याम् & \cbham सर्वयोः & \cpada सर्वासु \\ \hline 
सम्बोधन & हे सर्वे & हे सर्वे & हे सर्वाः \\ \hline 
\end{supertabular} 
\end{center} 

 \begin{center} 
 \begin{supertabular}{|c|c|c|c|}\hline 
 \multicolumn {4}{|c|}{\cellcolor{blue!10}  \nextTable \hypertarget{सर्व (नपु॰) All}{सर्व (नपु॰) All}}  \\ \hline  
 & एक॰ & द्वि॰ & बहु॰  \\ \hline 
 प्रथमा & \cpada सर्वम् & \cbham सर्वे & \csarva सर्वाणि \\ \hline 
 द्वितीया & \cbham सर्वम् & \cbham सर्वे & \csarva सर्वाणि \\ \hline 
सम्बोधन & हे सर्व & हे सर्वे & हे सर्वाणि \\ \hline 
 \multicolumn{4}{|c|}{ शेषं सर्व-पुंवत्} \\ \hline 
\end{supertabular} 
\end{center} 

 \begin{center} 
 \begin{supertabular}{|c|c|c|c|}\hline 
 \multicolumn {4}{|c|}{\cellcolor{blue!10}  \nextTable \hypertarget{भवत् (पु॰) You}{भवत् (पु॰) You}}  \\ \hline  
 & एक॰ & द्वि॰ & बहु॰  \\ \hline 
 प्रथमा & \csarva भवान् & \csarva भवन्तौ & \csarva भवन्तः \\ \hline 
 द्वितीया & \csarva भवन्तम् & \csarva भवन्तौ & \cbham भवतः \\ \hline 
 तृतीया & \cbham भवता & \cpada भवद्भ्याम् & \cpada भवद्भिः \\ \hline 
 चतुर्थी & \cbham भवते & \cpada भवद्भ्याम् & \cpada भवद्भ्यः \\ \hline 
 पञ्चमी & \cbham भवतः & \cpada भवद्भ्याम् & \cpada भवद्भ्यः \\ \hline 
 षष्ठी & \cbham भवतः & \cbham भवतोः & \cbham भवताम् \\ \hline 
सप्तमी & \cbham भवति & \cbham भवतोः & \cpada भवत्सु \\ \hline 
सम्बोधन & हे भवन् & हे भवन्तौ & हे भवन्तः \\ \hline 
\end{supertabular} 
\end{center} 

 \begin{center} 
 \begin{supertabular}{|c|c|c|c|}\hline 
 \multicolumn {4}{|c|}{\cellcolor{blue!10}  \nextTable \hypertarget{भवत् (स्त्री॰) You}{भवत् (स्त्री॰) You}}  \\ \hline  
 & एक॰ & द्वि॰ & बहु॰  \\ \hline 
 प्रथमा & \cpada भवती & \cbham भवत्यौ & \csarva भवत्यः \\ \hline 
 द्वितीया & \cbham भवतीम् & \cbham भवत्यौ & \csarva भवतीः \\ \hline 
सम्बोधन & हे भवति & हे भवती & हे भवतयः \\ \hline 
 \multicolumn{4}{|c|}{ शेषं नदी शब्दवत् } \\ \hline 
\end{supertabular} 
\end{center} 

 \begin{center} 
 \begin{supertabular}{|c|c|c|c|}\hline 
 \multicolumn {4}{|c|}{\cellcolor{blue!10}  \nextTable \hypertarget{भवत् (नपु॰) You}{भवत् (नपु॰) You}}  \\ \hline  
 & एक॰ & द्वि॰ & बहु॰  \\ \hline 
 प्रथमा & \cpada भवत् & \cbham भवती & \csarva भवन्ति \\ \hline 
 द्वितीया & \cbham भवत् & \cbham भवती & \csarva भवन्ति \\ \hline 
सम्बोधन & हे भवत् & हे भवती & हे भवन्ति \\ \hline 
 \multicolumn{4}{|c|}{ शेषं जगत् शब्दवत् } \\ \hline 
\end{supertabular} 
\end{center} 
 
 \begin{center} 
 \begin{supertabular}{|c|c|c|c|}\hline 
 \multicolumn {4}{|c|}{\cellcolor{blue!10}  \nextTable \hypertarget{अन्यत् (पु॰) Another}{अन्यत् (पु॰) Another}}  \\ \hline  
 & एक॰ & द्वि॰ & बहु॰  \\ \hline 
 प्रथमा & \csarva अन्यः & \csarva अन्यौ & \csarva अन्ये \\ \hline 
 द्वितीया & \csarva अन्यम् & \csarva अन्यौ & \cbham अन्यान् \\ \hline 
 तृतीया & \cbham अन्येन & \cpada अन्याभ्याम् & \cpada अन्यैः \\ \hline 
 चतुर्थी & \cbham अन्यस्मै & \cpada अन्याभ्याम् & \cpada अन्येभ्यः \\ \hline 
 पञ्चमी & \cbham अन्यस्मात् & \cpada अन्याभ्याम् & \cpada अन्येभ्यः \\ \hline 
 षष्ठी & \cbham अन्यस्य & \cbham अन्ययोः & \cbham अन्येषाम् \\ \hline 
सप्तमी & \cbham अन्यस्मिन् & \cbham अन्ययोः & \cpada अन्येषु \\ \hline 
%सम्बोधन &  & &  \\ \hline 
\end{supertabular} 
\end{center} 

\begin{center} 
 \begin{supertabular}{|c|c|c|c|}\hline 
 \multicolumn {4}{|c|}{\cellcolor{blue!10}  \nextTable \hypertarget{अन्यत् (स्त्री॰) Another}{अन्यत् (स्त्री॰) Another}}  \\ \hline  
 & एक॰ & द्वि॰ & बहु॰  \\ \hline 
 प्रथमा & \csarva अन्या & \csarva अन्ये & \csarva अन्याः \\ \hline 
 द्वितीया & \csarva अन्याम् & \csarva अन्ये & \cbham अन्याः \\ \hline 
 तृतीया & \cbham अन्यया & \cpada अन्याभ्याम् & \cpada अन्याभिः \\ \hline 
 चतुर्थी & \cbham अन्यस्यै & \cpada अन्याभ्याम् & \cpada अन्याभ्यः \\ \hline 
 पञ्चमी & \cbham अन्यस्याः & \cpada अन्याभ्याम् & \cpada अन्याभ्यः \\ \hline 
 षष्ठी & \cbham अन्यस्याः & \cbham अन्ययोः & \cbham अन्यासाम् \\ \hline 
सप्तमी & \cbham अन्यस्याम् & \cbham अन्ययोः & \cpada अन्यासु \\ \hline 
%सम्बोधन &  &  &  \\ \hline 
\end{supertabular} 
\end{center} 

 \begin{center} 
 \begin{supertabular}{|c|c|c|c|}\hline 
 \multicolumn {4}{|c|}{\cellcolor{blue!10}  \nextTable \hypertarget{अन्यत् (नपु॰) Other}{अन्यत् (नपु॰) Other}}  \\ \hline  
 & एक॰ & द्वि॰ & बहु॰  \\ \hline 
 प्रथमा & \cpada अन्यत् & \cbham अन्ये & \csarva अन्यानि \\ \hline 
 द्वितीया & \cbham अन्यत् & \cbham अन्ये & \csarva अन्यानि \\ \hline 
%सम्बोधन & &  &  \\ \hline 
 \multicolumn{4}{|c|}{ शेषं अन्यत्-पुंवत् शब्दवत् } \\ \hline 
\end{supertabular} 
\end{center} 
 
 \begin{center} 
 \begin{supertabular}{|c|c|c|c|}\hline 
 \multicolumn {4}{|c|}{\cellcolor{blue!10}  \nextTable \hypertarget{पूर्व (पु॰) Before}{पूर्व (पु॰) Before}}  \\ \hline  
 & एक॰ & द्वि॰ & बहु॰  \\ \hline 
 प्रथमा & \csarva पूर्वः & \csarva पूर्वौ & \csarva पूर्वे-पूर्वाः \\ \hline 
 द्वितीया & \csarva पूर्वम् & \csarva पूर्वौ & \cbham पूर्वान् \\ \hline 
 तृतीया & \cbham पूर्वेण & \cpada पूर्वाभ्याम् & \cpada पूर्वैः \\ \hline 
 चतुर्थी & \cbham पूर्वस्मै & \cpada पूर्वाभ्याम् & \cpada पूर्वेभ्यः \\ \hline 
 पञ्चमी & \cbham पूर्वस्मात्-पूर्वात् & \cpada पूर्वाभ्याम् & \cpada पूर्वेभ्यः \\ \hline 
 षष्ठी & \cbham पूर्वस्य & \cbham पूर्वयोः & \cbham पूर्वेषाम् \\ \hline 
सप्तमी & \cbham पूर्वस्मिन्-पूर्वे & \cbham पूर्वयोः & \cpada पूर्वेषु \\ \hline 
सम्बोधन & हे पूर्व & हे पूर्वौ & हे पूर्वे-पूर्वाः \\ \hline 
\end{supertabular} 
\end{center} 

\begin{center} 
 \begin{supertabular}{|c|c|c|c|}\hline 
 \multicolumn {4}{|c|}{\cellcolor{blue!10}  \nextTable \hypertarget{पूर्व (स्त्री॰) Before}{पूर्व (स्त्री॰) Before}}  \\ \hline  
 & एक॰ & द्वि॰ & बहु॰  \\ \hline 
 प्रथमा & \csarva पूर्वा & \csarva पूर्वे & \csarva पूर्वाः \\ \hline 
 द्वितीया & \csarva पूर्वाम् & \csarva पूर्वे & \cbham पूर्वाः \\ \hline 
 तृतीया & \cbham पूर्वया & \cpada पूर्वाभ्याम् & \cpada पूर्वाभिः \\ \hline 
 चतुर्थी & \cbham पूर्वस्यै & \cpada पूर्वाभ्याम् & \cpada पूर्वभ्यः \\ \hline 
 पञ्चमी & \cbham पूर्वस्याः & \cpada पूर्वाभ्याम् & \cpada पूर्वभ्यः \\ \hline 
 षष्ठी & \cbham पूर्वस्याः & \cbham पूर्वयोः & \cbham पूर्वासाम् \\ \hline 
सप्तमी & \cbham पूर्वस्याम् & \cbham पूर्वयोः & \cpada पूर्वासु \\ \hline 
सम्बोधन &  &  &  \\ \hline 
\end{supertabular} 
\end{center} 

 \begin{center} 
 \begin{supertabular}{|c|c|c|c|}\hline 
 \multicolumn {4}{|c|}{\cellcolor{blue!10}  \nextTable \hypertarget{पूर्व (नपु॰) Before}{पूर्व (नपु॰) Before}}  \\ \hline  
 & एक॰ & द्वि॰ & बहु॰  \\ \hline 
 प्रथमा & \cpada पूर्वम् & \cbham पूर्वे & \csarva पूर्वाणि \\ \hline 
 द्वितीया & \cbham पूर्वम् & \cbham पूर्वे & \csarva पूर्वाणि \\ \hline 
सम्बोधन & हे पूर्वम् & हे पूर्वे & हे पूर्वाणि \\ \hline 
 \multicolumn{4}{|c|}{ शेषं पूर्व-पुंवत्} \\ \hline 
\end{supertabular} 
\end{center} 

\begin{center} 
 \begin{supertabular}{|c|c|c|c|}\hline 
 \multicolumn {4}{|c|}{\cellcolor{blue!10}  \nextTable \hypertarget{त्वत् (पु॰) Other}{त्वत् (पु॰) Other}}  \\ \hline  
 & एक॰ & द्वि॰ & बहु॰  \\ \hline 
 प्रथमा & \cpada त्वत् & \cbham त्वतौ & \csarva त्वतः \\ \hline 
 द्वितीया & \cbham त्वतम् & \cbham त्वतौ & \csarva त्वतः \\ \hline 
सम्बोधन & हे त्वत् & हे त्वतौ & हे त्वतः \\ \hline 
 \multicolumn{4}{|c|}{ शेषं मरुत् शब्दवत् } \\ \hline 
\end{supertabular} 
\end{center} 

 \begin{center} 
 \begin{supertabular}{|c|c|c|c|}\hline 
 \multicolumn {4}{|c|}{\cellcolor{blue!10}  \nextTable \hypertarget{त्वत् (स्त्री॰) Other}{त्वत् (स्त्री॰) Other}}  \\ \hline  
 & एक॰ & द्वि॰ & बहु॰  \\ \hline 
 प्रथमा & \cpada त्वत् & \cbham त्वतौ & \csarva त्वतः \\ \hline 
 द्वितीया & \cbham त्वतम् & \cbham त्वतौ & \csarva त्वतः \\ \hline 
सम्बोधन & हे त्वत् & हे त्वतौ & हे त्वतः \\ \hline 
 \multicolumn{4}{|c|}{ शेषं सरित् शब्दवत् } \\ \hline 
\end{supertabular} 
\end{center} 
 
\begin{center} 
 \begin{supertabular}{|c|c|c|c|}\hline 
 \multicolumn {4}{|c|}{\cellcolor{blue!10}  \nextTable \hypertarget{त्वत् (नपु॰) Other}{त्वत् (नपु॰) Other}}  \\ \hline  
 & एक॰ & द्वि॰ & बहु॰  \\ \hline 
 प्रथमा & \cpada त्वत् & \cbham त्वती & \csarva त्वतन्ति \\ \hline 
 द्वितीया & \cbham त्वत् & \cbham त्वती & \csarva त्वतन्ति \\ \hline 
सम्बोधन & हे त्वत् & हे त्वती & हे त्वतन्ति \\ \hline 
 \multicolumn{4}{|c|}{ शेषं जगत् शब्दवत् } \\ \hline 
\end{supertabular} 
\end{center} 

 \begin{center} 
 \begin{supertabular}{|c|c|c|c|}\hline 
 \multicolumn {4}{|c|}{\cellcolor{blue!10}  \nextTable \hypertarget{उभ (द्वि॰) Both}{उभ (द्वि॰) Both}}  \\ \hline  
 & पु॰ & स्त्री॰ & नपु॰   \\ \hline 
 प्रथमा & \cnorm उभौ & \cnorm उभे &  \cnorm उभे \\ \hline 
 द्वितीया &  \cnorm उभौ & \cnorm उभे &  \cnorm उभे \\ \hline 
 तृतीया &  \cnorm उभाभ्याम् &  \cnorm उभाभ्याम् &  \cnorm उभाभ्याम् \\ \hline 
 चतुर्थी &  \cnorm उभाभ्याम् &  \cnorm उभाभ्याम् &  \cnorm उभाभ्याम् \\ \hline 
 पञ्चमी &  \cnorm उभाभ्याम् &  \cnorm उभाभ्याम् & \cnorm उभाभ्याम् \\ \hline 
 षष्ठी &  \cnorm उभयोः &  \cnorm उभयोः & \cnorm उभयोः \\ \hline 
सप्तमी &  \cnorm उभयोः &  \cnorm उभयोः & \cnorm उभयोः \\ \hline 
सम्बोधन & हे उभौ & हे उभे & हे उभे \\ \hline 
\end{supertabular} 
\end{center} 

\begin{center} 
 \begin{supertabular}{|c|c|c|c|}\hline 
 \multicolumn {4}{|c|}{\cellcolor{blue!10}  \nextTable \hypertarget{उभय (पु॰) Both}{उभय (पु॰) Both}}  \\ \hline  
 & एक॰ & द्वि॰ & बहु॰  \\ \hline 
 प्रथमा & \cnorm उभयः & \cnorm  & \cnorm उभये \\ \hline 
 द्वितीया & \cnorm उभयम् & \cnorm  & \cnorm उभयान् \\ \hline 
 तृतीया & \cnorm उभयेन & \cnorm  & \cnorm उभयैः \\ \hline 
 चतुर्थी & \cnorm उभयस्मै & \cnorm  & \cnorm उभयेभ्यः \\ \hline 
 पञ्चमी & \cnorm उभयस्मात् & \cnorm & \cnorm उभयेभ्यः \\ \hline 
 षष्ठी & \cnorm उभयस्य & \cnorm & \cnorm उभयेषाम् \\ \hline 
सप्तमी & \cnorm उभयस्मिन् & \cnorm & \cnorm उभयेषु \\ \hline 
सम्बोधन & हे उभयः &  & हे उभये \\ \hline 
\end{supertabular} 
\end{center} 

 \begin{center} 
 \begin{supertabular}{|c|c|c|c|}\hline 
 \multicolumn {4}{|c|}{\cellcolor{blue!10}  \nextTable \hypertarget{कति-यति-तति (पु॰, स्त्री॰, नपु॰)}{कति-यति-तति (पु॰, स्त्री॰, नपु॰)}}  \\ \hline  
 &  बहु॰  &  बहु॰ &  बहु॰  \\ \hline 
 प्रथमा & \cnorm  कति & \cnorm  यति & \cnorm तति \\ \hline 
 द्वितीया & \cnorm कति & \cnorm  यति & \cnorm  तति \\ \hline 
 तृतीया & \cnorm  कतिभिः & \cnorm  यतिभिः & \cnorm  ततिभिः \\ \hline 
 चतुर्थी & \cnorm  कतिभ्यः & \cnorm  यतिभ्यः & \cnorm  ततिभ्यः \\ \hline 
 पञ्चमी & \cnorm  कतिभ्यः & \cnorm यतिभ्यः & \cnorm  ततिभ्यः \\ \hline 
 षष्ठी & \cnorm  कतीनाम् & \cnorm  यतीनाम् & \cnorm  ततीनाम् \\ \hline 
सप्तमी & \cnorm  कतिषु & \cnorm  यतिषु & \cnorm  ततिषु \\ \hline 
%सम्बोधन &  &  &  \\ \hline 
\end{supertabular} 
\end{center} 
%

 \begin{center} 
 \begin{supertabular}{|c|c|c|c|}\hline 
 \multicolumn {4}{|c|}{\cellcolor{blue!10}  \nextTable \hypertarget{एक (एकवचनम्) One}{एक (एकवचनम्) One}}  \\ \hline  
 & पु॰ & स्त्री॰ & नपु॰  \\ \hline 
 प्रथमा & \cnorm एकः & \cnorm एका & \cnorm एकम् \\ \hline 
 द्वितीया & \cnorm एकम् & \cnorm एकाम् & \cnorm एकम् \\ \hline 
 तृतीया & \cnorm एकेन & \cnorm एकया & \cnorm एकेन \\ \hline 
 चतुर्थी & \cnorm एकस्मै & \cnorm एकस्यै & \cnorm एकस्मै \\ \hline 
 पञ्चमी & \cnorm एकस्मात् & \cnorm एकस्याः & \cnorm एकस्मात् \\ \hline 
 षष्ठी & \cnorm एकस्य & \cnorm एकस्याः & \cnorm एकस्य \\ \hline 
सप्तमी & \cnorm एकस्मिन् & \cnorm एकस्याम् & \cnorm एकस्मिन् \\ \hline 
%सम्बोधन &  &  &   \\ \hline 
\end{supertabular} 
\end{center} 

\medskip

 \begin{center} 
 \begin{supertabular}{|c|c|c|c|}\hline 
 \multicolumn {4}{|c|}{\cellcolor{blue!10}  \nextTable \hypertarget{द्वि (द्विवचनम्) Two}{द्वि (द्विवचनम्) Two}}  \\ \hline  
 & पु॰ & स्त्री॰ & नपु॰   \\ \hline 
 प्रथमा & \cnorm द्वौ & \cnorm द्वे & \cnorm द्वे \\ \hline 
 द्वितीया & \cnorm द्वौ & \cnorm द्वे & \cnorm द्वे \\ \hline 
 तृतीया & \cnorm द्वाभ्याम् & \cnorm द्वाभ्याम् & \cnorm द्वाभ्याम् \\ \hline 
 चतुर्थी & \cnorm द्वाभ्याम् & \cnorm द्वाभ्याम् & \cnorm द्वाभ्याम् \\ \hline 
 पञ्चमी & \cnorm द्वाभ्याम् & \cnorm द्वाभ्याम् & \cnorm द्वाभ्याम् \\ \hline 
 षष्ठी & \cnorm द्वयोः & \cnorm द्वयोः & \cnorm द्वयोः \\ \hline 
सप्तमी & \cnorm द्वयोः & \cnorm द्वयोः & \cnorm द्वयोः \\ \hline 
%सम्बोधन & हे द्वौ & हे द्वे & हे द्वे \\ \hline 
\end{supertabular} 
\end{center} 

 \begin{center} 
 \begin{supertabular}{|c|c|c|c|}\hline 
 \multicolumn {4}{|c|}{\cellcolor{blue!10}  \nextTable \hypertarget{त्रि (बहुवचनम्) Three}{त्रि (बहुवचनम्) Three}}  \\ \hline  
 & पु॰ & स्त्री॰ & नपु॰   \\ \hline 
 प्रथमा & \cnorm त्रयः & \cnorm तिस्रः & \cnorm त्रीणि \\ \hline 
 द्वितीया & \cnorm त्रिन् & \cnorm तिस्रः & \cnorm त्रीणि \\ \hline 
 तृतीया & \cnorm त्रिभिः & \cnorm तिसृभिः & \cnorm त्रिभिः \\ \hline 
 चतुर्थी & \cnorm त्रिभ्यः & \cnorm तिसृभ्यः & \cnorm त्रिभ्यः \\ \hline 
 पञ्चमी & \cnorm त्रिभ्यः & \cnorm तिसृभ्यः & \cnorm त्रिभ्यः \\ \hline 
 षष्ठी & \cnorm त्रयाणाम् & \cnorm तिसृणाम् & \cnorm त्रयाणाम् \\ \hline 
सप्तमी & \cnorm त्रिषु & \cnorm तिसृषु & \cnorm त्रिषु \\ \hline 
%सम्बोधन & हे त्रयः & हे तिस्रः & हे त्रयः \\ \hline 
\end{supertabular} 
\end{center} 
%\end{multicols}
%
%\clearpage

%\medskip

%
%\begin{multicols}{2}
%\TrickSupertabularIntoMulticols
 \begin{center} 
 \begin{supertabular}{|c|c|c|c|}\hline 
 \multicolumn {4}{|c|}{\cellcolor{blue!10}  \nextTable \hypertarget{चतुर् (बहुवचनम्) Four}{चतुर् (बहुवचनम्) Four}}  \\ \hline  
 & पु॰ & स्त्री॰ & नपु॰   \\ \hline 
 प्रथमा & \cnorm चत्वारः & \cnorm चतस्रः & \cnorm चत्वारि \\ \hline 
 द्वितीया & \cnorm चतुरः & \cnorm चतस्रः & \cnorm चत्वारि \\ \hline 
 तृतीया & \cnorm चतुर्भिः & \cnorm चतसृभिः & \cnorm चतुर्भिः \\ \hline 
 चतुर्थी & \cnorm चतुर्भ्यः & \cnorm चतसृभ्यः & \cnorm चतुर्भ्यः \\ \hline 
 पञ्चमी & \cnorm चतुर्भ्यः & \cnorm चतसृभ्यः & \cnorm चतुर्भ्यः \\ \hline 
 षष्ठी & \cnorm चतुर्णाम् & \cnorm चतसृणाम् & \cnorm चतुर्णाम् \\ \hline 
सप्तमी & \cnorm चतुर्षु & \cnorm चतसृषु & \cnorm चतुर्षु \\ \hline 
%सम्बोधन & हे चत्वारः & हे चतस्रः & हे चत्वारि \\ \hline 
\end{supertabular} 
\end{center} 

 \begin{center} 
 \begin{supertabular}{|c|c|c|c|}\hline 
 \multicolumn {4}{|c|}{\cellcolor{blue!10}  \nextTable \hypertarget{पञ्चन्-षष्-सप्तन् (पु॰, स्त्री॰, नपु॰) 5-6-7}{पञ्चन्-षष्-सप्तन् (पु॰, स्त्री॰, नपु॰) 5-6-7}}  \\ \hline  
 & बहु॰ & बहु॰ & बहु॰  \\ \hline 
 प्रथमा & \cnorm पञ्च & \cnorm षट्-षड् & \cnorm सप्त \\ \hline 
 द्वितीया & \cnorm पञ्च & \cnorm षट्-षड् & \cnorm सप्त \\ \hline 
 तृतीया & \cnorm पञ्चभिः & \cnorm षड्भिः & \cnorm सप्तभिः \\ \hline 
 चतुर्थी & \cnorm पञ्चभ्यः & \cnorm षड्भ्यः & \cnorm सप्तभ्यः \\ \hline 
 पञ्चमी & \cnorm पञ्चभ्यः & \cnorm षड्भ्यः & \cnorm सप्तभ्यः \\ \hline 
 षष्ठी & \cnorm पञ्चानाम् & \cnorm षण्णाम् & \cnorm सप्तानाम् \\ \hline 
सप्तमी & \cnorm पञ्चसु & \cnorm षट्सु & \cnorm सप्तसु \\ \hline 
%सम्बोधन & हे पञ्च & हे षट् & हे सप्त \\ \hline 
\end{supertabular} 
\end{center} 

 \begin{center} 
 \begin{supertabular}{|c|c|c|c|}\hline 
 \multicolumn {4}{|c|}{\cellcolor{blue!10}  \nextTable \hypertarget{अष्टन्-नवन्-दशन् (पु॰, स्त्री॰, नपु॰) 8-9-10}{अष्टन्-नवन्-दशन् (पु॰, स्त्री॰, नपु॰) 8-9-10}}  \\ \hline  
 & बहु॰ & बहु॰ & बहु॰  \\ \hline 
 प्रथमा & \cnorm अष्टौ-अष्ट & \cnorm नव & \cnorm दश \\ \hline 
 द्वितीया & \cnorm अष्टौ-अष्ट & \cnorm नव & \cnorm दश \\ \hline 
 तृतीया & \cnorm अष्टाभिः-अष्टभिः & \cnorm नवभिः & \cnorm दशभिः \\ \hline 
 चतुर्थी & \cnorm अष्टाभ्यः-अष्टभ्यः & \cnorm नवभ्यः & \cnorm दशभ्यः \\ \hline 
 पञ्चमी & \cnorm अष्टाभ्यः-अष्टभ्यः & \cnorm नवभ्यः & \cnorm दशभ्यः \\ \hline 
 षष्ठी & \cnorm अष्टानाम् & \cnorm नवानाम् & \cnorm दशानाम् \\ \hline 
सप्तमी & \cnorm अष्टासु-अष्टसु & \cnorm नवसु & \cnorm दशसु \\ \hline 
%सम्बोधन & हे अष्टौ-अष्ट & हे नव & हे दश \\ \hline 
\end{supertabular} 
\end{center}  

 \begin{center} 
 \begin{supertabular}{|c|c|c|c|}\hline 
 \multicolumn {4}{|c|}{\cellcolor{blue!10}  \nextTable \hypertarget{विंशति-त्रिंशत्-चात्वारिंशत् (स्त्री॰) 20-30-40}{विंशति-त्रिंशत्-चात्वारिंशत् (स्त्री॰) 20-30-40}}  \\ \hline  
 & एक॰ & एक॰ & एक॰  \\ \hline 
 प्रथमा & \cnorm विंशतिः & \cnorm त्रिंशत् & \cnorm चत्वारिंशत् \\ \hline 
 द्वितीया & \cnorm विंशतिम् & \cnorm त्रिंशतम् & \cnorm चत्वारिंशतम् \\ \hline 
 तृतीया & \cnorm विंशत्या & \cnorm त्रिंशता & \cnorm चत्वारिंशता \\ \hline 
 चतुर्थी & \cnorm विंशत्यै-विंशतये & \cnorm त्रिंशते & \cnorm चत्वारिंशते \\ \hline 
 पञ्चमी & \cnorm विंशत्यै-विंशतेः & \cnorm त्रिंशतः & \cnorm चत्वारिंशतः \\ \hline 
 षष्ठी & \cnorm विंशत्यै-विंशतये & \cnorm त्रिंशतः & \cnorm चत्वारिंशतः \\ \hline 
सप्तमी & \cnorm विंशत्याम्-विंशतौ & \cnorm त्रिंशति & \cnorm चत्वारिंशति \\ \hline 
%सम्बोधन &  &  &   \\ \hline 
\end{supertabular} 
\end{center} 

 \begin{center} 
 \begin{supertabular}{|c|c|c|c|}\hline 
 \multicolumn {4}{|c|}{\cellcolor{blue!10}  \nextTable \hypertarget{पञ्चाशत्-षष्ठि-सप्तति (स्त्री॰) 50-60-70}{पञ्चाशत्-षष्ठि-सप्तति (स्त्री॰) 50-60-70}}  \\ \hline  
 & एक॰ & एक॰ & एक॰  \\ \hline 
 प्रथमा & \cnorm पञ्चाशत् & \cnorm षष्ठिः & \cnorm सप्ततिः \\ \hline 
 द्वितीया & \cnorm पञ्चाशतम् & \cnorm षष्ठिम् & \cnorm सप्ततिम् \\ \hline 
 तृतीया & \cnorm पञ्चाशता & \cnorm षष्ठ्या & \cnorm सप्तत्या \\ \hline 
 चतुर्थी & \cnorm पञ्चाशते & \cnorm षष्ठ्यै-षष्ठये & \cnorm सप्तत्यै-सप्ततये \\ \hline 
 पञ्चमी & \cnorm पञ्चाशतः & \cnorm षष्ठ्याः-षष्ठेः & \cnorm सप्तत्याः-सप्ततेः \\ \hline 
 षष्ठी & \cnorm पञ्चाशतः & \cnorm षष्ठ्याः-षष्ठेः & \cnorm सप्तत्याः-सप्ततेः \\ \hline 
सप्तमी & \cnorm पञ्चाशति & \cnorm षष्ठ्याम्-षष्ठौ & \cnorm सप्तत्याम्-सप्ततौ \\ \hline 
%सम्बोधन &  &  &   \\ \hline 
\end{supertabular} 
\end{center} 

%\clearpage

 \begin{center} 
 \begin{supertabular}{|c|c|c|c|}\hline 
 \multicolumn {4}{|c|}{\cellcolor{blue!10}  \nextTable \hypertarget{अशीति-नवति (स्त्री॰), शतम् (नपु॰) 80-90-100}{अशीति-नवति (स्त्री॰), शतम् (नपु॰) 80-90-100}}  \\ \hline  
 & एक॰ & एक॰ & एक॰  \\ \hline 
 प्रथमा & \cnorm अशीतिः & \cnorm नवतिः & \cnorm शतम् \\ \hline 
 द्वितीया & \cnorm अशीतिम् & \cnorm नवतिम् & \cnorm शतम् \\ \hline 
 तृतीया & \cnorm अशीत्या & \cnorm नवत्या & \cnorm शतेन \\ \hline 
 चतुर्थी & \cnorm अशीत्यै-अशीतये & \cnorm नवत्यै-नवतये & \cnorm शताय \\ \hline 
 पञ्चमी & \cnorm अशीत्याः-अशीतेः & \cnorm नवत्याः-नवतेः & \cnorm शतात् \\ \hline 
 षष्ठी & \cnorm अशीत्या-अशीतेः & \cnorm नवत्याः-नवतेः & \cnorm शतस्य \\ \hline 
सप्तमी & \cnorm अशीत्याम्-अशीतौ & \cnorm नवत्याम्-नवतौ & \cnorm शते \\ \hline 
%सम्बोधन &  &  &   \\ \hline 
\end{supertabular} 
\end{center} 

सहस्त्रम् ($10^{3}$), अयुतम् ($10^{4}$), लक्षम् ($10^{5}$), प्रयुतम् ($10^{6}$), कोटिः (स्त्री॰) ($10^{7}$), अर्बुदम् ($10^{8}$), अब्जम् ($10^{9}$), खर्बम्  ($10^{10}$), निखर्बम् ($10^{11}$), महापद्मम् ($10^{12}$), शङ्कुः (पु॰) ($10^{13}$), जलधि (नपु॰)  ($10^{14}$), अन्त्यम्  ($10^{15}$), मध्यम्  ($10^{16}$), परार्धम् ($10^{17}$)|
%सहस्त्रम् (१०००), अयुतम् (१०,०००), लक्षम् (१००,०००), प्रयुतम् (१,०००,०००), कोटिः (स्त्री॰) (१०,०००,०००), अर्बुदम् (१००,०००,०००), अब्जम् (१,०००,०००,०००), खर्बम्  (१०,०००,०००,०००), निखर्बम् (१००,०००,०००,०००), महापद्मम् (१,०००,०००,०००,०००), शङ्कुः (पु॰) (१०,०००,०००,०००,०००), जलधि (नपु॰)  (१००,०००,०००,०००,०००), अन्त्यम्  (१,०००,०००,०००,०००,०००), मध्यम्  (१०,०००,०००,०००,०००,०००), परार्धम् (१००,०००,०००,०००,०००,०००)|

\begin{center} 
 \begin{supertabular}{|c|c|c|c|}\hline 
 \multicolumn {4}{|c|}{\cellcolor{blue!10}  \nextTable \hypertarget{ऐक्ष्वाक (पु॰) Descendant of Ishvaku}{ऐक्ष्वाक (पु॰) Descendant of Ishvaku}}  \\ \hline  
 & एक॰ & द्वि॰ & बहु॰  \\ \hline 
 प्रथमा & \csarva ऐक्ष्वाकः & \csarva ऐक्ष्वाकौ & \csarva इक्ष्वाकवः \\ \hline 
 द्वितीया & \csarva ऐक्ष्वाकम् & \csarva ऐक्ष्वाकौ & \cbham इक्ष्वाकून् \\ \hline 
सम्बोधन & हे ऐक्ष्वाक & हे ऐक्ष्वाकौ & हे इक्ष्वाक्वः \\ \hline 
 \multicolumn{4}{|c|}{ शेषं राम-एक॰-द्वि॰-गुरु-बहु॰ शब्दवत् } \\ \hline 
\end{supertabular} 
\end{center} 

  \begin{center} 
 \begin{supertabular}{|c|c|c|c|}\hline 
 \multicolumn {4}{|c|}{\cellcolor{blue!10}  \nextTable \hypertarget{हाहा (पु॰) A name}{हाहा (पु॰) A name}}  \\ \hline  
 & एक॰ & द्वि॰ & बहु॰  \\ \hline 
 प्रथमा & \csarva हाहाः & \csarva हाहौ & \csarva हाहाः \\ \hline 
 द्वितीया & \csarva हाहाम् & \csarva हाहौ & \cbham हाहान् \\ \hline 
 तृतीया & \cbham हाहा & \cpada हाहाभ्याम् & \cpada हाहाभिः \\ \hline 
 चतुर्थी & \cbham हाहै & \cpada हाहाभ्याम् & \cpada हाहाभ्यः \\ \hline 
 पञ्चमी & \cbham हाहाः & \cpada हाहाभ्याम् & \cpada हाहाभ्यः \\ \hline 
 षष्ठी & \cbham हाहाः & \cbham हाहौः & \cbham हाहाम् \\ \hline 
सप्तमी & \cbham हाहे & \cbham हाहौः & \cpada हाहासु \\ \hline 
सम्बोधन & हे हाहाः & हे हाहौः & हे हाहाः \\ \hline 
\end{supertabular} 
\end{center} 

 \begin{center} 
 \begin{supertabular}{|c|c|c|c|}\hline 
 \multicolumn {4}{|c|}{\cellcolor{blue!10}  \nextTable \hypertarget{औडुलोमि (पु॰) Descendant of Uduloman}{औडुलोमि (पु॰) Descendant of Uduloman}}  \\ \hline  
 & एक॰ & द्वि॰ & बहु॰  \\ \hline 
 प्रथमा & \csarva औडुलोमिः & \csarva औडुलोमी & \csarva उडुलोमाः \\ \hline 
 द्वितीया & \csarva औडुलोमिम् & \csarva औडुलोमी & \cbham उडुलोमान् \\ \hline 
सम्बोधन & हे औडुलोमे & हे औडुलोमी & हे उडुलोमाः \\ \hline 
 \multicolumn{4}{|c|}{ शेषं एक॰-द्वि॰-हरि-बहु॰-राम शब्दवत् } \\ \hline 
\end{supertabular} 
\end{center} 

 \begin{center} 
 \begin{supertabular}{|c|c|c|c|}\hline 
 \multicolumn {4}{|c|}{\cellcolor{blue!10}  \nextTable \hypertarget{सेनानी (पु॰) Warrior}{सेनानी (पु॰) Warrior}}  \\ \hline  
 & एक॰ & द्वि॰ & बहु॰  \\ \hline 
 प्रथमा & \csarva सेनानीः & \csarva सेनान्यौ & \csarva सेनान्यः \\ \hline 
 द्वितीया & \csarva सेनान्यम् & \csarva सेनान्यौ & \cbham सेनान्यः \\ \hline 
 तृतीया & \cbham सेनान्या & \cpada सेनानीभ्याम् & \cpada सेनानीभिः \\ \hline 
 चतुर्थी & \cbham सेनान्ये & \cpada सेनानीभ्याम् & \cpada सेनानीभ्यः \\ \hline 
 पञ्चमी & \cbham सेनान्यः & \cpada सेनानीभ्याम् & \cpada सेनानीभ्यः \\ \hline 
 षष्ठी & \cbham सेनान्यः & \cbham सेनान्योः & \cbham सेनान्याम् \\ \hline 
सप्तमी & \cbham सेनान्याम् & \cbham सेनान्योः & \cpada सेनानीषु \\ \hline 
सम्बोधन & हे सेनानीः & हे सेनान्यौ &  ने-सेनान्यः \\ \hline 
\end{supertabular} 
\end{center} 

\small
 \begin{center} 
 \begin{supertabular}{|c|c|c|c|}\hline 
 \multicolumn {4}{|c|}{\cellcolor{blue!10}  \nextTable \hypertarget{वातप्रमी (पु॰) Swift Antelope}{वातप्रमी (पु॰) Swift Antelope}}  \\ \hline  
 & एक॰ & द्वि॰ & बहु॰  \\ \hline 
 प्रथमा & \csarva वातप्रमीः & \csarva वातप्रम्यौ & \csarva वातप्रम्यः \\ \hline 
 द्वितीया & \csarva वातप्रमीम् & \csarva वातप्रम्यौ & \cbham वातप्रमीन् \\ \hline 
 तृतीया & \cbham वातप्रम्या & \cpada वातप्रमीभ्याम् & \cpada वातप्रमीभिः \\ \hline 
 चतुर्थी & \cbham वातप्रमये & \cpada वातप्रमीभ्याम् & \cpada वातप्रमीभ्यः \\ \hline 
 पञ्चमी & \cbham वातप्रम्यः & \cpada वातप्रमीभ्याम् & \cpada वातप्रमिभ्यः \\ \hline 
 षष्ठी & \cbham वातप्रम्यः & \cbham वातप्रम्योः & \cbham वातप्रम्याम् \\ \hline 
सप्तमी & \cbham वातप्रमी & \cbham वातप्रम्योः & \cpada वातप्रमीषु \\ \hline 
सम्बोधन & हे वातप्रमीः & हे वातप्रम्यौ & हे वातप्रम्यः \\ \hline 
\end{supertabular} 
\end{center} 

\normalsize 
\begin{center} 
 \begin{supertabular}{|c|c|c|c|}\hline 
 \multicolumn {4}{|c|}{\cellcolor{blue!10}  \nextTable \hypertarget{क्रोष्टु (पु॰) Jackal}{क्रोष्टु (पु॰) Jackal}}  \\ \hline  
 & एक॰ & द्वि॰ & बहु॰  \\ \hline 
 प्रथमा & \csarva क्रोष्टा & \csarva क्रोष्टारौ & \csarva क्रोष्टारः \\ \hline 
 द्वितीया & \csarva क्रोष्टारम् & \csarva क्रोष्टारौ & \cbham क्रोष्टून् \\ \hline 
 तृतीया & \cbham क्रोष्ट्रा-क्रोष्टुना & \cpada क्रोष्टुभ्याम् & \cpada क्रोष्टुभिः \\ \hline 
 चतुर्थी & \cbham क्रोष्ट्रे-क्रोष्टवे & \cpada क्रोष्ट्भ्याम् & \cpada क्रोष्टुभ्यः \\ \hline 
 पञ्चमी & \cbham क्रोष्टुः-क्रोष्टोः & \cpada क्रोष्टुभ्याम् & \cpada क्रोष्ट्रुभ्यः \\ \hline 
 षष्ठी & \cbham क्रोष्टुः-क्रोष्टोः & \cbham क्रोष्ट्रोः-क्रोष्ट्वोः & \cbham क्रोष्टूनाम् \\ \hline 
सप्तमी & \cbham क्रोष्टरि-क्रोष्टौ & \cbham क्रोष्ट्रोः-क्रोष्ट्वोः & \cpada क्रोष्टुषु \\ \hline 
सम्बोधन & हे क्रोष्टो & हे क्रोष्टारौ & हे क्रोष्टारः \\ \hline 
\end{supertabular} 
\end{center} 

 \begin{center} 
 \begin{supertabular}{|c|c|c|c|}\hline 
 \multicolumn {4}{|c|}{\cellcolor{blue!10}  \nextTable \hypertarget{वर्षाभू (पु॰) Frog}{वर्षाभू (पु॰) Frog}}  \\ \hline  
 & एक॰ & द्वि॰ & बहु॰  \\ \hline 
 प्रथमा & \csarva वर्षाभुः & \csarva वर्षाभ्वौ & \csarva वर्षाभ्वः \\ \hline 
 द्वितीया & \csarva वर्षाभ्वम् & \csarva वर्षाभ्वौ & \cbham वर्षाभ्वः \\ \hline 
 तृतीया & \cbham वर्षाभ्वा & \cpada वर्षाभूभ्याम् & \cpada वर्षाभूभिः \\ \hline 
 चतुर्थी & \cbham वर्षाभ्वे & \cpada वर्षाभूभ्याम् & \cpada वर्षाभूभ्यः \\ \hline 
 पञ्चमी & \cbham वर्षाभ्वः & \cpada वर्षाभ्याम् & \cpada वर्षाभूभ्यः \\ \hline 
 षष्ठी & \cbham वर्षाभ्वः & \cbham वर्षाभ्वोः & \cbham वर्षाभ्वाम् \\ \hline 
सप्तमी & \cbham वर्षाभ्वि & \cbham वर्षाभ्वोः & \cpada वर्षाभूषु \\ \hline 
सम्बोधन & हे वर्षाभूः & हे वर्षाभ्वौ & हे वर्षाभ्वः \\ \hline 
\end{supertabular} 
\end{center} 

 \begin{center} 
 \begin{supertabular}{|c|c|c|c|}\hline 
 \multicolumn {4}{|c|}{\cellcolor{blue!10}  \nextTable \hypertarget{हूहू (पु॰) A Name}{हूहू (पु॰) A Name}}  \\ \hline  
 & एक॰ & द्वि॰ & बहु॰  \\ \hline 
 प्रथमा & \csarva हूहूः & \csarva हूह्वौ & \csarva हूह्वः \\ \hline 
 द्वितीया & \csarva हूहूम् & \csarva हूह्वौ & \cbham हूहून् \\ \hline 
 तृतीया & \cbham हूह्वा & \cpada हूहूभ्याम् & \cpada हूहूभिः \\ \hline 
 चतुर्थी & \cbham हूह्वे & \cpada हूहूभ्याम् & \cpada हूहूभ्यः \\ \hline 
 पञ्चमी & \cbham हूह्वः & \cpada हूहूभ्याम् & \cpada हूहूभ्यः \\ \hline 
 षष्ठी & \cbham हूह्वः & \cbham हूह्वोः & \cbham हूह्वाम् \\ \hline 
सप्तमी & \cbham हूह्वि & \cbham हूह्वोः & \cpada हूहूषु \\ \hline 
सम्बोधन & हे हूहूः & हे हूह्वौ & हे हूह्वः \\ \hline 
\end{supertabular} 
\end{center} 

\small
 \begin{center} 
 \begin{supertabular}{|c|c|c|c|}\hline 
 \multicolumn {4}{|c|}{\cellcolor{blue!10}  \nextTable \hypertarget{जरा (स्त्री॰) Old Age}{जरा (स्त्री॰) Old Age}}  \\ \hline  
 & एक॰ & द्वि॰ & बहु॰  \\ \hline 
 प्रथमा & \csarva जरा & \csarva जरे-जरसौ & \csarva जराः-जरसः \\ \hline 
 द्वितीया & \csarva जराम्-जरसम् & \csarva जरे-जरसौ & \cbham जराः-जरसः \\ \hline 
 तृतीया & \cbham जरया-जरसा & \cpada जराभ्याम् & \cpada जराभिः \\ \hline 
 चतुर्थी & \cbham जरायै-जरसे & \cpada जराभ्याम् & \cpada जराभ्यः \\ \hline 
 पञ्चमी & \cbham जरायाः-जरसः & \cpada जराभ्याम् & \cpada जराभ्यः \\ \hline 
 षष्ठी & \cbham जरायाः-जरसः & \cbham जरयोः-जरसोः & \cbham जराणाम्-जरसाम् \\ \hline 
सप्तमी & \cbham जरायाम्-जरसि & \cbham जरयोः-जरसोः & \cpada जरासु \\ \hline 
सम्बोधन & हे जरे & हे जरे-जरसौ &  जे-जराः-जरसः \\ \hline 
\end{supertabular} 
\end{center} 

 \begin{center} 
 \begin{supertabular}{|c|c|c|c|}\hline 
 \multicolumn {4}{|c|}{\cellcolor{blue!10}  \nextTable \hypertarget{अजर (नपु॰) Ageless}{अजर (नपु॰) Ageless}}  \\ \hline  
 & एक॰ & द्वि॰ & बहु॰  \\ \hline 
 प्रथमा & \cpada अजरम् & \cbham अजरे-अजरसी & \csarva अजराणि-अजरांसि \\ \hline 
 द्वितीया & \cbham अजरम् & \cbham अजरे-अजरसी & \csarva अजराणि-अजरांसि \\ \hline 
सम्बोधन & हे अजर & हे अजरे-अजरसी & हे अजराणि-अजरांसि \\ \hline 
 \multicolumn{4}{|c|}{ शेषं निर्जर शब्दवत् } \\ \hline 
\end{supertabular} 
\end{center} 

 \begin{center} 
 \begin{supertabular}{|c|c|c|c|}\hline 
 \multicolumn {4}{|c|}{\cellcolor{blue!10}  \nextTable \hypertarget{प्राञ्च् (पु॰) Eastern}{प्राञ्च् (पु॰) Eastern}}  \\ \hline  
 & एक॰ & द्वि॰ & बहु॰  \\ \hline 
 प्रथमा & \csarva प्राङ् & \csarva प्राञ्चौ & \csarva प्राञ्चः \\ \hline 
 द्वितीया & \csarva प्राञ्चम् & \csarva प्राञ्चौ & \cbham प्राचः \\ \hline 
 तृतीया & \cbham प्राचा & \cpada प्राग्भ्याम् & \cpada प्राग्भिः \\ \hline 
 चतुर्थी & \cbham प्राचे & \cpada प्राग्भ्याम् & \cpada प्राग्भ्यः \\ \hline 
 पञ्चमी & \cbham प्राचः & \cpada प्राग्भ्याम् & \cpada प्राग्भ्यः \\ \hline 
 षष्ठी & \cbham प्राचः & \cbham प्राचोः & \cbham प्राचाम् \\ \hline 
सप्तमी & \cbham प्राचि & \cbham प्राचोः & \cpada प्राक्षु \\ \hline 
सम्बोधन & हे प्राङ् & हे प्राञ्चौ & हे प्राञ्चः \\ \hline 
\end{supertabular} 
\end{center} 

\normalsize
 \begin{center} 
 \begin{supertabular}{|c|c|c|c|}\hline 
 \multicolumn {4}{|c|}{\cellcolor{blue!10}  \nextTable \hypertarget{प्रत्यञ्च् (पु॰) Western}{प्रत्यञ्च् (पु॰) Western}}  \\ \hline  
 & एक॰ & द्वि॰ & बहु॰  \\ \hline 
 प्रथमा & \csarva प्रत्यङ् & \csarva प्रत्यञ्चौ & \csarva प्रत्यञ्चः \\ \hline 
 द्वितीया & \csarva प्रत्यञ्चम् & \csarva प्रत्यञ्चौ & \cbham प्रतीचः \\ \hline 
 तृतीया & \cbham प्रतीचा & \cpada प्रत्यग्भ्याम् & \cpada प्रत्यग्भिः \\ \hline 
 चतुर्थी & \cbham प्रतीचे & \cpada प्रत्यग्भ्याम् & \cpada प्रत्यग्भ्यः \\ \hline 
 पञ्चमी & \cbham प्रतीचः & \cpada प्रत्यग्भ्याम् & \cpada प्रत्यग्भ्यः \\ \hline 
 षष्ठी & \cbham प्रतीचः & \cbham प्रतीचोः & \cbham प्रतीचाम् \\ \hline 
सप्तमी & \cbham प्रतीचि & \cbham प्रतीचोः & \cpada प्रत्यक्षु \\ \hline 
सम्बोधन & हे प्रत्यङ् & हे प्रत्यञ्चौ & हे प्रत्यञ्चः \\ \hline 
\end{supertabular} 
\end{center} 

 \begin{center} 
 \begin{supertabular}{|c|c|c|c|}\hline 
 \multicolumn {4}{|c|}{\cellcolor{blue!10}  \nextTable \hypertarget{उदञ्च् (पु॰) Northern}{उदञ्च् (पु॰) Northern}}  \\ \hline  
 & एक॰ & द्वि॰ & बहु॰  \\ \hline 
 प्रथमा & \csarva उदङ् & \csarva उदञ्चौ & \csarva उदञ्चः \\ \hline 
 द्वितीया & \csarva उदञ्चम् & \csarva उदञ्चौ & \cbham उदैचः \\ \hline 
सम्बोधन & हे उदङ् & हे उदञ्चौ & हे उदञ्चः \\ \hline 
 \multicolumn{4}{|c|}{ शेषं उदञ्चम् शब्दवत् } \\ \hline 
\end{supertabular} 
\end{center} 

 \begin{center} 
 \begin{supertabular}{|c|c|c|c|}\hline 
 \multicolumn {4}{|c|}{\cellcolor{blue!10}  \nextTable \hypertarget{अन्वञ्च् (पु॰) Following}{अन्वञ्च् (पु॰) Following}}  \\ \hline  
 & एक॰ & द्वि॰ & बहु॰  \\ \hline 
 प्रथमा & \csarva अन्वङ् & \csarva अन्वञ्चौ & \csarva अन्वञ्चः \\ \hline 
 द्वितीया & \csarva अन्वञ्चम् & \csarva अन्वञ्चौ & \cbham अनूचः \\ \hline 
 तृतीया & \cbham अनूचा & \cpada अन्वग्भ्याम् & \cpada अन्वग्भिः \\ \hline 
 चतुर्थी & \cbham अनूचे & \cpada अन्वग्भ्याम् & \cpada अन्वग्भ्यः \\ \hline 
 पञ्चमी & \cbham अनूचः & \cpada अन्वग्भ्याम् & \cpada अन्वग्भ्यः \\ \hline 
 षष्ठी & \cbham अनूचः & \cbham अनूचोः & \cbham अनूचाम् \\ \hline 
सप्तमी & \cbham अनूचि & \cbham अनूचोः & \cpada अन्वक्षु \\ \hline 
सम्बोधन & हे अन्वङ् & हे अन्वञ्चौ & हे अन्वञ्चः \\ \hline 
\end{supertabular} 
\end{center} 

 \begin{center} 
 \begin{supertabular}{|c|c|c|c|}\hline 
 \multicolumn {4}{|c|}{\cellcolor{blue!10}  \nextTable \hypertarget{तिर्यञ्च् (पु॰) Horizontally}{तिर्यञ्च् (पु॰) Horizontally}}  \\ \hline  
 & एक॰ & द्वि॰ & बहु॰  \\ \hline 
 प्रथमा & \csarva तिर्यङ् & \csarva तिर्यञ्चौ & \csarva तिर्यञ्चः \\ \hline 
 द्वितीया & \csarva तिर्यञ्चम् & \csarva तिर्यञ्चौ & \cbham तिरश्चः \\ \hline 
 तृतीया & \cbham तिरश्चा & \cpada तिर्यग्भ्याम् & \cpada तिर्यग्भिः \\ \hline 
 चतुर्थी & \cbham तिरश्चे & \cpada तिर्यग्भ्याम् & \cpada तिर्यग्भ्यः \\ \hline 
 पञ्चमी & \cbham तिरश्चः & \cpada तिर्यग्भ्याम् & \cpada तिर्यग्भ्यः \\ \hline 
 षष्ठी & \cbham तिरश्चः & \cbham तिरश्चोः & \cbham तिरश्चाम् \\ \hline 
सप्तमी & \cbham तिरश्चि & \cbham तिरश्चोः & \cpada तिर्यक्षु \\ \hline 
सम्बोधन & हे तिर्यङ् & हे तिर्यञ्चौ & हे तिर्यञ्चः \\ \hline 
\end{supertabular} 
\end{center} 

 \begin{center} 
 \begin{supertabular}{|c|c|c|c|}\hline 
 \multicolumn {4}{|c|}{\cellcolor{blue!10}  \nextTable \hypertarget{विभ्राज् (पु॰) Bright}{विभ्राज् (पु॰) Bright}}  \\ \hline  
 & एक॰ & द्वि॰ & बहु॰  \\ \hline 
 प्रथमा & \csarva विभ्राट्-विभ्राक् & \csarva विभ्राजौ & \csarva विभ्राजः \\ \hline 
 द्वितीया & \csarva विभ्राजम् & \csarva विभ्राजौ & \cbham विभ्राजः \\ \hline 
सम्बोधन & हे विभ्राट्-विभ्राक् & हे विभ्राजौ & हे विभ्राजः \\ \hline 
 \multicolumn{4}{|c|}{ शेषं सम्राज् शब्दवत् } \\ \hline 
\end{supertabular} 
\end{center} 

 \begin{center} 
 \begin{supertabular}{|c|c|c|c|}\hline 
 \multicolumn {4}{|c|}{\cellcolor{blue!10}  \nextTable \hypertarget{युज् (पु॰) Sage}{युज् (पु॰) Sage}}  \\ \hline  
 & एक॰ & द्वि॰ & बहु॰  \\ \hline 
 प्रथमा & \csarva युक् & \csarva युजौ & \csarva युजः \\ \hline 
 द्वितीया & \csarva युजम् & \csarva युजौ & \cbham युजः \\ \hline 
 तृतीया & \cbham युजा & \cpada युग्भ्याम् & \cpada युग्भिः \\ \hline 
 चतुर्थी & \cbham युजे & \cpada युग्भ्याम् & \cpada युग्भ्यः \\ \hline 
 पञ्चमी & \cbham युजः & \cpada युग्भ्याम् & \cpada युग्भ्यः \\ \hline 
 षष्ठी & \cbham युजः & \cbham युजोः & \cbham युजाम् \\ \hline 
सप्तमी & \cbham युजि & \cbham युजोः & \cpada युक्षु \\ \hline 
सम्बोधन & हे युक् & हे युजौ & हे युजः \\ \hline 
\end{supertabular} 
\end{center} 

 \begin{center} 
 \begin{supertabular}{|c|c|c|c|}\hline 
 \multicolumn {4}{|c|}{\cellcolor{blue!10}  \nextTable \hypertarget{युञ्ज् (पु॰) United}{युञ्ज् (पु॰) United}}  \\ \hline  
 & एक॰ & द्वि॰ & बहु॰  \\ \hline 
 प्रथमा & \csarva युङ् & \csarva युञ्जौ & \csarva युञ्जः \\ \hline 
 द्वितीया & \csarva युञ्जम् & \csarva युञ्जौ & \cbham युजः \\ \hline 
सम्बोधन & हे युङ् & हे युञ्जौ & हे युञ्जः \\ \hline 
 \multicolumn{4}{|c|}{ शेषं युज् शब्दवत् } \\ \hline 
\end{supertabular} 
\end{center} 

 \begin{center} 
 \begin{supertabular}{|c|c|c|c|}\hline 
 \multicolumn {4}{|c|}{\cellcolor{blue!10}  \nextTable \hypertarget{सुपाद् (पु॰) Having Good Feet}{सुपाद् (पु॰) Having Good Feet}}  \\ \hline  
 & एक॰ & द्वि॰ & बहु॰  \\ \hline 
 प्रथमा & \csarva सुपात् & \csarva सुपादौ & \csarva सुपादः \\ \hline 
 द्वितीया & \csarva सुपादम् & \csarva सुपादौ & \cbham सुपादः \\ \hline 
 तृतीया & \cbham सुपदा & \cpada सुपाद्भ्याम् & \cpada सुपाद्भिः \\ \hline 
 चतुर्थी & \cbham सुपदे & \cpada सुपाद्भ्याम् & \cpada सुपाद्भ्यः \\ \hline 
 पञ्चमी & \cbham सुपदः & \cpada सुपाद्भ्याम् & \cpada सुपाद्भ्यः \\ \hline 
 षष्ठी & \cbham सुपदः & \cbham सुपदोः & \cbham सुपदाम् \\ \hline 
सप्तमी & \cbham सुपदि & \cbham सुपदोः & \cpada सुपत्सु \\ \hline 
सम्बोधन & हे सुपात् & हे सुपादौ & हे सुपादः \\ \hline 
\end{supertabular} 
\end{center} 

 \begin{center} 
 \begin{supertabular}{|c|c|c|c|}\hline 
 \multicolumn {4}{|c|}{\cellcolor{blue!10}  \nextTable \hypertarget{पूषन् (पु॰) Sun}{पूषन् (पु॰) Sun}}  \\ \hline  
 & एक॰ & द्वि॰ & बहु॰  \\ \hline 
 प्रथमा & \csarva पूषा & \csarva पूषणौ & \csarva पूषणः \\ \hline 
 द्वितीया & \csarva पूषणम् & \csarva पूषणौ & \cbham पूष्णः \\ \hline 
सम्बोधन & हे पूषन् & हे पूषणौ & हे पूषणः \\ \hline 
 \multicolumn{4}{|c|}{ शेषं राजन् शब्दवत् } \\ \hline 
\end{supertabular} 
\end{center} 

 \begin{center} 
 \begin{supertabular}{|c|c|c|c|}\hline 
 \multicolumn {4}{|c|}{\cellcolor{blue!10}  \nextTable \hypertarget{वृत्रहन् (पु॰) Indra}{वृत्रहन् (पु॰) Indra}}  \\ \hline  
 & एक॰ & द्वि॰ & बहु॰  \\ \hline 
 प्रथमा & \csarva वृत्रहा & \csarva वृत्रहणौ & \csarva वृत्रहणः \\ \hline 
 द्वितीया & \csarva वृत्रहणम् & \csarva वृत्रहणौ & \cbham वृत्रघ्नः \\ \hline 
 तृतीया & \cbham वृत्रघ्ना & \cpada वृत्रहभ्याम् & \cpada वृत्रहभिः \\ \hline 
 चतुर्थी & \cbham वृत्रघ्ने & \cpada वृत्रहभ्याम् & \cpada वृत्रहभ्यः \\ \hline 
 पञ्चमी & \cbham वृत्रघ्नः & \cpada वृत्रहभ्याम् & \cpada वृत्रहभ्यः \\ \hline 
 षष्ठी & \cbham वृत्रघ्नः & \cbham वृत्रघ्नोः & \cbham वृत्रघ्नाम् \\ \hline 
सप्तमी & \cbham वृत्रघ्नि-वृत्रहणि & \cbham वृत्रघ्नोः & \cpada वृत्रहसु \\ \hline 
सम्बोधन & हे वृत्रहन् & हे वृत्रहणौ & हे वृत्रहणः \\ \hline 
\end{supertabular} 
\end{center} 

 \begin{center} 
 \begin{supertabular}{|c|c|c|c|}\hline 
 \multicolumn {4}{|c|}{\cellcolor{blue!10}  \nextTable \hypertarget{दीर्घाहन् (पु॰) Summer}{दीर्घाहन् (पु॰) Summer}}  \\ \hline  
 & एक॰ & द्वि॰ & बहु॰  \\ \hline 
 प्रथमा & \csarva दीर्घाहाः & \csarva दीर्घाहणौ & \csarva दीर्घाहाणः \\ \hline 
 द्वितीया & \csarva दीर्घाहाणम् & \csarva दीर्घाहाणौ & \cbham दीर्घांह्णः \\ \hline 
 तृतीया & \cbham दीर्घाह्णा & \cpada दीर्घाहोभ्याम् & \cpada दीर्घाहोभिः \\ \hline 
 चतुर्थी & \cbham दीर्घाह्णे & \cpada दीर्घाहोभ्याम् & \cpada दीर्घाहोभ्यः \\ \hline 
 पञ्चमी & \cbham दीर्घाह्णः & \cpada दीर्घाहोभ्याम् & \cpada दीर्घाहोभ्यः \\ \hline 
 षष्ठी & \cbham दीर्घाह्णः & \cbham दीर्घाह्णोः & \cbham दीर्घाह्णाम् \\ \hline 
सप्तमी & \cbham दीर्घाह्णि-दीर्घाहणि & \cbham दीर्घाह्नोः & \cpada दीर्घाहस्सु \\ \hline 
सम्बोधन & हे दीर्घाहाः & हे दीर्घाहणौ & हे दीर्घाहाणः \\ \hline 
\end{supertabular} 
\end{center} 

\begin{center} 
 \begin{supertabular}{|c|c|c|c|}\hline 
 \multicolumn {4}{|c|}{\cellcolor{blue!10}  \nextTable \hypertarget{अर्वन् (पु॰) Horse}{अर्वन् (पु॰) Horse}}  \\ \hline  
 & एक॰ & द्वि॰ & बहु॰  \\ \hline 
 प्रथमा & \csarva अर्वा & \csarva अर्वन्तौ & \csarva अर्वन्तः \\ \hline 
 द्वितीया & \csarva अर्वन्तम् & \csarva अर्वन्तौ & \cbham अर्वतः \\ \hline 
सम्बोधन & हे अर्वन् & हे अर्वन्तौ & हे अर्वन्तः \\ \hline 
 \multicolumn{4}{|c|}{ शेषं धीमत् शब्दवत् } \\ \hline 
\end{supertabular} 
\end{center} 

 \begin{center} 
 \begin{supertabular}{|c|c|c|c|}\hline 
 \multicolumn {4}{|c|}{\cellcolor{blue!10}  \nextTable \hypertarget{ऋभुक्षिन् (पु॰) Indra}{ऋभुक्षिन् (पु॰) Indra}}  \\ \hline  
 & एक॰ & द्वि॰ & बहु॰  \\ \hline 
 प्रथमा & \csarva ऋभुक्षाः & \csarva ऋभुक्षाणौ & \csarva ऋभुक्षाणः \\ \hline 
 द्वितीया & \csarva ऋभुक्षाणम् & \csarva ऋभुक्षाणौ & \cbham ऋभुक्षः \\ \hline 
 तृतीया & \cbham ऋभुक्षा & \cpada ऋभुक्षिभ्याम् & \cpada ऋभुक्षिभिः \\ \hline 
 चतुर्थी & \cbham ऋभुक्षे & \cpada ऋभुक्षिभ्याम् & \cpada ऋभुक्षिभ्यः \\ \hline 
 पञ्चमी & \cbham ऋभुक्षः & \cpada ऋभुक्षिभ्याम् & \cpada ऋभुक्षिभ्यः \\ \hline 
 षष्ठी & \cbham ऋभुक्षः & \cbham ऋभुक्षोः & \cbham ऋभुक्षाम् \\ \hline 
सप्तमी & \cbham ऋभुक्षि & \cbham ऋभुक्षोः & \cpada ऋभुक्षिषु \\ \hline 
सम्बोधन & हे ऋभुक्षाः & हे ऋभुक्षाणौ & हे ऋभुक्षाणः \\ \hline 
\end{supertabular} 
\end{center} 

 \begin{center} 
 \begin{supertabular}{|c|c|c|c|}\hline 
 \multicolumn {4}{|c|}{\cellcolor{blue!10}  \nextTable \hypertarget{उशनस् (पु॰) Shukaacharya}{उशनस् (पु॰) Shukaacharya}}  \\ \hline  
 & एक॰ & द्वि॰ & बहु॰  \\ \hline 
 प्रथमा & \csarva उशना & \csarva उशनसौ & \csarva उशनसः \\ \hline 
 द्वितीया & \csarva उशनसम् & \csarva उशनसौ & \cbham उशनसः \\ \hline 
सम्बोधन & हे उशनन्-उशनः-उशन & हे उशनसौ & हे उशनसः \\ \hline 
 \multicolumn{4}{|c|}{ शेषं \hyperlink{वेधस् (पु॰) Brahma}{वेधस् (पु॰)} शब्दवत् } \\ \hline 
\end{supertabular} 
\end{center} 

 \begin{center} 
 \begin{supertabular}{|c|c|c|c|}\hline 
 \multicolumn {4}{|c|}{\cellcolor{blue!10}  \nextTable \hypertarget{अनेहस् (पु॰) Time}{अनेहस् (पु॰) Time}}  \\ \hline  
 & एक॰ & द्वि॰ & बहु॰  \\ \hline 
 प्रथमा & \csarva अनेहा & \csarva अनेहसौ & \csarva अनेहसः \\ \hline 
 द्वितीया & \csarva अनेहसम् & \csarva अनेहसौ & \cbham अनेहसः \\ \hline 
सम्बोधन & हे अनेहः & हे अनेहसौ & हे अनेहसः \\ \hline 
 \multicolumn{4}{|c|}{ शेषं वेधस् शब्दवत् } \\ \hline 
\end{supertabular} 
\end{center} 

 \begin{center} 
 \begin{supertabular}{|c|c|c|c|}\hline 
 \multicolumn {4}{|c|}{\cellcolor{blue!10}  \nextTable \hypertarget{विश्ववाह् (पु॰) Sustainer of the Universe}{विश्ववाह् (पु॰) Sustainer of the Universe}}  \\ \hline  
 & एक॰ & द्वि॰ & बहु॰  \\ \hline 
 प्रथमा & \csarva विश्ववाट् & \csarva विश्ववाहौ & \csarva विश्ववाहः \\ \hline 
 द्वितीया & \csarva विश्ववाहम् & \csarva विश्ववाहौ & \cbham विश्वौहः \\ \hline 
 तृतीया & \cbham विश्वौहा & \cpada विश्ववाड्भ्याम् & \cpada विश्ववाड्भिः \\ \hline 
 चतुर्थी & \cbham विश्वौहे & \cpada विश्वाड्भ्याम् & \cpada विश्ववाड्भ्यः \\ \hline 
 पञ्चमी & \cbham विश्वौहः & \cpada विश्ववाड्भ्याम् & \cpada विश्ववाड्भ्यः \\ \hline 
 षष्ठी & \cbham विश्वौहः & \cbham विश्वौहोः & \cbham विश्वौहास् \\ \hline 
सप्तमी & \cbham विश्वौहि & \cbham विश्वौहोः & \cpada विश्वाट्सु \\ \hline 
सम्बोधन & हे विश्ववाट् & हे विश्ववाहौ & हे विश्वाहः \\ \hline 
\end{supertabular} 
\end{center} 

 \begin{center} 
 \begin{supertabular}{|c|c|c|c|}\hline 
 \multicolumn {4}{|c|}{\cellcolor{blue!10}  \nextTable \hypertarget{तुरासाह् (पु॰) Indra}{तुरासाह् (पु॰) Indra}}  \\ \hline  
 & एक॰ & द्वि॰ & बहु॰  \\ \hline 
 प्रथमा & \csarva तुराषाट् & \csarva तुरासाहौ & \csarva तुरासाहः \\ \hline 
 द्वितीया & \csarva तुरासाहम् & \csarva तुरासाहौ & \cbham तुरासाहः \\ \hline 
 तृतीया & \cbham तुरासाहा & \cpada तुराषाड्भ्याम् & \cpada तुराषाड्भिः \\ \hline 
 चतुर्थी & \cbham तुरासाहे & \cpada तुराषाड्भ्याम् & \cpada तुराषाड्भ्यः \\ \hline 
 पञ्चमी & \cbham तुरासाहः & \cpada तुराषाड्भ्याम् & \cpada तुराषाड्भ्यः \\ \hline 
 षष्ठी & \cbham तुरासाहः & \cbham तुरासाहोः & \cbham तुरासाहाम् \\ \hline 
सप्तमी & \cbham तुरासाहि & \cbham तुरासाहोः & \cpada तुराषाट्सु \\ \hline 
सम्बोधन & हे तुराषाट् & हे तुरासाहौ & हे तुरासाहः \\ \hline 
\end{supertabular} 
\end{center} 

 \begin{center} 
 \begin{supertabular}{|c|c|c|c|}\hline 
 \multicolumn {4}{|c|}{\cellcolor{blue!10}  \nextTable \hypertarget{दुह् (पु॰) One who Milks}{दुह् (पु॰) One who Milks}}  \\ \hline  
 & एक॰ & द्वि॰ & बहु॰  \\ \hline 
 प्रथमा & \csarva धुक् & \csarva दुहौ & \csarva दुहः \\ \hline 
 द्वितीया & \csarva दुहम् & \csarva दुहौ & \cbham दुहः \\ \hline 
 तृतीया & \cbham दुहा & \cpada धुग्भ्याम् & \cpada धुग्भिः \\ \hline 
 चतुर्थी & \cbham दुहे & \cpada धुग्भ्याम् & \cpada धुग्भ्यः \\ \hline 
 पञ्चमी & \cbham दुहः & \cpada धुग्भ्याम् & \cpada धुग्भ्यः \\ \hline 
 षष्ठी & \cbham दुहः & \cbham दुहोः & \cbham दुहाम् \\ \hline 
सप्तमी & \cbham दुहि & \cbham दुहोः & \cpada धुक्षु \\ \hline 
सम्बोधन & हे धुक् & हे दुहौ & हे दुहः \\ \hline 
\end{supertabular} 
\end{center} 

 \begin{center} 
 \begin{supertabular}{|c|c|c|c|}\hline 
 \multicolumn {4}{|c|}{\cellcolor{blue!10}  \nextTable \hypertarget{द्रुह् (पु॰) One who bears hatred}{द्रुह् (पु॰) One who bears hatred}}  \\ \hline  
 & एक॰ & द्वि॰ & बहु॰  \\ \hline 
 प्रथमा & \csarva ध्रुक्-ध्रुट् & \csarva द्रुहौ & \csarva द्रुहः \\ \hline 
 द्वितीया & \csarva द्रुहम् & \csarva द्रुहौ & \cbham द्रुहः \\ \hline 
 तृतीया & \cbham द्रुहा & \cpada ध्रुग्भ्याम्-ध्रुड्भ्याम् & \cpada ध्रुग्भिः-ध्रुड्भिः \\ \hline 
 चतुर्थी & \cbham द्रुहे & \cpada ध्रुग्भ्याम्-ध्रुड्भ्याम् & \cpada ध्रुग्भ्यः-ध्रुड्भ्यः \\ \hline 
 पञ्चमी & \cbham द्रुहः & \cpada ध्रुग्भ्याम्-ध्रुड्भ्याम् & \cpada ध्रुग्भ्यः-ध्रुड्भ्यः \\ \hline 
 षष्ठी & \cbham द्रुहः & \cbham द्रुहोः & \cbham द्रुहाम् \\ \hline 
सप्तमी & \cbham द्रुहि & \cbham द्रुहोः & \cpada ध्रुक्षु-ध्रुट्सु \\ \hline 
सम्बोधन & हे ध्रुक्-ध्रुट् & हे द्रुहौ & हे द्रुहः \\ \hline 
\end{supertabular} 
\end{center} 

\begin{center} 
 \begin{supertabular}{|c|c|c|c|}\hline 
 \multicolumn {4}{|c|}{\cellcolor{blue!10}  \nextTable \hypertarget{अनडुह् (पु॰) Ox}{अनडुह् (पु॰) Ox}}  \\ \hline  
 & एक॰ & द्वि॰ & बहु॰  \\ \hline 
 प्रथमा & \csarva अनड्वान् & \csarva अनड्वाहौ & \csarva अनड्वाहः \\ \hline 
 द्वितीया & \csarva अनड्वाहम् & \csarva अनड्वाहौ & \cbham अनडुहः \\ \hline 
 तृतीया & \cbham अनडुहा & \cpada अनडुद्भ्याम् & \cpada अनडुद्भिः \\ \hline 
 चतुर्थी & \cbham अनडुहे & \cpada अनडुद्भ्याम् & \cpada अनडुद्भ्यः \\ \hline 
 पञ्चमी & \cbham अनडुहः & \cpada अनडुद्भ्याम् & \cpada अनडुद्भ्यः \\ \hline 
 षष्ठी & \cbham अनडुहः & \cbham अनडुहोः & \cbham अनडुहाम् \\ \hline 
सप्तमी & \cbham अनडुहि & \cbham अनडुहोः & \cpada अनडुत्सु \\ \hline 
सम्बोधन & हे अनड्वान् & हे अनड्वाहौ & हे अनड्वाहः \\ \hline 
\end{supertabular} 
\end{center} 

 \begin{center} 
 \begin{supertabular}{|c|c|c|c|}\hline 
 \multicolumn {4}{|c|}{\cellcolor{blue!10}  \nextTable \hypertarget{द्वार् (स्त्री॰) Door}{द्वार् (स्त्री॰) Door}}  \\ \hline  
 & एक॰ & द्वि॰ & बहु॰  \\ \hline 
 प्रथमा & \csarva द्वाः & \csarva द्वारौ & \csarva द्वारः \\ \hline 
 द्वितीया & \csarva द्वारम् & \csarva द्वारौ & \cbham द्वारः \\ \hline 
 तृतीया & \cbham द्वारा & \cpada द्वार्भ्याम् & \cpada द्वार्भिः \\ \hline 
 चतुर्थी & \cbham द्वारे & \cpada द्वार्भ्याम् & \cpada द्वार्भ्यः \\ \hline 
 पञ्चमी & \cbham द्वारः & \cpada द्वार्भ्याम् & \cpada द्वार्भ्यः \\ \hline 
 षष्ठी & \cbham द्वारः & \cbham द्वारोः & \cbham द्वाराम् \\ \hline 
सप्तमी & \cbham द्वारि & \cbham द्वारोः & \cpada द्वार्षु \\ \hline 
सम्बोधन & हे द्वाः & हे द्वारौ & हे द्वारः \\ \hline 
\end{supertabular} 
\end{center} 

 \begin{center} 
 \begin{supertabular}{|c|c|c|c|}\hline 
 \multicolumn {4}{|c|}{\cellcolor{blue!10}  \nextTable \hypertarget{अर्चिस् (स्त्री॰) Flame}{अर्चिस् (स्त्री॰) Flame}}  \\ \hline  
 & एक॰ & द्वि॰ & बहु॰  \\ \hline 
 प्रथमा & \csarva अर्चिः & \csarva अर्चिषौ & \csarva अर्चिषः \\ \hline 
 द्वितीया & \csarva अर्चिषम् & \csarva अर्चिषौ & \cbham अर्चिषः \\ \hline 
 तृतीया & \cbham अर्चिषा & \cpada अर्चिर्भ्याम् & \cpada अर्चिर्भिः \\ \hline 
 चतुर्थी & \cbham अर्चिषे & \cpada अर्चिर्भ्याम् & \cpada अर्चिर्भ्यः \\ \hline 
 पञ्चमी & \cbham अर्चिषः & \cpada अर्चिर्भ्याम् & \cpada अर्चिर्भ्यः \\ \hline 
 षष्ठी & \cbham अर्चिषः & \cbham अर्चिषोः & \cbham अर्चिषाम् \\ \hline 
सप्तमी & \cbham अर्चिषि & \cbham अर्चिषोः & \cpada अर्चिःषु \\ \hline 
सम्बोधन & हे अर्चिः & हे अर्चिषौ & हे अर्चिषः \\ \hline 
\end{supertabular} 
\end{center} 

 \begin{center} 
 \begin{supertabular}{|c|c|c|c|}\hline 
 \multicolumn {4}{|c|}{\cellcolor{blue!10}  \nextTable \hypertarget{सजुष् (स्त्री॰) Companion}{सजुष् (स्त्री॰) Companion}}  \\ \hline  
 & एक॰ & द्वि॰ & बहु॰  \\ \hline 
 प्रथमा & \csarva सजूः & \csarva सजुषौ & \csarva सजुषः \\ \hline 
 द्वितीया & \csarva सजुषम् & \csarva सजुषौ & \cbham सजुषः \\ \hline 
 तृतीया & \cbham सजुषा & \cpada सजूर्भ्याम् & \cpada सजूर्भिः \\ \hline 
 चतुर्थी & \cbham सजुषे & \cpada सजूर्भ्याम् & \cpada सजूर्भ्यः \\ \hline 
 पञ्चमी & \cbham सजुषः & \cpada सजूर्भ्याम् & \cpada सजूर्भ्यः \\ \hline 
 षष्ठी & \cbham सजुषः & \cbham सजुषोः & \cbham सजूषाम् \\ \hline 
सप्तमी & \cbham सजुषि & \cbham सजुषोः & \cpada सजूःषु \\ \hline 
सम्बोधन & हे सजूः & हे सजुषौ & हे सजुषः \\ \hline 
\end{supertabular} 
\end{center} 

 \begin{center} 
 \begin{supertabular}{|c|c|c|c|}\hline 
 \multicolumn {4}{|c|}{\cellcolor{blue!10}  \nextTable \hypertarget{उष्णिह् (स्त्री॰) A Metre}{उष्णिह् (स्त्री॰) A Metre}}  \\ \hline  
 & एक॰ & द्वि॰ & बहु॰  \\ \hline 
 प्रथमा & \csarva उष्णिक् & \csarva उष्णिहौ & \csarva उष्णिहः \\ \hline 
 द्वितीया & \csarva उष्णिहम् & \csarva उष्णिहौ & \cbham उष्णिहः \\ \hline 
 तृतीया & \cbham उष्णिहा & \cpada उष्णिग्भ्याम् & \cpada उष्णिग्भिः \\ \hline 
 चतुर्थी & \cbham उष्णिहे & \cpada उष्णिग्भ्याम् & \cpada उष्णिग्भ्यः \\ \hline 
 पञ्चमी & \cbham उष्णिहः & \cpada उष्णिग्भ्याम् & \cpada उष्णिग्भ्यः \\ \hline 
 षष्ठी & \cbham उष्णिहः & \cbham उषणिहोः & \cbham उषणिहाम् \\ \hline 
सप्तमी & \cbham उष्णिहि & \cbham उष्णिहोः & \cpada उष्णिक्षु \\ \hline 
सम्बोधन & हे उष्णिक् & हे उष्णिहौ & हे उष्णिहः \\ \hline 
\end{supertabular} 
\end{center} 

 \begin{center} 
 \begin{supertabular}{|c|c|c|c|}\hline 
 \multicolumn {4}{|c|}{\cellcolor{blue!10}  \nextTable \hypertarget{प्राञ्च् (नपु॰) Eastern}{प्राञ्च् (नपु॰) Eastern}}  \\ \hline  
 & एक॰ & द्वि॰ & बहु॰  \\ \hline 
 प्रथमा & \cpada प्राक् & \cbham प्राची & \csarva प्राञ्चि \\ \hline 
 द्वितीया & \cbham प्राक् & \cbham प्राची & \csarva प्राञ्चि \\ \hline 
सम्बोधन & हे प्राक् & हे प्राची & हे प्राञ्चि \\ \hline 
 \multicolumn{4}{|c|}{ शेषं प्राञ्च्-पुंवत् शब्दवत् } \\ \hline 
\end{supertabular} 
\end{center} 

 \begin{center} 
 \begin{supertabular}{|c|c|c|c|}\hline 
 \multicolumn {4}{|c|}{\cellcolor{blue!10}  \nextTable \hypertarget{प्रत्यञ्च् (नपु॰) Western}{प्रत्यञ्च् (नपु॰) Western}}  \\ \hline  
 & एक॰ & द्वि॰ & बहु॰  \\ \hline 
 प्रथमा & \cpada प्रत्यक् & \cbham प्रतीची & \csarva प्रत्यञ्चि \\ \hline 
 द्वितीया & \cbham प्रत्यक् & \cbham प्रतीची & \csarva प्रत्यञ्चि \\ \hline 
सम्बोधन & हे प्रत्यक् & हे प्रतीची & हे प्रत्यञ्चि \\ \hline 
 \multicolumn{4}{|c|}{ शेषं प्रत्यञ्च्-पुंवत् शब्दवत् } \\ \hline 
\end{supertabular} 
\end{center} 

 \begin{center} 
 \begin{supertabular}{|c|c|c|c|}\hline 
 \multicolumn {4}{|c|}{\cellcolor{blue!10}  \nextTable \hypertarget{अन्वञ्च् (नपु॰) Following}{अन्वञ्च् (नपु॰) Following}}  \\ \hline  
 & एक॰ & द्वि॰ & बहु॰  \\ \hline 
 प्रथमा & \cpada अन्वक् & \cbham अनूची & \csarva अन्वञ्चि \\ \hline 
 द्वितीया & \cbham अन्वक् & \cbham अनूची & \csarva अन्वञ्चि \\ \hline 
सम्बोधन & हे अन्वक् & हे अनूची & हे अन्वञ्चि \\ \hline 
 \multicolumn{4}{|c|}{ शेषं अन्वञ्च्-पुंवत् शब्दवत् } \\ \hline 
\end{supertabular} 
\end{center} 

 \begin{center} 
 \begin{supertabular}{|c|c|c|c|}\hline 
 \multicolumn {4}{|c|}{\cellcolor{blue!10}  \nextTable \hypertarget{उदञ्च् (नपु॰) Northern}{उदञ्च् (नपु॰) Northern}}  \\ \hline  
 & एक॰ & द्वि॰ & बहु॰  \\ \hline 
 प्रथमा & \cpada उदक् & \cbham उदीची & \csarva उदञ्चि \\ \hline 
 द्वितीया & \cbham उदक् & \cbham उदीची & \csarva उदञ्चि \\ \hline 
सम्बोधन & हे उदक् & हे उदीची & हे उदञ्चि \\ \hline 
 \multicolumn{4}{|c|}{ शेषं उदञ्च्-पुंवत् शब्दवत् } \\ \hline 
\end{supertabular} 
\end{center} 

 \begin{center} 
 \begin{supertabular}{|c|c|c|c|}\hline 
 \multicolumn {4}{|c|}{\cellcolor{blue!10}  \nextTable \hypertarget{तिर्यञ्च् (नपु॰) Horizontal}{तिर्यञ्च् (नपु॰) Horizontal}}  \\ \hline  
 & एक॰ & द्वि॰ & बहु॰  \\ \hline 
 प्रथमा & \cpada तिर्यक् & \cbham तिरञ्ची & \csarva तिर्यञ्चि \\ \hline 
 द्वितीया & \cbham तिर्यक् & \cbham तिरञ्ची & \csarva तिर्यञ्ची \\ \hline 
सम्बोधन & हे तिर्यक् & हे तिरञ्ची & हे तिर्यञ्ची \\ \hline 
 \multicolumn{4}{|c|}{ शेषं तिर्यञ्च्-पुंवत् शब्दवत् } \\ \hline 
\end{supertabular} 
\end{center} 

 \begin{center} 
 \begin{supertabular}{|c|c|c|c|}\hline 
 \multicolumn {4}{|c|}{\cellcolor{blue!10}  \nextTable \hypertarget{प्राञ्च् (नपु॰) Eastern}{प्राञ्च् (नपु॰) Eastern}}  \\ \hline  
 & एक॰ & द्वि॰ & बहु॰  \\ \hline 
 प्रथमा & \cpada प्राङ् & \cbham प्राञ्ची & \csarva प्राञ्चि \\ \hline 
 द्वितीया & \cbham प्राङ् & \cbham प्राञ्ची & \csarva प्राञ्चि \\ \hline 
सम्बोधन & हे प्राङ् & हे प्राञ्ची & हे प्राञ्चि \\ \hline 
 \multicolumn{4}{|c|}{ शेषं प्राञ्च्-पुंवत् शब्दवत् } \\ \hline 
\end{supertabular} 
\end{center} 

 \begin{center} 
 \begin{supertabular}{|c|c|c|c|}\hline 
 \multicolumn {4}{|c|}{\cellcolor{blue!10}  \nextTable \hypertarget{प्रत्यञ्च् (नपु॰) Western}{प्रत्यञ्च् (नपु॰) Western}}  \\ \hline  
 & एक॰ & द्वि॰ & बहु॰  \\ \hline 
 प्रथमा & \cpada प्रत्यङ् & \cbham प्रत्यञ्ची & \csarva प्रत्यञ्चि \\ \hline 
 द्वितीया & \cbham प्रत्यङ् & \cbham प्रत्यञ्ची & \csarva प्रत्यञ्चि \\ \hline 
सम्बोधन & हे प्रत्यङ् & हे प्रत्यञ्ची & हे प्रत्यञ्चि \\ \hline 
 \multicolumn{4}{|c|}{ शेषं प्रत्यञ्च्-पुंवत् शब्दवत् } \\ \hline 
\end{supertabular} 
\end{center} 

 \begin{center} 
 \begin{supertabular}{|c|c|c|c|}\hline 
 \multicolumn {4}{|c|}{\cellcolor{blue!10}  \nextTable \hypertarget{अन्वञ्च् (नपु॰) Following}{अन्वञ्च् (नपु॰) Following}}  \\ \hline  
 & एक॰ & द्वि॰ & बहु॰  \\ \hline 
 प्रथमा & \cpada अन्वङ् & \cbham अन्वञ्ची & \csarva अन्वञ्चि \\ \hline 
 द्वितीया & \cbham अन्वङ् & \cbham अन्वञ्ची & \csarva अन्वञ्चि \\ \hline 
सम्बोधन & हे अन्वङ् & हे अन्वञ्ची & हे अन्वञ्चि \\ \hline 
 \multicolumn{4}{|c|}{ शेषं अन्वञ्च्-पुंवत् शब्दवत् } \\ \hline 
\end{supertabular} 
\end{center} 

 \begin{center} 
 \begin{supertabular}{|c|c|c|c|}\hline 
 \multicolumn {4}{|c|}{\cellcolor{blue!10}  \nextTable \hypertarget{उदञ्च् (नपु॰) Northern}{उदञ्च् (नपु॰) Northern}}  \\ \hline  
 & एक॰ & द्वि॰ & बहु॰  \\ \hline 
 प्रथमा & \cpada उदङ् & \cbham उदीञ्ची & \csarva उदञ्चि \\ \hline 
 द्वितीया & \cbham उदङ् & \cbham उदीञ्ची & \csarva उदञ्चि \\ \hline 
सम्बोधन & हे उदङ् & हे उदीञ्ची & हे उदञ्चि \\ \hline 
 \multicolumn{4}{|c|}{ शेषं उदञ्च्-पुंवत् शब्दवत् } \\ \hline 
\end{supertabular} 
\end{center} 

%\columnbreak
 \begin{center} 
 \begin{supertabular}{|c|c|c|c|}\hline 
 \multicolumn {4}{|c|}{\cellcolor{blue!10}  \nextTable \hypertarget{तिर्यञ्च् (नपु॰) Horizontal}{तिर्यञ्च् (नपु॰) Horizontal}}  \\ \hline  
 & एक॰ & द्वि॰ & बहु॰  \\ \hline 
 प्रथमा & \cpada तिर्यङ् & \cbham तिर्यञ्ची & \csarva तिर्यञ्चि \\ \hline 
 द्वितीया & \cbham तिर्यङ् & \cbham तिर्यञ्ची & \csarva तिर्यञ्चि \\ \hline 
सम्बोधन & हे तिर्यङ् & हे तिर्यञ्ची & हे तिर्यञ्चि \\ \hline 
 \multicolumn{4}{|c|}{ शेषं तिर्यञ्च्-पुंवत् शब्दवत् } \\ \hline 
\end{supertabular} 
\end{center} 

 \begin{center} 
 \begin{supertabular}{|c|c|c|c|}\hline 
 \multicolumn {4}{|c|}{\cellcolor{blue!10}  \nextTable \hypertarget{दुह् (नपु॰) One who Milks}{दुह् (नपु॰) One who Milks}}  \\ \hline  
 & एक॰ & द्वि॰ & बहु॰  \\ \hline 
 प्रथमा & \cpada धुक् & \cbham दुही & \csarva दुंहि \\ \hline 
 द्वितीया & \cbham धुक् & \cbham दुही & \csarva दुंहि \\ \hline 
सम्बोधन & हे धुक् & हे दुही & हे दुंहि \\ \hline 
 \multicolumn{4}{|c|}{ शेषं दुह्-पुंवत् शब्दवत् } \\ \hline 
\end{supertabular} 
\end{center} 

 \begin{center} 
 \begin{supertabular}{|c|c|c|c|}\hline 
 \multicolumn {4}{|c|}{\cellcolor{blue!10}  \nextTable \hypertarget{द्रुह् (नपु॰) One who bears hatred}{द्रुह् (नपु॰) One who bears hatred}}  \\ \hline  
 & एक॰ & द्वि॰ & बहु॰  \\ \hline 
 प्रथमा & \cpada ध्रुक्-ध्रुट् & \cbham द्रुंही & \csarva द्रुंहि \\ \hline 
 द्वितीया & \cbham ध्रुक्-ध्रुट् & \cbham द्रुंही & \csarva द्रुंहि \\ \hline 
सम्बोधन & हे ध्रुक्-ध्रुट् & हे द्रुंही & हे द्रुंहि \\ \hline 
 \multicolumn{4}{|c|}{ शेषं द्रुह्-पुंवत् शब्दवत् } \\ \hline 
\end{supertabular} 
\end{center} 

 \begin{center} 
 \begin{supertabular}{|c|c|c|c|}\hline 
 \multicolumn {4}{|c|}{\cellcolor{blue!10}  \nextTable \hypertarget{स्वनडुह् (नपु॰) One with a good Ox}{स्वनडुह् (नपु॰) One with a good Ox}}  \\ \hline  
 & एक॰ & द्वि॰ & बहु॰  \\ \hline 
 प्रथमा & \cpada स्वनडुत् & \cbham स्वनडुही & \csarva स्वनड्वांहि \\ \hline 
 द्वितीया & \cbham स्वनडुत् & \cbham स्वनडुही & \csarva स्वनड्वांहि \\ \hline 
सम्बोधन & हे स्वनडुत् & हे स्वनडुही & हे स्वनड्वांहि \\ \hline 
 \multicolumn{4}{|c|}{ शेषं अनडुह्-पुंवत् शब्दवत् } \\ \hline 
\end{supertabular} 
\end{center} 

\small
\begin{center} 
 \begin{supertabular}{|c|c|c|c|}\hline 
 \multicolumn {4}{|c|}{\cellcolor{blue!10}  \nextTable \hypertarget{निर्जर (पु॰) God}{निर्जर (पु॰) God}}  \\ \hline  
 & एक॰ & द्वि॰ & बहु॰  \\ \hline 
 प्रथमा & \csarva निर्जरः & \csarva निर्जरौ-रसौ & \csarva निर्जराः-रसः \\ \hline 
 द्वितीया & \csarva निर्जरम् & \csarva निर्जरौ-रसौ & \cbham निर्जराः-रसः \\ \hline 
 तृतीया & \cbham निर्जरेण-रसा & \cpada निरजाभ्याम् & \cpada निर्जरैः \\ \hline 
 चतुर्थी & \cbham निर्जराय-रसे & \cpada निर्जराभ्याम् & \cpada निर्जरेभ्यः \\ \hline 
 पञ्चमी & \cbham निर्जरात्-रसः & \cpada निर्जराभ्याम् & \cpada निर्जरेभ्यः \\ \hline 
 षष्ठी & \cbham निर्जरस्य-रसः & \cbham निर्जरयोः-रसोः & \cbham निर्जराणाम्-रसाम् \\ \hline 
सप्तमी & \cbham निर्जरे-रसि & \cbham निर्जरयोः-रसोः & \cpada निर्जरेषु \\ \hline 
सम्बोधन & हे निर्जर & हे निर्जरौ-रसौ &  हे निर्जराः-रसः \\ \hline 
\end{supertabular} 
\end{center} 
\end{multicols}

\normalsize
\clearpage
\section{सङ्ख्येयशब्दाः (Ordinals)}

%\the\abovedisplayskip \the\abovedisplayskip
%\begin{multicols}{2}
%\TrickSupertabularIntoMulticols

 \begin{center} 
 %Files to generate the below table are: 
 %c:/wattle/programs/python/cardinals.txt and
 %c:/wattle/programs/python/Ordinals.py
 \begin{supertabular}{|lc|c|c|c|}\hline  
 \multicolumn {5}{|c|}{\cellcolor{blue!10}  \nextTable \hypertarget{सङ्ख्येयशब्दाः (Ordinals)}{सङ्ख्येयशब्दाः (Ordinals)}}  \\ \hline  
 & & पु॰ & स्त्री॰ & नपु॰   \\ \hline 
 १ & एकः एका एकम् & प्रथमः & प्रथमा & प्रथमम्  \\ \hline 
 २ & द्वौ द्वे द्वे & द्वितीयः & द्वितीया & द्वितीयम् \\ \hline
 ३ & त्रयः तिस्रः त्रीणि & तृतीयः & तृतीया & तृतीयम् \\ \hline
 ४ & चत्वारः चतस्रः चत्वारि & चतुर्थः & चतुर्थी & चतुर्थम् \\ \hline
 ४ & चत्वारः चतस्रः चत्वारि & तुरीयः & तुरीया & तुरीयम् \\ \hline
 ४ & चत्वारः चतस्रः चत्वारि & तुर्यः & तुर्या & तुर्यम् \\ \hline
 ५ & पञ्च & पञ्चमः & पञ्चमी & पञ्चमम् \\ \hline
 ६ & षट् & षष्ठः & षष्ठी & षष्ठम् \\ \hline
 ७ & सप्त & सप्तमः & सप्तमी & सप्तमम् \\ \hline
 ८ & अष्ट, अष्टौ & अष्टमः & अष्टमी & अष्टमम् \\ \hline
 ९ & नव & नवमः & नवमी & नवमम् \\ \hline
 १० & दश & दशमः & दशमी & दशमम् \\ \hline
 ११ & एकादश & एकादशः & एकादशी & एकादशम् \\ \hline
 १२ & द्वादश & द्वादशः & द्वादशी & द्वादशम् \\ \hline
 १३ & त्रयोदश & त्रयोदशः & त्रयोदशी & त्रयोदशम् \\ \hline
 १४ & चतुर्दश & चतुर्दशः & चतुर्दशी & चतुर्दशम् \\ \hline
 १५ & पञ्चदश & पञ्चदशः & पञ्चदशी & पञ्चदशम् \\ \hline
१६ & षोडश & षोडशः & षोडशी & षोडशम् \\ \hline 
१७ & सप्तदश & सप्तदशः & सप्तदशी & सप्तदशम् \\ \hline 
१८ & अष्टादश & अष्टादशः & अष्टादशी & अष्टादशम् \\ \hline 
१९ & नवदश & नवदशः & नवदशी & नवदशम् \\ \hline 
१९ & एकोनविंशः & एकोनविंशः & एकोनविंशी & एकोनविंशम् \\ \hline 
१९ & एकोन्नविंशतिः & एकोन्नविंशतः, एकोन्नविंशतितमः & एकोन्नविंशती, एकोन्नविंशतितमी & एकोन्नविंशतम्, एकोन्नविंशतितमम् \\ \hline 
१९ & उनविंशतिः & उनविंशतः, उनविंशतितमः & उनविंशती, उनविंशतितमी & उनविंशतम्, उनविंशतितमम् \\ \hline 
२० & विंशतिः & विंशतः, विंशतितमः & विंशती, विंशतितमी & विंशतम्, विंशतितमम् \\ \hline 
२१ & एकविंशतिः & एकविंशतः, एकविंशतितमः & एकविंशती, एकविंशतितमी & एकविंशतम्, एकविंशतितमम् \\ \hline 
२२ & द्वाविंशतिः & द्वाविंशतः, द्वाविंशतितमः & द्वाविंशती, द्वाविंशतितमी & द्वाविंशतम्, द्वाविंशतितमम् \\ \hline 
२३ & त्रयोविंशतिः & त्रयोविंशतः, त्रयोविंशतितमः & त्रयोविंशती, त्रयोविंशतितमी & त्रयोविंशतम्, त्रयोविंशतितमम् \\ \hline 
२४ & चतुर्विंशतिः & चतुर्विंशतः, चतुर्विंशतितमः & चतुर्विंशती, चतुर्विंशतितमी & चतुर्विंशतम्, चतुर्विंशतितमम् \\ \hline 
२५ & पञ्चविंशतिः & पञ्चविंशतः, पञ्चविंशतितमः & पञ्चविंशती, पञ्चविंशतितमी & पञ्चविंशतम्, पञ्चविंशतितमम् \\ \hline 
२६ & षड्विंशतिः & षड्विंशतः, षड्विंशतितमः & षड्विंशती, षड्विंशतितमी & षड्विंशतम्, षड्विंशतितमम् \\ \hline 
२७ & सप्तविंशतिः & सप्तविंशतः, सप्तविंशतितमः & सप्तविंशती, सप्तविंशतितमी & सप्तविंशतम्, सप्तविंशतितमम् \\ \hline 
२८ & अष्टाविंशतिः & अष्टाविंशतः, अष्टाविंशतितमः & अष्टाविंशती, अष्टाविंशतितमी & अष्टाविंशतम्, अष्टाविंशतितमम् \\ \hline 
२९ & नवविंशतिः & नवविंशतः, नवविंशतितमः & नवविंशती, नवविंशतितमी & नवविंशतम्, नवविंशतितमम् \\ \hline 
३० & त्रिंशत् & त्रिंशः, त्रिंशत्तमः & त्रिंशी, त्रिंशत्तमी & त्रिंशम्, त्रिंशत्तमम् \\ \hline 
३१ & एकत्रिंशत् & एकत्रिंशः, एकत्रिंशत्तमः & एकत्रिंशी, एकत्रिंशत्तमी & एकत्रिंशम्, एकत्रिंशत्तमम् \\ \hline 
३२ & द्वात्रिंशत् & द्वात्रिंशः, द्वात्रिंशत्तमः & द्वात्रिंशी, द्वात्रिंशत्तमी & द्वात्रिंशम्, द्वात्रिंशत्तमम् \\ \hline 
३३ & त्रयत्रिंशत् & त्रयत्रिंशः, त्रयत्रिंशत्तमः & त्रयत्रिंशी, त्रयत्रिंशत्तमी & त्रयत्रिंशम्, त्रयत्रिंशत्तमम् \\ \hline 
३४ & चतुस्त्रिंशत् & चतुस्त्रिंशः, चतुस्त्रिंशत्तमः & चतुस्त्रिंशी, चतुस्त्रिंशत्तमी & चतुस्त्रिंशम्, चतुस्त्रिंशत्तमम् \\ \hline 
३५ & पञ्चत्रिंशत् & पञ्चत्रिंशः, पञ्चत्रिंशत्तमः & पञ्चत्रिंशी, पञ्चत्रिंशत्तमी & पञ्चत्रिंशम्, पञ्चत्रिंशत्तमम् \\ \hline 
३६ & षट्त्रिंशत् & षट्त्रिंशः, षट्त्रिंशत्तमः & षट्त्रिंशी, षट्त्रिंशत्तमी & षट्त्रिंशम्, षट्त्रिंशत्तमम् \\ \hline 
३७ & सप्तत्रिंशत् & सप्तत्रिंशः, सप्तत्रिंशत्तमः & सप्तत्रिंशी, सप्तत्रिंशत्तमी & सप्तत्रिंशम्, सप्तत्रिंशत्तमम् \\ \hline 
३८ & अष्टात्रिंशत् & अष्टात्रिंशः, अष्टात्रिंशत्तमः & अष्टात्रिंशी, अष्टात्रिंशत्तमी & अष्टात्रिंशम्, अष्टात्रिंशत्तमम् \\ \hline 
३९ & नवत्रिंशत् & नवत्रिंशः, नवत्रिंशत्तमः & नवत्रिंशी, नवत्रिंशत्तमी & नवत्रिंशम्, नवत्रिंशत्तमम् \\ \hline 
४० & चत्वरिंशत् & चत्वरिंशः, चत्वरिंशत्तमः & चत्वरिंशी, चत्वरिंशत्तमी & चत्वरिंशम्, चत्वरिंशत्तमम् \\ \hline 
४१ & एकचत्वरिंशत् & एकचत्वरिंशः, एकचत्वरिंशत्तमः & एकचत्वरिंशी, एकचत्वरिंशत्तमी & एकचत्वरिंशम्, एकचत्वरिंशत्तमम् \\ \hline 
४२ & द्विचत्वारिंशत् & द्विचत्वारिंशः, द्विचत्वारिंशत्तमः & द्विचत्वारिंशी, द्विचत्वारिंशत्तमी & द्विचत्वारिंशम्, द्विचत्वारिंशत्तमम् \\ \hline 
४२ & द्वाचत्वरिंशत् & द्वाचत्वरिंशः, द्वाचत्वरिंशत्तमः & द्वाचत्वरिंशी, द्वाचत्वरिंशत्तमी & द्वाचत्वरिंशम्, द्वाचत्वरिंशत्तमम् \\ \hline 
४३ & त्रीचत्वरिंशत् & त्रीचत्वरिंशः, त्रीचत्वरिंशत्तमः & त्रीचत्वरिंशी, त्रीचत्वरिंशत्तमी & त्रीचत्वरिंशम्, त्रीचत्वरिंशत्तमम् \\ \hline 
४३ & त्रयस्चत्वरिंशत् & त्रयस्चत्वरिंशः, त्रयस्चत्वरिंशत्तमः & त्रयस्चत्वरिंशी, त्रयस्चत्वरिंशत्तमी & त्रयस्चत्वरिंशम्, त्रयस्चत्वरिंशत्तमम् \\ \hline 
४४ & चतुश्चत्वरिंशत् & चतुश्चत्वरिंशः, चतुश्चत्वरिंशत्तमः & चतुश्चत्वरिंशी, चतुश्चत्वरिंशत्तमी & चतुश्चत्वरिंशम्, चतुश्चत्वरिंशत्तमम् \\ \hline 
४५ & पञ्चचत्वरिंशत् & पञ्चचत्वरिंशः, पञ्चचत्वरिंशत्तमः & पञ्चचत्वरिंशी, पञ्चचत्वरिंशत्तमी & पञ्चचत्वरिंशम्, पञ्चचत्वरिंशत्तमम् \\ \hline 
४६ & षट्चत्वरिंशत् & षट्चत्वरिंशः, षट्चत्वरिंशत्तमः & षट्चत्वरिंशी, षट्चत्वरिंशत्तमी & षट्चत्वरिंशम्, षट्चत्वरिंशत्तमम् \\ \hline 
४७ & सप्तचत्वरिंशत् & सप्तचत्वरिंशः, सप्तचत्वरिंशत्तमः & सप्तचत्वरिंशी, सप्तचत्वरिंशत्तमी & सप्तचत्वरिंशम्, सप्तचत्वरिंशत्तमम् \\ \hline 
४८ & अष्टचत्वरिंशत् & अष्टचत्वरिंशः, अष्टचत्वरिंशत्तमः & अष्टचत्वरिंशी, अष्टचत्वरिंशत्तमी & अष्टचत्वरिंशम्, अष्टचत्वरिंशत्तमम् \\ \hline 
४८ & अष्टाचत्वरिंशत् & अष्टाचत्वरिंशः, अष्टाचत्वरिंशत्तमः & अष्टाचत्वरिंशी, अष्टाचत्वरिंशत्तमी & अष्टाचत्वरिंशम्, अष्टाचत्वरिंशत्तमम् \\ \hline 
४९ & नवचत्वरिंशत् & नवचत्वरिंशः, नवचत्वरिंशत्तमः & नवचत्वरिंशी, नवचत्वरिंशत्तमी & नवचत्वरिंशम्, नवचत्वरिंशत्तमम् \\ \hline 
५० & पञ्चाशत् & पञ्चाशः, पञ्चाशत्तमः & पञ्चाशी, पञ्चाशत्तमी & पञ्चाशम्, पञ्चाशत्तमम् \\ \hline 
५१ & एकपञ्चाशत् & एकपञ्चाशः, एकपञ्चाशत्तमः & एकपञ्चाशी, एकपञ्चाशत्तमी & एकपञ्चाशम्, एकपञ्चाशत्तमम् \\ \hline 
५२ & द्विपञ्चाशत् & द्विपञ्चाशः, द्विपञ्चाशत्तमः & द्विपञ्चाशी, द्विपञ्चाशत्तमी & द्विपञ्चाशम्, द्विपञ्चाशत्तमम् \\ \hline 
५२ & द्वापञ्चाशत् & द्वापञ्चाशः, द्वापञ्चाशत्तमः & द्वापञ्चाशी, द्वापञ्चाशत्तमी & द्वापञ्चाशम्, द्वापञ्चाशत्तमम् \\ \hline 
५३ & त्रिपञ्चाशत् & त्रिपञ्चाशः, त्रिपञ्चाशत्तमः & त्रिपञ्चाशी, त्रिपञ्चाशत्तमी & त्रिपञ्चाशम्, त्रिपञ्चाशत्तमम् \\ \hline 
५३ & त्रयःपञ्चाशत् & त्रयःपञ्चाशः, त्रयःपञ्चाशत्तमः & त्रयःपञ्चाशी, त्रयःपञ्चाशत्तमी & त्रयःपञ्चाशम्, त्रयःपञ्चाशत्तमम् \\ \hline 
५४ & चतुःपञ्चाशत् & चतुःपञ्चाशः, चतुःपञ्चाशत्तमः & चतुःपञ्चाशी, चतुःपञ्चाशत्तमी & चतुःपञ्चाशम्, चतुःपञ्चाशत्तमम् \\ \hline 
५५ & पञ्चपञ्चाशत् & पञ्चपञ्चाशः, पञ्चपञ्चाशत्तमः & पञ्चपञ्चाशी, पञ्चपञ्चाशत्तमी & पञ्चपञ्चाशम्, पञ्चपञ्चाशत्तमम् \\ \hline 
५६ & षट्पञ्चाशत् & षट्पञ्चाशः, षट्पञ्चाशत्तमः & षट्पञ्चाशी, षट्पञ्चाशत्तमी & षट्पञ्चाशम्, षट्पञ्चाशत्तमम् \\ \hline 
५७ & सप्तपञ्चाशत् & सप्तपञ्चाशः, सप्तपञ्चाशत्तमः & सप्तपञ्चाशी, सप्तपञ्चाशत्तमी & सप्तपञ्चाशम्, सप्तपञ्चाशत्तमम् \\ \hline 
५८ & अष्टपञ्चाशत् & अष्टपञ्चाशः, अष्टपञ्चाशत्तमः & अष्टपञ्चाशी, अष्टपञ्चाशत्तमी & अष्टपञ्चाशम्, अष्टपञ्चाशत्तमम् \\ \hline 
५८ & अष्टापञ्चाशत् & अष्टापञ्चाशः, अष्टापञ्चाशत्तमः & अष्टापञ्चाशी, अष्टापञ्चाशत्तमी & अष्टापञ्चाशम्, अष्टापञ्चाशत्तमम् \\ \hline 
५९ & नवपञ्चाशत् & नवपञ्चाशः, नवपञ्चाशत्तमः & नवपञ्चाशी, नवपञ्चाशत्तमी & नवपञ्चाशम्, नवपञ्चाशत्तमम् \\ \hline 
६० & षष्टिः & षष्टितमः & षष्टितमी & षष्टितमम् \\ \hline 
६१ & एकषष्टिः & एकषष्टः, एकषष्टितमः & एकषष्टी, एकषष्टितमी & एकषष्टम्, एकषष्टितमम् \\ \hline 
६२ & द्विषष्टिः & द्विषष्टः, द्विषष्टितमः & द्विषष्टी, द्विषष्टितमी & द्विषष्टम्, द्विषष्टितमम् \\ \hline 
६२ & द्वाषष्टिः & द्वाषष्टः, द्वाषष्टितमः & द्वाषष्टी, द्वाषष्टितमी & द्वाषष्टम्, द्वाषष्टितमम् \\ \hline 
६३ & त्रिषष्टिः & त्रिषष्टः, त्रिषष्टितमः & त्रिषष्टी, त्रिषष्टितमी & त्रिषष्टम्, त्रिषष्टितमम् \\ \hline 
६३ & त्रयःषष्टिः & त्रयःषष्टः, त्रयःषष्टितमः & त्रयःषष्टी, त्रयःषष्टितमी & त्रयःषष्टम्, त्रयःषष्टितमम् \\ \hline 
६४ & चतुःषष्टिः & चतुःषष्टः, चतुःषष्टितमः & चतुःषष्टी, चतुःषष्टितमी & चतुःषष्टम्, चतुःषष्टितमम् \\ \hline 
६५ & पञ्चषष्टिः & पञ्चषष्टः, पञ्चषष्टितमः & पञ्चषष्टी, पञ्चषष्टितमी & पञ्चषष्टम्, पञ्चषष्टितमम् \\ \hline 
६६ & षट्षष्टिः & षट्षष्टः, षट्षष्टितमः & षट्षष्टी, षट्षष्टितमी & षट्षष्टम्, षट्षष्टितमम् \\ \hline 
६७ & सप्तषष्टिः & सप्तषष्टः, सप्तषष्टितमः & सप्तषष्टी, सप्तषष्टितमी & सप्तषष्टम्, सप्तषष्टितमम् \\ \hline 
६८ & अष्टषष्टिः & अष्टषष्टः, अष्टषष्टितमः & अष्टषष्टी, अष्टषष्टितमी & अष्टषष्टम्, अष्टषष्टितमम् \\ \hline 
६८ & अष्टाषष्टिः & अष्टाषष्टः, अष्टाषष्टितमः & अष्टाषष्टी, अष्टाषष्टितमी & अष्टाषष्टम्, अष्टाषष्टितमम् \\ \hline 
६९ & नवषष्टिः & नवषष्टः, नवषष्टितमः & नवषष्टी, नवषष्टितमी & नवषष्टम्, नवषष्टितमम् \\ \hline 
७० & सप्ततिः & सप्ततितमः & सप्ततितमी & सप्ततितमम् \\ \hline 
७१ & एकसप्ततिः & एकसप्ततः, एकसप्ततितमः & एकसप्तती, एकसप्ततितमी & एकसप्ततम्, एकसप्ततितमम् \\ \hline 
७२ & द्विसप्ततिः & द्विसप्ततः, द्विसप्ततितमः & द्विसप्तती, द्विसप्ततितमी & द्विसप्ततम्, द्विसप्ततितमम् \\ \hline 
७२ & द्वासप्ततिः & द्वासप्ततः, द्वासप्ततितमः & द्वासप्तती, द्वासप्ततितमी & द्वासप्ततम्, द्वासप्ततितमम् \\ \hline 
७३ & त्रिसप्ततिः & त्रिसप्ततः, त्रिसप्ततितमः & त्रिसप्तती, त्रिसप्ततितमी & त्रिसप्ततम्, त्रिसप्ततितमम् \\ \hline 
७३ & त्रयःसप्ततिः & त्रयःसप्ततः, त्रयःसप्ततितमः & त्रयःसप्तती, त्रयःसप्ततितमी & त्रयःसप्ततम्, त्रयःसप्ततितमम् \\ \hline 
७४ & चतुःसप्ततिः & चतुःसप्ततः, चतुःसप्ततितमः & चतुःसप्तती, चतुःसप्ततितमी & चतुःसप्ततम्, चतुःसप्ततितमम् \\ \hline 
७५ & पञ्चसप्ततिः & पञ्चसप्ततः, पञ्चसप्ततितमः & पञ्चसप्तती, पञ्चसप्ततितमी & पञ्चसप्ततम्, पञ्चसप्ततितमम् \\ \hline 
७६ & षट्सप्ततिः & षट्सप्ततः, षट्सप्ततितमः & षट्सप्तती, षट्सप्ततितमी & षट्सप्ततम्, षट्सप्ततितमम् \\ \hline 
७७ & सप्तसप्ततिः & सप्तसप्ततः, सप्तसप्ततितमः & सप्तसप्तती, सप्तसप्ततितमी & सप्तसप्ततम्, सप्तसप्ततितमम् \\ \hline 
७८ & अष्टसप्ततिः & अष्टसप्ततः, अष्टसप्ततितमः & अष्टसप्तती, अष्टसप्ततितमी & अष्टसप्ततम्, अष्टसप्ततितमम् \\ \hline 
७८ & अष्टासप्ततिः & अष्टासप्ततः, अष्टासप्ततितमः & अष्टासप्तती, अष्टासप्ततितमी & अष्टासप्ततम्, अष्टासप्ततितमम् \\ \hline 
७९ & नवसप्ततिः & नवसप्ततः, नवसप्ततितमः & नवसप्तती, नवसप्ततितमी & नवसप्ततम्, नवसप्ततितमम् \\ \hline 
८० & अशीतिः & अशीतितमः & अशीतितमी & अशीतितमम् \\ \hline 
८१ & एकाशीतिः & एकाशीतः, एकाशीतितमः & एकाशीती, एकाशीतितमी & एकाशीतम्, एकाशीतितमम् \\ \hline 
८२ & द्व्यशीतिः & द्व्यशीतः, द्व्यशीतितमः & द्व्यशीती, द्व्यशीतितमी & द्व्यशीतम्, द्व्यशीतितमम् \\ \hline 
८३ & त्र्यशीतिः & त्र्यशीतः, त्र्यशीतितमः & त्र्यशीती, त्र्यशीतितमी & त्र्यशीतम्, त्र्यशीतितमम् \\ \hline 
८४ & चतुरशीतिः & चतुरशीतः, चतुरशीतितमः & चतुरशीती, चतुरशीतितमी & चतुरशीतम्, चतुरशीतितमम् \\ \hline 
८५ & पञ्चाशीतिः & पञ्चाशीतः, पञ्चाशीतितमः & पञ्चाशीती, पञ्चाशीतितमी & पञ्चाशीतम्, पञ्चाशीतितमम् \\ \hline 
८६ & षटशीतिः & षटशीतः, षटशीतितमः & षटशीती, षटशीतितमी & षटशीतम्, षटशीतितमम् \\ \hline 
८७ & सप्ताशीतिः & सप्ताशीतः, सप्ताशीतितमः & सप्ताशीती, सप्ताशीतितमी & सप्ताशीतम्, सप्ताशीतितमम् \\ \hline 
८८ & अष्टाशीतिः & अष्टाशीतः, अष्टाशीतितमः & अष्टाशीती, अष्टाशीतितमी & अष्टाशीतम्, अष्टाशीतितमम् \\ \hline 
८९ & नवाशीतिः & नवाशीतः, नवाशीतितमः & नवाशीती, नवाशीतितमी & नवाशीतम्, नवाशीतितमम् \\ \hline 
९० & नवतिः & नवतितमः & नवतितमी & नवतितमम् \\ \hline 
९१ & एकनवतिः & एकनवतः, एकनवतितमः & एकनवती, एकनवतितमी & एकनवतम्, एकनवतितमम् \\ \hline 
९२ & द्विनवतिः & द्विनवतः, द्विनवतितमः & द्विनवती, द्विनवतितमी & द्विनवतम्, द्विनवतितमम् \\ \hline 
९२ & द्वानवतिः & द्वानवतः, द्वानवतितमः & द्वानवती, द्वानवतितमी & द्वानवतम्, द्वानवतितमम् \\ \hline 
९३ & त्रिनवतिः & त्रिनवतः, त्रिनवतितमः & त्रिनवती, त्रिनवतितमी & त्रिनवतम्, त्रिनवतितमम् \\ \hline 
९३ & त्रयोनवतिः & त्रयोनवतः, त्रयोनवतितमः & त्रयोनवती, त्रयोनवतितमी & त्रयोनवतम्, त्रयोनवतितमम् \\ \hline 
९४ & चतुर्नवतिः & चतुर्नवतः, चतुर्नवतितमः & चतुर्नवती, चतुर्नवतितमी & चतुर्नवतम्, चतुर्नवतितमम् \\ \hline 
९५ & पञ्चनवतिः & पञ्चनवतः, पञ्चनवतितमः & पञ्चनवती, पञ्चनवतितमी & पञ्चनवतम्, पञ्चनवतितमम् \\ \hline 
९६ & षण्नवतिः & षण्नवतः, षण्नवतितमः & षण्नवती, षण्नवतितमी & षण्नवतम्, षण्नवतितमम् \\ \hline 
९७ & सप्तनवतिः & सप्तनवतः, सप्तनवतितमः & सप्तनवती, सप्तनवतितमी & सप्तनवतम्, सप्तनवतितमम् \\ \hline 
९८ & अष्टनवतिः & अष्टनवतः, अष्टनवतितमः & अष्टनवती, अष्टनवतितमी & अष्टनवतम्, अष्टनवतितमम् \\ \hline 
९८ & अष्टानवतिः & अष्टानवतः, अष्टानवतितमः & अष्टानवती, अष्टानवतितमी & अष्टानवतम्, अष्टानवतितमम् \\ \hline 
९९ & नवनवतिः & नवनवतः, नवनवतितमः & नवनवती, नवनवतितमी & नवनवतम्, नवनवतितमम् \\ \hline 
१०० & शतम् & शततमः & शततमी & शततमम् \\ \hline 
१००० & सहस्रम् & सहस्रतमः & सहस्रतमी & सहस्रतमम् \\ \hline 
१०००० & अयुतम् & अयुततमः & अयुततमी & अयुततमम् \\ \hline 
१००००० & लक्षा,लक्षम् & लक्षतमः & लक्षतमी & लक्षतमम् \\ \hline 
१०००००० & प्रयुतम् & प्रयुततमः & प्रयुततमी & प्रयुततमम् \\ \hline 
१००००००० & कोटिः & कोटितमः & कोटितमी & कोटितमम् \\ \hline 
\end{supertabular} 
\end{center} 
% & & & \\ \hline

%\end{multicols}


\clearpage
\section{Word Index}
%\nextWord{नामन्}  नामन् नकारान्तः नपुंसकलिङ्गः शब्दः \\ \wref{नामन्}
In the list below the first 25 words are in the order in which they are listed in the पञ्चविंशति-प्रकार-शब्दानां रूपावलिः book. After that words with regular declension are organised in the Devanagari alphabetical order with respect to their endings; this is followed by pronouns and then words for numbers; finally special words are listed as per the शब्द मञ्जरी book. Remember, in most pdf viewers ``Alt + left-arrow'' takes back after visiting a link. The number to the left of the words is its विभक्ति-table number.

The words and comments in the list below are taken from various sources; mainly from the two books: बृहद्-अनुवाद-चन्द्रिका and शब्द मञ्जरी.
%॥ ।
\begin{multicols}{3}
\resetWord
 \nextWord{नामन् (नपु॰) Name} \hyperlink{नामन् (नपु॰) Name}{नामन् (नपु॰) Name}, व्योमन्, धामन्, सामन्, दामन्, प्रेमन् आदि \\ %\wref{नामन् (नपु॰) Name} \\
 \nextWord{जन्मन् (नपु॰) Birth} \hyperlink{जन्मन् (नपु॰) Birth}{जन्मन् (नपु॰) Birth}, शर्मन्, पर्वन्, ब्रह्मन्, वर्मन्, वर्त्मन्, चर्मन् आदि \\ 
 \nextWord{वारि (नपु॰) Water} \hyperlink{वारि (नपु॰) Water}{वारि (नपु॰) Water}, except the following four, अस्थि, सक्थि, अक्षि, and दधि, all  इकारन्तः नपुंसकलिङ् have the same declension as वारि \\ 
%\nextWord{दधि इकारान्तः नपुंसकलिङ्गः शब्दः} \hyperlink{दधि इकारान्तः नपुंसकलिङ्गः शब्दः}{दधि इकारान्तः नपुंसकलिङ्गः शब्दः} \\ 
 \nextWord{दधि (नपु॰) Curd} \hyperlink{दधि (नपु॰) Curd}{दधि (नपु॰) Curd}, अस्थि, सक्थि, and अक्षि  \\ 
 \nextWord{मधु (नपु॰) Honey} \hyperlink{मधु (नपु॰) Honey}{मधु (नपु॰) Honey}, जानु, दारु, जतु, जत्रु, तालु, वस्तु, सानु, अम्बु, अश्रु आदि \\ 
 \nextWord{जगत् (नपु॰) World} \hyperlink{जगत् (नपु॰) World}{जगत् (नपु॰) World} \\ 
 \nextWord{मनस् (नपु॰) Mind} \hyperlink{मनस् (नपु॰) Mind}{मनस् (नपु॰) Mind}, नभस्, अम्भस्, आगस्, उरस्, पयस्, रजस्, वयस्, वक्षस्, अयस्, तमस्, वचस्, यशस्, तपस्, सरस्, शिरस् आदि \\ 
 \nextWord{ज्योतिस् (नपु॰) Light} \hyperlink{ज्योतिस् (नपु॰) Light}{ज्योतिस् (नपु॰) Light}, धनुस्, वपुस्, चक्षुस्, आयुस्, यजुस् इत्यादयः उस्-अन्ताः शब्दाः \\ 
 \nextWord{फल (नपु॰) Fruit} \hyperlink{फल (नपु॰) Fruit}{फल (नपु॰) Fruit}, वन, अरण्य, मुख, कुसुम, पुष्प, कमल, वर्ण, मित्र, नक्षत्र, पत्र, तृण, बीज, जल, गगन, शरीर, ज्ञान, पुस्तक आदि \\ 
 \nextWord{अभिजित् (पु॰) Victorious} \hyperlink{अभिजित् (पु॰) Victorious}{अभिजित् (पु॰) Victorious} \\ 
 \nextWord{धीमत् (पु॰) Intelligent} \hyperlink{धीमत् (पु॰) Intelligent}{धीमत् (पु॰) Intelligent}, बुद्धिमत्, भानुमत्, श्रीमत्, सानुमत्, अंशुमत्, विद्यावत्, धनुष्मत्, बलवत्, भगवत्, भाग्यवत्, उक्तवत्, गतवत्, श्रुतवत् आदि; धीमत्, बुद्धिमत्, बलवत् स्त्रीलिङ्गप्रयोगे धीमती, बुद्धिमती, बलवती, इत्यादयः भवन्ति, नदी शब्दवत् रूपाणि च भवन्ति \\ 
 \nextWord{सरित् (स्त्री॰) River} \hyperlink{सरित् (स्त्री॰) River}{सरित् (स्त्री॰) River}, विद्युत्, हरित्, योषित्, त्रिंशत्, चत्वारिंशत् आदि \\ 
 \nextWord{खादत् (पु॰) The Eating One} \hyperlink{खादत् (पु॰) The Eating One}{खादत् (पु॰) The Eating One}, गच्छत्, धावत्, वदत्, पठत्, पश्यत्, पतत्, गृह्णत्, शोचत्, भवत्, पिबत्, गमिष्यत्, हरिष्यत्, करिष्यत्, कथयिष्यत् \\ 
 \nextWord{ज्ञानिन् (पु॰) Wise} \hyperlink{ज्ञानिन् (पु॰) Wise}{ज्ञानिन् (पु॰) Wise}, गुणिन्, करिन्, हस्तिन्, मन्त्रिन्, पक्षिन्, शशिन्, धनिन्, वाजिन्, तपस्विन्, बलिन्, सुखिन्, एकाकिन्, सत्यवादिन् आदि; स्त्रीलिङ्गे ज्ञानिनी, गुणिनी, करिणी, हस्तिनी नदीवत् \\ 
 \nextWord{राम (पु॰) Ram} \hyperlink{राम (पु॰) Ram}{राम (पु॰) Ram}, नरः, बालः, जनकः, नृपः, भक्तः, शिष्यः, सूर्यः, चन्द्रः, सुरः, खगः, मयूरः, प्रश्नः, क्रोशः, लोकः, धर्मः, अनलः, प्राज्ञः, सज्जनः, दुर्जनः, खलः, करः, पिकः, वंशः, वानरः, गजः, अनिलः, वृकः, नक्रः, रासभः, उपहारः। \\ 
 \nextWord{लता (स्त्री॰) Creeper} \hyperlink{लता (स्त्री॰) Creeper}{लता (स्त्री॰) Creeper}, रमा, बाला, ललना, कन्या, निशा, भार्या, बडवा, सुमित्रा, राधा, तारा, कौशल्या, क्षमा, लज्जा, कला आदि \\ 
 \nextWord{रवि (पु॰) Sun} \hyperlink{रवि (पु॰) Sun}{रवि (पु॰) Sun}, कवि, मुनि, कपि, ऋषि, यति, विरञ्चि, विधि, निधि, गिरि, अग्नि, अरि, वह्नि, सप्ति, रवि, नृपति, उदधि, अतिथि, असि, पाणि, मरीचि, व्याधि, सेनापति, प्रजापति, भूपति, महीपति, नरपति, लोकपति, सुरपति, गजपति, अधिपति, जगत्पति, बृहस्पति, पृथ्वीपति, गृहपति आदि \\ 
 \nextWord{मति (स्त्री॰) Idea} \hyperlink{मति (स्त्री॰) Idea}{मति (स्त्री॰) Idea}, धूलि, बुद्धि, शुद्धि, गति, भक्ति, शक्ति, स्मृति, रुचि, शान्ति, रीति, नीति, रात्रि, पङ्क्ति, जाति, गीति आदि; मति, श्रुति, रुचि, etc., have two forms in the singular of चतुर्थी, पञ्चमी, षष्ठी, and सप्तमी \\ 
 \nextWord{नदी (स्त्री॰) River} \hyperlink{नदी (स्त्री॰) River}{नदी (स्त्री॰) River}, राज्ञी, पार्वती, गौरी, जानकी, नटी, पृथ्वी, अरुन्धती, नन्दिनी, द्रौपदी, देवी, कैकेयी, पाञ्चाली, त्रिलोकी, पञ्चवटी, अटवी, गान्धारी, कादम्बरी, कौमुदी, माद्री, कुन्ती, देवकी, सावित्री, गायत्री, कमलिनी, नलिनी, वाणी आदि \\ 
 \nextWord{साधु (पु॰) Saint} \hyperlink{साधु (पु॰) Saint}{साधु (पु॰) Saint}, भानु, कृशानु, विधु, रिपु, शत्रु, विष्णु, शम्भु, शिशु, गुरु, ऊरु, प्रभु, वेणु, पांशु, वायु, मृत्यु, बाहु, मृदु, पृथु, लघु आदि \\ 
 \nextWord{धेनु (स्त्री॰) Cow} \hyperlink{धेनु (स्त्री॰) Cow}{धेनु (स्त्री॰) Cow}, तनु, रेणु, हनु, इषु, रज्जु, चञ्चु आदि \\ 
 \nextWord{वधू (स्त्री॰) Bride} \hyperlink{वधू (स्त्री॰) Bride}{वधू (स्त्री॰) Bride}, चमू, तनू, रज्जू, श्वश्रू, कर्कन्धू, जम्बू आदि \\ 
 \nextWord{पितृ (पु॰) Father} \hyperlink{पितृ (पु॰) Father}{पितृ (पु॰) Father}, भ्रातृ, जामातृ, देवृ, सव्येष्टृ आदि \\ 
 \nextWord{कर्तृ (पु॰) Doer} \hyperlink{कर्तृ (पु॰) Doer}{कर्तृ (पु॰) Doer}, वक्तृ, धातृ, दातृ, गन्तृ, नेतृ, श्रोतृ, नप्तृ, सवितृ, भर्तृ, द्रष्टृ, आदि। विशेष - तृन् और तृच् प्रत्ययान्त शब्दों के एवं स्वसृ, नेष्टृ, नप्तृ, त्वष्टृ, क्षत्तृ, प्रशास्तृ, होतृ और पोतृ के आगे जब प्रथमा और द्वितीया विभक्ति के प्रत्यय आवें तब ऋ के आदिष्ट रूप अ को दीर्घ हो जाता है। स्त्रीलिङ्गे दात्री धात्री कर्त्री नदीवत् \\ 
 \nextWord{आत्मन् (पु॰) Self} \hyperlink{आत्मन् (पु॰) Self}{आत्मन् (पु॰) Self}, अश्मन्, यज्वन्, अध्वन्, ब्रह्मन्, सुशर्मन्, कृतवर्मन् आदि \\ 
 \nextWord{भवादृश (पु॰) Like You} \hyperlink{भवादृश (पु॰) Like You}{भवादृश (पु॰) Like You}, तादृश, मादृश, त्वादृश, यादृश, एतादृश आदि \\ 
 \nextWord{विश्वपा (पु॰) Protector of the Universe} \hyperlink{विश्वपा (पु॰) Protector of the Universe}{विश्वपा (पु॰) Protector of the Universe}, सोमपा, धूम्रपा, गोपा, शंखध्मा, बलदा, आदि \\ 
 \nextWord{पति (पु॰) Husband} \hyperlink{पति (पु॰) Husband}{पति (पु॰) Husband}; पति and सखि are irregular bases in इकारान्त. The compound words ending with पति however are declined like रवि, e.g., श्रीपति, सीतापति, भूपति, नृपाति \\ 
 \nextWord{सखि (पु॰) Friend} \hyperlink{सखि (पु॰) Friend}{सखि (पु॰) Friend} \\ 
 \nextWord{अक्षि (नपु॰) Eye} \hyperlink{अक्षि (नपु॰) Eye}{अक्षि (नपु॰) Eye} \\ 
 \nextWord{शुचि (नपु॰) Clean} \hyperlink{शुचि (नपु॰) Clean}{शुचि (नपु॰) Clean}, for इकारान्त and उकारान्त  नपुंसकलिङ्गः adjectives, the corresponding पुँल्लिङ्गः forms are used in चतुर्थी, पञ्चमी, षष्ठी, and सप्तमी एकवचन \\ 
 \nextWord{प्रधी (पु॰) Great Intelligence} \hyperlink{प्रधी (पु॰) Great Intelligence}{प्रधी (पु॰) Great Intelligence} \\ 
 \nextWord{सुधी (पु॰) Intelligence} \hyperlink{सुधी (पु॰) Intelligence}{सुधी (पु॰) Intelligence}, शुद्धधी, परमधी, सुश्री, शुष्की, पक्वी आदि \\ 
 \nextWord{श्री (स्त्री॰) Prosperity} \hyperlink{श्री (स्त्री॰) Prosperity}{श्री (स्त्री॰) Prosperity}; अवधेयम् - प्रथमा एकवचनम् - अवीः, तन्त्रीः, तरीः, लक्ष्मीः, ह्रीः, धीः, श्रीः \twolineshloka*{अवी-तन्त्री-तरी-लक्ष्मी-द्री-धी-श्रीणामुणादिषु}{सप्तानामपि शब्दानां सुलोपो न कदाचन}
 \nextWord{स्त्री (स्त्री॰) Woman} \hyperlink{स्त्री (स्त्री॰) Woman}{स्त्री (स्त्री॰) Woman} \\ 
 \nextWord{सखी (पु॰) Friend} \hyperlink{सखी (पु॰) Friend}{सखी (पु॰) Friend} \\ 
 \nextWord{बहु (नपु॰) Many} \hyperlink{बहु (नपु॰) Many}{बहु (नपु॰) Many}, कटु, मृदु, लघु, पटु आदि \\ 
 \nextWord{स्वयम्भू (पु॰) Brahma} \hyperlink{स्वयम्भू (पु॰) Brahma}{स्वयम्भू (पु॰) Brahma}, स्वभू, सुभ्रू, प्रतिभू आदि \\ 
 \nextWord{भू (स्त्री॰) Earth} \hyperlink{भू (स्त्री॰) Earth}{भू (स्त्री॰) Earth} \\ 
 \nextWord{नृ (पु॰) Man} \hyperlink{नृ (पु॰) Man}{नृ (पु॰) Man} \\ 
 \nextWord{स्वसृ (स्त्री॰) Sister} \hyperlink{स्वसृ (स्त्री॰) Sister}{स्वसृ (स्त्री॰) Sister} \\ 
 \nextWord{मातृ (स्त्री॰) Mother} \hyperlink{मातृ (स्त्री॰) Mother}{मातृ (स्त्री॰) Mother}, दुहितृ, यातृ आदि \\ 
 \nextWord{दातृ (नपु॰) Giver} \hyperlink{दातृ (नपु॰) Giver}{दातृ (नपु॰) Giver}; दातृ, कर्तृ, धातृ, नेतृ, रक्षितृ, etc., are used as adjectives and so they take forms in all three genders, here the नपुंसकलिङ्गः forms are given.\\ 
 \nextWord{रै (पु॰) Wealth} \hyperlink{रै (पु॰) Wealth}{रै (पु॰) Wealth} \\ 
 \nextWord{गो (पु॰) Cow} \hyperlink{गो (पु॰) Cow}{गो (पु॰) Cow}; ऐकारान्त स्त्रीलिङ्गः and ओकारान्त स्त्रीलिङ्गः words decline like the ऐकारान्त (रै) and ओकारान्त (गो) पुँल्लिङ्ग forms, respectively; नपुंसके एकारान्ताः ऐकारन्ताः ओकारान्ताः औकारान्ताः शब्दाः न विद्यन्ते \\ 
 \nextWord{ग्लौ (पु॰) Moon} \hyperlink{ग्लौ (पु॰) Moon}{ग्लौ (पु॰) Moon} \\ 
 \nextWord{नौ (स्त्री॰) Boat} \hyperlink{नौ (स्त्री॰) Boat}{नौ (स्त्री॰) Boat} \\ 
\nextWord{प्राञ्च् (पु॰) East} \hyperlink{प्राञ्च् (पु॰) East}{प्राञ्च् (पु॰) East} \\ 
 \nextWord{जलमुच् (पु॰) Cloud} \hyperlink{जलमुच् (पु॰) Cloud}{जलमुच् (पु॰) Cloud}, सत्यवाच् आदि but प्राञ्च्, प्रत्यञ्च्, उदञ्च्, and तिर्यञ्च् form differently as they are derived from the अञ्च् धातुः\\ 
 \nextWord{वाच् (स्त्री॰) Speech} \hyperlink{वाच् (स्त्री॰) Speech}{वाच् (स्त्री॰) Speech}, त्वच्, शुच्, रुच्, ऋच् आदि; हलन्तेषु केचन शब्दाः स्त्रियाम् ईकारान्ता भवन्ति। तेषां नदीशब्दवत् रूपाणि। ये शब्दा ईकारान्ता न भवन्ति ते तत्तदन्ताः पुँल्लिङ्गशब्दा इव द्रष्टव्याः। तथा हि - वाच् (स्त्री॰) जल्मुच् (पु॰) शब्दवत्, स्रज् (स्त्री॰) वणिज् (पु॰) शब्दवत्, सरित् (स्त्री॰) मरुत् (पु॰) शब्दवत्, शरत्  (स्त्री॰) सम्पद् (स्त्री॰) आपद् (स्त्री॰) मृद् (स्त्री॰) प्रभृतयः सुहृद् (पु॰) शब्दवत्\\ 
 \nextWord{सुवाच् (नपु॰) Praiseworthy} \hyperlink{सुवाच् (नपु॰) Praiseworthy}{सुवाच् (नपु॰) Praiseworthy} \\ 
 \nextWord{वणिज् (पु॰) Trader} \hyperlink{वणिज् (पु॰) Trader}{वणिज् (पु॰) Trader} \\ 
 \nextWord{सम्राज् (पु॰) Sovereign} \hyperlink{सम्राज् (पु॰) Sovereign}{सम्राज् (पु॰) Sovereign}, विश्वसृज्, विराज्, परिव्राज् आदि \\ 
 \nextWord{ऋत्विज् (पु॰) Priest} \hyperlink{ऋत्विज् (पु॰) Priest}{ऋत्विज् (पु॰) Priest}, हुतभुज्, भूभुज्, भिषज्, वणिज् आदि \\ 
 \nextWord{स्रज् (स्त्री॰) Garland} \hyperlink{स्रज् (स्त्री॰) Garland}{स्रज् (स्त्री॰) Garland} \\ 
 \nextWord{असृज् (नपु॰) Saffron} \hyperlink{असृज् (नपु॰) Saffron}{असृज् (नपु॰) Saffron} \\ 
\nextWord{सरट् (स्त्री॰) Lizard} \hyperlink{सरट् (स्त्री॰) Lizard}{सरट् (स्त्री॰) Lizard} \\ 
 \nextWord{मरुत् (पु॰) Wind} \hyperlink{मरुत् (पु॰) Wind}{मरुत् (पु॰) Wind}, भूभृत्, महीभृत्, शशभृत्, दिनकृत्, परभृत्, विश्वजित् आदि \\ 
 \nextWord{जगत् (नपु॰) World} \hyperlink{जगत् (नपु॰) World}{जगत् (नपु॰) World}; मत्वन्ता, वत्वन्ताः, स्वादि-तनादि-रुधादि-क्र्यादि धातुभ्यो निष्पन्नः शत्रन्ताः, अदादौ अकारान्तवर्जितधातुभ्यो निष्पन्नः शन्नन्ताः, बृहत्, पृषत् प्रभृतयः शब्दाश्च एवं ज्ञेयाः \\ 
 \nextWord{ददत् (नपु॰) The Giving One} \hyperlink{ददत् (नपु॰) The Giving One}{ददत् (नपु॰) The Giving One}, जुह्वत्, शंसत्, जक्षत्, चकासत्, दरिद्रत्, जाग्रत् प्रभृतयः; अभ्यस्तधातुभ्यः (जुहोत्यादिभ्यः यङ्लुगन्तेभ्यस्च) निष्पन्नाः श्त्रन्ताः शासत् प्रभृतयश्च एवं द्रष्टव्याः \\ 
 \nextWord{तुदत् (नपु॰) The Troubling One} \hyperlink{तुदत् (नपु॰) The Troubling One}{तुदत् (नपु॰) The Troubling One}, एवं पृच्छत् मुञ्चत् यात् भात् करिष्यत् प्रभृतयः; तुदादिभ्यः अदादौ अकारान्तेभ्यश्च धातुभ्योः निष्पन्नः शत्रन्ताः लृटो निष्पन्ना शत्रन्ताश्च अस्मिन् वर्गेऽन्तर्भवन्ति। \\ 
 \nextWord{पचत् (नपु॰) The Cooking One} \hyperlink{पचत् (नपु॰) The Cooking One}{पचत् (नपु॰) The Cooking One}, गच्छत्, धावत्, वदत्, पठत्, पश्यत्, पतत्, गृह्णत्, शोचत्, भवत्, पिबत्, गमिष्यत्, हरिष्यत्, करिष्यत्, कथयिष्यत्; स्त्रीलिङ्गप्रयोगे गच्छन्ती, धावन्ती इत्यादि नदीवत्; एवं भवत्, दीव्यत्, चोरयत्, चिकीर्षत्, पुत्रीयत्, प्रभृतयः; भ्वादि-दिवादि-चुरादि-सन्नन्त-नामधातुभ्यो निष्पन्नाः शत्रन्ताः एवंविधाः। \\ 
 \nextWord{महत् (नपु॰) Great} \hyperlink{महत् (नपु॰) Great}{महत् (नपु॰) Great} \\ 
 \nextWord{महत् (पु॰) Great} \hyperlink{महत् (पु॰) Great}{महत् (पु॰) Great} \\ 
 \nextWord{दत् (पु॰) Tooth} \hyperlink{दत् (पु॰) Tooth}{दत् (पु॰) Tooth} \\ 
\nextWord{शरद् (स्त्री॰) Winter} \hyperlink{शरद् (स्त्री॰) Winter}{शरद् (स्त्री॰) Winter} \\ 
 \nextWord{पद् (पु॰) Foot} \hyperlink{पद् (पु॰) Foot}{पद् (पु॰) Foot} \\ 
 \nextWord{हृद् (नपु॰) Heart} \hyperlink{हृद् (नपु॰) Heart}{हृद् (नपु॰) Heart} \\ 
 \nextWord{सुहृद् (पु॰) Kind-hearted} \hyperlink{सुहृद् (पु॰) Kind-hearted}{सुहृद् (पु॰) Kind-hearted}, मर्मभिद्, सभासद्, तमोनुद्, धर्मविद्, हृदयच्छिद्, हृदयन्तुद्, दिविषद्, शास्त्रविद्, तमोनुद् आदि \\ 
 \nextWord{समिध् (स्त्री॰) Firewood} \hyperlink{समिध् (स्त्री॰) Firewood}{समिध् (स्त्री॰) Firewood}, क्षुध्, युध्, क्रुध्, विरुध् आदि \\ 
 \nextWord{क्षुध् (स्त्री॰) Hunger} \hyperlink{क्षुध् (स्त्री॰) Hunger}{क्षुध् (स्त्री॰) Hunger} \\ 
 \nextWord{राजन् (पु॰) King} \hyperlink{राजन् (पु॰) King}{राजन् (पु॰) King}; स्त्रीलिङ्गे राज्ञी नदीवत् \\ 
 \nextWord{मघवन् (पु॰) Indra} \hyperlink{मघवन् (पु॰) Indra}{मघवन् (पु॰) Indra} \\ 
 \nextWord{मघवन् (पु॰) Indra} \hyperlink{मघवन् (पु॰) Indra}{मघवन् (पु॰) Indra} \\ 
 \nextWord{युवन् (पु॰) Youth} \hyperlink{युवन् (पु॰) Youth}{युवन् (पु॰) Youth}; युवन् शब्दस्य स्त्रियां युवतिः इति मतिशब्दवत् किन्तु युवती इति अपि रूपमस्ति \\ 
 \nextWord{श्वन् (पु॰) Dog} \hyperlink{श्वन् (पु॰) Dog}{श्वन् (पु॰) Dog} \\ 
 \nextWord{अर्वन् (पु॰) Horse} \hyperlink{अर्वन् (पु॰) Horse}{अर्वन् (पु॰) Horse} \\ 
 \nextWord{मूर्धन् (पु॰) Head} \hyperlink{मूर्धन् (पु॰) Head}{मूर्धन् (पु॰) Head} \\ 
 \nextWord{लघिमन् (पु॰) Smaller} \hyperlink{लघिमन् (पु॰) Smaller}{लघिमन् (पु॰) Smaller} \\ 
 \nextWord{महिमन् (पु॰) Greatness} \hyperlink{महिमन् (पु॰) Greatness}{महिमन् (पु॰) Greatness}, सीमन् (स्त्री॰), मूर्धन्, गरिमन्, अणिमन्, लघिमन्, शुक्लिमन्, कालिमन्, अश्वत्थामन्, द्रढिमन् आदि \\ 
 \nextWord{अस्थायिन् (नपु॰) Temporary} \hyperlink{अस्थायिन् (नपु॰) Temporary}{अस्थायिन् (नपु॰) Temporary} \\ 
 \nextWord{गुणिन् (नपु॰) With Good Qualities} \hyperlink{गुणिन् (नपु॰) With Good Qualities}{गुणिन् (नपु॰) With Good Qualities}, एवं कुशलिन्, वाग्मिन्, दण्डिन् प्रभृतयः \\ 
 \nextWord{पथिन् (पु॰) Traveller} \hyperlink{पथिन् (पु॰) Traveller}{पथिन् (पु॰) Traveller} \\ 
 \nextWord{अहन् (नपु॰) Day} \hyperlink{अहन् (नपु॰) Day}{अहन् (नपु॰) Day} \\ 
 \nextWord{भाविन् (नपु॰) Imminent} \hyperlink{भाविन् (नपु॰) Imminent}{भाविन् (नपु॰) Imminent} \\ 
 \nextWord{अप् (स्त्री॰) Water} \hyperlink{अप् (स्त्री॰) Water}{अप् (स्त्री॰) Water} \\ 
 \nextWord{ककुभ् (स्त्री॰) Splendour} \hyperlink{ककुभ् (स्त्री॰) Splendour}{ककुभ् (स्त्री॰) Splendour} \\ 
 \nextWord{वार् (नपु॰) Pond} \hyperlink{वार् (नपु॰) Pond}{वार् (नपु॰) Pond} \\ 
 \nextWord{गिर् (स्त्री॰) Voice} \hyperlink{गिर् (स्त्री॰) Voice}{गिर् (स्त्री॰) Voice} \\ 
 \nextWord{पुर् (स्त्री॰) City} \hyperlink{पुर् (स्त्री॰) City}{पुर् (स्त्री॰) City}, धुर् \\ 
 \nextWord{दिव् (स्त्री॰) Sky} \hyperlink{दिव् (स्त्री॰) Sky}{दिव् (स्त्री॰) Sky} \\ 
 \nextWord{दिश् (स्त्री॰) Direction} \hyperlink{दिश् (स्त्री॰) Direction}{दिश् (स्त्री॰) Direction} \\ 
 \nextWord{विश् (पु॰) House} \hyperlink{विश् (पु॰) House}{विश् (पु॰) House} \\ 
 \nextWord{भवादृश् (पु॰) Like You} \hyperlink{भवादृश् (पु॰) Like You}{भवादृश् (पु॰) Like You}, यादृश्, मादृश्, तादृश्, त्वादृश्, एतादृश्, आदि; स्त्रीलिङ्गे भवादृशी, यादृशी, मादृशी नदीवत् किन्तु तेषां तु स्त्रियां तादृशा ईदृशा इत्येवं लताशब्दवत् रूपाणि \\ 
 \nextWord{भवादृश् (नपु॰) Like You} \hyperlink{भवादृश् (नपु॰) Like You}{भवादृश् (नपु॰) Like You} \\ 
 \nextWord{निश् (स्त्री॰) Night} \hyperlink{निश् (स्त्री॰) Night}{निश् (स्त्री॰) Night} \\ 
 \nextWord{तादृश् (नपु॰) Like That} \hyperlink{तादृश् (नपु॰) Like That}{तादृश् (नपु॰) Like That} \\ 
 \nextWord{सुत्विष् (नपु॰) Very Lustrous} \hyperlink{सुत्विष् (नपु॰) Very Lustrous}{सुत्विष् (नपु॰) Very Lustrous} \\ 
 \nextWord{द्विष् (पु॰) Dislike} \hyperlink{द्विष् (पु॰) Dislike}{द्विष् (पु॰) Dislike} \\ 
 \nextWord{प्रावृष् (स्त्री॰) Rain} \hyperlink{प्रावृष् (स्त्री॰) Rain}{प्रावृष् (स्त्री॰) Rain} \\ 
 \nextWord{विद्वस् (पु॰) Learned} \hyperlink{विद्वस् (पु॰) Learned}{विद्वस् (पु॰) Learned}; ऊचिवस्, उपेयिवस्, सेदिवस्, तस्थिवस् स्त्रीलिङ्गे ऊचुषी, उपेयुषी, सेदुषी, तस्थुषी, विदुषी नदीवत् \\ 
 \nextWord{कनीयस् (पु॰) Younger} \hyperlink{कनीयस् (पु॰) Younger}{कनीयस् (पु॰) Younger}, गरीयस्, लघीयस्, द्रढीयस्, प्रथीयस्, द्राघीयस्, श्रेयस् इत्यादयः ईयस् प्रत्ययान्ताः शब्दाः स्त्रीलिङ्गे कनीयसी, लघीयसी, गरीयसी, द्रढीयसी नदीवत् \\ 
 \nextWord{अप्सरस् (स्त्री॰) Fairy} \hyperlink{अप्सरस् (स्त्री॰) Fairy}{अप्सरस् (स्त्री॰) Fairy}; अप्सरस् शब्दः प्रायः बहुवचने प्रयुज्यते \\ 
 \nextWord{वेधस् (पु॰) Brahma} \hyperlink{वेधस् (पु॰) Brahma}{वेधस् (पु॰) Brahma} \\ 
 \nextWord{चन्द्रमस् (पु॰) Moon} \hyperlink{चन्द्रमस् (पु॰) Moon}{चन्द्रमस् (पु॰) Moon}, महौजस्, दिवौकस्, सुमनस्, महायशस्, वेधस्, महातेजस्, वनौकस्, विशालवक्षस्, दुर्वासस्, प्रचेतस् आदि \\ 
 \nextWord{मास् (पु॰) Month} \hyperlink{मास् (पु॰) Month}{मास् (पु॰) Month} \\ 
 \nextWord{श्रेयस् (पु॰) Better} \hyperlink{श्रेयस् (पु॰) Better}{श्रेयस् (पु॰) Better} \\ 
 \nextWord{पुंस् (पु॰) Person} \hyperlink{पुंस् (पु॰) Person}{पुंस् (पु॰) Person} \\ 
 \nextWord{दोस् (पु॰) Arm} \hyperlink{दोस् (पु॰) Arm}{दोस् (पु॰) Arm} \\ 
 \nextWord{भास् (स्त्री॰) Light} \hyperlink{भास् (स्त्री॰) Light}{भास् (स्त्री॰) Light} \\ 
 \nextWord{आशिस् (स्त्री॰) Wish} \hyperlink{आशिस् (स्त्री॰) Wish}{आशिस् (स्त्री॰) Wish} \\ 
 \nextWord{हविस् (नपु॰) Oblation} \hyperlink{हविस् (नपु॰) Oblation}{हविस् (नपु॰) Oblation}, एवं सर्पिस्, ज्योतिस्, रोचिस्, प्रभृतयः \\ 
 \nextWord{वपुस् (नपु॰) Body} \hyperlink{वपुस् (नपु॰) Body}{वपुस् (नपु॰) Body} \\ 
 \nextWord{तस्थिवस् (नपु॰) Standing} \hyperlink{तस्थिवस् (नपु॰) Standing}{तस्थिवस् (नपु॰) Standing} \\ 
 \nextWord{लिह् (पु॰) Licking} \hyperlink{लिह् (पु॰) Licking}{लिह् (पु॰) Licking} \\ 
 \nextWord{उपानह् (स्त्री॰) Shoe} \hyperlink{उपानह् (स्त्री॰) Shoe}{उपानह् (स्त्री॰) Shoe} \\ 
 \nextWord{अम्भोरुह् (नपु॰) Lotus} \hyperlink{अम्भोरुह् (नपु॰) Lotus}{अम्भोरुह् (नपु॰) Lotus} \\ 
 \nextWord{अस्मद्} \hyperlink{अस्मद्}{अस्मद्}; the optional short forms of युष्मद् and अस्मद् viz., त्वा, वां, वः, ते and मा, नौ, नः, मे are never used at the beginning of a sentence or of a foot (पाद) of a श्लोक, nor can they be used immediately before particles च, ह, हा, अह and एव \\ 
 \nextWord{युष्मद्} \hyperlink{युष्मद्}{युष्मद्} \\ 
\nextWord{तत् (पु॰) That} \hyperlink{तद् (पु॰) That}{तद् (पु॰) That}\twolineshloka*{इदमस्तु सन्निकृष्टं समीपतरवर्ति चैतदो रूपम्}{अदसस्तु विप्रकृष्टं तदिति परोक्षे विजानीयात्}
\nextWord{तत् (स्त्री॰) That} \hyperlink{तद् (स्त्री॰) That}{तद् (स्त्री॰) That} \\ 
 \nextWord{तत् (नपु॰) That} \hyperlink{तद् (नपु॰) That}{तद् (नपु॰) That} \\ 
 \nextWord{इदम् (पु॰) This} \hyperlink{इदम् (पु॰) This}{इदम् (पु॰) This} \\ 
 \nextWord{इदम् (स्त्री॰) This} \hyperlink{इदम् (स्त्री॰) This}{इदम् (स्त्री॰) This} \\ 
 \nextWord{इदम् (नपु॰) This} \hyperlink{इदम् (नपु॰) This}{इदम् (नपु॰) This} \\ 
 \nextWord{एतत् (पु॰) This} \hyperlink{एतद् (पु॰) This}{एतद् (पु॰) This}; The optional forms of एतद् and इदम्, viz., एनम्, एनेन, etc., are to be used when there is अन्वादेश, i.e., when their proper forms have already been used in a previous clause; एतेन व्याकरणमधीतम् (he has studied grammar), एनं छन्दोऽध्यापय (teach him छन्दस् (prosody)), etc. अन्वादेश means the subsequent mention of a thing already mentioned.\\ 
 \nextWord{एतद् (स्त्री॰) This} \hyperlink{एतद् (स्त्री॰) This}{एतद् (स्त्री॰) This} \\ 
 \nextWord{एतद् (नपु॰) This} \hyperlink{एतद् (नपु॰) This}{एतद् (नपु॰) This} \\ 
 \nextWord{अदस् (पु॰) That} \hyperlink{अदस् (पु॰) That}{अदस् (पु॰) That} \\ 
 \nextWord{अदस् (स्त्री॰) This} \hyperlink{अदस् (स्त्री॰) This}{अदस् (स्त्री॰) This} \\ 
 \nextWord{अदस् (नपु॰) This} \hyperlink{अदस् (नपु॰) This}{अदस् (नपु॰) This} \\ 
 \nextWord{यत् (पु॰) Who} \hyperlink{यद् (पु॰) Who}{यद् (पु॰) Who} \\ 
 \nextWord{यद् (स्त्री॰) Who} \hyperlink{यद् (स्त्री॰) Who}{यद् (स्त्री॰) Who} \\ 
 \nextWord{यद् (नपु॰) Who} \hyperlink{यद् (नपु॰) Who}{यद् (नपु॰) Who} \\ 
 \nextWord{किम् (पु॰) What} \hyperlink{किम् (पु॰) What}{किम् (पु॰) What}; Indefinite pronouns are formed by the addition of चित्, चन् or अपि to the various cases of this word in all the genders, e.g., कश्चित् कश्चन कोऽपि केनचित् कयाचन कस्यापि कस्मिंश्चित् कयोश्चित्, etc. \\ 
 \nextWord{किम् (स्त्री॰) What} \hyperlink{किम् (स्त्री॰) What}{किम् (स्त्री॰) What} \\ 
 \nextWord{किम् (नपु॰) What} \hyperlink{किम् (नपु॰) What}{किम् (नपु॰) What} \\ 
 \nextWord{सर्व (पु॰) All} \hyperlink{सर्व (पु॰) All}{सर्व (पु॰) All} \\ 
 \nextWord{सर्वा (स्त्री॰) All} \hyperlink{सर्वा (स्त्री॰) All}{सर्वा (स्त्री॰) All} \\ 
 \nextWord{सर्व (नपु॰) All} \hyperlink{सर्व (नपु॰) All}{सर्व (नपु॰) All} \\ 
 \nextWord{भवत् (पु॰) You} \hyperlink{भवत् (पु॰) You}{भवत् (पु॰) You} \\ 
 \nextWord{भवत् (स्त्री॰) You} \hyperlink{भवत् (स्त्री॰) You}{भवत् (स्त्री॰) You} \\ 
 \nextWord{भवत् (नपु॰) You} \hyperlink{भवत् (नपु॰) You}{भवत् (नपु॰) You} \\ 
 \nextWord{अन्यत् (पु॰) Another} \hyperlink{अन्यत् (पु॰) Another}{अन्यत् (पु॰) Another} \\ 
 \nextWord{अन्यत् (स्त्री॰) Another} \hyperlink{अन्यत् (स्त्री॰) Another}{अन्यत् (स्त्री॰) Another} \\ 
 \nextWord{अन्यत् (नपु॰) Other} \hyperlink{अन्यत् (नपु॰) Other}{अन्यत् (नपु॰) Other} \\ 
 \nextWord{पूर्व (पु॰) Before} \hyperlink{पूर्व (पु॰) Before}{पूर्व (पु॰) Before} \\ 
 \nextWord{पूर्व (स्त्री॰) Before} \hyperlink{पूर्व (स्त्री॰) Before}{पूर्व (स्त्री॰) Before} \\ 
 \nextWord{पूर्व (नपु॰) Before} \hyperlink{पूर्व (नपु॰) Before}{पूर्व (नपु॰) Before} \\ 
 \nextWord{त्वत् (पु॰) Other} \hyperlink{त्वत् (पु॰) Other}{त्वत् (पु॰) Other} \\ 
 \nextWord{त्वत् (स्त्री॰) Other} \hyperlink{त्वत् (स्त्री॰) Other}{त्वत् (स्त्री॰) Other} \\ 
 \nextWord{त्वत् (नपु॰) Other} \hyperlink{त्वत् (नपु॰) Other}{त्वत् (नपु॰) Other} \\ 
 \nextWord{उभ (द्वि॰) Both} \hyperlink{उभ (द्वि॰) Both}{उभ (द्वि॰) Both} \\ 
 \nextWord{उभय (पु॰) Both} \hyperlink{उभय (पु॰) Both}{उभय (पु॰) Both} \\ 
 \nextWord{कति-यति-तति (पु॰, स्त्री॰, नपु॰)} \hyperlink{कति-यति-तति (पु॰, स्त्री॰, नपु॰)}{कति-यति-तति (पु॰, स्त्री॰, नपु॰)} \\ 
 \nextWord{एक (एकवचनम्) One} \hyperlink{एक (एकवचनम्) One}{एक (एकवचनम्) One} \\ 
 \nextWord{द्वि (द्विवचनम्) Two} \hyperlink{द्वि (द्विवचनम्) Two}{द्वि (द्विवचनम्) Two} \\ 
 \nextWord{त्रि (बहुवचनम्) Three} \hyperlink{त्रि (बहुवचनम्) Three}{त्रि (बहुवचनम्) Three} \\ 
 \nextWord{चतुर् (बहुवचनम्) Four} \hyperlink{चतुर् (बहुवचनम्) Four}{चतुर् (बहुवचनम्) Four} \\ 
 \nextWord{पञ्चन्-षष्-सप्तन् (पु॰, स्त्री॰, नपु॰) 5-6-7} \hyperlink{पञ्चन्-षष्-सप्तन् (पु॰, स्त्री॰, नपु॰) 5-6-7}{पञ्चन्-षष्-सप्तन् (पु॰, स्त्री॰, नपु॰) 5-6-7} \\ 
 \nextWord{अष्टन्-नवन्-दशन् (पु॰, स्त्री॰, नपु॰) 8-9-10} \hyperlink{अष्टन्-नवन्-दशन् (पु॰, स्त्री॰, नपु॰) 8-9-10}{अष्टन्-नवन्-दशन् (पु॰, स्त्री॰, नपु॰) 8-9-10} \\ 
 \nextWord{विंशति-त्रिंशत्-चात्वारिंशत् (स्त्री॰) 20-30-40} \hyperlink{विंशति-त्रिंशत्-चात्वारिंशत् (स्त्री॰) 20-30-40}{विंशति-त्रिंशत्-चात्वारिंशत् (स्त्री॰) 20-30-40} विंशत्यादयः --- Numerals 20 and above --- विंशतिः, त्रिंशत्, चत्वारिंशत्, पञ्चाशत्, षष्टिः, सप्ततिः, अशीतिः, नवतिः एते स्त्रीलिङ्गाः। त्रिंशत्, चत्वारिंशत्, पञ्चाशत् एते त्रयः तकारान्ताः, अन्ये इकारान्ताः इति विशेषः। एतेषाम् द्विधा प्रयोगः संभवति, विशेषणतया विशेष्यतया च। विशेषणतया प्रयोगे एकवचनमेव। यथा `विंशतिः कुमाराः' इत्यादि। विशेष्यतया प्रयोगे तु द्विवचन-बहुवचने अपि स्तः। यथा `रूप्यकाणां द्वे विंशती', गवां तिस्त्रः विंशतयः॑ इत्यादि। अत्र इकारान्तानां `मति' शब्दवत्, तकारान्तानां `सरित्' शब्दवत् च रूपाणि द्रष्टव्यानि। शत सहस्र-अयुत-लक्ष-प्रयुत शब्दाः अकारान्त-नपुंसकलिङ्गाः, कोटिशब्दः इकारान्तस्त्रीलिङ्गः। एतेषामपि विशेषणतया प्रयोगे एकवचनमेव, यथा `शतं जनाः' `सहस्रं स्त्रियः' इत्यादि। विशेष्यतया प्रयोगे तु द्विवचन-बहुवचने अपि भवतः, यथा `गवां द्वे शते,' `रूप्यकाणां त्रीणि सहस्राणि' इत्यादि। \\ 
 \nextWord{पञ्चाशत्-षष्ठि-सप्तति (स्त्री॰) 50-60-70} \hyperlink{पञ्चाशत्-षष्ठि-सप्तति (स्त्री॰) 50-60-70}{पञ्चाशत्-षष्ठि-सप्तति (स्त्री॰) 50-60-70} \\ 
 \nextWord{अशीति-नवति (स्त्री॰), शतम् (नपु॰) 80-90-100} \hyperlink{अशीति-नवति (स्त्री॰), शतम् (नपु॰) 80-90-100}{अशीति-नवति (स्त्री॰), शतम् (नपु॰) 80-90-100} \\ 
 \nextWord{ऐक्ष्वाक (पु॰) Descendant of Ishvaku} \hyperlink{ऐक्ष्वाक (पु॰) Descendant of Ishvaku}{ऐक्ष्वाक (पु॰) Descendant of Ishvaku} \\ 
 \nextWord{हाहा (पु॰) A name} \hyperlink{हाहा (पु॰) A name}{हाहा (पु॰) A name} \\ 
 \nextWord{औडुलोमि (पु॰) Descendant of Uduloman} \hyperlink{औडुलोमि (पु॰) Descendant of Uduloman}{औडुलोमि (पु॰) Descendant of Uduloman} \\ 
 \nextWord{सेनानी (पु॰) Warrior} \hyperlink{सेनानी (पु॰) Warrior}{सेनानी (पु॰) Warrior}, ग्रामणी, उन्नी \\ 
 \nextWord{वातप्रमी (पु॰) Swift Antelope} \hyperlink{वातप्रमी (पु॰) Swift Antelope}{वातप्रमी (पु॰) Swift Antelope} \\ 
 \nextWord{क्रोष्टु (पु॰) Jackal} \hyperlink{क्रोष्टु (पु॰) Jackal}{क्रोष्टु (पु॰) Jackal} \\ 
 \nextWord{वर्षाभू (पु॰) Frog} \hyperlink{वर्षाभू (पु॰) Frog}{वर्षाभू (पु॰) Frog} \\ 
 \nextWord{हूहू (पु॰) A Name} \hyperlink{हूहू (पु॰) A Name}{हूहू (पु॰) A Name} \\ 
 \nextWord{जरा (स्त्री॰) Old Age} \hyperlink{जरा (स्त्री॰) Old Age}{जरा (स्त्री॰) Old Age} \\ 
 \nextWord{अजर (नपु॰) Ageless} \hyperlink{अजर (नपु॰) Ageless}{अजर (नपु॰) Ageless} \\ 
 \nextWord{प्राञ्च् (पु॰) Eastern} \hyperlink{प्राञ्च् (पु॰) Eastern}{प्राञ्च् (पु॰) Eastern} \\ 
 \nextWord{प्रत्यञ्च् (पु॰) Western} \hyperlink{प्रत्यञ्च् (पु॰) Western}{प्रत्यञ्च् (पु॰) Western} \\ 
 \nextWord{उदञ्च् (पु॰) Northern} \hyperlink{उदञ्च् (पु॰) Northern}{उदञ्च् (पु॰) Northern} \\ 
 \nextWord{अन्वञ्च् (पु॰) Following} \hyperlink{अन्वञ्च् (पु॰) Following}{अन्वञ्च् (पु॰) Following} \\ 
 \nextWord{तिर्यञ्च् (पु॰) Horizontally} \hyperlink{तिर्यञ्च् (पु॰) Horizontally}{तिर्यञ्च् (पु॰) Horizontally} \\ 
 \nextWord{विभ्राज् (पु॰) Bright} \hyperlink{विभ्राज् (पु॰) Bright}{विभ्राज् (पु॰) Bright} \\ 
 \nextWord{युज् (पु॰) Sage} \hyperlink{युज् (पु॰) Sage}{युज् (पु॰) Sage} \\ 
 \nextWord{युञ्ज् (पु॰) United} \hyperlink{युञ्ज् (पु॰) United}{युञ्ज् (पु॰) United} \\ 
 \nextWord{सुपाद् (पु॰) Having Good Feet} \hyperlink{सुपाद् (पु॰) Having Good Feet}{सुपाद् (पु॰) Having Good Feet} \\ 
 \nextWord{पूषन् (पु॰) Sun} \hyperlink{पूषन् (पु॰) Sun}{पूषन् (पु॰) Sun} \\ 
 \nextWord{वृत्रहन् (पु॰) Indra} \hyperlink{वृत्रहन् (पु॰) Indra}{वृत्रहन् (पु॰) Indra} \\ 
 \nextWord{दीर्घाहन् (पु॰) Summer} \hyperlink{दीर्घाहन् (पु॰) Summer}{दीर्घाहन् (पु॰) Summer} \\ 
 \nextWord{अर्वन् (पु॰) Horse} \hyperlink{अर्वन् (पु॰) Horse}{अर्वन् (पु॰) Horse} \\ 
 \nextWord{ऋभुक्षिन् (पु॰) Indra} \hyperlink{ऋभुक्षिन् (पु॰) Indra}{ऋभुक्षिन् (पु॰) Indra} \\ 
 \nextWord{उशनस् (पु॰) Shukaacharya} \hyperlink{उशनस् (पु॰) Shukaacharya}{उशनस् (पु॰) Shukaacharya} \\ 
 \nextWord{अनेहस् (पु॰) Time} \hyperlink{अनेहस् (पु॰) Time}{अनेहस् (पु॰) Time} \\ 
 \nextWord{विश्ववाह् (पु॰) Sustainer of the Universe} \hyperlink{विश्ववाह् (पु॰) Sustainer of the Universe}{विश्ववाह् (पु॰) Sustainer of the Universe} \\ 
 \nextWord{तुरासाह् (पु॰) Indra} \hyperlink{तुरासाह् (पु॰) Indra}{तुरासाह् (पु॰) Indra} \\ 
 \nextWord{दुह् (पु॰) One who Milks} \hyperlink{दुह् (पु॰) One who Milks}{दुह् (पु॰) One who Milks} \\ 
 \nextWord{द्रुह् (पु॰) One who bears hatred} \hyperlink{द्रुह् (पु॰) One who bears hatred}{द्रुह् (पु॰) One who bears hatred} \\ 
 \nextWord{अनडुह् (पु॰) Ox} \hyperlink{अनडुह् (पु॰) Ox}{अनडुह् (पु॰) Ox} \\ 
 \nextWord{द्वार् (स्त्री॰) Door} \hyperlink{द्वार् (स्त्री॰) Door}{द्वार् (स्त्री॰) Door} \\ 
 \nextWord{अर्चिस् (स्त्री॰) Flame} \hyperlink{अर्चिस् (स्त्री॰) Flame}{अर्चिस् (स्त्री॰) Flame} \\ 
 \nextWord{सजुष् (स्त्री॰) Companion} \hyperlink{सजुष् (स्त्री॰) Companion}{सजुष् (स्त्री॰) Companion} \\ 
 \nextWord{उष्णिह् (स्त्री॰) A Metre} \hyperlink{उष्णिह् (स्त्री॰) A Metre}{उष्णिह् (स्त्री॰) A Metre} \\ 
 \nextWord{प्राञ्च् (नपु॰) Eastern} \hyperlink{प्राञ्च् (नपु॰) Eastern}{प्राञ्च् (नपु॰) Eastern} \\ 
 \nextWord{प्रत्यञ्च् (नपु॰) Western} \hyperlink{प्रत्यञ्च् (नपु॰) Western}{प्रत्यञ्च् (नपु॰) Western} \\ 
 \nextWord{अन्वञ्च् (नपु॰) Following} \hyperlink{अन्वञ्च् (नपु॰) Following}{अन्वञ्च् (नपु॰) Following} \\ 
 \nextWord{उदञ्च् (नपु॰) Northern} \hyperlink{उदञ्च् (नपु॰) Northern}{उदञ्च् (नपु॰) Northern} \\ 
 \nextWord{तिर्यञ्च् (नपु॰) Horizontal} \hyperlink{तिर्यञ्च् (नपु॰) Horizontal}{तिर्यञ्च् (नपु॰) Horizontal} \\ 
 \nextWord{प्राञ्च् (नपु॰) Eastern} \hyperlink{प्राञ्च् (नपु॰) Eastern}{प्राञ्च् (नपु॰) Eastern} \\ 
 \nextWord{प्रत्यञ्च् (नपु॰) Western} \hyperlink{प्रत्यञ्च् (नपु॰) Western}{प्रत्यञ्च् (नपु॰) Western} \\ 
 \nextWord{अन्वञ्च् (नपु॰) Following} \hyperlink{अन्वञ्च् (नपु॰) Following}{अन्वञ्च् (नपु॰) Following} \\ 
 \nextWord{उदञ्च् (नपु॰) Northern} \hyperlink{उदञ्च् (नपु॰) Northern}{उदञ्च् (नपु॰) Northern} \\ 
 \nextWord{तिर्यञ्च् (नपु॰) Horizontal} \hyperlink{तिर्यञ्च् (नपु॰) Horizontal}{तिर्यञ्च् (नपु॰) Horizontal} \\ 
 \nextWord{दुह् (नपु॰) One who Milks} \hyperlink{दुह् (नपु॰) One who Milks}{दुह् (नपु॰) One who Milks} \\ 
 \nextWord{द्रुह् (नपु॰) One who bears hatred} \hyperlink{द्रुह् (नपु॰) One who bears hatred}{द्रुह् (नपु॰) One who bears hatred} \\ 
 \nextWord{स्वनडुह् (नपु॰) One with a good Ox} \hyperlink{स्वनडुह् (नपु॰) One with a good Ox}{स्वनडुह् (नपु॰) One with a good Ox} \\ 
 \nextWord{निर्जर (पु॰) God} \hyperlink{निर्जर (पु॰) God}{निर्जर (पु॰) God} \\
\nextWord{सङ्ख्येयशब्दाः (Ordinals)} \hyperlink{सङ्ख्येयशब्दाः (Ordinals)}{सङ्ख्येयशब्दाः (Ordinals)} The ordinals from विंशतिः and above have two forms; the first form is by adding तम invariably; e.g., विंशतितम, त्रिंशत्तम, षष्टितम, etc. The second form is: (a) by dropping ति of विंशति, group and forming its compounds, e.g., विंश, एकविंश, etc., (b) by dropping त् of त्रिंशत्, चत्वारिंशत्, and पञ्चाशत् groups; त्रिंशः, चत्वारिंशः, and पञ्चाशः, (c) by changing the final इ to अ of एकषष्टि up to नवनवतिः, e.g., एकषष्ट, एकसप्तत, etc. Note that the following 4 cardinals and have only the first form, e.g., षष्टितम, सप्ततितम, अशीतितम, and नवन्तितम। In the compounds of सङ्ख्याशब्दाः (१२, २२, २३, २४, etc.), द्वि, त्रि and अष्टन्, are changed to द्वा, त्रयः and अष्टा necessarily before दशन्, विंशति and त्रिंशत् and optionally before चत्वारिंशत्, पञ्चाशत्, षष्टि, सप्तति and नवति. Before अशीति they remain unchanged.\\
\end{multicols}
%\nextWord{अशीति-नवति}  \hyperlink{अशीति-नवति}{अशीति-नवति} (स्त्री॰), शतम् (नपु॰)
%\clearpage


\section{Useful Resources}
\begin{ewosp}
\item \url{http://www.tinyurl.com/samskritam} (``vyakaranam/temolat'' folder) पाणिनीयप्रवेशः - भावबोधकः, २६-मे-२०१०।
\item सरलसंस्कृतसरणिः, प्रथमो भागः, द्वितियोऽन्तिमश्च भागः, जगन्नाथो वेदालङ्कारः डॉ॰ नरेन्द्रः च, संस्कृतकार्यालयः श्रीअरविन्दाश्रमः पुदुच्चेरी २००३ (http://sabda.sriaurobindoashram.org).
\item \href{http://www.sanskritebooks.org/2009/06/sabda-manjari/}{Sabda Manjari}, K.L.V. Sastri \& Pandit L. Anantharam Sastri, Sanskrit Made Easy Series, Sanskrit Education Society, 148-150 (98/99), Luz Church Road, Mylapore, Chennai - 600 004, India, 2008.
\item बृहद्-अनुवाद चन्द्रिका (अनुवाद-व्याकरण-निबन्धादिविषय-सम्वलिता), चक्रधर नौटियाल `हंस' शास्त्री, मोतीलाल बनारसीदास, १९७२।
\item शब्द-रूपावली (बिना रटे शब्द-रूपों का ज्ञान करानेवाली), पं॰ युधिष्ठिर मीमांसक, रामलाल कपूर ट्रस्ट २०००।
\item The Tested Easiest Method of Learning and Teaching Sanskrit: the study of Sanskrit by the Ashtadhyayi system in six months without cramming, Brahmadatta Jijñāsu; Rāmalāla Kapūra Ṭrasṭa, Sonepat, Haryana : Ram Lal Kapoor Trust, 1982, (Address: 2596 Nai Sarak, Delhi, Phone 0130 - 3290276, 2100285).
\item संस्कृत पठन-पाठन की अनुभूत सरलतम विधि (बिना रटे ६ मास में अष्टाध्यायी-पद्धति से संस्कृत का पठन-पाठन) - (प्रथम भाग), श्री पं॰ ब्रह्मदत्त जिज्ञासु, रामलाल कपूर ट्रस्ट ग्राम रवली, पोस्ट शाहपुरतुर्क, जिला सोनीपत - १३१००१ (हरियाणा)।
\item संस्कृत पठन-पाठन की अनुभूत सरलतम विधि (बिना रटे ६ मास में अष्टाध्यायी-पद्धति से संस्कृत का पठन-पाठन) - (द्वितीय भाग), पं॰ युधिष्ठिर मीमांसक, रामलाल कपूर ट्रस्ट ग्राम रवली, पोस्ट शाहपुरतुर्क, जिला सोनीपत - १३१००१ (हरियाणा)।
\item \url{http://www.advaita-vedanta.org/archives/advaita-l/2006-February/015904.html} --- List of Sanskrit Grammar Books
\item अष्टाध्यायी-भाष्य (3 Vols) - प्रथमावृत्ति, पं॰ ब्रह्मदत्त जिज्ञासु, रामलाल कपूर ट्रस्ट, २०००।
\item महर्षिपाणिनिविरचिता अष्टाध्यायी, चन्द्रलेखा - हिन्दीव्याख्यायुताऽनेकोपयोगिविषयैरुपबृंहिता, सूत्रपाठ - सूत्रवार्तिकोदाहरणानां सूच्या चान्विता, व्याख्याकारः पं॰ ईश्वरचन्द्रः, चौखम्बा संस्कृत प्रतिष्ठान, दिल्लि २००४।
\item \href{https://archive.org/details/TheAshtadhyayiOfPanini-RamNathSharma}{The Astadhyayi of Panini (English Commentary in 6 Vols), Rama Nath Sharma, Munshiram Manoharlal, Delhi, 2002} 
\item लघु-सिद्धान्त-कौमुदी-भैमीव्याख्या, (प्रथमः - षष्ठः भागाः), भीमसेन शास्त्री, भैमी प्रकाशन, ५३७ लाजपतराय मार्केट, दिल्ली-११०००६, २००५।
\item These sites can be very useful - \url{http://sanskrit.jnu.ac.in/index.jsp} and \url{http://tdil-dc.in/san/skt_gen/generators.html}
\item Dictionaries: (i) \url{http://www.aupasana.com/stardict}, (ii) \url{http://www.sanskrit-lexicon.uni-koeln.de/}, and (iii) \url{http://dsal.uchicago.edu/}
\item The AVG-Sanskrit site - \url{http://avg-sanskrit.org/about/}
\item A Most Useful Site - \url{http://sanskritdocuments.org/learning_tools/}
\end{ewosp}

\section{Document Home}
The latest copy of this document can be downloaded from \url{http://sanskritdocuments.org/learning_tools/SubantaRuupaNi.pdf}, the \LaTeX{} file used to create the pdf is at: \url{http://sanskritdocuments.org/learning_tools/SubantaRuupaNi.tex}. This document is housed with many other wonderful Sanskrit learning documents at: \url{http://sanskritdocuments.org/learning_tools/}.
\end{document}
