%%%%%%%%%%%%%%%%%%%%%%%%%%%%%%%%%%%%%%%%%%%%%%%%%%%%%%%%%%%%%%%%%%%%%%%%%%%%%%%%%%
\begin{frame}[fragile]\frametitle{}
\begin{center}
{\Large Introduction}
\end{center}
\end{frame}


%%%%%%%%%%%%%%%%%%%%%%%%%%%%%%%%%%%%%%%%%%%%%%%%%%%%%%%%%%%%%%%%%%%%%%%%%%%%%%%%%%
\begin{frame}[fragile]\frametitle{Introduction to Yoganidra}
    \textbf{Yoga Nidra (योगनिद्रा)} is a deep relaxation technique that:
    \begin{itemize}
        \item Relieves stress.
        \item Improves sleep.
        \item Accesses the bliss state (Ananda आनन्द).
    \end{itemize}
    \textbf{Inspired by the Bihar School of Yoga}, this script follows the inward journey through the Koshas.
\end{frame}

%%%%%%%%%%%%%%%%%%%%%%%%%%%%%%%%%%%%%%%%%%%%%%%%%%%%%%%%%%%%%%%%%%%%%%%%%%%%%%%%%%
\begin{frame}[fragile]\frametitle{Nidra vs Yoganidra}
    \textbf{Nidra (निद्रा)}: \\
    \begin{itemize}
        \item Unaware, only physical relaxation.
        \item Unconscious state.
    \end{itemize}
    \vspace{5mm}
    \textbf{Yoganidra (योगनिद्रा)}: \\
    \begin{itemize}
        \item Aware relaxation (physical, mental, and emotional).
        \item Conscious of subconscious mind.
    \end{itemize}
\end{frame}

%%%%%%%%%%%%%%%%%%%%%%%%%%%%%%%%%%%%%%%%%%%%%%%%%%%%%%%%%%%%%%%%%%%%%%%%%%%%%%%%%%
\begin{frame}[fragile]\frametitle{8 Stages of Yoganidra}
    \begin{enumerate}
        \item \textbf{Preparation (Shavasana)}: Deep breaths in Shavasana (शवासन).
        \item \textbf{Resolve (Sankalpa)}: Optional positive affirmation (संकल्प).
        \item \textbf{Body Awareness (Rotation)}: Relax body parts.
        \item \textbf{Breath Awareness}: Relaxation through breath.
        \item \textbf{Opposite Sensations}: Experience and release emotions.
        \item \textbf{Visualization}: Reach the subconscious with imagery.
        \item \textbf{Resolve (Sankalpa)}: Repeat the Sankalpa again.
        \item \textbf{Exiting}: Return awareness to external surroundings.
    \end{enumerate}
\end{frame}

%%%%%%%%%%%%%%%%%%%%%%%%%%%%%%%%%%%%%%%%%%%%%%%%%%%%%%%%%%%%%%%%%%%%%%%%%%%%%%%%%%
\begin{frame}[fragile]\frametitle{Key Instructions}
    \begin{itemize}
        \item No movement during Yoganidra.
        \item Stay awake, do not fall asleep.
        \item Do not think, just follow the instructions.
    \end{itemize}
\end{frame}


%%%%%%%%%%%%%%%%%%%%%%%%%%%%%%%%%%%%%%%%%%%%%%%%%%%%%%%%%%%%%%%%%%%%%%%%%%%%%%%%%%
\begin{frame}[fragile]\frametitle{The Koshas (कोश)}
    \begin{itemize}
        \item \textbf{Annamaya Kosha (अन्नमयकोश)} - Physical Body
        \item \textbf{Pranamaya Kosha (प्राणमयकोश)} - Energy Body
        \item \textbf{Manomaya Kosha (मनोमयकोश)} - Emotional Body
        \item \textbf{Vijnanamaya Kosha (विज्ञानमयकोश)} - Wisdom Body
        \item \textbf{Anandamaya Kosha (आनन्दमयकोश)} - Bliss Body
    \end{itemize}
\end{frame}

%%%%%%%%%%%%%%%%%%%%%%%%%%%%%%%%%%%%%%%%%%%%%%%%%%%%%%%%%%%%%%%%%%%%%%%%%%%%%%%%%%
\begin{frame}[fragile]\frametitle{Koshas in Yoganidra}
    
    \begin{itemize}
        \item Body Awareness (Rotation): \textbf{Annamayakosha (अन्नमयकोश) - Physical Body:} Focus on different body parts (right palm, right arm, legs, back, etc.).
        \item Breath Awareness: \textbf{Pranamayakosha (प्राणमयकोश) - Breath Awareness:} Reverse breath count from 27.
        \item Opposite Sensations: \textbf{Manomayakosha (मनोमयकोश) - Emotional Body:} Experience opposite sensations (hot/cold, wet/dry).
        \item Visualization: \textbf{Vijnanamayakosha (विज्ञानमयकोश) - Subconscious Visualization:} Visualize calming scenes like deserts, lakes, and waves.
    \end{itemize}
\end{frame}

%%%%%%%%%%%%%%%%%%%%%%%%%%%%%%%%%%%%%%%%%%%%%%%%%%%%%%%%%%%%%%%%%%%%%%%%%%%%%%%%%%
\begin{frame}[fragile]\frametitle{Tips for Practicing Yoganidra}
    \begin{itemize}
        \item Use simple and precise language in the script.
        \item Speak in a clear and even tone.
        \item Sit comfortably and be still during facilitation.
        \item Practice in a warm, comfortable space. Use props (pillows, blankets) to support the body.
        \item Remain still, but do not fall asleep.
    \end{itemize}
\end{frame}
