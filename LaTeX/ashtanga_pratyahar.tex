%%%%%%%%%%%%%%%%%%%%%%%%%%%%%%%%%%%%%%%%%%%%%%%%%%%%%%%%%%%%%%%%%%%%%%%%%%%%%%%%%%
\begin{frame}[fragile]\frametitle{}
\begin{center}
{\Large Pratyahar प्रत्याहार}
\end{center}
\end{frame}

%%%%%%%%%%%%%%%%%%%%%%%%%%%%%%%%%%%%%%%%%%%%%%%%%%%%%%%%%%%
\begin{frame}[fragile]\frametitle{Introduction}

Swavishasamprayoge Chittvaswarupanukar eevendrayanang Pratyaharah

स्वविषयासम्प्रयोगे चित्तस्वरुपानुकार इवेन्द्रियाणां प्रत्याहार:||

	\begin{itemize}
	\item Pratyahara  is  withdrawing 
the  senses  or  organs  from 
their  contact  with  the 
objects  in  the  external 
world.  
	\item Sri  Ramakrishna  has 
explained  it  thus  :  the 
moment  an  elephant 
stretches out its trunk to eat 
neighbor’s  garden,  it  gets  a 
blow  from  the  iron  goad  of 
driver 
	\end{itemize}

\end{frame}



%%%%%%%%%%%%%%%%%%%%%%%%%%%%%%%%%%%%%%%%%%%%%%%%%%%%%%%%%%%
\begin{frame}[fragile]\frametitle{What is Pratyahara?}


	\begin{itemize}
	\item Pratyahara is fifth of the eight limbs – yama, 
niyama, asana, pranayama, pratyahara, 
dharana, dhyana, samadhi
	\item One of the most important and yet, the least 
discussed, taught or practiced limbs of yoga!
	\item Most often referred to as an “external limb” 
along with yamas, niyamas, asana and 
pranayama
	\item Yamas and Niyamas (truth, non-violence, 
purity etc.) help purify the mind
	\end{itemize}

\tiny{(Ref: Eight Limbs of Yoga - Subhash Mittal)}

\end{frame}

%%%%%%%%%%%%%%%%%%%%%%%%%%%%%%%%%%%%%%%%%%%%%%%%%%%%%%%%%%%
\begin{frame}[fragile]\frametitle{What is Pratyahara?}


	\begin{itemize}
	\item Pratyahara is fifth of the eight limbs – yama, 
niyama, asAsanas strengthen the body and make it free 
of disease
	\item Pranayama energizes the vital body, calms the 
mind and aids in controlling the senses
	\item To achieve stability, mind needs to be cut off 
from the five senses
	\item  Mind is then ready for meditation and 
samadhi
	\end{itemize}

\tiny{(Ref: Eight Limbs of Yoga - Subhash Mittal)}

\end{frame}

%%%%%%%%%%%%%%%%%%%%%%%%%%%%%%%%%%%%%%%%%%%%%%%%%%%%%%%%%%%
\begin{frame}[fragile]\frametitle{What is Pratyahara?}


	\begin{itemize}
	\item  When separated from their corresponding 
objects, the organs follow, as it were, the nature 
of the mind -Sutra 2.54
	\item   That brings supreme control of the sense organs -
Sutra 2.55
	\item   Pratyahara = ''prati'' (against/away) + ''ahara'' 
(food/inputs)  = withdrawing away from the 
sense inputs.
	\end{itemize}

\tiny{(Ref: Eight Limbs of Yoga - Subhash Mittal)}

\end{frame}

%%%%%%%%%%%%%%%%%%%%%%%%%%%%%%%%%%%%%%%%%%%%%%%%%%%%%%%%%%%
\begin{frame}[fragile]\frametitle{What is Pratyahara?}


	\begin{itemize}
	\item The conscious withdrawal of energy from the senses
	\item Participate in the task at hand, I have a space between the world around me and my responses to that world.
	\item Responding to the world isn't a problem in and of itself; the problem comes when response is with knee-jerk reactions rather than with actions that one chooses.
	\item Act not React.
	\item Practice pratyahara by withdrawing energy from thoughts about the pose and focusing instead on the pose itself.
	\end{itemize}

{\tiny (Ref : Patanjali's Yoga Sutra: How to Live by the Yama - Judith Lasater)}
\end{frame}



%%%%%%%%%%%%%%%%%%%%%%%%%%%%%%%%%%%%%%%%%%%%%%%%%%%%%%%%%%%
\begin{frame}[fragile]\frametitle{Pratyahara: outside-in Approach}


	\begin{itemize}
	\item  Senses are withdrawn away from their respective 
objects
	\item   They remain in their own natural state
	\item   Once the senses are not in contact with their 
objects, the mind also has no contact with them. 
It then stays in its own natural state
	\item   The senses then seem to resemble the state of 
the mind. 
	\item   Such a control of the senses and of the mind is 
called Pratyahara.
	\end{itemize}

\tiny{(Ref: Eight Limbs of Yoga - Subhash Mittal)}

\end{frame}

%%%%%%%%%%%%%%%%%%%%%%%%%%%%%%%%%%%%%%%%%%%%%%%%%%%%%%%%%%%
\begin{frame}[fragile]\frametitle{Pratyahara: inside-out Approach}


	\begin{itemize}
	\item When stilled through vairagya (detachment) 
and other practices, mind is no longer 
interested in the five senses
	\item  Senses continue to receive input thru the 
sense organs (eyes, ears etc)
	\item Mind is unwilling to retrieve these inputs
	\item  The senses, unable to distract the mind, 
appear as if they are following the mind
	\end{itemize}

\tiny{(Ref: Eight Limbs of Yoga - Subhash Mittal)}

\end{frame}


%%%%%%%%%%%%%%%%%%%%%%%%%%%%%%%%%%%%%%%%%%%%%%%%%%%%%%%%%%%
\begin{frame}[fragile]\frametitle{Sensory Overload}


	\begin{itemize}
	\item  Senses constantly bombarded with attractions 
thru the media – TV, newspaper, internet etc
	\item  It becomes hard to control mind and senses
	\item  Senses begin to control the mind, our 
thoughts and behavior patterns
	\item  Mind can benefit from a ``sensory fast'' similar 
to the body benefitting from a ``food fast''
	\end{itemize}

\tiny{(Ref: Eight Limbs of Yoga - Subhash Mittal)}

\end{frame}

%%%%%%%%%%%%%%%%%%%%%%%%%%%%%%%%%%%%%%%%%%%%%%%%%%%%%%%%%%%
\begin{frame}[fragile]\frametitle{Pratyahara Techniques}


	\begin{itemize}
	\item   Pranayama – senses follow prana (vital life force) so we energize 
prana thru pranayama
	\begin{itemize}

	\item    Deep, slow breathing techniques calm the nerves, mind
	\item    Duirng Kumbhaka, the yogi withdraws his awareness from the five 
senses
	\end{itemize}

	\item    Karma Pratyahara (control of actions) – selfless service, offering all 
actions and fruits thereof to Lord
	\item    Focus on one sense impression – blue sky, ocean, tree etc (like 
mono-diet can heal the body)
	\item    Creating positive impressions – meditate on nature, visiting 
temples, offering incense, flowers etc
	\item    Visualization techniques
	\item    In the practice of Yoga Nidra, we are able to turn the senses inward
	\end{itemize}

\tiny{(Ref: Eight Limbs of Yoga - Subhash Mittal)}

\end{frame}

%%%%%%%%%%%%%%%%%%%%%%%%%%%%%%%%%%%%%%%%%%%%%%%%%%%%%%%%%%%
\begin{frame}[fragile]\frametitle{Pratyahara Techniques}


	\begin{itemize}
	\item   Practice Yoni Mudra – symbolically shutting off 
the senses and breathing out with humming 
sound – after pranayama
	\item  Trataka (gazing) – normally practiced with a 
lighted candle
	\item   Pratyahara meditation – connecting with and 
withdrawing from senses one at a time
	\end{itemize}

\tiny{(Ref: Eight Limbs of Yoga - Subhash Mittal)}

\end{frame}

%%%%%%%%%%%%%%%%%%%%%%%%%%%%%%%%%%%%%%%%%%%%%%%%%%%%%%%%%%%
\begin{frame}[fragile]\frametitle{Pratyahara in Bhagavad Gita}


	\begin{itemize}
	\item  The five senses are compared to the five horses of 
a chariot:
	\begin{itemize}

	\item   The mind is the reins 
	\item  the soul (the atman) is the passenger
	\item   In order to keep the chariot (the human being) on 
course, it is important to keep the horses in control. 
	\end{itemize}
	\item   Just like a turtle which withdraws its limbs away, 
the yogi, by turning senses away from their 
objects, attains steady wisdom
	\end{itemize}

\tiny{(Ref: Eight Limbs of Yoga - Subhash Mittal)}

\end{frame}

%%%%%%%%%%%%%%%%%%%%%%%%%%%%%%%%%%%%%%%%%%%%%%%%%%%%%%%%%%%
\begin{frame}[fragile]\frametitle{Pratyahara in Bhagavad Gita}


	\begin{itemize}
	\item  When one thinks of objects, attachment is 
born which leads to the following “ladder of 
destruction”: 
	\item   attachment $\rightarrow$desires $\rightarrow$ anger $\rightarrow$ delusion $\rightarrow$ loss 
of memory $\rightarrow$ loss of intellect and discriminatory 
power $\rightarrow$ total annihilation. 
	\item   One who is self-controlled, even though 
moving among the objects, attains peace 
since the senses remain under control and 
free from attraction
	\end{itemize}

\tiny{(Ref: Eight Limbs of Yoga - Subhash Mittal)}

\end{frame}

%%%%%%%%%%%%%%%%%%%%%%%%%%%%%%%%%%%%%%%%%%%%%%%%%%%%%%%%%%%
\begin{frame}[fragile]\frametitle{Summary}


	\begin{itemize}
	\item   Pratyahara is a critically important limb on the 
path to meditation and samadhi
	\item   Techniques like pranayama, yoga nidra and 
`pratyahara meditation' help attain the state 
of pratyahara (sense withdrawal)
	\item   It is easier to control the mind when it is not 
distracted by the senses and their objects
	\end{itemize}

\tiny{(Ref: Eight Limbs of Yoga - Subhash Mittal)}

\end{frame}
