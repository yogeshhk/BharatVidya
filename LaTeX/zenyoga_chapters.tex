%%%%%%%%%%%%%%%%%%%%%%%%%%%%%%%%%%%%%%%%%%%%%%%%%%%%%%%%%%%%%%%%
\begin{frame}[fragile]\frametitle{}
\begin{center}
{\Large ZenYoga Chapters}


{\tiny (Based on ``Zen Yoga'' by P J Saher)}
\end{center}
\end{frame}


%%%%%%%%%%%%%%%%%%%%%%%%%%%%%%%%%%%%%%%%%%%%%%%%%%%%%%%%%%%%%%%%
\begin{frame}[fragile]\frametitle{}
\begin{center}
{\Large Chapter 1}

{\tiny (Based on ``Life Simplified'' YouTube Channel by Abhishen)}

\end{center}
\end{frame}

%%%%%%%%%%%%%%%%%%%%%%%%%%%%%%%%%%%%%%%%%%%%%%%%%%%%%%%%%%%%%%%%
\begin{frame}[fragile]\frametitle{Introduction}

\begin{itemize}
    \item The chapter discusses the mind, its thinking process, and how both interact.
    \item How thoughts influence the mind and how the mind gets distracted.
    \item Examples are provided to show how our thoughts drift and how we lose track of the original subject.
\end{itemize}

\end{frame}


%%%%%%%%%%%%%%%%%%%%%%%%%%%%%%%%%%%%%%%%%%%%%%%%%%%%%%%%%%%%%%%%
\begin{frame}[fragile]\frametitle{Types of Drifting Thoughts}

\begin{itemize}
    \item Drifting thoughts can be classified into:
    \begin{itemize}
        \item Controllable drifts
        \item Uncontrollable drifts
        \item Unrecognized drifts
    \end{itemize}
    \item Controllable drifts: When we realize the mind has drifted and we bring it back to the original subject.
    \item Uncontrollable drifts: When we completely forget the original subject and cannot recall how we got there.
\end{itemize}

\end{frame}

%%%%%%%%%%%%%%%%%%%%%%%%%%%%%%%%%%%%%%%%%%%%%%%%%%%%%%%%%%%%%%%%
\begin{frame}[fragile]\frametitle{Drifting Thoughts: Examples}

\begin{itemize}
    \item Example of controllable drift: 
        \begin{itemize}
            \item We start thinking about one subject, but our mind wanders. 
            \item Realizing the drift, we bring ourselves back to the original thought.
        \end{itemize}
    \item Example of uncontrollable drift: 
        \begin{itemize}
            \item The mind jumps from one thought to another, and eventually, we forget what we were originally thinking about.
        \end{itemize}
\end{itemize}

\end{frame}

%%%%%%%%%%%%%%%%%%%%%%%%%%%%%%%%%%%%%%%%%%%%%%%%%%%%%%%%%%%%%%%%
\begin{frame}[fragile]\frametitle{Impact of Thoughts on Actions}

\begin{itemize}
    \item Our thoughts are directly connected to our actions.
    \item Understanding the thinking process helps shape our life.
    \item If we don’t control our thoughts, we may end up in a difficult mental state.
    \item We are responsible for our thoughts, and in turn, our actions and life direction.
\end{itemize}

\end{frame}

%%%%%%%%%%%%%%%%%%%%%%%%%%%%%%%%%%%%%%%%%%%%%%%%%%%%%%%%%%%%%%%%
\begin{frame}[fragile]\frametitle{Mind and Thought Control}

\begin{itemize}
    \item If we cannot control our thoughts, it can lead to an uncontrolled mental state, or "mind drift."
    \item By understanding the nature of mind and thoughts, we can learn to control and redirect them.
    \item Meditation can help in recognizing the thought processes and bring the mind back to focus.
\end{itemize}

\end{frame}

%%%%%%%%%%%%%%%%%%%%%%%%%%%%%%%%%%%%%%%%%%%%%%%%%%%%%%%%%%%%%%%%
\begin{frame}[fragile]\frametitle{Meditation and Mind Control}

\begin{itemize}
    \item Meditation involves observing the drifting thoughts without attachment.
    \item It helps in understanding which part of the mind is active and dominant.
    \item Achieving a focused, clear state of mind is the goal of meditation.
\end{itemize}

\end{frame}

%%%%%%%%%%%%%%%%%%%%%%%%%%%%%%%%%%%%%%%%%%%%%%%%%%%%%%%%%%%%%%%%
\begin{frame}[fragile]
\frametitle{Drift Order}
\begin{itemize}
\item Drifts happen in the section of mind that deals with pictures, day-dreams.
\item Typically the come in this order:
\begin{itemize}
\item Sexual
\item Anger
\item Ego
\item Greed
\item Jealousy
\item Arrogance
\item Ignorance
\item Courage
\item Doubt
\item Day dreaming
\end{itemize}
\end{itemize}
\end{frame}


%%%%%%%%%%%%%%%%%%%%%%%%%%%%%%%%%%%%%%%%%%%%%%%%%%%%%%%%%%%%%%%%
\begin{frame}[fragile]\frametitle{Summary}

\begin{itemize}
	\item Drift: Change in thoughts away from the main subject.
	\item Types:
		\begin{itemize}
			\item Un-controllable: Unaware of it, can not recollect
			\item Controllable: Can detect and come back to the main subject.
		\end{itemize}
	\item Mind, if not trained, can not channelize drifts.
	\item This drift is not property of the whole mind but one section within it.
	\item So, learn to separate sections of mind and understand the functionality.
    \item Understanding mind and thought drifts is essential for self-awareness and personal growth.
    \item Thoughts lead to actions, and actions shape our lives.
    \item Meditation and mindfulness can help control the mind and direct it towards constructive actions.
    \item Mastery over thoughts leads to a harmonious and controlled life.
\end{itemize}

\end{frame}

%%%%%%%%%%%%%%%%%%%%%%%%%%%%%%%%%%%%%%%%%%%%%%%%%%%%%%%%%%%%%%%%
\begin{frame}[fragile]\frametitle{}
\begin{center}
{\Large Chapter 2}

{\tiny (Based on ``Life Simplified'' YouTube Channel by Abhishen)}
\end{center}
\end{frame}


%%%%%%%%%%%%%%%%%%%%%%%%%%%%%%%%%%%%%%%%%%%%%%%%%%%%%%%%%%%%%%%%
\begin{frame}[fragile]\frametitle{Introduction}
  \begin{itemize}
    \item This chapter focuses on three main concepts in Zenyoga: Simple Consciousness, Self-Consciousness, and Cosmic Consciousness.
    \item Simple Consciousness: Present in all living beings, including animals and humans.
    \item Self-Consciousness: Exclusive to humans, which helps us observe and control our mind's state.
    \item Cosmic Consciousness: A higher state of awareness that transcends individual identity.
  \end{itemize}
\end{frame}

%%%%%%%%%%%%%%%%%%%%%%%%%%%%%%%%%%%%%%%%%%%%%%%%%%%%%%%%%%%%%%%%
\begin{frame}[fragile]\frametitle{Simple Consciousness}
  \begin{itemize}
    \item Simple consciousness is a state present in all living beings.
    \item It allows us to be aware of our surroundings and react accordingly.
    \item This awareness is crucial for survival and adaptation to the environment.
    \item Animals also experience simple consciousness but lack the ability to reflect on their mental states.
  \end{itemize}
\end{frame}

%%%%%%%%%%%%%%%%%%%%%%%%%%%%%%%%%%%%%%%%%%%%%%%%%%%%%%%%%%%%%%%%
\begin{frame}[fragile]\frametitle{Self-Consciousness}
  \begin{itemize}
    \item Self-consciousness is unique to humans, allowing us to reflect on and control our mental states.
    \item Humans can observe their mood swings and emotions throughout the day.
    \item This ability to reflect and evolve one's mind is a key aspect of human nature.
    \item Unlike animals, humans can identify their identity and separate it from the environment.
  \end{itemize}
\end{frame}

%%%%%%%%%%%%%%%%%%%%%%%%%%%%%%%%%%%%%%%%%%%%%%%%%%%%%%%%%%%%%%%%
\begin{frame}[fragile]\frametitle{Challenges in Self-Consciousness}
  \begin{itemize}
    \item While humans have the potential for self-awareness, it is often difficult to concentrate and meditate.
    \item Our comfort zone and habits prevent us from evolving our minds effectively.
    \item The difficulty lies in sustaining concentration and overcoming habitual patterns.
  \end{itemize}
\end{frame}

%%%%%%%%%%%%%%%%%%%%%%%%%%%%%%%%%%%%%%%%%%%%%%%%%%%%%%%%%%%%%%%%
\begin{frame}[fragile]\frametitle{Cosmic Consciousness}
  \begin{itemize}
    \item Cosmic consciousness is a higher state of awareness that transcends individual identity.
    \item It is a state where one feels a deep connection with the universe.
    \item Achieving this state requires overcoming the distractions of the mind and focusing on a higher subject.
    \item It is a state that few have achieved due to the complexity of human tendencies.
  \end{itemize}
\end{frame}

%%%%%%%%%%%%%%%%%%%%%%%%%%%%%%%%%%%%%%%%%%%%%%%%%%%%%%%%%%%%%%%%
\begin{frame}[fragile]\frametitle{The Story of Narada and Krishna}
  \begin{itemize}
    \item Narada Muni asks Lord Krishna why humans remain trapped in the illusion of the physical world.
    \item Krishna explains that humans have the potential to achieve enlightenment but are distracted by the mind’s tendencies.
    \item Narada Muni forgets his mission to bring water for Krishna after being distracted by a woman at the river.
    \item The woman reveals herself to be Krishna, demonstrating how even enlightened beings can be distracted by worldly matters.
  \end{itemize}
\end{frame}



%%%%%%%%%%%%%%%%%%%%%%%%%%%%%%%%%%%%%%%%%%%%%%%%%%%%%%%%%%%%%%%%
\begin{frame}[fragile]\frametitle{Exercise}
\begin{itemize}
\item Put aside 15-30 minutes in a day to watch the drifts and note them down.
\item Take one thought as the main subject.
\item Let the mind drift.
\item Classify and note the drift.
\item Never try to make your mind ``blank''.
\item Analyze over 3 months, the persistent patterns.
\item After 15 days of consistent practice, you will start recognizing which mental centers dominate your mind.
\item Over time, this awareness will help you overcome mental distractions and develop a deeper connection with the 
\end{itemize}
\end{frame}




%%%%%%%%%%%%%%%%%%%%%%%%%%%%%%%%%%%%%%%%%%%%%%%%%%%%%%%%%%%%%%%%
\begin{frame}[fragile]\frametitle{Summary}
Forms of Consciousness:
\begin{itemize}
\item Simple: like animals, awareness of body. Needed for daily survival.
\item Self: Aware of mental states, sense of self, thoughts.
\item Cosmic: Higher state. Not achieved by logic, reasoning, etc but by stronger awareness.
\item No one has shown practical steps to go from Self to Cosmic consciousness.
\item Even if we get to know the steps, they are difficult to follow. Old patterns are hard to break away from.
\item The journey from simple consciousness to cosmic consciousness is challenging but achievable with dedication.
\item Regular practice, patience, and awareness are key to evolving the mind and reaching higher states of consciousness.
\item Zenyoga provides tools to understand and transcend the mind's limitations, leading to enlightenment.
\end{itemize}


\end{frame}


%%%%%%%%%%%%%%%%%%%%%%%%%%%%%%%%%%%%%%%%%%%%%%%%%%%%%%%%%%%%%%%%
\begin{frame}[fragile]\frametitle{}
\begin{center}
{\Large Chapter 3}

{\tiny (Based on ``Life Simplified'' YouTube Channel by Abhishen)}

\end{center}
\end{frame}


%%%%%%%%%%%%%%%%%%%%%%%%%%%%%%%%%%%%%%%%%%%%%%%%%%%%%%%%%%%%%%%%
\begin{frame}[fragile]
\frametitle{Chapter 3}
\begin{itemize}
\item Brain is a physical entity.
\item Impulses come from various sensory organs, which perturb the nerves/areas in brain. These are called as `thoughts'.
\item Collection of thoughts (or the software/platform, in IMHO) is called as `mind'.
\item Reactions to other minds can be of types:
\begin{itemize}
\item Affinity: love
\item Repulsion: hate
\item Indifference
\end{itemize}
\end{itemize}


\end{frame}

%%%%%%%%%%%%%%%%%%%%%%%%%%%%%%%%%%%%%%%%%%%%%%%%%%%%%%%%%%%%%%%%
\begin{frame}[fragile]
\frametitle{Mind as a Cloth}
\begin{itemize}
\item Thoughts are strands
\item Emotions are the colors
\item Repetition or habit strengthens and makes cloth durable.
\item Quality of thoughts correspond to smoothness or roughness of the cloth.
\item Grey matter, attitude, thought process: Shape of the cloth
\item Likes/Dislikes: fashion of the cloth.
\end{itemize}

Constant daily practice can change the properties of the cloth.
\end{frame}


%%%%%%%%%%%%%%%%%%%%%%%%%%%%%%%%%%%%%%%%%%%%%%%%%%%%%%%%%%%%%%%%
\begin{frame}[fragile]\frametitle{}
\begin{center}
{\Large Chapter 4}

{\tiny (Based on ``Life Simplified'' YouTube Channel by Abhishen)}

\end{center}
\end{frame}

%%%%%%%%%%%%%%%%%%%%%%%%%%%%%%%%%%%%%%%%%%%%%%%%%%%%%%%%%%%%%%%%
\begin{frame}[fragile]
\frametitle{Chapter 4}
\begin{itemize}
\item Incoming impulses are decoded in brain as Pure Mind Energy, which are divided into:

\begin{itemize}
\item Held: thoughts suppressed
\item Words: expressed as words or mental pictures, day dreaming.
\item Actions: Something is done by them, movements.
\end{itemize}
\item As per Patanjali, there are 5 states (`vrutti') of mind (`chitta'):
\begin{itemize}
\item Correct Knowledge
\item Incorrect Knowledge
\item Fancy
\item Passivity (sleep)
\item Memory
\end{itemize}
\item Mind can be controlled (``chitta vrutti nirodh:'') by efforts, detachment.
\item Peace of mind can be brought by regulation of ``prana'' (breath).
\end{itemize}

\end{frame}

%%%%%%%%%%%%%%%%%%%%%%%%%%%%%%%%%%%%%%%%%%%%%%%%%%%%%%%%%%%%%%%%
\begin{frame}[fragile]\frametitle{}
\begin{center}
{\Large Chapter 5}
\end{center}
\end{frame}


%%%%%%%%%%%%%%%%%%%%%%%%%%%%%%%%%%%%%%%%%%%%%%%%%%%%%%%%%%%%%%%%
\begin{frame}[fragile]
\frametitle{Chapter 5: Sleep}
\begin{itemize}
\item Types of Sleep are:
\resizebox{\textwidth}{!}{%
  \begin{tabular}{|c|c|c|c|c|}
  \hline
  Type & Beneficial? &Timing&Vibrational Color built in Body\\
  \hline
  Very Intense 	& 	Very beneficial	&	Midnight to 4am	&	Pale Blue\\
  Intense     	&	Beneficial		& 	11pm to Midnight, 4 to 5am	&	Pink\\
  Indifferent  	&	Does not add to energy	& 9 to 11pm, 5 to 7am &Green\\
  Wasting		& 	Losing energy & 7am to 12 noon & Yellow (dark) \\
  Damaging		& 	Bad to nerves &  12 noon to 4 pm & Orange (deep)\\
  Highly Damaging & Sickness, disease & 4pm to 9pm&  Red (deep)\\
  
  
  \hline
  \end{tabular}
} % end of scope of "\resizebox"  directive

\item Resting and sleeping are two different things.
\item 11pm to 5pm should be considered as a good time for sleep.
\end{itemize}
\end{frame}


%%%%%%%%%%%%%%%%%%%%%%%%%%%%%%%%%%%%%%%%%%%%%%%%%%%%%%%%%%%%%%%%
\begin{frame}[fragile]\frametitle{}
\begin{center}
{\Large Notes from Other References}
\end{center}
\end{frame}


%%%%%%%%%%%%%%%%%%%%%%%%%%%%%%%%%%%%%%%%%%%%%%%%%%%%%%%%%%%%%%%%
\begin{frame}[fragile]
\frametitle{``Introduction To Three Step Rhythmic Breathing(3SRB)'' by Mr.Deepak Dhingra}


\begin{itemize}
\item Yaksh : ``What's the most delusion that humans have?''
\item Yudhishthiara: ``People live life as if they are not going to die''.
\item Its inevitable. No one has control. 
\item Only sometimes externally.
\item But we have no control internally. 
\end{itemize}
\end{frame}


%%%%%%%%%%%%%%%%%%%%%%%%%%%%%%%%%%%%%%%%%%%%%%%%%%%%%%%%%%%%%%%%
\begin{frame}[fragile]
\frametitle{Mind}
\begin{itemize}
\item None of the internal organs behave the way we want. 
\item Including brain!! although we believe we have control on that
\item Brain is just another organ, a hardware. And mind is the energy/software that runs on it.
\item Proof: After death, hardware remains, but the software/energy is shutdown, so cant function.
\end{itemize}
\end{frame}



%%%%%%%%%%%%%%%%%%%%%%%%%%%%%%%%%%%%%%%%%%%%%%%%%%%%%%%%%%%%%%%%
\begin{frame}[fragile]
\frametitle{Free Will?}

\begin{itemize}
\item We have no control internally. Many centers are producing impulses. Most of the organ functioning is autonomous
\item Thus we cannot control mind, but we have to bring them to rhythm, balance.
\item As per Pantanjali, daily, out of 13k impulses 120 go to brain. Rest are used to keep system working. Out of 120 to brain, 12 are per second, Only one becomes a thought. This just one more theory. May ignore. Just a model.
% \item We are interpreting life our way and not how it is
\end{itemize}
\end{frame}

%%%%%%%%%%%%%%%%%%%%%%%%%%%%%%%%%%%%%%%%%%%%%%%%%%%%%%%%%%%%%%%%
\begin{frame}[fragile]
\frametitle{Mind}
\begin{itemize}
\item As per Patanjali, Breathing is one way to control the mind.
\item Breathing happens as per emotions, eg. anger.
\item Can controlling breathing control emotions? Patanjali says, Yes.
\item Mind (Software) is run by breath. When Breath stops, its called death.
\item Need Deep, Rhythmic and Belly breathing
\end{itemize}
\end{frame}

%%%%%%%%%%%%%%%%%%%%%%%%%%%%%%%%%%%%%%%%%%%%%%%%%%%%%%%%%%%%%%%%
\begin{frame}[fragile]
\frametitle{3SRB (3 Step Rhythmic Breathing )}

Technique

\begin{itemize}
\item Both chest and abdomen are raised and lowered together equally.
\item Note: You can lie down in front of the mirror with two heavy books one on the chest and the other on the abdomen. 
\item Check whether both move together.
\end{itemize}

\tiny{(Ref: https://www.3srb.org/3srb/how-to-practise-3srb.html )}
\end{frame}

%%%%%%%%%%%%%%%%%%%%%%%%%%%%%%%%%%%%%%%%%%%%%%%%%%%%%%%%%%%%%%%%
\begin{frame}[fragile]
\frametitle{3SRB (3 Step Rhythmic Breathing )}

Volume

\begin{itemize}
\item The breath is full from neck to naval, i.e. the upper, middle and lower abdomen are filled to normal capacity
\item The quantum of air inhaled and exhaled is what is usually normal to us neither too much nor too forceful, since normal 3SRB is not an exercise but a process of correct breathing. 
\item Note: Initially, to establish the rhythm your breath will be deeper. Once, the technique and volume is mastered, then the volume of the breath will be normal as you breath today.
\end{itemize}
\end{frame}


%%%%%%%%%%%%%%%%%%%%%%%%%%%%%%%%%%%%%%%%%%%%%%%%%%%%%%%%%%%%%%%%
\begin{frame}[fragile]
\frametitle{3SRB (3 Step Rhythmic Breathing )}

Rhythm

\begin{itemize}
\item In 3 sec out 2 sec | one hand chest one hand on belly| equal volume. No jerks. 
\item  While counting its 1-2-3 (4) 5-6. Here, at ``4'' it is the turn of breath. Is not counted but just understood. Later, same but with deep breathing, and fast
\item 12 cycles a minute to start with then to 24 and 36.
\item  Increase the duration of practice by 5 minutes every fortnight until 6 months time and until one hour of conscious 3SRB is reached.
\end{itemize}
\end{frame}


%%%%%%%%%%%%%%%%%%%%%%%%%%%%%%%%%%%%%%%%%%%%%%%%%%%%%%%%%%%%%%%%
\begin{frame}[fragile]
\frametitle{3SRB by ShriRajen Vakil}


\begin{itemize}
\item Deep Chest breathing: Both hands on chest. Let breath come in and push chest out. 36 times a minute. Only for a minute. On rhythm 123-56. Area of anger, jealousy stored so far.
\item Deep Stomach breathing: Both hands on belly. Let breath come in and push belly out. On exhale push belly in. 36 times a minute. Only for a minute. On rhythm 123-56. Area of fear and worry for future.
\item Paschimottanasan breathing: Head up. Both chest and stomach should come out on breath in. 36 times a minute. Only for a minute. On rhythm 123-56. For backbone, the channel for energy to go up.
\item Take breath in 5 installments and breath in through mouth forcefully. 12345-Out
\item Breath in 5 seconds, hold 5, empty 5, hold 5. 3 such rounds.
\item Inhale, hold breath, touch chin, swallow 5 times. Area of Pain.

\end{itemize}
\end{frame}
