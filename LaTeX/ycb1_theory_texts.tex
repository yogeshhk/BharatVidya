%%%%%%%%%%%%%%%%%%%%%%%%%%%%%%%%%%%%%%%%%%%%%%%%%%%%%%%%%%%%%%%%%%%%%%%%%%%%%%%%%%
\begin{frame}[fragile]\frametitle{}
\begin{center}
{\Large Introduction to Yoga Texts }
\end{center}
\end{frame}

%%%%%%%%%%%%%%%%%%%%%%%%%%%%%%%%%%%%%%%%%%%%%%%%%%%%%%%%%%%
\begin{frame}[fragile]\frametitle{Syllabus}

\begin{itemize}
\item 2.1  Introduction  and  study  of  Patanjala  Yoga  Sutra  including  memorization  of  selected Sutras (Chapter I- 1-12). 
\item 2.2  Introduction  and  study  of  Bhagavad  Gita  including  memorization  of  selected  Slokas (Chapter II -47, 48, 49, 50 and 70). 
\item 2.3  Introduction and study of Hathpradipika.  
\item 2.4  General Introduction to Prasthanatrayee. 
\item 2.5  Concepts  and  principles  of  Aahara  (Diet)  in  Hathapradipika  and  Bhagawadgita  (Mitahara and Yuktahara). 
\item 2.6  Significance of Hatha Yoga practices in health and well being. 
\item 2.7  Concept of mental wellbeing according to Patanjala Yoga. 
\item 2.8  Yogic practices of Patanjala Yoga: Bahiranga and Antaranga Yoga. 
\item 2.9  Concepts of healthy living in Bhagwad Gita. 
\item 2.10  Importance of subjective experience in daily Yoga practice. 
\end{itemize}
	  
\end{frame}

%%%%%%%%%%%%%%%%%%%%%%%%%%%%%%%%%%%%%%%%%%%%%%%%%%%%%%%%%%%
\begin{frame}[fragile]\frametitle{The Vedas and the Upanishads}
      \begin{itemize}
        \item \textbf{Vedas Sections}: 
          
          The वेद (Vedas) are divided into two sections:
          \begin{itemize}
            \item \textbf{Karma Kanda} (Ritual portion)
            \item \textbf{Jnana Kanda} (Knowledge portion)
          \end{itemize}
          
        \item \textbf{Upanishads}: 
          
          The Upanishads are contained in the knowledge portion of the Vedas. They describe the inner vision of reality through self-inquiry and expound upon three subjects:
          \begin{itemize}
            \item \textbf{The Jiva} - (Embodied soul)
            \item \textbf{Jagat} - (The World)
            \item \textbf{Ishwara} - (God or the creator of the universe)
          \end{itemize}
          The climax of the enquiry is the experience of the essential identity of आत्मन् (Atman) within with ब्रह्मन् (Brahman).
          
        \item \textbf{Vedic Yoga}: 
          
          The Vedas contain the oldest known yogic teachings called Vedic Yoga. Vedic Yogis (ऋषि) taught how to live in divine harmony and see the ultimate reality through intensive spiritual practice.
          
      \end{itemize}
\end{frame}



%%%%%%%%%%%%%%%%%%%%%%%%%%%%%%%%%%%%%%%%%%%%%%%%%%%%%%%%%%%%%%%%%%%%%%%%%%%%%%%%%%
\begin{frame}[fragile]\frametitle{}
\begin{center}
{\Large 2.1  Introduction  and  study  of  Patanjala  Yoga  Sutra  including  memorization  of  selected Sutras (Chapter I- 1-12)}
\end{center}
\end{frame}

%%%%%%%%%%%%%%%%%%%%%%%%%%%%%%%%%%%%%%%%%%%%%%%%%%%%%%%%%%%%%%%%%%%%%%%%%%%%%%%%%%
\begin{frame}[fragile]\frametitle{Key Verses of Yoga Sutra}

    \begin{itemize}
        \item अथ योगानुशासनम् ॥ १ . १ ॥ - Introduction to Yoga and its practice.
        \item योगिश्चत्तवृित्तिनरोधः ॥ १ . २ ॥ - Yoga is controlling mental fluctuations.
        \item तदा द्रष्टुः स्वरूपेऽवस्थानम् ॥ १ . ३ ॥ - Perceiver returns to true self.
        \item वृत्तसारूप्यिमतरत्र ॥ १ . ४ ॥ - Mental states conform to thoughts.
        \item वृत्तयः पञ्चतय्यः क्लिष्टाऽक्लिष्टाः ॥ १ . ५ ॥ - Five types,  painful - non-painful.
        \item प्रमाणिविपरीयिवकल्पिनद्रास्मृतयः ॥ १ . ६ ॥ - Types: perception, error, imagination, sleep, memory.
        \item प्रत्यक्षानुमानागमाः प्रमाणानि ॥ १ . ७ ॥ - Sources of valid knowledge: direct perception, inference, testimony.
        \item विपरीयो मिथ्याज्ञानमतद्रूपप्रतिष्ठम् ॥ १ . ८ ॥ - Incorrect knowledge is based on false information.
        \item शब्दज्ञानानुपाती वस्तुशून्यो विकल्पः ॥ १ . ९ ॥ - Imagination is based on words without reality.
        \item अभावप्रत्ययालम्बना वृत्तिनिद्रा ॥ १ . १० ॥ - Sleep is absence of objective awareness.
        \item अनुभूतिवषयासंप्रमोषः स्मृतिः ॥ १ . ११ ॥ - Memory is retention of experienced impressions.
        \item अभ्यासवैराग्याभ्यां तन्निरोधः ॥ १ . १२ ॥ - Control of mental states through practice and detachment.
    \end{itemize}

\end{frame}

%%%%%%%%%%%%%%%%%%%%%%%%%%%%%%%%%%%%%%%%%%%%%%%%%%%%%%%%%%%%%%%%%%%%%%%%%%%%%%%%%%
\begin{frame}[fragile]\frametitle{Summary of Patanjali Yoga Sutra}

    \begin{itemize}
        \item \textbf{Yoga Meaning:} Derived from 'Yuj' – union and concentration.
        \item \textbf{Union Aspect:} Integration of body, mind, and spirit.
        \item \textbf{Concentration Aspect:} Yoga as focused awareness and ultimate goal.
        \item \textbf{Definition:} \textit{“Yogaḥ cittavṛtti nirodhaḥ”} - Stoppage of mental modifications.
        \item \textbf{Vrittis:} Mental modifications or thought waves.
        \item \textbf{Goal:} Liberate from suffering and cycle of rebirth by controlling vrittis.
        \item \textbf{Samadhi:} Ultimate limb of Ashtanga Yoga, representing deep concentration.
        \item \textbf{Mind Functions:} Misery arises from false identification at the mental level.
    \end{itemize}

\end{frame}

%%%%%%%%%%%%%%%%%%%%%%%%%%%%%%%%%%%%%%%%%%%%%%%%%%%%%%%%%%%%%%%%%%%%%%%%%%%%%%%%%%
\begin{frame}[fragile]\frametitle{Definition of Yoga and Patanjali's Ashtanga Yoga}

    \begin{itemize}
        \item \textbf{Chitta:} The mind or mind stuff.
        \item \textbf{Vritti:} Modifications or fluctuations of the mind.
        \item \textbf{Nirodhah:} Cessation or control of mind modifications.
        \item \textbf{Basis:} Yoga Darshana is based on this aphorism.
        \item \textbf{Ashtanga Yoga:} Propounded by Patanjali as the Royal (Kingly) path.
        \item \textbf{Supreme Yoga:} Incorporates fundamental tenets of other Yoga systems (Jnana, Bhakti, Karma, Hatha, Mantra).
    \end{itemize}

\end{frame}

%%%%%%%%%%%%%%%%%%%%%%%%%%%%%%%%%%%%%%%%%%%%%%%%%%%%%%%%%%%%%%%%%%%%%%%%%%%%%%%%%%
\begin{frame}[fragile]\frametitle{Aim of Patanjali’s Ashtanga Yoga and Concept of Chitta}

    \begin{itemize}
        \item \textbf{Aim:} \textit{Kaivalya} (liberation) through mind management.
        \item \textbf{Focus:} Concentration to end all miseries and suffering.
        \item \textbf{Physical Postures:} Support stability for prolonged meditation.
        \item \textbf{Chitta:} The Western term for mind; has four aspects:
        \begin{itemize}
            \item \textbf{Manas:} \textit{मनस} (thinking, doubting, willing).
            \item \textbf{Chitta:} \textit{चित्त} (past experiences, memory).
            \item \textbf{Buddhi:} \textit{बुद्धि} (discrimination, decision-making).
            \item \textbf{Ahamkara:} \textit{अहंकार} (self-identification, separation).
        \end{itemize}
    \end{itemize}

\end{frame}

%%%%%%%%%%%%%%%%%%%%%%%%%%%%%%%%%%%%%%%%%%%%%%%%%%%%%%%%%%%%%%%%%%%%%%%%%%%%%%%%%%
\begin{frame}[fragile]\frametitle{Concept of Chitta and Chitta Bhumis; Chitta Vrittis and Chitta Vrittinirodhopaya}

    \begin{itemize}
        \item \textbf{Antahkarana:} 
        \begin{itemize}
            \item \textit{चित्त} (Chitta): Storehouse of Samskaras
            \item \textit{बुद्धि} (Buddhi): Decision-making faculty
            \item \textit{अहंकार} (Ahamkara): Ego
            \item \textit{मनस्} (Manas): Synthesizing faculty
        \end{itemize}
        \item \textbf{Chitta Bhumi:} Condition/state of mind in concentration
        \item \textbf{Chitta Qualities:} \textit{सत्त्विक} (Sattvic), \textit{राजसिक} (Rajasic), \textit{तामसिक} (Tamasic)
        \item \textbf{Chitta Bhumis:} Five modes of manifestation
        \begin{itemize}
            \item \textbf{क्षिप्त} (Kshipta): Scattered, anxious (Rajasic)
            \item \textbf{मूढ} (Mudha): Dull, stupid (Tamasic)
            \item \textbf{विक्षिप्त} (Vikshipta): Occasionally centered (Rajasic)
            \item \textbf{एकाग्रता} (Ekagrata): One-pointed, concentrated (Sattvic)
            \item \textbf{निरुद्ध} (Niruddha): Suspended mental activity (Sattvic, obstructed Rajas and Tamas)
        \end{itemize}
    \end{itemize}

\end{frame}

%%%%%%%%%%%%%%%%%%%%%%%%%%%%%%%%%%%%%%%%%%%%%%%%%%%%%%%%%%%%%%%%%%%%%%%%%%%%%%%%%%
\begin{frame}[fragile]\frametitle{Chitta Levels in Yoga}

    \begin{itemize}
        \item First 3 levels of \textit{चित्त} (Chitta) are not considered Yoga:
        \begin{itemize}
            \item \textit{क्षिप्त} (Kshipta)
            \item \textit{मूढ} (Mudha)
            \item \textit{विक्षिप्त} (Vikshipta)
        \end{itemize}
        \item \textit{एकाग्रता} (Ekagrata) and \textit{निरुद्ध} (Niruddha) are considered Yoga.
        \item Passing through \textit{Ekagrata} and \textit{Niruddha} leads to \textit{समाधि} (Samadhi).
        \item \textbf{Samskaras:}
        \begin{itemize}
            \item \textit{प्रारब्धसंस्कार} (Praarabdha Samskara): Accumulated impressions from previous births
            \item \textit{वासनासंस्कार} (Vasana Samskara)
        \end{itemize}
        \item \textit{निरोधसंस्कार} (Nirodha Samskara) remains in Chitta when others are restrained.
    \end{itemize}

\end{frame}

%%%%%%%%%%%%%%%%%%%%%%%%%%%%%%%%%%%%%%%%%%%%%%%%%%%%%%%%%%%%%%%%%%%%%%%%%%%%%%%%%%
\begin{frame}[fragile]\frametitle{Chitta-Vrittis}
    \begin{itemize}
        \item \textit{प्रमाणविपर्ययविकल्पनिद्रास्मृतयः} (1.6)
        \item Five modifications of mind (\textit{Vrittis}):
        \begin{itemize}
            \item \textit{प्रमाण} (Pramana): Knowing correctly
            \item \textit{विपर्यय} (Viparyaya): Incorrect knowledge
            \item \textit{विकल्प} (Vikalpa): Fantasy or imagination
            \item \textit{निद्रा} (Nidra): Deep sleep
            \item \textit{स्मृति} (Smriti): Recollection of memory
        \end{itemize}
    \end{itemize}
\end{frame}

%%%%%%%%%%%%%%%%%%%%%%%%%%%%%%%%%%%%%%%%%%%%%%%%%%%%%%%%%%%%%%%%%%%%%%%%%%%%%%%%%%
\begin{frame}[fragile]\frametitle{Pramana and Viparyaya}
    \begin{itemize}
        \item \textbf{Pramana}: Sources of right knowledge
        \begin{itemize}
            \item \textit{प्रत्यक्ष} (Pratyaksa): Direct cognition
            \item \textit{अनुमान} (Anumana): Inference
            \item \textit{आगम} (Agama/Shabda): Testimony, revelation by Guru
        \end{itemize}
        \item \textbf{Viparyaya}: Misconception, incorrect knowledge
        \item \textit{विपर्ययो मिथ्याज्ञानमतद्रूपप्रतिष्ठम्} (1.8)
        \item False knowledge not based on its own form
    \end{itemize}
\end{frame}

%%%%%%%%%%%%%%%%%%%%%%%%%%%%%%%%%%%%%%%%%%%%%%%%%%%%%%%%%%%%%%%%%%%%%%%%%%%%%%%%%%
\begin{frame}[fragile]\frametitle{Vikalpa, Nidra, and Smriti}
    \begin{itemize}
        \item \textbf{Vikalpa}: Unfounded belief
        \item \textit{शब्दज्ञानानुपाती वस्तुशून्यो विकल्पः} (1.9)
        \item Knowledge through words but empty of an object is fantasy
        \item \textbf{Nidra}: State of deep sleep
        \item \textit{अभावप्रत्ययालम्बना वृत्तिर्निद्रा} (1.10)
        \item Vritti of absence of mental contents for support
        \item \textbf{Smriti}: Memory
        \item \textit{अनुभूतविषयासंप्रमोषः स्मृतिः} (1.11)
        \item Not letting experienced objects escape from the mind
    \end{itemize}
\end{frame}

%%%%%%%%%%%%%%%%%%%%%%%%%%%%%%%%%%%%%%%%%%%%%%%%%%%%%%%%%%%%%%%%%%%%%%%%%%%%%%%%%%
\begin{frame}[fragile]\frametitle{Vrittis and Chitta-Vritti Nirodhopaya}
    \begin{itemize}
        \item Vrittis: Mental responses to stimuli
        \item Ego identifies with thought waves
        \item Wrong identification with 'I' causes miseries
        \item Enlightenment: Control thought waves
        \item Abhyasa (\textit{अभ्यास}): Practice
        \item Vairagya (\textit{वैराग्य}): Non-attachment
        \item Practice:
        \begin{itemize}
            \item Disciplines, diet, pranayama (\textit{प्राणायाम}), asana (\textit{आसन}), meditation
        \end{itemize}
        \item Let go of attachments and aversions
        \item Practice long, uninterrupted, sincere, and firmly rooted
    \end{itemize}
\end{frame}

%%%%%%%%%%%%%%%%%%%%%%%%%%%%%%%%%%%%%%%%%%%%%%%%%%%%%%%%%%%%%%%%%%%%%%%%%%%%%%%%%%
\begin{frame}[fragile]\frametitle{Kleshas and Their Management}
    \begin{itemize}
        \item Kleshas: Causes of pain
        \item अविद्यास्मितारागद्वेषाभिनिवेशाः (\textit{2.3})
        \item 1. Avidya: Ignorance (अविद्या)
        \item 2. Asmita: Egoism (I-am-ness) (अस्मिता)
        \item 3. Raga: Attachment (Liking) (राग)
        \item 4. Dvesha: Aversion (Disliking) (द्वेष)
        \item 5. Abhinivesha: Fear of death (Clinging to life) (अभिनिवेश)
        \item अविद्या is the root of other Kleshas
        \item Degrees of manifestation:
        \begin{itemize}
            \item Prasupta: Dormant (प्रसुप्त)
            \item Tanu: Weak (तनु)
            \item Vichhina: Oscillating (विच्छिन्न)
            \item Udara: Abundant (उद्र)
        \end{itemize}
        \item Managing Kleshas:
        \begin{itemize}
            \item Kriya Yoga (Tapa, Swadhyaya, Ishwara Pranidhana) (तपः, स्वाध्याय, ईश्वरप्रणिधान)
            \item Dhyana (Meditation) (ध्यान)
            \item Pratiprasava (Involution) (प्रतिप्रसव)
        \end{itemize}
    \end{itemize}
\end{frame}

%%%%%%%%%%%%%%%%%%%%%%%%%%%%%%%%%%%%%%%%%%%%%%%%%%%%%%%%%%%%%%%%%%%%%%%%%%%%%%%%%%
\begin{frame}[fragile]\frametitle{Concept of Ishwara and Ishwara Pranidhana}
    \begin{itemize}
        \item ईश्वरप्रणिधानाद्वा (\textit{1.23})
        \item Devotion to Ishwara leads to Samadhi
        \item क्लेशकर्मविपाकाशयैरपरामृष्टः पुरुषविशेष ईश्वरः (\textit{1.24})
        \item Ishwara: Special soul, untouched by afflictions
        \item तत्र निरतिशयं सावर्ज्ञबीजम् (\textit{1.25})
        \item Ishwara: Seed of limitless omniscience
        \item स पूवेर्षामपि गुरुः कालेनानवच्छेदात् (\textit{1.26})
        \item Ishwara: Guru of all ancient gurus
    \end{itemize}
\end{frame}

%%%%%%%%%%%%%%%%%%%%%%%%%%%%%%%%%%%%%%%%%%%%%%%%%%%%%%%%%%%%%%%%%%%%%%%%%%%%%%%%%%
\begin{frame}[fragile]\frametitle{Concept of Ishwara and Ishwara Pranidhana (contd.)}
    \begin{itemize}
        \item तस्य वाचकः प्रणवः (\textit{1.27})
        \item AUM denotes Ishwara
        \item तज्जपस्तदर्थभावनम् (\textit{1.28})
        \item Recite AUM with understanding
        \item ततः प्रत्यक्चेतनािधगमोऽप्यन्तरायाभावश्च (\textit{1.29})
        \item Practice turns consciousness inward, removes obstacles
        \item Ishwara: Not a religious god, Yoga: Not a religion
        \item Ishwara Pranidhana: Complete surrender to Ishwara
        \item Optional technique in Kriya Yoga (तपः, स्वाध्याय, ईश्वरप्रणिधान)
        \item Key to overcoming ego, leading to Samadhi
    \end{itemize}
\end{frame}

%%%%%%%%%%%%%%%%%%%%%%%%%%%%%%%%%%%%%%%%%%%%%%%%%%%%%%%%%%%%%%%%%%%%%%%%%%%%%%%%%%
\begin{frame}[fragile]\frametitle{}
\begin{center}
{\Large 2.2 Introduction and study of Bhagwad Gita including memorisation of selected Shlokas (Chapter 2 - 47,48,49,50 and 70)}
\end{center}
\end{frame}

%%%%%%%%%%%%%%%%%%%%%%%%%%%%%%%%%%%%%%%%%%%%%%%%%%%%%%%%%%%
\begin{frame}[fragile]\frametitle{Introduction to Bhagavad Gita}

      \begin{itemize}
		\item \textit{Bhagavad Gita} - A 700-verse Hindu scripture part of the \textit{Mahabharata}.
		\item Dialog between Prince Arjuna and Lord Krishna on the battlefield of Kurukshetra.
		\item Focuses on duty (\textit{Dharma}), righteousness, and the path to spiritual wisdom.
		\item Addresses the moral and philosophical dilemmas faced by Arjuna.
		\item Revered as a key philosophical and spiritual text in Hinduism.
	  \end{itemize}

\end{frame}

%%%%%%%%%%%%%%%%%%%%%%%%%%%%%%%%%%%%%%%%%%%%%%%%%%%%%%%%%%%
\begin{frame}[fragile]\frametitle{Chapter 2: Selected Shlokas}

      \begin{itemize}
		\item \textbf{Verse 47} (\textit{Karma Yoga}) - 
		
			\textit{कर्मण्येवाधिकारस्ते मा फलेषु कदाचन |} \\
			\textit{मा कर्मफलहेतुर्भूर्मा ते सङ्गोऽस्त्वकर्मणि ||}
		
		\item \textbf{Verse 48} (\textit{Karma Yoga}) - 
		
			\textit{योगस्थः कुरु कर्माणि सङ्गं त्यक्त्वा धनञ्जय |} \\
			\textit{सिद्ध्यसिद्ध्योः समो भूत्वा समत्वं योग उच्यते ||}
		
		\item \textbf{Verse 49} (\textit{Karma Yoga}) - 
		
			\textit{यस्त्विन्द्रियाणि मनसा नियाम्यारभते नरः |} \\
			\textit{मुञ्जते तस्य योगिनोऽन्यः ||}
		
		\item \textbf{Verse 50} (\textit{Karma Yoga}) - 
		
			\textit{ब्रह्मण्याधाय कर्माणि सङ्गं त्यक्त्वा धनञ्जय |} \\
			\textit{सिद्ध्यसिद्ध्योः समो भूत्वा समत्वं योग उच्यते ||}
		
		\item \textbf{Verse 70} (\textit{Self-Realization}) - 
		
			\textit{अपण्यतं तु तद्वृत्तमन्तरायामुक्तं सदा} \\
			\textit{तन्मया न संशय ||}
		
	  \end{itemize}

\end{frame}

%%%%%%%%%%%%%%%%%%%%%%%%%%%%%%%%%%%%%%%%%%%%%%%%%%%%%%%%%%%
\begin{frame}[fragile]\frametitle{Study of Bhagavad Gita: Key Themes}

      \begin{itemize}
		\item \textit{Dharma} - The concept of duty and righteousness.
		\item \textit{Karma Yoga} - Path of selfless action and duty.
		\item \textit{Bhakti Yoga} - Path of devotion and love towards God.
		\item \textit{Jnana Yoga} - Path of knowledge and wisdom.
		\item \textit{Self-Realization} - Understanding the true nature of self and existence.
	  \end{itemize}

\end{frame}

%%%%%%%%%%%%%%%%%%%%%%%%%%%%%%%%%%%%%%%%%%%%%%%%%%%%%%%%%%%
\begin{frame}[fragile]\frametitle{Memorization of Selected Shlokas}

      \begin{itemize}
		\item \textbf{Verse 47} (\textit{Karma Yoga}) - \textit{कर्मण्येवाधिकारस्ते मा फलेषु कदाचन | मा कर्मफलहेतुर्भूर्मा ते सङ्गोऽस्त्वकर्मणि ||}
		\item \textbf{Verse 48} (\textit{Karma Yoga}) - \textit{योगस्थः कुरु कर्माणि सङ्गं त्यक्त्वा धनञ्जय | सिद्ध्यसिद्ध्योः समो भूत्वा समत्वं योग उच्यते ||}
		\item \textbf{Verse 49} (\textit{Karma Yoga}) - \textit{यस्त्विन्द्रियाणि मनसा नियाम्यारभते नरः | मुञ्जते तस्य योगिनोऽन्यः ||}
		\item \textbf{Verse 50} (\textit{Karma Yoga}) - \textit{ब्रह्मण्याधाय कर्माणि सङ्गं त्यक्त्वा धनञ्जय | सिद्ध्यसिद्ध्योः समो भूत्वा समत्वं योग उच्यते ||}
		\item \textbf{Verse 70} (\textit{Self-Realization}) - \textit{अपण्यतं तु तद्वृत्तमन्तरायामुक्तं सदा तन्मया न संशय ||}
	  \end{itemize}

\end{frame}

%%%%%%%%%%%%%%%%%%%%%%%%%%%%%%%%%%%%%%%%%%%%%%%%%%%%%%%%%%%%%%%%%%%%%%%%%%%%%%%%%%
\begin{frame}[fragile]\frametitle{}
\begin{center}
{\Large 2.3 Introduction and study of Hathapradipika}
\end{center}
\end{frame}

%%%%%%%%%%%%%%%%%%%%%%%%%%%%%%%%%%%%%%%%%%%%%%%%%%%%%%%%%%%
\begin{frame}[fragile]\frametitle{Introduction to Hatha Pradipika}

      \begin{itemize}
		\item \textit{Hatha Pradipika} - A classical text on Hatha Yoga.
		\item Written by Swami Svatmarama in the 15th century CE.
		\item Focuses on physical postures (\textit{Asanas}), breath control (\textit{Pranayama}), and meditation.
		\item Aims to prepare the body and mind for higher spiritual practices.
		\item Provides detailed instructions on various Hatha Yoga techniques.
	  \end{itemize}

\end{frame}

%%%%%%%%%%%%%%%%%%%%%%%%%%%%%%%%%%%%%%%%%%%%%%%%%%%%%%%%%%%
\begin{frame}[fragile]\frametitle{Key Concepts in Hatha Pradipika}

      \begin{itemize}
		\item \textit{Asanas} - Physical postures for physical stability and health.
		\item \textit{Pranayama} - Techniques for controlling the breath and vital energy.
		\item \textit{Mudras} - Hand gestures to control energy flow.
		\item \textit{Bandhas} - Body locks to channel energy within.
		\item \textit{Shatkarma} - Six purification techniques to cleanse the body.
	  \end{itemize}

\end{frame}

%%%%%%%%%%%%%%%%%%%%%%%%%%%%%%%%%%%%%%%%%%%%%%%%%%%%%%%%%%%
\begin{frame}[fragile]\frametitle{Study of Asanas in Hatha Pradipika}

      \begin{itemize}
		\item Describes various \textit{Asanas} for physical health and spiritual progress.
		\item Emphasizes proper alignment, stability, and breath control.
		\item Includes postures like \textit{Padmasana} (Lotus Pose), \textit{Shirshasana} (Headstand), and \textit{Sarvangasana} (Shoulder Stand).
		\item Focuses on achieving physical strength, flexibility, and concentration.
		\item Prepares the practitioner for deeper meditative practices.
	  \end{itemize}

\end{frame}

%%%%%%%%%%%%%%%%%%%%%%%%%%%%%%%%%%%%%%%%%%%%%%%%%%%%%%%%%%%
\begin{frame}[fragile]\frametitle{Study of Pranayama in Hatha Pradipika}

      \begin{itemize}
		\item Details various \textit{Pranayama} techniques for controlling breath and energy.
		\item Includes practices such as \textit{Kapalabhati} (Skull Shining Breath) and \textit{Nadi Shodhana} (Alternate Nostril Breathing).
		\item Aims to purify the body, calm the mind, and increase vital energy.
		\item Techniques are used to balance the prana (\textit{vital energy}) and support meditation.
		\item Essential for mastering advanced Hatha Yoga practices.
	  \end{itemize}

\end{frame}

%%%%%%%%%%%%%%%%%%%%%%%%%%%%%%%%%%%%%%%%%%%%%%%%%%%%%%%%%%%
\begin{frame}[fragile]\frametitle{Significance of Hatha Pradipika}

      \begin{itemize}
		\item Foundation of Hatha Yoga practices - Essential for practitioners seeking deeper understanding.
		\item Integrates physical and spiritual practices to enhance overall well-being.
		\item Offers practical guidance for practitioners of all levels.
		\item Highlights the importance of discipline, perseverance, and correct practice.
		\item Continues to influence modern Yoga practices and teachings.
	  \end{itemize}

\end{frame}


%%%%%%%%%%%%%%%%%%%%%%%%%%%%%%%%%%%%%%%%%%%%%%%%%%%%%%%%%%%%%%%%%%%%%%%%%%%%%%%%%%
\begin{frame}[fragile]\frametitle{}
\begin{center}
{\Large 2.4 General Introduction to Prasthanatrayee}
\end{center}
\end{frame}

%%%%%%%%%%%%%%%%%%%%%%%%%%%%%%%%%%%%%%%%%%%%%%%%%%%%%%%%%%%
\begin{frame}[fragile]\frametitle{Introduction to Prasthanatrayee}

      \begin{itemize}
		\item \textit{Prasthanatrayee} - The three foundational texts of Vedanta philosophy.
		\item Comprises:
		  \begin{itemize}
		      \item \textit{Upanishads} - Core philosophical texts exploring the nature of reality and self.
		      \item \textit{Bhagavad Gita} - A dialogue between Arjuna and Krishna on duty, righteousness, and spirituality.
		      \item \textit{Brahma Sutras} - Philosophical aphorisms systematizing Vedantic thought.
		  \end{itemize}
		\item Together, they form the basis of Vedantic study and practice.
		\item Essential for understanding key concepts in Hindu philosophy.
	  \end{itemize}

\end{frame}

%%%%%%%%%%%%%%%%%%%%%%%%%%%%%%%%%%%%%%%%%%%%%%%%%%%%%%%%%%%
\begin{frame}[fragile]\frametitle{The Upanishads}

      \begin{itemize}
		\item Ancient texts that form the core of Vedic wisdom.
		\item Focus on spiritual knowledge and philosophical inquiry.
		\item Discuss the nature of ultimate reality (\textit{Brahman}) and the individual soul (\textit{Atman}).
		\item Key Upanishads include \textit{Isha}, \textit{Kena}, \textit{Katha}, and \textit{Mandukya}.
		\item Emphasize meditation, self-realization, and the unity of all existence.
	  \end{itemize}

\end{frame}

%%%%%%%%%%%%%%%%%%%%%%%%%%%%%%%%%%%%%%%%%%%%%%%%%%%%%%%%%%%
\begin{frame}[fragile]\frametitle{The Bhagavad Gita}

      \begin{itemize}
		\item A 700-verse Hindu scripture part of the \textit{Mahabharata}.
		\item Dialogue between Prince Arjuna and Lord Krishna.
		\item Addresses the nature of duty (\textit{Dharma}), action, and devotion.
		\item Explores paths of Karma Yoga (action), Bhakti Yoga (devotion), and Jnana Yoga (knowledge).
		\item Provides guidance on ethical and spiritual living.
	  \end{itemize}

\end{frame}

%%%%%%%%%%%%%%%%%%%%%%%%%%%%%%%%%%%%%%%%%%%%%%%%%%%%%%%%%%%
\begin{frame}[fragile]\frametitle{The Brahma Sutras}

      \begin{itemize}
		\item Philosophical texts attributed to Sage Vyasa.
		\item Comprises 555 sutras (aphorisms) summarizing the teachings of the Upanishads.
		\item Systematizes Vedantic thought and addresses key metaphysical questions.
		\item Divided into four chapters: \textit{Sutras on the Nature of Brahman}, \textit{Sutras on the Universe}, \textit{Sutras on the Path of Knowledge}, and \textit{Sutras on the Liberation}.
		\item Focuses on the unity of Brahman and the self, and the nature of liberation.
	  \end{itemize}

\end{frame}

%%%%%%%%%%%%%%%%%%%%%%%%%%%%%%%%%%%%%%%%%%%%%%%%%%%%%%%%%%%
\begin{frame}[fragile]\frametitle{Significance of Prasthanatrayee}

      \begin{itemize}
		\item Provides comprehensive understanding of Vedantic philosophy.
		\item Forms the basis for various schools of Vedanta and spiritual practices.
		\item Guides ethical, spiritual, and philosophical aspects of life.
		\item Essential for deep study of Hindu philosophy and theology.
		\item Continues to influence spiritual thought and practice today.
	  \end{itemize}

\end{frame}


%%%%%%%%%%%%%%%%%%%%%%%%%%%%%%%%%%%%%%%%%%%%%%%%%%%%%%%%%%%%%%%%%%%%%%%%%%%%%%%%%%
\begin{frame}[fragile]\frametitle{}
\begin{center}
{\Large 2.5  Concepts  and  principles  of  Aahara  (Diet)  in  Hathapradipika  and  Bhagawadgita  (Mitahara and Yuktahara)}
\end{center}
\end{frame}

%%%%%%%%%%%%%%%%%%%%%%%%%%%%%%%%%%%%%%%%%%%%%%%%%%%%%%%%%%%
\begin{frame}[fragile]\frametitle{Concepts of Aahara in Hatha Pradipika}

      \begin{itemize}
		\item \textit{Aahara} - Diet and its role in Yoga practice.
		\item Emphasizes moderation and the impact of diet on physical and mental health.
		\item Advocates for simple, pure, and balanced food.
		\item Recommends avoidance of heavy, spicy, or overly processed foods.
		\item Stresses the importance of regular and timely meals.
		\item Highlights the role of diet in supporting physical strength and stamina for Yoga.
	  \end{itemize}

\end{frame}

%%%%%%%%%%%%%%%%%%%%%%%%%%%%%%%%%%%%%%%%%%%%%%%%%%%%%%%%%%%
\begin{frame}[fragile]\frametitle{Concepts of Aahara in Bhagavad Gita}

      \begin{itemize}
		\item \textit{Mitahara} - Moderate eating; balanced and moderate in quantity.
		\item Recommends a diet that is:
		  \begin{itemize}
		      \item \textit{Sattvic} - Pure, clean, and nourishing.
		      \item \textit{Rajasic} - Overly stimulating, often leading to restlessness.
		      \item \textit{Tamasic} - Stale, impure, and harmful.
		  \end{itemize}
		\item Emphasizes the impact of food on the mind and consciousness.
		\item Advocates for moderation and awareness in eating habits.
		\item Suggests that the right diet supports spiritual and physical well-being.
	  \end{itemize}

\end{frame}

%%%%%%%%%%%%%%%%%%%%%%%%%%%%%%%%%%%%%%%%%%%%%%%%%%%%%%%%%%%
\begin{frame}[fragile]\frametitle{Principles of Mitahara in Bhagavad Gita}

      \begin{itemize}
		\item \textit{Mitahara} - Eating in moderation and balance.
		\item Consumes food that is:
		  \begin{itemize}
		      \item Fresh and wholesome.
		      \item Prepared with love and devotion.
		      \item Conducive to physical health and mental clarity.
		  \end{itemize}
		\item Avoids excessive or insufficient eating.
		\item Focuses on maintaining harmony between body and mind through diet.
		\item Supports overall spiritual and physical health.
	  \end{itemize}

\end{frame}

%%%%%%%%%%%%%%%%%%%%%%%%%%%%%%%%%%%%%%%%%%%%%%%%%%%%%%%%%%%
\begin{frame}[fragile]\frametitle{Principles of Yuktahara in Bhagavad Gita}

      \begin{itemize}
		\item \textit{Yuktahara} - Proper and disciplined eating.
		\item Involves:
		  \begin{itemize}
		      \item Consuming food at appropriate times.
		      \item Eating in moderation, neither too much nor too little.
		      \item Aligning diet with one's physical and spiritual needs.
		  \end{itemize}
		\item Emphasizes mindful eating and awareness of the impact of food.
		\item Integrates diet with other aspects of Yoga practice for holistic well-being.
		\item Enhances both physical health and spiritual development.
	  \end{itemize}

\end{frame}



%%%%%%%%%%%%%%%%%%%%%%%%%%%%%%%%%%%%%%%%%%%%%%%%%%%%%%%%%%%%%%%%%%%%%%%%%%%%%%%%%%
\begin{frame}[fragile]\frametitle{}
\begin{center}
{\Large 2.6  Significance of Hatha Yoga practices in health and well being}
\end{center}
\end{frame}

%%%%%%%%%%%%%%%%%%%%%%%%%%%%%%%%%%%%%%%%%%%%%%%%%%%%%%%%%%%
\begin{frame}[fragile]\frametitle{Significance of Hatha Yoga Practices}

      \begin{itemize}
		\item Hatha Yoga - A system focusing on physical postures, breath control, and meditation.
		\item Aims to balance physical, mental, and spiritual health.
		\item Includes practices like \textit{Asanas} (postures), \textit{Pranayama} (breath control), \textit{Mudras} (hand gestures), and \textit{Bandhas} (body locks).
		\item Enhances physical strength, flexibility, and endurance.
		\item Supports mental clarity, stress reduction, and emotional stability.
	  \end{itemize}

\end{frame}

%%%%%%%%%%%%%%%%%%%%%%%%%%%%%%%%%%%%%%%%%%%%%%%%%%%%%%%%%%%
\begin{frame}[fragile]\frametitle{Physical Health Benefits}

      \begin{itemize}
		\item Improves flexibility and muscle tone.
		\item Enhances strength and stamina.
		\item Aids in weight management and improves posture.
		\item Boosts circulation and respiratory function.
		\item Promotes detoxification through sweating and improved digestion.
	  \end{itemize}

\end{frame}

%%%%%%%%%%%%%%%%%%%%%%%%%%%%%%%%%%%%%%%%%%%%%%%%%%%%%%%%%%%
\begin{frame}[fragile]\frametitle{Mental and Emotional Well-being}

      \begin{itemize}
		\item Reduces stress and anxiety through relaxation techniques.
		\item Enhances focus and concentration.
		\item Improves mood and emotional resilience.
		\item Supports mental clarity and cognitive function.
		\item Encourages mindfulness and self-awareness.
	  \end{itemize}

\end{frame}

%%%%%%%%%%%%%%%%%%%%%%%%%%%%%%%%%%%%%%%%%%%%%%%%%%%%%%%%%%%
\begin{frame}[fragile]\frametitle{Spiritual Growth}

      \begin{itemize}
		\item Facilitates deeper meditation and self-realization.
		\item Helps in achieving inner peace and balance.
		\item Promotes a sense of connection to self and higher consciousness.
		\item Supports spiritual development through disciplined practice.
		\item Integrates physical health with spiritual practice for holistic growth.
	  \end{itemize}

\end{frame}

%%%%%%%%%%%%%%%%%%%%%%%%%%%%%%%%%%%%%%%%%%%%%%%%%%%%%%%%%%%
\begin{frame}[fragile]\frametitle{Overall Well-being}

      \begin{itemize}
		\item Combines physical, mental, and spiritual practices for comprehensive health.
		\item Encourages a balanced lifestyle and regular practice.
		\item Provides tools for managing daily stress and enhancing quality of life.
		\item Fosters a harmonious relationship between body, mind, and spirit.
		\item Contributes to long-term health and vitality.
	  \end{itemize}

\end{frame}

%%%%%%%%%%%%%%%%%%%%%%%%%%%%%%%%%%%%%%%%%%%%%%%%%%%%%%%%%%%


%%%%%%%%%%%%%%%%%%%%%%%%%%%%%%%%%%%%%%%%%%%%%%%%%%%%%%%%%%%%%%%%%%%%%%%%%%%%%%%%%%
\begin{frame}[fragile]\frametitle{}
\begin{center}
{\Large 2.7  Concept of mental wellbeing according to Patanjala Yoga}
\end{center}
\end{frame}

%%%%%%%%%%%%%%%%%%%%%%%%%%%%%%%%%%%%%%%%%%%%%%%%%%%%%%%%%%%
\begin{frame}[fragile]\frametitle{Concept of Mental Well-being in Patanjali Yoga}

      \begin{itemize}
		\item Patanjali's Yoga - Focuses on achieving mental clarity and stability.
		\item Central text: \textit{Yoga Sutras of Patanjali}.
		\item Emphasizes the importance of controlling the mind (\textit{Chitta}) for well-being.
		\item Aims to cultivate a state of \textit{Sattva} (mental purity and balance).
		\item Addresses mental disturbances and provides techniques to overcome them.
	  \end{itemize}

\end{frame}

%%%%%%%%%%%%%%%%%%%%%%%%%%%%%%%%%%%%%%%%%%%%%%%%%%%%%%%%%%%
\begin{frame}[fragile]\frametitle{Role of Mind Control (Chitta Vritti Nirodha)}

      \begin{itemize}
		\item \textit{Chitta Vritti Nirodha} - Control of the fluctuations of the mind.
		\item Essential for achieving mental stability and peace.
		\item Involves restraining mental patterns and disturbances.
		\item Focuses on reducing \textit{Vrittis} (mental modifications) that cause suffering.
		\item Achieved through practice of \textit{Yamas} (ethical restraints) and \textit{Niyamas} (personal observances).
	  \end{itemize}

\end{frame}

%%%%%%%%%%%%%%%%%%%%%%%%%%%%%%%%%%%%%%%%%%%%%%%%%%%%%%%%%%%
\begin{frame}[fragile]\frametitle{Key Practices for Mental Well-being}

      \begin{itemize}
		\item \textit{Dhyana} (Meditation) - Regular practice to cultivate concentration and inner peace.
		\item \textit{Pranayama} (Breath Control) - Regulates mental and emotional states through breath.
		\item \textit{Asanas} (Postures) - Physical practice to stabilize the mind and body.
		\item \textit{Self-Discipline} - Adherence to \textit{Yamas} and \textit{Niyamas} for mental clarity.
		\item \textit{Mindfulness} - Awareness of thoughts and emotions to maintain balance.
	  \end{itemize}

\end{frame}

%%%%%%%%%%%%%%%%%%%%%%%%%%%%%%%%%%%%%%%%%%%%%%%%%%%%%%%%%%%
\begin{frame}[fragile]\frametitle{Achieving Mental Clarity (Sattva)}

      \begin{itemize}
		\item \textit{Sattva} - The quality of purity and harmony in the mind.
		\item Promotes inner peace, clarity, and wisdom.
		\item Cultivated through regular practice of Yoga and meditation.
		\item Helps in overcoming mental disturbances and achieving higher states of consciousness.
		\item Supports overall mental and emotional stability.
	  \end{itemize}

\end{frame}

%%%%%%%%%%%%%%%%%%%%%%%%%%%%%%%%%%%%%%%%%%%%%%%%%%%%%%%%%%%
\begin{frame}[fragile]\frametitle{Overcoming Mental Disturbances}

      \begin{itemize}
		\item \textit{Kleshas} - Mental afflictions such as ignorance, egoism, attachment, aversion, and fear of death.
		\item Addressed through disciplined practice and self-awareness.
		\item Use of \textit{Vairagya} (detachment) and \textit{Abhyasa} (practice) to manage mental challenges.
		\item Achieving \textit{Samadhi} - A state of perfect mental equilibrium.
		\item Focus on reducing negative thought patterns and promoting mental resilience.
	  \end{itemize}

\end{frame}

%%%%%%%%%%%%%%%%%%%%%%%%%%%%%%%%%%%%%%%%%%%%%%%%%%%%%%%%%%%%%%%%%%%%%%%%%%%%%%%%%%
\begin{frame}[fragile]\frametitle{}
\begin{center}
{\Large 2.8  Yogic practices of Patanjala Yoga: Bahiranga and Antaranga Yoga}
\end{center}
\end{frame}

%%%%%%%%%%%%%%%%%%%%%%%%%%%%%%%%%%%%%%%%%%%%%%%%%%%%%%%%%%%
\begin{frame}[fragile]\frametitle{Yogic Practices in Patanjali Yoga}

      \begin{itemize}
		\item Patanjali’s Yoga - Divided into \textbf{Bahiranga} (external) and \textbf{Antaranga} (internal) practices.
		\item Aims to achieve holistic development and spiritual realization.
		\item \textbf{Bahiranga} Yoga focuses on external practices.
		\item \textbf{Antaranga} Yoga emphasizes internal, meditative practices.
		\item Both are essential for achieving the ultimate goal of Yoga.
	  \end{itemize}

\end{frame}

%%%%%%%%%%%%%%%%%%%%%%%%%%%%%%%%%%%%%%%%%%%%%%%%%%%%%%%%%%%
\begin{frame}[fragile]\frametitle{Bahiranga Yoga}

      \begin{itemize}
		\item \textbf{Bahiranga} - External practices of Yoga.
		\item Includes:
		  \begin{itemize}
		      \item \textit{Yamas} (यम) - Ethical restraints (non-violence, truthfulness, non-stealing, etc.).
		      \item \textit{Niyamas} (नियम) - Personal observances (cleanliness, contentment, self-discipline, etc.).
		      \item \textit{Asanas} (आसन) - Physical postures to prepare the body for meditation.
		      \item \textit{Pranayama} (प्राणायाम) - Breath control to regulate vital energy and calm the mind.
		  \end{itemize}
		\item Focuses on ethical and physical preparation for deeper practices.
		\item Establishes a foundation for internal practices.
	  \end{itemize}

\end{frame}

%%%%%%%%%%%%%%%%%%%%%%%%%%%%%%%%%%%%%%%%%%%%%%%%%%%%%%%%%%%
\begin{frame}[fragile]\frametitle{Antaranga Yoga}

      \begin{itemize}
		\item \textbf{Antaranga} - Internal practices of Yoga.
		\item Includes:
		  \begin{itemize}
		      \item \textit{Pratyahara} (प्रत्याहार) - Withdrawal of the senses from external objects.
		      \item \textit{Dharana} (धारणा) - Concentration on a single point or object.
		      \item \textit{Dhyana} (ध्यान) - Meditation; sustained and uninterrupted flow of consciousness.
		      \item \textit{Samadhi} (समाधि) - Enlightenment; a state of profound inner peace and realization.
		  \end{itemize}
		\item Focuses on deepening the inner experience and achieving spiritual insight.
		\item Cultivates mental clarity, inner peace, and ultimate realization of the self.
	  \end{itemize}

\end{frame}

%%%%%%%%%%%%%%%%%%%%%%%%%%%%%%%%%%%%%%%%%%%%%%%%%%%%%%%%%%%
\begin{frame}[fragile]\frametitle{Integration of Bahiranga and Antaranga Yoga}

      \begin{itemize}
		\item \textbf{Bahiranga} and \textbf{Antaranga} practices are interdependent.
		\item External practices prepare and purify the body and mind for internal practices.
		\item Internal practices build on the discipline established by external practices.
		\item Both are necessary for comprehensive development and achieving Yoga’s ultimate goals.
		\item Harmonizing both aspects leads to a balanced and integrated approach to Yoga.
	  \end{itemize}

\end{frame}


%%%%%%%%%%%%%%%%%%%%%%%%%%%%%%%%%%%%%%%%%%%%%%%%%%%%%%%%%%%%%%%%%%%%%%%%%%%%%%%%%%
\begin{frame}[fragile]\frametitle{}
\begin{center}
{\Large 2.9  Concepts of healthy living in Bhagwad Gita}
\end{center}
\end{frame}

%%%%%%%%%%%%%%%%%%%%%%%%%%%%%%%%%%%%%%%%%%%%%%%%%%%%%%%%%%%
\begin{frame}[fragile]\frametitle{Concepts of Healthy Living in Bhagavad Gita}

      \begin{itemize}
		\item \textbf{Bhagavad Gita} - Provides guidance on living a balanced and healthy life.
		\item Emphasizes the harmony of body, mind, and spirit.
		\item Encourages living in accordance with \textbf{धर्म} (Dharma) and righteousness.
		\item Focuses on maintaining balance in daily activities and lifestyle.
		\item Highlights the importance of self-discipline and moderation.
	  \end{itemize}

\end{frame}

%%%%%%%%%%%%%%%%%%%%%%%%%%%%%%%%%%%%%%%%%%%%%%%%%%%%%%%%%%%
\begin{frame}[fragile]\frametitle{Principles of Healthy Living}

      \begin{itemize}
		\item \textbf{Moderation in Eating} - \textit{मिताहार} (Mitahara): Eat in moderation; balanced diet and mindful eating.
		\item \textbf{Balanced Lifestyle} - Maintain a balanced routine; avoid extremes in work and rest.
		\item \textbf{Regular Practice} - Engage in daily practice of Yoga, meditation, or self-discipline.
		\item \textbf{Right Action} - Perform actions in line with one’s duty and ethical principles.
		\item \textbf{Mental Peace} - Cultivate a peaceful and steady mind through mindfulness and self-awareness.
	  \end{itemize}

\end{frame}

%%%%%%%%%%%%%%%%%%%%%%%%%%%%%%%%%%%%%%%%%%%%%%%%%%%%%%%%%%%
\begin{frame}[fragile]\frametitle{Concepts of Diet and Behavior}

      \begin{itemize}
		\item \textbf{Healthy Diet} - Food that is:
		  \begin{itemize}
		      \item Fresh and wholesome.
		      \item Prepared with care and respect.
		      \item Not overly spicy or heavy.
		  \end{itemize}
		\item \textbf{Behavior and Attitude} - Approach life with:
		  \begin{itemize}
		      \item \textit{संतोष} (Santosha) - Contentment.
		      \item \textit{वैराग्य} (Vairagya) - Non-attachment.
		      \item Equanimity in success and failure.
		  \end{itemize}
		\item \textbf{Self-Control} - Exercise self-control over desires and impulses.
		\item \textbf{Spiritual Focus} - Align daily actions with spiritual growth and self-realization.
	  \end{itemize}

\end{frame}

%%%%%%%%%%%%%%%%%%%%%%%%%%%%%%%%%%%%%%%%%%%%%%%%%%%%%%%%%%%
\begin{frame}[fragile]\frametitle{Living in Harmony with Nature}

      \begin{itemize}
		\item \textbf{Natural Rhythm} - Live in harmony with natural cycles and rhythms.
		\item \textbf{Sattvic Living} - Adopt a lifestyle that promotes purity and tranquility (\textit{सात्त्विक}).
		\item \textbf{Avoid Excesses} - Avoid excessive indulgence and self-denial.
		\item \textbf{Holistic Approach} - Integrate physical health, mental peace, and spiritual well-being.
		\item \textbf{Mindful Living} - Practice mindfulness in all aspects of life.
	  \end{itemize}

\end{frame}


%%%%%%%%%%%%%%%%%%%%%%%%%%%%%%%%%%%%%%%%%%%%%%%%%%%%%%%%%%%%%%%%%%%%%%%%%%%%%%%%%%
\begin{frame}[fragile]\frametitle{}
\begin{center}
{\Large 2.10  Importance of subjective experience in daily Yoga practice}
\end{center}
\end{frame}

%%%%%%%%%%%%%%%%%%%%%%%%%%%%%%%%%%%%%%%%%%%%%%%%%%%%%%%%%%%
\begin{frame}[fragile]\frametitle{Importance of Subjective Experience in Daily Yoga Practice}

      \begin{itemize}
		\item \textbf{Subjective Experience} - Personal, internal perception of Yoga practice.
		\item Focuses on individual feelings, sensations, and inner states.
		\item Essential for understanding the impact of practice on body and mind.
		\item Promotes self-awareness and deeper connection to one’s practice.
		\item Enhances the effectiveness and benefits of Yoga practice.
	  \end{itemize}

\end{frame}

%%%%%%%%%%%%%%%%%%%%%%%%%%%%%%%%%%%%%%%%%%%%%%%%%%%%%%%%%%%
\begin{frame}[fragile]\frametitle{Self-Awareness and Mindfulness}

      \begin{itemize}
		\item \textbf{Self-Awareness} - Being conscious of physical and mental states during practice.
		\item Encourages observation of subtle changes and progress.
		\item \textbf{Mindfulness} - Paying attention to the present moment.
		\item Enhances focus and concentration in practice.
		\item Supports mental clarity and emotional stability.
	  \end{itemize}

\end{frame}

%%%%%%%%%%%%%%%%%%%%%%%%%%%%%%%%%%%%%%%%%%%%%%%%%%%%%%%%%%%
\begin{frame}[fragile]\frametitle{Personalization of Practice}

      \begin{itemize}
		\item \textbf{Personalization} - Adapting Yoga practices to individual needs and conditions.
		\item Allows for modifications based on subjective experience and feedback.
		\item Facilitates alignment with personal goals and limitations.
		\item Enhances comfort and effectiveness of the practice.
		\item Supports long-term adherence and progress in Yoga practice.
	  \end{itemize}

\end{frame}

%%%%%%%%%%%%%%%%%%%%%%%%%%%%%%%%%%%%%%%%%%%%%%%%%%%%%%%%%%%
\begin{frame}[fragile]\frametitle{Integration of Mind and Body}

      \begin{itemize}
		\item \textbf{Mind-Body Connection} - Awareness of the interplay between mental and physical aspects.
		\item Helps in achieving balance and harmony.
		\item Enhances the holistic benefits of Yoga.
		\item Supports deeper meditative and reflective states.
		\item Facilitates a more profound understanding of oneself.
	  \end{itemize}

\end{frame}

%%%%%%%%%%%%%%%%%%%%%%%%%%%%%%%%%%%%%%%%%%%%%%%%%%%%%%%%%%%
\begin{frame}[fragile]\frametitle{Reflective Practice and Growth}

      \begin{itemize}
		\item \textbf{Reflective Practice} - Regularly reviewing personal experiences and progress.
		\item Encourages continuous learning and improvement.
		\item Aids in identifying areas for growth and development.
		\item Supports emotional and spiritual evolution.
		\item Fosters a deeper commitment to Yoga practice.
	  \end{itemize}

\end{frame}

