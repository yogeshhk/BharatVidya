
%%%%%%%%%%%%%%%%%%%%%%%%%%%%%%%%%%%%%%%%%%%%%%%%%%%%%%%%%%%%%%%%%%%%%%%%%%%%%%%%%%
\begin{frame}[fragile]\frametitle{}
\begin{center}
{\Large Introduction to Yoga Texts }
\end{center}
\end{frame}

%%%%%%%%%%%%%%%%%%%%%%%%%%%%%%%%%%%%%%%%%%%%%%%%%%%%%%%%%%%
\begin{frame}[fragile]\frametitle{Syllabus}

\begin{itemize}
\item 2.1  Introduction  and  study  of  Patanjala  Yoga  Sutra  including  memorization  of  selected Sutras (Chapter I- 1-12). 
\item 2.2  Introduction  and  study  of  Bhagavad  Gita  including  memorization  of  selected  Slokas (Chapter II -47, 48, 49, 50 and 70). 
\item 2.3  Introduction and study of Hathpradipika.  
\item 2.4  General Introduction to Prasthanatrayee. 
\item 2.5  Concepts  and  principles  of  Aahara  (Diet)  in  Hathapradipika  and  Bhagawadgita  (Mitahara and Yuktahara). 
\item 2.6  Significance of Hatha Yoga practices in health and well being. 
\item 2.7  Concept of mental wellbeing according to Patanjala Yoga. 
\item 2.8  Yogic practices of Patanjala Yoga: Bahiranga and Antaranga Yoga. 
\item 2.9  Concepts of healthy living in Bhagwad Gita. 
\item 2.10  Importance of subjective experience in daily Yoga practice. 
\end{itemize}
	  
\end{frame}

% %%%%%%%%%%%%%%%%%%%%%%%%%%%%%%%%%%%%%%%%%%%%%%%%%%%%%%%%%%%
% \begin{frame}[fragile]\frametitle{Introduction to Yoga Texts}

% \begin{itemize}
% \item उपनिषद्: वेदांचा भाग. लगभग 108 है, किन्तु मुख्य उपनिषद 13 हैं। उपनिषदों में मुख्य रूप से 'आत्मविद्या' का प्रतिपादन है, जिसके अन्तर्गत ब्रह्म और आत्मा के स्वरूप, उसकी प्राप्ति के साधन और आवश्यकता की समीक्षा की गयी है. कुल -- १०८ उपनिषद्. प्राचीनतम  १. ईश, २. ऐतरेय, ३. छान्दोग्य, ४. प्रश्न, ५. तैत्तिरीय, ६. बृहदारण्यक, ७. माण्डूक्य और ८. मुण्डक
% प्राचीन १. कठ, २. केन
% अवान्तरकालीन १. कौषीतकि, २. मैत्री (मैत्राणयी) तथा ३. श्वेताश्वतर
% \item पतंजली योगसूत्र :
% समाधी पाद : समाधीचे विविध प्रकार, थेट फलाचे विवरण ,
% साधना पाद : क्लेश आणि तो कमी करण्याचे मार्ग ,
% विभूती पाद : ध्यान, विशेष सिद्धी ,
% कैवल्य पाद : मोक्ष अनुभव ,

% \item प्रस्थान त्रयी :
% उपनिषद .
% भगवद्गीता : १-६ कर्मयोग, ७-१२ भक्ती योग, १३-१८ ज्ञान योग .
% ब्रह्मसूत्र 

% \end{itemize}
	  
% \end{frame}

% %%%%%%%%%%%%%%%%%%%%%%%%%%%%%%%%%%%%%%%%%%%%%%%%%%%%%%%%%%%
% \begin{frame}[fragile]\frametitle{\textit{वेद} (Vedas) and the \textit{उपनिषद} (Upanishads)}
      % \begin{itemize}
        % \item \textbf{\textit{वेद} (Vedas) Sections}: 
          
          % The \textit{वेद} (Vedas) are divided into two sections:
          % \begin{itemize}
            % \item \textbf{\textit{कर्मकाण्ड} (Karma Kanda)} (Ritual portion)
            % \item \textbf{\textit{ज्ञानकाण्ड} (Jnana Kanda)} (Knowledge portion)
          % \end{itemize}
          
        % \item \textbf{\textit{उपनिषद} (Upanishads)}: 
          
          % The \textit{उपनिषद्स} (Upanishads) are contained in the knowledge portion of the Vedas. They describe the inner vision of reality through self-inquiry and expound upon three subjects:
          % \begin{itemize}
            % \item \textbf{\textit{जीव} (Jiva)} - (Embodied soul)
            % \item \textbf{\textit{जगत} (Jagat)} - (The World)
            % \item \textbf{\textit{ईश्वर} (Ishwara)} - (God or the creator of the universe)
          % \end{itemize}
          % The climax of the enquiry is the experience of the essential identity of \textit{आत्मन्} (Atman) within with \textit{ब्रह्मन्} (Brahman).
          
        % \item \textbf{\textit{वैदिक योग} (Vedic Yoga)}: 
          
          % The Vedas contain the oldest known yogic teachings called \textit{वैदिक योग} (Vedic Yoga). Vedic Yogis (ऋषि) taught how to live in divine harmony and see the ultimate reality through intensive spiritual practice.
          
      % \end{itemize}
% \end{frame}



%%%%%%%%%%%%%%%%%%%%%%%%%%%%%%%%%%%%%%%%%%%%%%%%%%%%%%%%%%%%%%%%%%%%%%%%%%%%%%%%%%
\begin{frame}[fragile]\frametitle{}
\begin{center}
{\Large 2.1 Introduction and study of \textit{Patanjala Yoga Sutra} (पातंजल योग सूत्र) including memorization of selected Sutras (Chapter I- 1-12)}
\end{center}
\end{frame}

% %%%%%%%%%%%%%%%%%%%%%%%%%%%%%%%%%%%%%%%%%%%%%%%%%%%%%%%%%%%%%%%%%%%%%%%%%%%%%%%%%%
% \begin{frame}[fragile]\frametitle{\textit{Muni Patanjali} (मुनी पतंजलि)}

    % \begin{itemize}
        % \item Birth shrouded in mystery; various legends exist
		% \item One lore: पतंजली : पत (fallen into) + अंजली (two hands): from sky a baby snake was fallen in to yogi woman, which later became a baby body, that's Patanjali.
        % \item Another lore: Born to \textit{Atri} (अत्रि) and \textit{Anusuya} (अनसूया); incarnation of \textit{Adishesha} (आदीशेष) (cosmic serpent)
        % % \item Another tradition: Fell from heaven as a small snake into the palms (\textit{anjali} - अञ्जलि) of his virgin mother
        % \item Believed to have lived between 400 BCE and 200 CE, exact dates uncertain
        % \item 2nd to 5th century BC
        % \item Incarnation of \textit{Adishesha} (आदीशेष)
        % \item Also known as: \textit{Naganath} (नागनाथ), \textit{Gonika Putra} (गोनिका पुत्र), \textit{Phani Bhrt} (फणि भर्त)
        % \item He was a physician, grammarian, and who systematically compiled \textit{Yoga Sutra} (योग सूत्र).
        % \item \textit{Raja Bhoj} (राजा भोज) got the sutras written
    % \end{itemize}

% \end{frame}

%%%%%%%%%%%%%%%%%%%%%%%%%%%%%%%%%%%%%%%%%%%%%%%%%%%%%%%%%%%%%%%%%%%%%%%%%%%%%%%%%%
\begin{frame}[fragile]\frametitle{\textit{Muni Patanjali} (मुनी पतंजलि)}

    \begin{itemize}
        \item Birth shrouded in mystery, several legends.
        \item One legend: \textit{Pat} (fallen) + \textit{Anjali} (hands) - baby snake fell from sky into yogi woman’s hands, turned into baby Patanjali.
        \item Another: Born to \textit{Atri} (अत्रि) and \textit{Anusuya} (अनसूया), incarnation of \textit{Adishesha} (आदीशेष).
        \item Lived between 400 BCE – 200 CE; some suggest 2nd to 5th century BCE.
        \item Also known as \textit{Naganath} (नागनाथ), \textit{Gonika Putra} (गोनिका पुत्र), \textit{Phani Bhrt} (फणि भर्त).
        \item Physician, grammarian, compiled \textit{Yoga Sutra} (योग सूत्र).
        \item \textit{Raja Bhoj} (राजा भोज) got the sutras written.
    \end{itemize}

\end{frame}


% %%%%%%%%%%%%%%%%%%%%%%%%%%%%%%%%%%%%%%%%%%%%%%%%%%%%%%%%%%%
% \begin{frame}[fragile]\frametitle{Patanjali Yoga (Raja Yoga)}

      % \begin{itemize}
		% \item Focuses on the eight limbs of Yoga (\textit{Ashtanga Yoga}).
		% \item Aims for mental discipline and spiritual insight.
		% \item Key practices: Ethical guidelines, physical postures, breath control, and meditation.
		% \item Major text: \textit{Yoga Sutras of Patanjali}.
		% \item Emphasizes systematic approach to achieving higher states of consciousness.
	  % \end{itemize}

% \end{frame}

%%%%%%%%%%%%%%%%%%%%%%%%%%%%%%%%%%%%%%%%%%%%%%%%%%%%%%%%%%%
\begin{frame}[fragile]\frametitle{Patanjali's Yoga Sutra Overview}
    \begin{itemize}
        \item Divided into four chapters:
        \begin{itemize}
            \item \textbf{Samadhi Pada} (समाधिपाद): Describes the goal of life and nature of \textbf{Samadhi} (समाधि).
            \item \textbf{Sadhana Pada} (साधनपाद): Explains practices to achieve the goal, including the Eight Limbs of Yoga.
            \item \textbf{Vibhuti Pada} (विभूतिपाद): Discusses supernatural powers (\textbf{Siddhis} सिद्धि) gained through practice.
            \item \textbf{Kaivalya Pada} (कैवल्यपाद): Centers on liberation or \textbf{Moksha} (मोक्ष), the ultimate freedom.
        \end{itemize}
    \end{itemize}
\end{frame}

%%%%%%%%%%%%%%%%%%%%%%%%%%%%%%%%%%%%%%%%%%%%%%%%%%%%%%%%%%%%%%%%%%%%%%%%%%%%%%%%%%
\begin{frame}[fragile]\frametitle{\textit{Yoga Sutras}: \textit{Samadhi Pada} (समाधि पाद)}

    \begin{itemize}
        \item First chapter of the \textit{Yoga Sutras} (योग सूत्र), 51 sutras
        \item Focuses on the nature and aim of Yoga
        \item Introduces key concepts like \textit{Citta} (चित्त) (mind-stuff) and 5 \textit{Vrittis} (वृत्तियाँ) (mental modifications)
        \item Defines Yoga as "\textit{Yogas citta vrtti nirodhah}" (योगस चित्त वृत्ति निरोध) (cessation of mind fluctuations)
        \item \textbf{Abhyasa} (अभ्यास) (Practice)
        \item \textbf{Vairagya} (वैराग्य) (Detachment)
        \item \textbf{Antarayas} (अन्तराय) (obstacles on the path of Yoga)
        \item \textbf{Sahabhavas} (सहभव) (accomplishments)
        \item \textbf{Chitta Prasadana} (चित्त प्रसादना)
        \item Describes various forms of \textit{Samadhi} (समाधि) (meditative absorption), \textit{Samprajñata} (संप्रज्ञात) and \textit{Asamprajñata} (असंप्रज्ञात)
    \end{itemize}

\end{frame}



%%%%%%%%%%%%%%%%%%%%%%%%%%%%%%%%%%%%%%%%%%%%%%%%%%%%%%%%%%%%%%%%%%%%%%%%%%%%%%%%%%
\begin{frame}[fragile]\frametitle{\textit{Definition of Yoga} (योग) and \textit{Patanjali's Ashtanga Yoga} (पातंजल अष्टांग योग)}

    \begin{itemize}
        \item \textbf{Chitta:} The mind or mind stuff, of \textit{sanskara} (संस्कार) (past life experience)
        \item \textbf{Vritti:} (\textit{वृत्ति}) Modifications or fluctuations of the mind .
        \item \textbf{Nirodhah:}  (\textit{निरोधः}) Cessation or control of mind modifications.
        \item \textbf{Basis:} \textit{Yoga Darshana} (योग दर्शन) is based on this aphorism.
        \item \textbf{Ashtanga Yoga:}  (\textit{अष्टांग योग}) Propounded by Patanjali as the Royal (Kingly) path.
        \item \textbf{Supreme Yoga:} Incorporates fundamental tenets of other Yoga systems (\textit{Jnana} (ज्ञान), \textit{Bhakti} (भक्ति), \textit{Karma} (कर्म), \textit{Hatha} (हठ), \textit{Mantra} (मंत्र)).
    \end{itemize}

\end{frame}


% %%%%%%%%%%%%%%%%%%%%%%%%%%%%%%%%%%%%%%%%%%%%%%%%%%%%%%%%%%%%%%%%%%%%%%%%%%%%%%%%%%
% \begin{frame}[fragile]\frametitle{Aim of Patanjali’s Ashtanga Yoga and Concept of Chitta}

    % \begin{itemize}
        % \item \textbf{Aim:} \textit{Kaivalya} (liberation) through mind management.
        % \item \textbf{Focus:} Concentration to end all miseries and suffering.
        % \item \textbf{Physical Postures:} Support stability for prolonged meditation.
        % \item \textbf{Antahkarana/Chitta:} The Western term for mind; has four aspects:
        % \begin{itemize}
            % \item \textit{चित्त} (Chitta): Storehouse of Samskaras
            % \item \textit{बुद्धि} (Buddhi): Decision-making faculty
            % \item \textit{अहंकार} (Ahamkara): Ego
            % \item \textit{मनस्} (Manas): Synthesizing faculty
        % \end{itemize}
    % \end{itemize}

% \end{frame}

		
% %%%%%%%%%%%%%%%%%%%%%%%%%%%%%%%%%%%%%%%%%%%%%%%%%%%%%%%%%%%%%%%%%%%%%%%%%%%%%%%%%%
% \begin{frame}[fragile]\frametitle{Concept of Chitta and Chitta Bhumis; Chitta Vrittis and Chitta Vrittinirodhopaya}

    % \begin{itemize}
        % \item \textbf{Chitta Bhumi:} Condition/state of mind in concentration
        % \item \textbf{Chitta Qualities:} \textit{सत्त्विक} (Sattvic), \textit{राजसिक} (Rajasic), \textit{तामसिक} (Tamasic)
        % \item \textbf{Chitta Bhumis:} Five modes of manifestation
        % \begin{itemize}
            % \item \textbf{क्षिप्त} (Kshipta): Scattered, anxious (Rajasic)
            % \item \textbf{मूढ} (Mudha): Dull, stupid (Tamasic)
            % \item \textbf{विक्षिप्त} (Vikshipta): Occasionally centered (Rajasic)
            % \item \textbf{एकाग्रता} (Ekagrata): One-pointed, concentrated (Sattvic)
            % \item \textbf{निरुद्ध} (Niruddha): Suspended mental activity (Sattvic, obstructed Rajas and Tamas)
        % \end{itemize}
    % \end{itemize}

% \end{frame}

% %%%%%%%%%%%%%%%%%%%%%%%%%%%%%%%%%%%%%%%%%%%%%%%%%%%%%%%%%%%%%%%%%%%%%%%%%%%%%%%%%%
% \begin{frame}[fragile]\frametitle{Chitta Levels in Yoga}

    % \begin{itemize}
        % \item First 3 levels of \textit{चित्त} (Chitta) are not considered Yoga:
        % \begin{itemize}
            % \item \textit{क्षिप्त} (Kshipta) 100\% rajas
            % \item \textit{मूढ} (Mudha) 100\% tamas
            % \item \textit{विक्षिप्त} (Vikshipta) 75\% rajas 25\% satvik
        % \end{itemize}
        % \item \textit{एकाग्रता} (Ekagrata) and \textit{निरुद्ध} (Niruddha) are considered Yoga. 100 \% satvik
        % \item Passing through \textit{Ekagrata} and \textit{Niruddha} leads to \textit{समाधि} (Samadhi).
        % \item \textbf{Samskaras:}
        % \begin{itemize}
            % \item \textit{प्रारब्धसंस्कार} (Praarabdha Samskara): Accumulated impressions from previous births
            % \item \textit{वासनासंस्कार} (Vasana Samskara)
        % \end{itemize}
        % \item \textit{निरोधसंस्कार} (Nirodha Samskara) remains in Chitta when others are restrained.
    % \end{itemize}

% \end{frame}

%%%%%%%%%%%%%%%%%%%%%%%%%%%%%%%%%%%%%%%%%%%%%%%%%%%%%%%%%%%%%%%%%%%%%%%%%%%%%%%%%%
\begin{frame}[fragile]\frametitle{Aim of Patanjali’s Ashtanga Yoga and Concept of Chitta}

    \begin{itemize}
        \item \textbf{Aim:} \textit{Kaivalya} (liberation) through mind management.
        \item \textbf{Focus:} Concentration to end all suffering.
        \item \textbf{Chitta:} Four aspects:
        \begin{itemize}
            \item \textit{चित्त} (Chitta): Storehouse of Samskaras
            \item \textit{बुद्धि} (Buddhi): Decision-making
            \item \textit{अहंकार} (Ahamkara): Ego
            \item \textit{मनस्} (Manas): Synthesizing faculty
        \end{itemize}
    \end{itemize}

\end{frame}

%%%%%%%%%%%%%%%%%%%%%%%%%%%%%%%%%%%%%%%%%%%%%%%%%%%%%%%%%%%%%%%%%%%%%%%%%%%%%%%%%%
\begin{frame}[fragile]\frametitle{Chitta Bhumis and Chitta Levels}

    \begin{itemize}
        \item \textbf{Chitta Bhumis:} Five states:
        \begin{itemize}
            \item \textit{क्षिप्त} (Kshipta): Scattered (Rajasic)
            \item \textit{मूढ} (Mudha): Dull (Tamasic)
            \item \textit{विक्षिप्त} (Vikshipta): Occasionally centered (Rajasic)
            \item \textit{एकाग्रता} (Ekagrata): One-pointed (Sattvic)
            \item \textit{निरुद्ध} (Niruddha): Suspended mental activity (Sattvic)
        \end{itemize}
        \item \textbf{Chitta Levels in Yoga:} \textit{Ekagrata} and \textit{Niruddha} lead to \textit{Samadhi}.
    \end{itemize}

\end{frame}


%%%%%%%%%%%%%%%%%%%%%%%%%%%%%%%%%%%%%%%%%%%%%%%%%%%%%%%%%%%%%%%%%%%%%%%%%%%%%%%%%%
\begin{frame}[fragile]\frametitle{Chitta-Vrittis}
    \begin{itemize}
        \item \textit{प्रमाणविपर्ययविकल्पनिद्रास्मृतयः} (1.6)
        \item Five modifications of mind (\textit{Vrittis}):
        \begin{itemize}
            \item \textit{प्रमाण} (Pramana): Knowing correctly
            \item \textit{विपर्यय} (Viparyaya): Incorrect knowledge
            \item \textit{विकल्प} (Vikalpa): Fantasy or imagination
            \item \textit{निद्रा} (Nidra): Deep sleep
            \item \textit{स्मृति} (Smriti): Recollection of memory
        \end{itemize}
    \end{itemize}
\end{frame}

%%%%%%%%%%%%%%%%%%%%%%%%%%%%%%%%%%%%%%%%%%%%%%%%%%%%%%%%%%%%%%%%%%%%%%%%%%%%%%%%%%
\begin{frame}[fragile]\frametitle{Pramana and Viparyaya}
    \begin{itemize}
        \item \textbf{Pramana}: Sources of right knowledge
        \begin{itemize}
            \item \textit{प्रत्यक्ष} (Pratyaksa): Direct cognition
            \item \textit{अनुमान} (Anumana): Inference
            \item \textit{आगम} (Agama/Shabda): Testimony, revelation by Guru
        \end{itemize}
        \item \textbf{Viparyaya}: Misconception, incorrect knowledge
        \item \textit{विपर्ययो मिथ्याज्ञानमतद्रूपप्रतिष्ठम्} (1.8)
        \item False knowledge not based on its own form
    \end{itemize}
\end{frame}

%%%%%%%%%%%%%%%%%%%%%%%%%%%%%%%%%%%%%%%%%%%%%%%%%%%%%%%%%%%%%%%%%%%%%%%%%%%%%%%%%%
\begin{frame}[fragile]\frametitle{Vikalpa, Nidra, and Smriti}
    \begin{itemize}
        \item \textbf{Vikalpa}: Unfounded belief
        \item \textit{शब्दज्ञानानुपाती वस्तुशून्यो विकल्पः} (1.9)
        \item Knowledge through words but empty of an object is fantasy
        \item \textbf{Nidra}: State of deep sleep
        \item \textit{अभावप्रत्ययालम्बना वृत्तिर्निद्रा} (1.10)
        \item Vritti of absence of mental contents for support
        \item \textbf{Smriti}: Memory
        \item \textit{अनुभूतविषयासंप्रमोषः स्मृतिः} (1.11)
        \item Not letting experienced objects escape from the mind
    \end{itemize}
\end{frame}

% %%%%%%%%%%%%%%%%%%%%%%%%%%%%%%%%%%%%%%%%%%%%%%%%%%%%%%%%%%%%%%%%%%%%%%%%%%%%%%%%%%
% \begin{frame}[fragile]\frametitle{Vrittis and Chitta-Vritti Nirodhopaya}
    % \begin{itemize}
        % \item Vrittis: Mental responses to stimuli
        % \item Ego identifies with thought waves
        % \item Wrong identification with 'I' causes miseries
        % \item Enlightenment: Control thought waves
        % \item Abhyasa (\textit{अभ्यास}): Practice
        % \item Vairagya (\textit{वैराग्य}): Non-attachment
        % \item Practice:
        % \begin{itemize}
            % \item Disciplines, diet, pranayama (\textit{प्राणायाम}), asana (\textit{आसन}), meditation
        % \end{itemize}
        % \item Let go of attachments and aversions
        % \item Practice long, uninterrupted, sincere, and firmly rooted
    % \end{itemize}
% \end{frame}

%%%%%%%%%%%%%%%%%%%%%%%%%%%%%%%%%%%%%%%%%%%%%%%%%%%%%%%%%%%%%%%%%%%%%%%%%%%%%%%%%% 
\begin{frame}[fragile]\frametitle{Vrittis and Chitta-Vritti Nirodhopaya} 
	\begin{itemize} 
	\item \textbf{Vrittis:} Mental responses; ego misidentifies with thought waves, causing misery. 
	\item \textbf{Nirodhopaya:} Control thought waves for enlightenment. 
	\item \textbf{Key Practices:} 
		\begin{itemize} 
			\item \textit{Abhyasa} (अभ्यास) - Persistent practice. 
			\item \textit{Vairagya} (वैराग्य) - Non-attachment. 
			\item Practice: Discipline, diet, \textit{प्राणायाम Pranayama}, \textit{आसन Asana}, meditation. 
		\end{itemize} 
	\item \textbf{Key Principle:} Practice must be long, uninterrupted, sincere, and firmly rooted. 
	\end{itemize} 
\end{frame}

%%%%%%%%%%%%%%%%%%%%%%%%%%%%%%%%%%%%%%%%%%%%%%%%%%%%%%%%%%%%%%%%%%%%%%%%%%%%%%%%%%
\begin{frame}[fragile]\frametitle{Kleshas and Their Management}
    \begin{itemize}
        \item Kleshas: Causes of pain
        \item अविद्यास्मितारागद्वेषाभिनिवेशाः (\textit{2.3})
        \item 1. Avidya: Ignorance (अविद्या)
        \item 2. Asmita: Egoism (I-am-ness) (अस्मिता)
        \item 3. Raga: Attachment (Liking) (राग)
        \item 4. Dvesha: Aversion (Disliking) (द्वेष)
        \item 5. Abhinivesha: Fear of death (Clinging to life) (अभिनिवेश)
        \item अविद्या is the root of other Kleshas
        \item Degrees of manifestation:
        \begin{itemize}
            \item Prasupta: Dormant (प्रसुप्त)
            \item Tanu: Weak (तनु)
            \item Vichhina: Oscillating (विच्छिन्न)
            \item Udara: Abundant (उद्र)
        \end{itemize}
        \item Managing Kleshas:
        \begin{itemize}
            \item Kriya Yoga (Tapa, Swadhyaya, Ishwara Pranidhana) (तपः, स्वाध्याय, ईश्वरप्रणिधान)
            \item Dhyana (Meditation) (ध्यान)
            \item Pratiprasava (Involution) (प्रतिप्रसव)
        \end{itemize}
    \end{itemize}
\end{frame}

%%%%%%%%%%%%%%%%%%%%%%%%%%%%%%%%%%%%%%%%%%%%%%%%%%%%%%%%%%%%%%%%%%%%%%%%%%%%%%%%%%
\begin{frame}[fragile]\frametitle{Concept of Ishwara and Ishwara Pranidhana}
    \begin{itemize}
        \item ईश्वरप्रणिधानाद्वा (\textit{1.23})
        \item Devotion to Ishwara leads to Samadhi
        \item क्लेशकर्मविपाकाशयैरपरामृष्टः पुरुषविशेष ईश्वरः (\textit{1.24})
        \item Ishwara: Special soul, untouched by afflictions
        \item तत्र निरतिशयं सावर्ज्ञबीजम् (\textit{1.25})
        \item Ishwara: Seed of limitless omniscience
        \item स पूवेर्षामपि गुरुः कालेनानवच्छेदात् (\textit{1.26})
        \item Ishwara: Guru of all ancient gurus
    \end{itemize}
\end{frame}

%%%%%%%%%%%%%%%%%%%%%%%%%%%%%%%%%%%%%%%%%%%%%%%%%%%%%%%%%%%%%%%%%%%%%%%%%%%%%%%%%%
\begin{frame}[fragile]\frametitle{Concept of Ishwara and Ishwara Pranidhana (contd.)}
    \begin{itemize}
        \item तस्य वाचकः प्रणवः (\textit{1.27})
        \item AUM denotes Ishwara
        \item तज्जपस्तदर्थभावनम् (\textit{1.28})
        \item Recite AUM with understanding
        \item ततः प्रत्यक्चेतनािधगमोऽप्यन्तरायाभावश्च (\textit{1.29})
        \item Practice turns consciousness inward, removes obstacles
        \item Ishwara: Not a religious god, Yoga: Not a religion
        \item Ishwara Pranidhana: Complete surrender to Ishwara
        \item Optional technique in Kriya Yoga (तपः, स्वाध्याय, ईश्वरप्रणिधान)
        \item Key to overcoming ego, leading to Samadhi
    \end{itemize}
\end{frame}

%%%%%%%%%%%%%%%%%%%%%%%%%%%%%%%%%%%%%%%%%%%%%%%%%%%%%%%%%%%%%%%%%%%%%%%%%%%%%%%%%%
\begin{frame}[fragile]\frametitle{\textit{Yoga Sutras}: \textit{Sadhana Pada} (साधन पाद)}

    \begin{itemize}
        \item Second chapter of the \textit{Yoga Sutras} (योग सूत्र), 55 sutras
        \item Outlines the practice (\textit{Sadhana} - साधना) of Yoga
        \item Introduces \textit{Kriya Yoga} (क्रिया योग): \textit{Tapas} (तपस्) (discipline), \textit{Svadhyaya} (स्वाध्याय) (self-study), \textit{Ishvara Pranidhana} (ईश्वर प्राणिधान) (surrender to the divine)
        \item Describes the five \textit{Kleshas} (क्लेशा) (afflictions)
        \item Introduces the eight limbs of Yoga (\textit{Ashtanga Yoga} - अष्टांग योग)
    \end{itemize}

\end{frame}

%%%%%%%%%%%%%%%%%%%%%%%%%%%%%%%%%%%%%%%%%%%%%%%%%%%%%%%%%%%%%%%%%%%%%%%%%%%%%%%%%%
\begin{frame}[fragile]\frametitle{\textit{Yoga Sutras}: \textit{Vibhuti Pada} (विभूतिपाद)}

    \begin{itemize}
        \item Third chapter of the \textit{Yoga Sutras} (योग सूत्र), 55 sutras
        \item Focuses on the supernormal powers (\textit{Vibhutis} - विभूति) that may arise from yoga practice
        \item Describes various forms of \textit{Samyama} (संयम) - controls (combination of concentration, meditation, and \textit{Samadhi} (समाधि))
        \item Warns against attachment to these powers, \textit{siddhis} (सिद्धि)
        \item Emphasizes the importance of discernment and detachment
    \end{itemize}

\end{frame}


%%%%%%%%%%%%%%%%%%%%%%%%%%%%%%%%%%%%%%%%%%%%%%%%%%%%%%%%%%%%%%%%%%%%%%%%%%%%%%%%%%
\begin{frame}[fragile]\frametitle{\textit{Yoga Sutras}: \textit{Kaivalya Pada} (कैवल्य पाद)}

    \begin{itemize}
        \item Fourth and final chapter of the \textit{Yoga Sutras} (योग सूत्र), 34 sutras
        \item Describes \textit{Kaivalya} (कैवल्य) (liberation), the ultimate goal of Yoga
        \item \textit{Dharmamegha Samadhi} (धर्ममेघ समाधि)
        \item \textit{Pratiprasava} (प्रतिप्रसव ), journey of return
    \end{itemize}

\end{frame}



%%%%%%%%%%%%%%%%%%%%%%%%%%%%%%%%%%%%%%%%%%%%%%%%%%%%%%%%%%%
\begin{frame}[fragile]\frametitle{Ashtanga Yoga (Eight Limbs of Yoga)}
    \begin{itemize}
        \item \textbf{Yama} (यम): Social constraints (e.g., अहिंसा (Ahimsa), सत्य (Satya), अस्तेय (Asteya)).
        \item \textbf{Niyama} (नियम): Personal constraints (e.g., शौच (Shaucha), संतोष (Santosha), तप (Tapas)).
        \item \textbf{Asana} (आसन): A posture that is steady and comfortable.
        \item \textbf{Pranayama} (प्राणायाम): Control of breath (modifications of inhalation, exhalation, and breath retention).
        \item \textbf{Pratyahara} (प्रत्याहार): Withdrawal of senses from external objects.
        \item \textbf{Dharana} (धारणा): Concentration on a single object.
        \item \textbf{Dhyana} (ध्यान): Meditation, uninterrupted flow of thoughts towards an object.
        \item \textbf{Samadhi} (समाधि): A transcendental state of being one with the object.
    \end{itemize}
\end{frame}

%%%%%%%%%%%%%%%%%%%%%%%%%%%%%%%%%%%%%%%%%%%%%%%%%%%%%%%%%%%
\begin{frame}[fragile]\frametitle{Key Concepts in Yoga Sutra}
    \begin{itemize}
        \item \textbf{Viveka Khyati} (विवेकख्याति): Discriminative knowledge between \textbf{Purusha} (पुरुष) (consciousness) and \textbf{Prakriti} (प्रकृति) (matter).
        \item \textbf{Chitta Vrittis} (चित्तवृत्तियाँ): Modifications of the mind (right knowledge, misconception, imagination, sleep, memory).
        \item \textbf{Kleshas} (क्लेश): Afflictions causing suffering (\textbf{Avidya} (अविद्या), \textbf{Asmita} (अस्मिता), \textbf{Raga} (राग), \textbf{Dvesha} (द्वेष), \textbf{Abhinivesha} (अभिनिवेश)).
        \item \textbf{Chitta Bhumi} (चित्तभूमि): States of mind ranging from dull to one-pointed focus.
    \end{itemize}
\end{frame}

%%%%%%%%%%%%%%%%%%%%%%%%%%%%%%%%%%%%%%%%%%%%%%%%%%%%%%%%%%%
\begin{frame}[fragile]\frametitle{Chitta Prasadanam and Other Concepts}
    \begin{itemize}
        \item \textbf{Chitta Prasadanam} (चित्तप्रसादनम्): Cultivating a peaceful mind through four attitudes:
        \begin{itemize}
            \item \textbf{Maitri} (मैत्री): Friendliness towards the happy.
            \item \textbf{Karuna} (करुणा): Compassion towards the unhappy.
            \item \textbf{Mudita} (मुदिता): Gladness towards the virtuous.
            \item \textbf{Upeksha} (उपेक्षा): Indifference towards the wicked.
        \end{itemize}
        \item \textbf{Ishvara Pranidhana} (ईश्वरप्रणिधान): Complete surrender to the Supreme (Universal Power).
        \item \textbf{Kriya Yoga} (क्रियायोग): Practice combining Tapas (तप), Svadhyaya (स्वाध्याय), and Ishvara Pranidhana (ईश्वरप्रणिधान) to reduce Kleshas and achieve Samadhi.
    \end{itemize}
\end{frame}


%%%%%%%%%%%%%%%%%%%%%%%%%%%%%%%%%%%%%%%%%%%%%%%%%%%%%%%%%%%%%%%%%%%%%%%%%%%%%%%%%%
\begin{frame}[fragile]\frametitle{Key Verses of Yoga Sutra}

    \begin{itemize}
        \item अथ योगानुशासनम् ॥ १ . १ ॥ - Introduction to Yoga and its practice.
        \item योगिश्चत्तवृित्तिनरोधः ॥ १ . २ ॥ - Yoga is controlling mental fluctuations.
        \item तदा द्रष्टुः स्वरूपेऽवस्थानम् ॥ १ . ३ ॥ - Perceiver returns to true self.
        \item वृत्तसारूप्यिमतरत्र ॥ १ . ४ ॥ - Mental states conform to thoughts.
        \item वृत्तयः पञ्चतय्यः क्लिष्टाऽक्लिष्टाः ॥ १ . ५ ॥ - Five types,  painful - non-painful.
        \item प्रमाणिविपरीयिवकल्पिनद्रास्मृतयः ॥ १ . ६ ॥ - Types: perception, error, imagination, sleep, memory.
        \item प्रत्यक्षानुमानागमाः प्रमाणानि ॥ १ . ७ ॥ - Sources of valid knowledge: direct perception, inference, testimony.
        \item विपरीयो मिथ्याज्ञानमतद्रूपप्रतिष्ठम् ॥ १ . ८ ॥ - Incorrect knowledge is based on false information.
        \item शब्दज्ञानानुपाती वस्तुशून्यो विकल्पः ॥ १ . ९ ॥ - Imagination is based on words without reality.
        \item अभावप्रत्ययालम्बना वृत्तिनिद्रा ॥ १ . १० ॥ - Sleep is absence of objective awareness.
        \item अनुभूतिवषयासंप्रमोषः स्मृतिः ॥ १ . ११ ॥ - Memory is retention of experienced impressions.
        \item अभ्यासवैराग्याभ्यां तन्निरोधः ॥ १ . १२ ॥ - Control of mental states through practice and detachment.
    \end{itemize}

\end{frame}


%%%%%%%%%%%%%%%%%%%%%%%%%%%%%%%%%%%%%%%%%%%%%%%%%%%%%%%%%%%%%%%%%%%%%%%%%%%%%%%%%%
\begin{frame}[fragile]\frametitle{\textit{Summary of Patanjali Yoga Sutra} (पातंजल योग सूत्र)}

    \begin{itemize}
        \item \textbf{Yoga Meaning:} Derived from 'Yuj' (युज) – union and concentration.
        \item \textbf{Union Aspect:} Integration of body, mind, and spirit.
        \item \textbf{Concentration Aspect:} Yoga as focused awareness and ultimate goal.
        \item \textbf{Definition:} \textit{“Yogaḥ cittavṛtti nirodhaḥ”} (योगः चित्तवृत्ति निरोधः) - Stoppage of mental modifications.
        \item \textbf{Vrittis:} Mental modifications or thought waves (\textit{वृत्ति}).
        \item \textbf{Goal:} Liberate from suffering and cycle of rebirth by controlling \textit{vrittis} (वृत्ति).
        \item \textbf{समाधि Samadhi:} Ultimate limb of \textit{Ashtanga Yoga} (अष्टांग योग), representing deep concentration.
        \item \textbf{Mind Functions:} Misery arises from false identification at the mental level.
    \end{itemize}

\end{frame}


%%%%%%%%%%%%%%%%%%%%%%%%%%%%%%%%%%%%%%%%%%%%%%%%%%%%%%%%%%%%%%%%%%%%%%%%%%%%%%%%%%
\begin{frame}[fragile]\frametitle{}
\begin{center}
{\Large 2.2 Introduction and study of Bhagwad Gita including memorization of selected Shlokas (Chapter 2 - 47,48,49,50 and 70)}
\end{center}
\end{frame}

%%%%%%%%%%%%%%%%%%%%%%%%%%%%%%%%%%%%%%%%%%%%%%%%%%%%%%%%%%% 
\begin{frame}[fragile]\frametitle{\textit{Introduction to Bhagavad Gita} (भगवद गीता)} 
	\begin{itemize} 
		\item \textit{Bhagavad Gita} (भगवद गीता): 700-verse scripture, part of the \textit{Mahabharata} (महाभारत). 
		\item Dialogue between Arjuna (अर्जुन) and Krishna (कृष्ण) on the कुरुक्षेत्र  Kurukshetra battlefield. 
		\item Central themes: Duty (\textit{Dharma} - धर्म), righteousness, and spiritual wisdom. 
		\item Addresses Arjuna's moral and philosophical dilemmas. 
		\item A revered philosophical and spiritual text in Hinduism. 
	\end{itemize} 
\end{frame}

%%%%%%%%%%%%%%%%%%%%%%%%%%%%%%%%%%%%%%%%%%%%%%%%%%%%%%%%%%%
\begin{frame}[fragile]\frametitle{Chapter 2: Selected Shlokas}

      \begin{itemize}
		\item \textbf{Verse 47} (\textit{Karma Yoga}) - 
		
			\textit{कर्मण्येवाधिकारस्ते मा फलेषु कदाचन |} \\
			\textit{मा कर्मफलहेतुर्भूर्मा ते सङ्गोऽस्त्वकर्मणि ||}
		
		\item \textbf{Verse 48} (\textit{Karma Yoga}) - 
		
			\textit{योगस्थः कुरु कर्माणि सङ्गं त्यक्त्वा धनञ्जय |} \\
			\textit{सिद्ध्यसिद्ध्योः समो भूत्वा समत्वं योग उच्यते ||}
		
		\item \textbf{Verse 49} (\textit{Karma Yoga}) - 
		
			\textit{यस्त्विन्द्रियाणि मनसा नियाम्यारभते नरः |} \\
			\textit{मुञ्जते तस्य योगिनोऽन्यः ||}
		
		\item \textbf{Verse 50} (\textit{Karma Yoga}) - 
		
			\textit{ब्रह्मण्याधाय कर्माणि सङ्गं त्यक्त्वा धनञ्जय |} \\
			\textit{सिद्ध्यसिद्ध्योः समो भूत्वा समत्वं योग उच्यते ||}
		
		\item \textbf{Verse 70} (\textit{Self-Realization}) - 
		
			\textit{अपण्यतं तु तद्वृत्तमन्तरायामुक्तं सदा} \\
			\textit{तन्मया न संशय ||}
		
	  \end{itemize}

\end{frame}

%%%%%%%%%%%%%%%%%%%%%%%%%%%%%%%%%%%%%%%%%%%%%%%%%%%%%%%%%%%
\begin{frame}[fragile]\frametitle{Study of Bhagavad Gita: Key Themes}

      \begin{itemize}
		\item \textit{Dharma} (धर्म) - The concept of duty and righteousness.
		\item \textit{Karma Yoga} (कर्म योग) - Path of selfless action and duty.
		\item \textit{Bhakti Yoga} (भक्ति योग) - Path of devotion and love towards God.
		\item \textit{Jnana Yoga} (ज्ञान योग) - Path of knowledge and wisdom.
		\item \textit{Self-Realization} - Understanding the true nature of self and existence.
	  \end{itemize}

\end{frame}


%%%%%%%%%%%%%%%%%%%%%%%%%%%%%%%%%%%%%%%%%%%%%%%%%%%%%%%%%%%
\begin{frame}[fragile]\frametitle{Memorization of Selected Shlokas}

      \begin{itemize}
		\item \textbf{Verse 47} (\textit{Karma Yoga}) - \textit{कर्मण्येवाधिकारस्ते मा फलेषु कदाचन | मा कर्मफलहेतुर्भूर्मा ते सङ्गोऽस्त्वकर्मणि ||}
		\item \textbf{Verse 48} (\textit{Karma Yoga}) - \textit{योगस्थः कुरु कर्माणि सङ्गं त्यक्त्वा धनञ्जय | सिद्ध्यसिद्ध्योः समो भूत्वा समत्वं योग उच्यते ||}
		\item \textbf{Verse 49} (\textit{Karma Yoga}) - \textit{यस्त्विन्द्रियाणि मनसा नियाम्यारभते नरः | मुञ्जते तस्य योगिनोऽन्यः ||}
		\item \textbf{Verse 50} (\textit{Karma Yoga}) - \textit{ब्रह्मण्याधाय कर्माणि सङ्गं त्यक्त्वा धनञ्जय | सिद्ध्यसिद्ध्योः समो भूत्वा समत्वं योग उच्यते ||}
		\item \textbf{Verse 70} (\textit{Self-Realization}) - \textit{अपण्यतं तु तद्वृत्तमन्तरायामुक्तं सदा तन्मया न संशय ||}
	  \end{itemize}

\end{frame}

%%%%%%%%%%%%%%%%%%%%%%%%%%%%%%%%%%%%%%%%%%%%%%%%%%%%%%%%%%%%%%%%%%%%%%%%%%%%%%%%%%
\begin{frame}[fragile]\frametitle{}
\begin{center}
{\Large 2.3 Introduction and study of \textit{Hatha Pradipika} (हठ प्रदीपिका)}
\end{center}
\end{frame}

%%%%%%%%%%%%%%%%%%%%%%%%%%%%%%%%%%%%%%%%%%%%%%%%%%%%%%%%%%%
\begin{frame}[fragile]\frametitle{Hatha Yoga}

      \begin{itemize}
		\item Focuses on physical practices and techniques.
		\item Aims to balance the body and mind through postures (\textit{Asanas}) and breath control (\textit{Pranayama}).
		\item Emphasizes purification of the body to prepare for higher practices.
		\item Major texts: \textit{Hatha Yoga Pradipika}, \textit{Gheranda Samhita}.
		\item Often serves as a preparatory practice for deeper meditative techniques.
	  \end{itemize}

\end{frame}

%%%%%%%%%%%%%%%%%%%%%%%%%%%%%%%%%%%%%%%%%%%%%%%%%%%%%%%%%%%
\begin{frame}[fragile]\frametitle{Introduction to \textit{Hatha Pradipika} (हठ प्रदीपिका)}

      \begin{itemize}
		\item \textit{Hatha Pradipika} (हठ प्रदीपिका) - A classical text on Hatha Yoga.
		\item Written by Swami Svatmarama (स्वामी स्वात्माराम) in the 15th century CE.
		\item Focuses on physical postures (\textit{Asanas} (आसन)), breath control (\textit{Pranayama} (प्राणायाम)), and meditation.
		\item Aims to prepare the body and mind for higher spiritual practices.
		\item Provides detailed instructions on various Hatha Yoga techniques.
	  \end{itemize}

\end{frame}

%%%%%%%%%%%%%%%%%%%%%%%%%%%%%%%%%%%%%%%%%%%%%%%%%%%%%%%%%%%
\begin{frame}[fragile]\frametitle{Key Concepts in \textit{Hatha Pradipika} (हठ प्रदीपिका)}

      \begin{itemize}
		\item \textit{Asanas} (आसन) - Physical postures for physical stability and health.
		\item \textit{Pranayama} (प्राणायाम) - Techniques for controlling the breath and vital energy.
		\item \textit{Mudras} (मुद्रा) - Hand gestures to control energy flow.
		\item \textit{Bandhas} (बंधन) - Body locks to channel energy within.
		\item \textit{Shatkarma} (षटकर्म ) - Six purification techniques to cleanse the body.
	  \end{itemize}

\end{frame}

%%%%%%%%%%%%%%%%%%%%%%%%%%%%%%%%%%%%%%%%%%%%%%%%%%%%%%%%%%%
\begin{frame}[fragile]\frametitle{Hatha Yoga Pradipika: Chapter 1 - आसन  Asana}

Study of \textit{Asanas} (आसन) in \textit{Hatha Pradipika} (हठ प्रदीपिका)

\begin{itemize}
    \item Overview of Hatha Yoga and its significance.
    \item Qualifications of a Hatha Yogi.
    \item Ideal environment for practice.
    \item Brief on Yamas and Niyamas.
    \item 15 key asanas with benefits and techniques.
    \item Key postures: \textit{पद्मासन Padmasana}, \textit{शीर्षासन Shirshasana}, \textit{सर्वांगासन Sarvangasana}.
    \item Emphasis on strength, flexibility, and concentration.
    \item Prepares for deeper meditation.
\end{itemize}

\end{frame}

% %%%%%%%%%%%%%%%%%%%%%%%%%%%%%%%%%%%%%%%%%%%%%%%%%%%%%%%%%%%
% \begin{frame}[fragile]\frametitle{Hatha Yoga Pradipika: Chapter 1 - आसन  Asana}

% Study of \textit{Asanas} (आसन) in \textit{Hatha Pradipika} (हठ प्रदीपिका)

% \begin{itemize}
    % \item Introduction to Hatha Yoga and its importance
    % \item Qualifications of a Hatha Yogi
    % \item Description of the proper place for yoga practice
    % \item Yamas and Niyamas briefly mentioned
    % \item 15 important asanas described in detail
    % \item Benefits and techniques of each asana
	% \item Describes various \textit{Asanas} (आसन) for physical health and spiritual progress.
	% \item Emphasizes proper alignment, stability, and breath control.
	% \item Includes postures like \textit{Padmasana} (पद्मासन) (Lotus Pose), \textit{Shirshasana} (शीर्षासन) (Headstand), and \textit{Sarvangasana} (सर्वांगासन) (Shoulder Stand).
	% \item Focuses on achieving physical strength, flexibility, and concentration.
	% \item Prepares the practitioner for deeper meditative practices.
  % \end{itemize}

% \end{frame}

%%%%%%%%%%%%%%%%%%%%%%%%%%%%%%%%%%%%%%%%%%%%%%%%%%%%%%%%%%%
\begin{frame}[fragile]\frametitle{Hatha Yoga Pradipika: Chapter 2 - प्राणायाम  Pranayama}
Study of \textit{Pranayama} (प्राणायाम) in \textit{Hatha Pradipika} (हठ प्रदीपिका)

\begin{itemize}
    \item Importance of Pranayama in Hatha Yoga
    \item Purification of nadis (energy channels)
    \item Description of various Pranayama techniques:
    \begin{itemize}
        \item सूर्य भेदन Surya Bhedana
        \item उज्जयि Ujjayi
        \item सितकारि Sitkari
        \item शीतली Shitali
        \item भस्त्रिका Bhastrika
        \item भ्रामरी Bhramari
        \item मूर्च्छा Murccha
        \item प्लविनि Plavini
    \end{itemize}
    \item Benefits and cautions for each प्राणायाम Pranayama
	\item Details various \textit{Pranayama} (प्राणायाम) techniques for controlling breath and energy.
	\item Includes practices such as \textit{Kapalabhati} (कपालभाति) (Skull Shining Breath) and \textit{Nadi Shodhana} (नाडी शोधन) (Alternate Nostril Breathing).
	\item Aims to purify the body, calm the mind, and increase vital energy.
	\item Techniques are used to balance the prana (प्राण) (vital energy) and support meditation.
	\item Essential for mastering advanced Hatha Yoga practices.
  \end{itemize}

\end{frame}


%%%%%%%%%%%%%%%%%%%%%%%%%%%%%%%%%%%%%%%%%%%%%%%%%%%%%%%%%%%
\begin{frame}[fragile]\frametitle{Hatha Yoga Pradipika: Chapter 3 - मुद्रा  Mudra and बन्ध Bandha}
\begin{itemize}
    \item Introduction to Mudras and Bandhas
    \item 10 important Mudras described:
    \begin{itemize}
        \item महा मुद्रा Maha Mudra
        \item महा बन्ध Maha Bandha
        \item महा  वेध  Maha Vedha
        \item खेचरी Khechari
        \item उड्डीयान बन्ध Uddiyana Bandha
        \item मूलबन्ध Mula Bandha
        \item जालन्धर बन्ध Jalandhara Bandha
        \item विपरीत करणीViparita Karani
        \item वज्रोली Vajroli
        \item शक्ति चालना Shakti Chalana
    \end{itemize}
    \item Techniques and benefits of each Mudra and Bandha
\end{itemize}
\end{frame}



%%%%%%%%%%%%%%%%%%%%%%%%%%%%%%%%%%%%%%%%%%%%%%%%%%%%%%%%%%%
\begin{frame}[fragile]\frametitle{Hatha Yoga Pradipika: Chapter 4 - समाधि  Samadhi}
\begin{itemize}
    \item Introduction to नाद  Nada (inner sound) and its importance
    \item Stages of Nada and their characteristics
    \item Description of लय  Laya Yoga (absorption through sound)
    \item Techniques for awakening कुण्डलिनि  Kundalini
    \item Four stages of Yoga:
    \begin{itemize}
        \item आरम्भ  Arambha
        \item घट Ghata
        \item परिचय Parichaya
        \item निष्पत्ति Nishpatti
    \end{itemize}
    % \item Signs of progress in yoga practice
\end{itemize}
\end{frame}

%%%%%%%%%%%%%%%%%%%%%%%%%%%%%%%%%%%%%%%%%%%%%%%%%%%%%%%%%%%
\begin{frame}[fragile]\frametitle{Hatha Yoga Pradipika: Chapter 5 - लय  Laya Yoga}
\begin{itemize}
    \item Detailed explanation of लय  Laya Yoga
    \item Importance of dissolving the mind in the Absolute
    \item Techniques for achieving Laya
    \item Role of Kundalini in Laya Yoga
    \item Relationship between Prana and mind
    \item Signs of successful Laya practice
\end{itemize}
\end{frame}

%%%%%%%%%%%%%%%%%%%%%%%%%%%%%%%%%%%%%%%%%%%%%%%%%%%%%%%%%%%
\begin{frame}[fragile]\frametitle{Hatha Yoga Pradipika: Chapter 6 - Liberation}
\begin{itemize}
    \item Nature of Samadhi and liberation
    \item Differences between जीवनमुक्त  Jivanmukta and विदेहमुक्त  Videhamukta
    \item Characteristics of a liberated being
    \item Obstacles on the path to liberation
    \item Final instructions for attaining liberation
    \item Importance of Guru's grace in achieving liberation
\end{itemize}
\end{frame}

%%%%%%%%%%%%%%%%%%%%%%%%%%%%%%%%%%%%%%%%%%%%%%%%%%%%%%%%%%%
\begin{frame}[fragile]\frametitle{Significance of Hatha Pradipika}

      \begin{itemize}
		\item Foundation of Hatha Yoga practices - Essential for practitioners seeking deeper understanding.
		\item Integrates physical and spiritual practices to enhance overall well-being.
		\item Offers practical guidance for practitioners of all levels.
		\item Highlights the importance of discipline, perseverance, and correct practice.
		\item Continues to influence modern Yoga practices and teachings.
	  \end{itemize}

\end{frame}



%%%%%%%%%%%%%%%%%%%%%%%%%%%%%%%%%%%%%%%%%%%%%%%%%%%%%%%%%%%%%%%%%%%%%%%%%%%%%%%%%%
\begin{frame}[fragile]\frametitle{}
\begin{center}
{\Large घेरण्ड संहिता  Gheranda Samhita}
\end{center}
\end{frame}



%%%%%%%%%%%%%%%%%%%%%%%%%%%%%%%%%%%%%%%%%%%%%%%%%%%%%%%%%%%
\begin{frame}[fragile]\frametitle{घेरण्ड संहिता  Gheranda Samhita: Overview}
\begin{itemize}
    \item 7 chapters covering the "seven-limbed yoga"
    \item Systematic approach to purification and yoga practice
    \item Emphasis on physical purification as a foundation
    \item Detailed descriptions of various techniques
    \item Goal: to achieve the "divine body" (दिव्य देह divya deha)
\end{itemize}
\end{frame}

%%%%%%%%%%%%%%%%%%%%%%%%%%%%%%%%%%%%%%%%%%%%%%%%%%%%%%%%%%%
\begin{frame}[fragile]\frametitle{घेरण्ड संहिता Gheranda Samhita: Chapter 1 - षटकर्म  Shatkarma}
\begin{itemize}
    \item Six purification techniques (षटकर्म  Shatkarma):
    \begin{itemize}
        \item धौति  Dhauti (cleansing)
        \item बस्ति  Basti (enema)
        \item नेति  Neti (nasal cleansing)
        \item त्राटक  Trataka (gazing)
        \item नौलि Nauli (abdominal massaging)
        \item कपालभाति  Kapalabhati (skull shining breath)
    \end{itemize}
    \item Benefits and methods for each technique
    \item Importance of purification before other practices
\end{itemize}
\end{frame}

%%%%%%%%%%%%%%%%%%%%%%%%%%%%%%%%%%%%%%%%%%%%%%%%%%%%%%%%%%%
\begin{frame}[fragile]\frametitle{Gheranda Samhita: Chapter 2 - Asana}
\begin{itemize}
    \item 32 asanas described in detail
    \item Classification of asanas:
    \begin{itemize}
        \item Meditative
        \item Relaxation
        \item Cultural
    \end{itemize}
    \item Techniques and benefits of each asana
    \item Emphasis on steady and comfortable posture
\end{itemize}
\end{frame}

%%%%%%%%%%%%%%%%%%%%%%%%%%%%%%%%%%%%%%%%%%%%%%%%%%%%%%%%%%%
\begin{frame}[fragile]\frametitle{Gheranda Samhita: Chapter 3 - मुद्राः Mudra}
\begin{itemize}
    \item 25 \textit{Mudras} (मुद्राः) described
    \item Importance of \textit{Mudras} (मुद्राः) in directing \textit{Prana} (प्राण)
    \item Notable \textit{Mudras} (मुद्राः):
    \begin{itemize}
        \item \textit{Maha Mudra} (महामुद्रा)
        \item \textit{Nabho Mudra} (नभोमुद्रा)
        \item \textit{Khechari Mudra} (खेचरीमुद्रा)
        \item \textit{Viparita Karani Mudra} (विपरीतकरणीमुद्रा)
        \item \textit{Yoni Mudra} (योनीमुद्रा)
    \end{itemize}
    \item Techniques and benefits of each \textit{Mudra} (मुद्रा)
\end{itemize}
\end{frame}

%%%%%%%%%%%%%%%%%%%%%%%%%%%%%%%%%%%%%%%%%%%%%%%%%%%%%%%%%%%
\begin{frame}[fragile]\frametitle{Gheranda Samhita: Chapter 4 - Pratyahara}
\begin{itemize}
    \item Techniques for sense withdrawal
    \item 5 types of \textit{Pratyahara} (प्रत्याहार) described:
    \begin{itemize}
        \item \textit{Yoni Mudra} (योनीमुद्रा)
        \item \textit{Sambhavi Mudra} (शांभवी मुद्रा)
        \item Five \textit{Dharanas} (धारणाः) (concentrations on elements)
        \item Sound absorption
        \item Withdrawing senses from objects
    \end{itemize}
    \item Importance of \textit{Pratyahara} (प्रत्याहार) in preparing for meditation
\end{itemize}
\end{frame}


%%%%%%%%%%%%%%%%%%%%%%%%%%%%%%%%%%%%%%%%%%%%%%%%%%%%%%%%%%%
\begin{frame}[fragile]\frametitle{Gheranda Samhita: Chapter 5 - \textit{Pranayama} (प्राणायाम)}
\begin{itemize}
    \item Importance of proper diet before \textit{Pranayama} (प्राणायाम)
    \item 8 types of \textit{Pranayama} (प्राणायाम) described:
    \begin{itemize}
        \item \textit{Sahita Kumbhaka} (सहित कुम्भक)
        \item \textit{Surya Bheda} (सूर्यभेद)
        \item \textit{Ujjayi} (उज्जायी)
        \item \textit{Sitali} (सीतली)
        \item \textit{Bhastrika} (भस्त्रिका)
        \item \textit{Bhramari} (भ्रामरी)
        \item \textit{Murchha} (मूर्छा)
        \item \textit{Kevali} (केवली)
    \end{itemize}
    \item Techniques and benefits of each \textit{Pranayama} (प्राणायाम)
    \item Signs of success in \textit{Pranayama} (प्राणायाम) practice
\end{itemize}
\end{frame}


%%%%%%%%%%%%%%%%%%%%%%%%%%%%%%%%%%%%%%%%%%%%%%%%%%%%%%%%%%%
\begin{frame}[fragile]\frametitle{Gheranda Samhita: Chapter 6 - \textit{Dhyana} (ध्यान)}
\begin{itemize}
    \item 3 types of \textit{Dhyana} (ध्यान) (meditation) described:
    \begin{itemize}
        \item \textit{Sthula} (स्थूल) (gross)
        \item \textit{Jyotis} (ज्योतिष) (luminous)
        \item \textit{Sukshma} (सूक्ष्म) (subtle)
    \end{itemize}
    \item Techniques for each type of meditation
    \item Importance of concentration on specific objects or concepts
    \item Benefits of regular meditation practice
\end{itemize}
\end{frame}

%%%%%%%%%%%%%%%%%%%%%%%%%%%%%%%%%%%%%%%%%%%%%%%%%%%%%%%%%%%
\begin{frame}[fragile]\frametitle{Gheranda Samhita: Chapter 7 - \textit{Samadhi} (समाधि)}
\begin{itemize}
    \item 6 types of \textit{Samadhi} (समाधि) described:
    \begin{itemize}
        \item \textit{Dhyana Yoga Samadhi} (ध्यान योग समाधि)
        \item \textit{Nada Yoga Samadhi} (नाद योग समाधि)
        \item \textit{Rasananda Samadhi} (रसानन्द समाधि)
        \item \textit{Laya Sidhi Samadhi} (लय सिद्धि समाधि)
        \item \textit{Bhakti Yoga Samadhi} (भक्ति योग समाधि)
        \item \textit{Raja Yoga Samadhi} (राज योग समाधि)
    \end{itemize}
    \item Techniques for achieving each type of \textit{Samadhi} (समाधि)
    \item Signs of progress and success in \textit{Samadhi} (समाधि)
    \item Ultimate goal: liberation and realization of the Self
\end{itemize}
\end{frame}



%%%%%%%%%%%%%%%%%%%%%%%%%%%%%%%%%%%%%%%%%%%%%%%%%%%%%%%%%%%%%%%%%%%%%%%%%%%%%%%%%%
\begin{frame}[fragile]\frametitle{}
\begin{center}
{\Large Concepts in Hatha Yoga}
\end{center}
\end{frame}

%%%%%%%%%%%%%%%%%%%%%%%%%%%%%%%%%%%%%%%%%%%%%%%%%%%%%%%%%%%%%%%%%%%%%%%%%%%%%%%%%%
\begin{frame}[fragile]\frametitle{Causes of Success (Sādhaka Tattwa) in Hatha Yoga Sādhanā}
\begin{itemize}
    \item \textbf{Enthusiasm (उत्साहः - Utsāha):} Positive attitude and constant inspiration.
    \item \textbf{Courage (साहसः - Sāhasa):} Face inner visions and realizations as they dawn.
    \item \textbf{Perseverance (धैर्यम् - Dhairya):} Maintain regular practice despite challenges.
    \item \textbf{Discriminating Knowledge (तत्त्वज्ञानम् - Tattvajnāna):} Align actions with spiritual goals.
    \item \textbf{Determination (निश्चयः - Niśchaya):} Resolve to continue practice under all circumstances.
    \item \textbf{Aloofness from Company (जनसङ्गपरित्यागः - Janasaṅgha Parityāga):} Avoid social distractions and negative influences.
\end{itemize}
\end{frame}

%%%%%%%%%%%%%%%%%%%%%%%%%%%%%%%%%%%%%%%%%%%%%%%%%%%%%%%%%%%%%%%%%%%%%%%%%%%%%%%%%%
\begin{frame}[fragile]\frametitle{Causes of Failure (Bādhaka Tattwa) in Hatha Yoga Sādhanā}
\begin{itemize}
    \item \textbf{Over-eating (अत्याहारः - Atyāhāra):} Consuming more food than needed.
    \item \textbf{Exertion (प्रयासः - Prayāsa):} Excessive physical or mental effort.
    \item \textbf{Talkativeness (प्रजल्पः - Prajalpa):} Talking more than necessary.
    \item \textbf{Attachment to Rules (नियमाग्रहः - Niyamāgraha):} Over-adherence to rigid rules (e.g., cold baths, strict diets).
    \item \textbf{Social Company (जनसङ्गः - Janasaṅgha):} Associating with people can disturb mental focus.
    \item \textbf{Fickle-mindedness (लोल्यम् - Laulya):} Instability of the senses leading to distractions.
\end{itemize}
\end{frame}

%%%%%%%%%%%%%%%%%%%%%%%%%%%%%%%%%%%%%%%%%%%%%%%%%%%%%%%%%%%%%%%%%%%%%%%%%%%%%%%%%%
\begin{frame}[fragile]\frametitle{Concept of Ghata (घटः) and Ghata Shuddhi (घटशुद्धिः)}
\begin{itemize}
    \item \textbf{Ghata (घटः - Vessel):} The body and mind are likened to a vessel in Gheranda Samhita.
    \item \textbf{Tempering by Fire of Yoga:} The body (ghata) must be tempered or purified through yoga practice.
    \item \textbf{Ghata Shuddhi (घटशुद्धिः):} Purification of the psycho-physiological structure to prepare for higher yogic practices.
    \item \textbf{Saptanga Yoga (सप्ताङ्गयोगः):} Seven limbs of yoga are essential for Ghata Shuddhi, as outlined by Sage Gheranda.
    \item \textbf{Goal:} Achieving self-realization through purification of the body and mind.
\end{itemize}
\end{frame}

%%%%%%%%%%%%%%%%%%%%%%%%%%%%%%%%%%%%%%%%%%%%%%%%%%%%%%%%%%%%%%%%%%%%%%%%%%%%%%%%%%
\begin{frame}[fragile]\frametitle{Saptanga Yoga (सप्ताङ्गयोगः) in Ghata Shuddhi}
\begin{itemize}
    \item \textbf{Kriyas (क्रियाः):} Cleansing actions to purify the body.
    \item \textbf{Asanas (आसनाः):} Physical postures to prepare the body.
    \item \textbf{Pratyahara (प्रत्याहारः):} Withdrawal of senses from external objects.
    \item \textbf{Pranayama (प्राणायामः):} Breath control to purify the pranic energies.
    \item \textbf{Mudras (मुद्राः):} Gestures that direct energy flows in the body.
    \item \textbf{Dhyana (ध्यानम्):} Meditation to purify the mind.
    \item \textbf{Samadhi (समाधिः):} The final state of self-realization and union.
\end{itemize}
\end{frame}


%%%%%%%%%%%%%%%%%%%%%%%%%%%%%%%%%%%%%%%%%%%%%%%%%%%%%%%%%%%%%%%%%%%%%%%%%%%%%%%%%%
\begin{frame}[fragile]\frametitle{}
\begin{center}
{\Large 2.4 General Introduction to \textit{Prasthanatrayee} (प्रस्थानत्रयी)}
\end{center}
\end{frame}

%%%%%%%%%%%%%%%%%%%%%%%%%%%%%%%%%%%%%%%%%%%%%%%%%%%%%%%%%%%%%%%%%%%%%%%%%%%%%%%%%%
\begin{frame}[fragile]\frametitle{\textit{Prasthanatrayee} (प्रस्थानत्रयी)}
    \begin{itemize}
        \item \textbf{Upanishads (उपनिषद्):}
        \begin{itemize}
            \item Known as \textit{Śruti Prasthāna} (श्रुति प्रवस्था), meaning "that which is heard."
            \item Composed during deep meditation by Rishis, recorded knowledge received.
        \end{itemize}
        \item \textbf{Bhagavad Gita (भगवद्गीता):}
        \begin{itemize}
            \item Known as \textit{Smriti Prasthāna} (स्मृति प्रवस्था), meaning "that which is remembered."
            \item Central text of the Mahabharata, offering philosophical and practical guidance.
        \end{itemize}
        \item \textbf{Brahma Sutra (ब्रह्मसूत्र):}
        \begin{itemize}
            \item Known as \textit{Nyaya Prasthāna} (न्याय प्रवस्था), meaning "the path of logic."
            \item Provides a systematic exposition of Upanishadic teachings.
        \end{itemize}
    \end{itemize}
\end{frame}

% %%%%%%%%%%%%%%%%%%%%%%%%%%%%%%%%%%%%%%%%%%%%%%%%%%%%%%%%%%%
% \begin{frame}[fragile]\frametitle{Introduction to \textit{Prasthanatrayee} (प्रस्थानत्रयी)}

      % \begin{itemize}
		% \item \textit{Prasthanatrayee} (प्रस्थानत्रयी) - The three foundational texts of Vedanta philosophy.
		% \item Comprises:
		  % \begin{itemize}
		      % \item \textit{Upanishads} (उपनिषद्) - Core philosophical texts exploring the nature of reality and self.
		      % \item \textit{Bhagavad Gita} (भगवद् गीता) - A dialogue between Arjuna and Krishna on duty, righteousness, and spirituality.
		      % \item \textit{Brahma Sutras} (ब्रह्म सूत्र) - Philosophical aphorisms systematizing Vedantic thought.
		  % \end{itemize}
		% \item Together, they form the basis of Vedantic study and practice.
		% \item Essential for understanding key concepts in Hindu philosophy.
	  % \end{itemize}

% \end{frame}

%%%%%%%%%%%%%%%%%%%%%%%%%%%%%%%%%%%%%%%%%%%%%%%%%%%%%%%%%%%
\begin{frame}[fragile]\frametitle{The \textit{Upanishads} (उपनिषद्)}

      \begin{itemize}
		\item Ancient texts that form the core of Vedic wisdom. Focus on spiritual knowledge and philosophical inquiry.
        \item \textbf{11 Principal Upanishads:}
        \begin{itemize}
            \item \textit{Chandogya (छांडोग्य)} 
            \item \textit{Kena (केन)}
            \item \textit{Katha (काठ)}
            \item \textit{Isha (ईशा)}
            \item \textit{Taittiriya (तैत्तिरीय)}
            \item \textit{Aitareya (ऐतरेय)}
            \item \textit{Brihadaranyaka (बृहदारण्यक)}
            \item \textit{Mandukya (माण्डूक्य)}
            \item \textit{Prashna (प्रश्न)}
            \item \textit{Munda (मुण्ड)}
            \item \textit{Svetasvatara (स्वेताश्वतार)}
        \end{itemize}		
		\item Discuss the nature of ultimate reality (\textit{Brahman} - ब्रह्मन्) and the individual soul (\textit{Atman} - आत्मन्).
		\item Key Upanishads include \textit{Isha} (ईशा), \textit{Kena} (केन), \textit{Katha} (कठ), and \textit{Mandukya} (माण्डूक्य).
		\item Emphasize meditation, self-realization, and the unity of all existence.
	  \end{itemize}

\end{frame}


%%%%%%%%%%%%%%%%%%%%%%%%%%%%%%%%%%%%%%%%%%%%%%%%%%%%%%%%%%%
\begin{frame}[fragile]\frametitle{The Bhagavad Gita}

      \begin{itemize}
		\item A 700-verse Hindu scripture part of the \textit{Mahabharata}.
		\item Dialogue between Prince Arjuna and Lord Krishna.
		\item Addresses the nature of duty (\textit{Dharma}), action, and devotion.
		\item Explores paths of Karma Yoga (action), Bhakti Yoga (devotion), and Jnana Yoga (knowledge).
		\item Provides guidance on ethical and spiritual living.
	  \end{itemize}

\end{frame}

%%%%%%%%%%%%%%%%%%%%%%%%%%%%%%%%%%%%%%%%%%%%%%%%%%%%%%%%%%%
\begin{frame}[fragile]\frametitle{The Brahma Sutras}

      \begin{itemize}
		\item Philosophical texts attributed to Sage Vyasa.
		\item Comprises 555 sutras (aphorisms) summarizing the teachings of the Upanishads.
		\item Systematizes Vedantic thought and addresses key metaphysical questions.
		\item Divided into four chapters: \textit{Sutras on the Nature of Brahman}, \textit{Sutras on the Universe}, \textit{Sutras on the Path of Knowledge}, and \textit{Sutras on the Liberation}.
		\item Focuses on the unity of Brahman and the self, and the nature of liberation.
	  \end{itemize}

\end{frame}

%%%%%%%%%%%%%%%%%%%%%%%%%%%%%%%%%%%%%%%%%%%%%%%%%%%%%%%%%%%
\begin{frame}[fragile]\frametitle{Significance of Prasthanatrayee}

      \begin{itemize}
		\item Provides comprehensive understanding of Vedantic philosophy.
		\item Forms the basis for various schools of Vedanta and spiritual practices.
		\item Guides ethical, spiritual, and philosophical aspects of life.
		\item Essential for deep study of Hindu philosophy and theology.
		\item Continues to influence spiritual thought and practice today.
	  \end{itemize}

\end{frame}


%%%%%%%%%%%%%%%%%%%%%%%%%%%%%%%%%%%%%%%%%%%%%%%%%%%%%%%%%%%%%%%%%%%%%%%%%%%%%%%%%%
\begin{frame}[fragile]\frametitle{}
\begin{center}
{\Large 2.5 Concepts and Principles of \textit{ahara} (Diet) in \textit{Hatha Pradipika} (हठप्रदीपिका) and \textit{Bhagavad Gita} (भगवद् गीता) (\textit{Mitahara} and \textit{Yuktahara})}
\end{center}
\end{frame}

% %%%%%%%%%%%%%%%%%%%%%%%%%%%%%%%%%%%%%%%%%%%%%%%%%%%%%%%%%%%%%%%%%%%%%%%%%%%%%%%%%%
% \begin{frame}[fragile]\frametitle{Concept of Mita Ahara}
    % \begin{itemize}
        % \item \textbf{Mita Ahara} refers to a moderate diet, as described in the \textit{Hatha Yoga Pradipika} by \textbf{स्वात्माराम} (Swatma Ram).
        % \item The concept involves eating food that is \textbf{moderate} in quantity and quality.
    % \end{itemize}
% \end{frame}

% %%%%%%%%%%%%%%%%%%%%%%%%%%%%%%%%%%%%%%%%%%%%%%%%%%%%%%%%%%%%%%%%%%%%%%%%%%%%%%%%%%
% \begin{frame}[fragile]\frametitle{Characteristics of Mita Ahara}
    % \begin{itemize}
        % \item \textbf{Rich in Natural Oils}: The food should be rich in natural oils and not dry.
        % \item \textbf{Naturally Sweet}: Food should be naturally sweet and enjoyable, not forced.
        % \item \textbf{Proper Quantity}: Food should be consumed in a way that it leaves one-fourth of the stomach empty.
    % \end{itemize}
% \end{frame}

% %%%%%%%%%%%%%%%%%%%%%%%%%%%%%%%%%%%%%%%%%%%%%%%%%%%%%%%%%%%%%%%%%%%%%%%%%%%%%%%%%%
% \begin{frame}[fragile]\frametitle{Definition}
    % \begin{quote}
        % \textit{सुस्निग्धमधुरं आहारं चतुर्थांशात् चतुर्थांशात् \\ 
        % विवर्ततस्नेहपरीतं मधुरं शीतं चतुर्थांशात्} \\ 
        % \textit{कर्षणं चात्र चतुर्थांशात् युक्ता आहारयुक्ता आहारः}
    % \end{quote}
    % \begin{itemize}
        % \item \textbf{Translation:}
        % \begin{itemize}
            % \item Food should be \textbf{rich and lubricating} (\textit{सुस्निग्ध}) with \textbf{natural oil}, 
            % \item \textbf{sweet} (\textit{मधुर}) and \textbf{cool} (\textit{शीत}) in nature.
            % \item It should be consumed in such a way that one-fourth of the stomach is left empty.
        % \end{itemize}
    % \end{itemize}
% \end{frame}


%%%%%%%%%%%%%%%%%%%%%%%%%%%%%%%%%%%%%%%%%%%%%%%%%%%%%%%%%%%%%%%%%%%%%%%%%%%%%%%%%%
\begin{frame}[fragile]\frametitle{Concept and Characteristics of Mita Ahara}

\begin{itemize}
    \item \textbf{Mita Ahara}: Moderate diet as per \textit{Hatha Yoga Pradipika} by \textbf{स्वात्माराम} (Swatma Ram); involves moderate quantity and quality.
    \item \textbf{Characteristics:}
        \begin{itemize}
            \item \textbf{Rich in Natural Oils}: Foods should be lubricating, not dry.
            \item \textbf{Naturally Sweet}: Enjoyable sweetness, not forced.
            \item \textbf{Proper Quantity}: Leave one-fourth of the stomach empty.
        \end{itemize}
    \item \textbf{Definition:}
        \begin{quote}
            \textit{सुस्निग्धमधुरं आहारं चतुर्थांशात् चतुर्थांशात \\ 
            विवर्ततस्नेहपरीतं मधुरं शीतं चतुर्थांशात्}
        \end{quote}
    \item \textbf{Translation:} Food should be rich, sweet, cool, and leave one-fourth of the stomach empty.
\end{itemize}

\end{frame}


% %%%%%%%%%%%%%%%%%%%%%%%%%%%%%%%%%%%%%%%%%%%%%%%%%%%%%%%%%%%%%%%%%%%%%%%%%%%%%%%%%%
% \begin{frame}[fragile]\frametitle{Eating as an Offering}
    % \begin{itemize}
        % \item Food should be eaten with full attention, almost as an offering or sacred act.
        % \item Chew food at least 32 times to aid digestion, which begins in the mouth.
        % \item Practice eating in silence to enhance mindfulness and appreciation of food.
    % \end{itemize}
% \end{frame}


% %%%%%%%%%%%%%%%%%%%%%%%%%%%%%%%%%%%%%%%%%%%%%%%%%%%%%%%%%%%%%%%%%%%%%%%%%%%%%%%%%%
% \begin{frame}[fragile]\frametitle{Foods to Avoid}
    % \begin{itemize}
        % \item \textbf{Bitter, Sour, and Acidic}: Includes \textbf{कटु} (Katuka), \textbf{अम्ल} (Amla), \textbf{तीक्ष्ण} (Tikshna), \textbf{लवण} (Lavana), and \textbf{उष्ण} (Ushna) foods.
        % \item \textbf{Fermented and Oily Foods}: Avoid fermented foods, oily foods, and those mixed with mustard seeds, fish, or meat of goats.
        % \item \textbf{Stale Foods}: Foods that are reheated, dry, excessively salty, or acidic.
    % \end{itemize}
% \end{frame}

% %%%%%%%%%%%%%%%%%%%%%%%%%%%%%%%%%%%%%%%%%%%%%%%%%%%%%%%%%%%%%%%%%%%%%%%%%%%%%%%%%%
% \begin{frame}[fragile]\frametitle{Prescribed Foods}
    % \begin{itemize}
        % \item \textbf{Mita Ahara}: Foods that are naturally sweet, rich in natural oils, and enjoyable to eat.
        % \item \textbf{Fresh Foods}: Includes fresh vegetables, grains like rice and barley, and other wholesome, nourishing foods.
    % \end{itemize}
% \end{frame}


% %%%%%%%%%%%%%%%%%%%%%%%%%%%%%%%%%%%%%%%%%%%%%%%%%%%%%%%%%%%%%%%%%%%%%%%%%%%%%%%%%%
% \begin{frame}[fragile]\frametitle{Nutritional Guidelines from Hatha Yoga Pradipika}
    % \begin{itemize}
        % \item Foods like fresh butter, clarified butter (\textbf{घृत} (Ghee)), honey, and certain grains are considered beneficial.
        % \item Five green vegetables are mentioned for their beneficial properties.
    % \end{itemize}
% \end{frame}

% %%%%%%%%%%%%%%%%%%%%%%%%%%%%%%%%%%%%%%%%%%%%%%%%%%%%%%%%%%%%%%%%%%%%%%%%%%%%%%%%%%
% \begin{frame}[fragile]\frametitle{Avoiding Harmful Foods}
    % \begin{itemize}
        % \item Foods that are excessively salty, sour, or mixed improperly.
        % \item Reheating and stale foods are considered detrimental.
    % \end{itemize}
% \end{frame}


% %%%%%%%%%%%%%%%%%%%%%%%%%%%%%%%%%%%%%%%%%%%%%%%%%%%%%%%%%%%%%%%%%%%%%%%%%%%%%%%%%%
% \begin{frame}[fragile]\frametitle{Concept of Yuktahara (Balanced Diet)}
    % \begin{itemize}
        % \item \textbf{Yuktahara} is the concept of balanced eating, as mentioned in the \textit{Bhagavad Gita}.
        % \item This involves moderation not only in eating but also in activities like recreation and sleep.
    % \end{itemize}
% \end{frame}

% %%%%%%%%%%%%%%%%%%%%%%%%%%%%%%%%%%%%%%%%%%%%%%%%%%%%%%%%%%%%%%%%%%%%%%%%%%%%%%%%%%
% \begin{frame}[fragile]\frametitle{Key Verse from Bhagavad Gita}
    % \begin{quote}
        % \textit{युक्ताहारविहारस्य युक्तचेतस्य कर्मसु \\
        % युक्तस्वप्नावबोधस्य योगो भवति दुखः}
    % \end{quote}
    % \begin{itemize}
        % \item Translation: Balanced eating, recreation, effort in work, and balanced sleep lead to the practice of yoga which mitigates sorrow.
    % \end{itemize}
% \end{frame}



% %%%%%%%%%%%%%%%%%%%%%%%%%%%%%%%%%%%%%%%%%%%%%%%%%%%%%%%%%%%
% \begin{frame}[fragile]\frametitle{Concepts of \textit{ahara} (आहार) in \textit{Hatha Pradipika} (हठप्रदीपिका)}

      % \begin{itemize}
		% \item \textit{ahara} (आहार) - Diet and its role in Yoga practice.
		% \item Emphasizes moderation and the impact of diet on physical and mental health.
		% \item Advocates for simple, pure, and balanced food.
		% \item Recommends avoidance of heavy, spicy, or overly processed foods.
		% \item Stresses the importance of regular and timely meals.
		% \item Highlights the role of diet in supporting physical strength and stamina for Yoga.
	  % \end{itemize}

% \end{frame}

% %%%%%%%%%%%%%%%%%%%%%%%%%%%%%%%%%%%%%%%%%%%%%%%%%%%%%%%%%%%
% \begin{frame}[fragile]\frametitle{Concepts of \textit{ahara} (आहार) in \textit{Bhagavad Gita} (भगवद् गीता)}

      % \begin{itemize}
		% \item \textit{Mitahara} (मिताहार) - Moderate eating; balanced and moderate in quantity.
		% \item Recommends a diet that is:
		  % \begin{itemize}
		      % \item \textit{Sattvic} (सात्त्विक) - Pure, clean, and nourishing.
		      % \item \textit{Rajasic} (राजसिक) - Overly stimulating, often leading to restlessness.
		      % \item \textit{Tamasic} (तामसिक) - Stale, impure, and harmful.
		  % \end{itemize}
		% \item Emphasizes the impact of food on the mind and consciousness.
		% \item Advocates for moderation and awareness in eating habits.
		% \item Suggests that the right diet supports spiritual and physical well-being.
	  % \end{itemize}

% \end{frame}


% %%%%%%%%%%%%%%%%%%%%%%%%%%%%%%%%%%%%%%%%%%%%%%%%%%%%%%%%%%%
% \begin{frame}[fragile]\frametitle{Principles of \textit{Mitahara} (मिताहार) in \textit{Bhagavad Gita} (भगवद् गीता)}

      % \begin{itemize}
		% \item \textit{Mitahara} (मिताहार) - Eating in moderation and balance.
		% \item Consumes food that is:
		  % \begin{itemize}
		      % \item Fresh and wholesome.
		      % \item Prepared with love and devotion.
		      % \item Conducive to physical health and mental clarity.
		  % \end{itemize}
		% \item Avoids excessive or insufficient eating.
		% \item Focuses on maintaining harmony between body and mind through diet.
		% \item Supports overall spiritual and physical health.
	  % \end{itemize}

% \end{frame}

% %%%%%%%%%%%%%%%%%%%%%%%%%%%%%%%%%%%%%%%%%%%%%%%%%%%%%%%%%%%
% \begin{frame}[fragile]\frametitle{Principles of \textit{Yuktahara} (युक्ताहार) in \textit{Bhagavad Gita} (भगवद् गीता)}

      % \begin{itemize}
		% \item \textit{Yuktahara} (युक्ताहार) - Proper and disciplined eating.
		% \item Involves:
		  % \begin{itemize}
		      % \item Consuming food at appropriate times.
		      % \item Eating in moderation, neither too much nor too little.
		      % \item Aligning diet with one's physical and spiritual needs.
		  % \end{itemize}
		% \item Emphasizes mindful eating and awareness of the impact of food.
		% \item Integrates diet with other aspects of Yoga practice for holistic well-being.
		% \item Enhances both physical health and spiritual development.
	  % \end{itemize}

% \end{frame}

%%%%%%%%%%%%%%%%%%%%%%%%%%%%%%%%%%%%%%%%%%%%%%%%%%%%%%%%%%%%%%%%%%%%%%%%%%%%%%%%%%
\begin{frame}[fragile]\frametitle{Eating as an Offering and Food Guidelines}

\begin{itemize}
    \item Eating as an Offering: 
        \begin{itemize}
            \item Eat with attention as a sacred act; chew food at least 32 times for digestion.
            \item Practice silence during meals for mindfulness and appreciation.
        \end{itemize}
    \item Foods to Avoid:
        \begin{itemize}
            \item \textbf{Bitter, Sour, and Acidic}: Avoid \textbf{कटु} (Katuka), \textbf{अम्ल} (Amla), \textbf{तीक्ष्ण} (Tikshna) foods.
            \item \textbf{Fermented and Oily Foods}: Avoid fermented, oily foods, and those mixed with strong flavors or meat.
            \item \textbf{Stale Foods}: Avoid reheated, dry, and excessively salty foods.
        \end{itemize}
    \item Prescribed Foods:
        \begin{itemize}
            \item \textbf{Mita Ahara}: Foods that are naturally sweet, rich in oils, and fresh.
            \item Include beneficial foods like fresh butter, ghee, and green vegetables.
        \end{itemize}
\end{itemize}
\begin{itemize}
    \item \textbf{References}:
        \begin{itemize}
            \item Hatha Yoga Pradipika: Guidelines on dietary practices.
            \item Bhagavad Gita (3.16): Discusses the importance of diet in maintaining harmony.
        \end{itemize}
\end{itemize}

\end{frame}

%%%%%%%%%%%%%%%%%%%%%%%%%%%%%%%%%%%%%%%%%%%%%%%%%%%%%%%%%%%%%%%%%%%%%%%%%%%%%%%%%%
\begin{frame}[fragile]\frametitle{Concepts of Ahara in Hatha Yoga and Bhagavad Gita}

\begin{itemize}
    \item Concept of Ahara:
        \begin{itemize}
            \item Role of diet in Yoga: emphasizes moderation, simplicity, and regularity.
            \item Avoid heavy, spicy, or processed foods for optimal physical and mental health.
        \end{itemize}
    \item Yuktahara (Balanced Diet):
        \begin{itemize}
            \item Balanced eating involves moderation in diet and activities like sleep.
            \item \textit{Key Verse}: \textit{युक्ताहारविहारस्य ... योगो भवति दुखः} (Bhagavad Gita 6.16-17) - Balanced lifestyle leads to yoga that mitigates sorrow.
        \end{itemize}
    \item Mitahara (Moderate Eating):
        \begin{itemize}
            \item Consists of pure, wholesome foods that support health and clarity.
            \item Advocates for mindful eating that aligns with spiritual well-being.
        \end{itemize}
\end{itemize}
\begin{itemize}
    \item \textbf{References}:
        \begin{itemize}
            \item Hatha Yoga Pradipika: Emphasizes the significance of a balanced diet.
            \item Bhagavad Gita (17.7-10): Discusses the types of foods conducive to spiritual and physical health.
        \end{itemize}
\end{itemize}

\end{frame}

%%%%%%%%%%%%%%%%%%%%%%%%%%%%%%%%%%%%%%%%%%%%%%%%%%%%%%%%%%%%%%%%%%%%%%%%%%%%%%%%%%
\begin{frame}[fragile]\frametitle{}
\begin{center}
{\Large 2.6  Significance of \textit{Hatha Yoga} (हठ योग) Practices in Health and Well-Being}
\end{center}
\end{frame}

% %%%%%%%%%%%%%%%%%%%%%%%%%%%%%%%%%%%%%%%%%%%%%%%%%%%%%%%%%%%
% \begin{frame}[fragile]\frametitle{Significance of \textit{Hatha Yoga} (हठ योग) Practices}

      % \begin{itemize}
		% \item \textit{Hatha Yoga} (हठ योग) - A system focusing on physical postures (\textit{Asanas} - आसन), breath control (\textit{Pranayam} - प्राणायाम), and meditation (\textit{Dhyana} - ध्यान).
		% \item Aims to balance physical, mental, and spiritual health.
		% \item Includes practices like \textit{Asanas} (postures - आसन), \textit{Pranayam} (breath control - प्राणायाम), \textit{Mudras} (hand gestures - मुद्रा), and \textit{Bandhas} (body locks - बन्धा).
		% \item Enhances physical strength, flexibility, and endurance.
		% \item Supports mental clarity, stress reduction, and emotional stability.
	  % \end{itemize}

% \end{frame}

% %%%%%%%%%%%%%%%%%%%%%%%%%%%%%%%%%%%%%%%%%%%%%%%%%%%%%%%%%%%
% \begin{frame}[fragile]\frametitle{Physical Health Benefits}

      % \begin{itemize}
		% \item Improves flexibility and muscle tone.
		% \item Enhances strength and stamina.
		% \item Aids in weight management and improves posture.
		% \item Boosts circulation and respiratory function.
		% \item Promotes detoxification through sweating and improved digestion.
	  % \end{itemize}

% \end{frame}

% %%%%%%%%%%%%%%%%%%%%%%%%%%%%%%%%%%%%%%%%%%%%%%%%%%%%%%%%%%%
% \begin{frame}[fragile]\frametitle{Mental and Emotional Well-being}

      % \begin{itemize}
		% \item Reduces stress and anxiety through relaxation techniques.
		% \item Enhances focus and concentration.
		% \item Improves mood and emotional resilience.
		% \item Supports mental clarity and cognitive function.
		% \item Encourages mindfulness and self-awareness.
	  % \end{itemize}

% \end{frame}

% %%%%%%%%%%%%%%%%%%%%%%%%%%%%%%%%%%%%%%%%%%%%%%%%%%%%%%%%%%%
% \begin{frame}[fragile]\frametitle{Spiritual Growth}

      % \begin{itemize}
		% \item Facilitates deeper meditation and self-realization.
		% \item Helps in achieving inner peace and balance.
		% \item Promotes a sense of connection to self and higher consciousness.
		% \item Supports spiritual development through disciplined practice.
		% \item Integrates physical health with spiritual practice for holistic growth.
	  % \end{itemize}

% \end{frame}

% %%%%%%%%%%%%%%%%%%%%%%%%%%%%%%%%%%%%%%%%%%%%%%%%%%%%%%%%%%%
% \begin{frame}[fragile]\frametitle{Overall Well-being}

      % \begin{itemize}
		% \item Combines physical, mental, and spiritual practices for comprehensive health.
		% \item Encourages a balanced lifestyle and regular practice.
		% \item Provides tools for managing daily stress and enhancing quality of life.
		% \item Fosters a harmonious relationship between body, mind, and spirit.
		% \item Contributes to long-term health and vitality.
	  % \end{itemize}

% \end{frame}


%%%%%%%%%%%%%%%%%%%%%%%%%%%%%%%%%%%%%%%%%%%%%%%%%%%%%%%%%%%
\begin{frame}[fragile]\frametitle{Significance of \textit{Hatha Yoga} (हठ योग) Practices}

\begin{itemize}
    \item Overview: Focuses on physical postures (\textit{Asanas} - आसन), breath control (\textit{Pranayam} - प्राणायाम), and meditation (\textit{Dhyana} - ध्यान) for balanced health.
    \item Physical Benefits: Improves flexibility, strength, stamina, posture, circulation, respiratory function, and detoxification.
    \item Mental Well-being: Reduces stress and anxiety, enhances focus, mood, cognitive function, and promotes mindfulness.
    \item Spiritual Growth: Facilitates meditation, inner peace, self-realization, and holistic integration of physical and spiritual practices.
    \item Overall Well-being: Combines physical, mental, and spiritual practices for comprehensive health and long-term vitality.
\end{itemize}

\end{frame}

%%%%%%%%%%%%%%%%%%%%%%%%%%%%%%%%%%%%%%%%%%%%%%%%%%%%%%%%%%%%%%%%%%%%%%%%%%%%%%%%%%
\begin{frame}[fragile]\frametitle{}
\begin{center}
{\Large 2.7  Concept of Mental Well-being according to \textit{Patanjali Yoga} (पातञ्जलि योग)}
\end{center}
\end{frame}

% %%%%%%%%%%%%%%%%%%%%%%%%%%%%%%%%%%%%%%%%%%%%%%%%%%%%%%%%%%%
% \begin{frame}[fragile]\frametitle{Concept of Mental Well-being in \textit{Patanjali Yoga} (पातञ्जलि योग)}

      % \begin{itemize}
		% \item \textit{Patanjali's Yoga} (पातञ्जलि योग) - Focuses on achieving mental clarity and stability.
		% \item Central text: \textit{Yoga Sutras of Patanjali} (पातञ्जलि योग सूत्र).
		% \item Emphasizes the importance of controlling the mind (\textit{Chitta} - चित्त) for well-being.
		% \item Aims to cultivate a state of \textit{Sattva} (सत्त्व) - mental purity and balance.
		% \item Addresses mental disturbances and provides techniques to overcome them.
	  % \end{itemize}

% \end{frame}

% %%%%%%%%%%%%%%%%%%%%%%%%%%%%%%%%%%%%%%%%%%%%%%%%%%%%%%%%%%%
% \begin{frame}[fragile]\frametitle{Role of Mind Control (\textit{Chitta Vritti Nirodha} - चित्त वृत्ति निरोध)}

      % \begin{itemize}
		% \item \textit{Chitta Vritti Nirodha} (चित्त वृत्ति निरोध) - Control of the fluctuations of the mind.
		% \item Essential for achieving mental stability and peace.
		% \item Involves restraining mental patterns and disturbances.
		% \item Focuses on reducing \textit{Vrittis} (वृत्ति) - mental modifications that cause suffering.
		% \item Achieved through practice of \textit{Yamas} (यमा:) - ethical restraints and \textit{Niyamas} (नियमा:) - personal observances.
	  % \end{itemize}

% \end{frame}

% %%%%%%%%%%%%%%%%%%%%%%%%%%%%%%%%%%%%%%%%%%%%%%%%%%%%%%%%%%%
% \begin{frame}[fragile]\frametitle{Key Practices for Mental Well-being}

      % \begin{itemize}
		% \item \textit{Dhyana} (ध्यान) - Meditation - Regular practice to cultivate concentration and inner peace.
		% \item \textit{Pranayama} (प्राणायाम) - Breath Control - Regulates mental and emotional states through breath.
		% \item \textit{Asanas} (आसन) - Postures - Physical practice to stabilize the mind and body.
		% \item \textit{Self-Discipline} - Adherence to \textit{Yamas} (यमा:) and \textit{Niyamas} (नियमा:) for mental clarity.
		% \item \textit{Mindfulness} - Awareness of thoughts and emotions to maintain balance.
	  % \end{itemize}

% \end{frame}

% %%%%%%%%%%%%%%%%%%%%%%%%%%%%%%%%%%%%%%%%%%%%%%%%%%%%%%%%%%%
% \begin{frame}[fragile]\frametitle{Achieving Mental Clarity (\textit{Sattva} - सत्त्व)}

      % \begin{itemize}
		% \item \textit{Sattva} (सत्त्व) - The quality of purity and harmony in the mind.
		% \item Promotes inner peace, clarity, and wisdom.
		% \item Cultivated through regular practice of Yoga and meditation.
		% \item Helps in overcoming mental disturbances and achieving higher states of consciousness.
		% \item Supports overall mental and emotional stability.
	  % \end{itemize}

% \end{frame}

% %%%%%%%%%%%%%%%%%%%%%%%%%%%%%%%%%%%%%%%%%%%%%%%%%%%%%%%%%%%
% \begin{frame}[fragile]\frametitle{Overcoming Mental Disturbances}

      % \begin{itemize}
		% \item \textit{Kleshas} (क्लेशा:) - Mental afflictions such as ignorance, egoism, attachment, aversion, and fear of death.
		% \item Addressed through disciplined practice and self-awareness.
		% \item Use of \textit{Vairagya} (वैराग्य) - detachment and \textit{Abhyasa} (अभ्यास) - practice to manage mental challenges.
		% \item Achieving \textit{Samadhi} (समाधि) - A state of perfect mental equilibrium.
		% \item Focus on reducing negative thought patterns and promoting mental resilience.
	  % \end{itemize}

% \end{frame}


% %%%%%%%%%%%%%%%%%%%%%%%%%%%%%%%%%%%%%%%%%%%%%%%%%%%%%%%%%%%
% \begin{frame}[fragile]\frametitle{प्रतिपक्ष भावना  Pratipaksha Bhavana (Thinking of the Opposite)}
    % \begin{itemize}
        % \item \textbf{प्रतिपक्ष भावना  Pratipaksha Bhavana}: Consciously replacing negative emotions with positive ones.
        % \item For example, when experiencing anger or hatred, consciously think of something that brings joy or love.
        % \item Swami Vivekananda’s example: A couple fighting will instantly forget their anger when a child does something funny.
    % \end{itemize}
% \end{frame}

% %%%%%%%%%%%%%%%%%%%%%%%%%%%%%%%%%%%%%%%%%%%%%%%%%%%%%%%%%%%
% \begin{frame}[fragile]\frametitle{चित्त प्रसादन  Attitude of Chitta Prasadana}
    % \begin{itemize}
        % \item \textbf{चित्त प्रसादन  Chitta Prasadana}: Cultivating attitudes towards different types of people.
        % \begin{itemize}
            % \item \textbf{Maitri} (\textbf{मैत्री}): Friendship with happy people.
            % \item \textbf{Karuna} (\textbf{करुणा}): Compassion towards unhappy people.
            % \item \textbf{Mudita} (\textbf{मुदिता}): Joy for virtuous people.
            % \item \textbf{Upeksha} (\textbf{उपेक्षा}): Indifference towards negative or evil people.
        % \end{itemize}
    % \end{itemize}
% \end{frame}

%%%%%%%%%%%%%%%%%%%%%%%%%%%%%%%%%%%%%%%%%%%%%%%%%%%%%%%%%%%
\begin{frame}[fragile]\frametitle{Concept of Mental Well-being in Patanjali Yoga (पातञ्जलि योग)}

      \begin{itemize}
		\item Focuses on achieving mental clarity and stability.
		\item Central text: Yoga Sutras of Patanjali (पातञ्जलि योग सूत्र).
		\item Importance of controlling the mind (Chitta - चित्त) for well-being.
		\item Cultivation of Sattva (सत्त्व) - mental purity and balance.
		\item Techniques to overcome mental disturbances.
		\item Sattva (सत्त्व) - Quality of purity, harmony, and clarity.
		\item Overcoming Kleshas (क्लेशा:) - Mental afflictions through disciplined practice.
		\item प्रतिपक्ष भावना Pratipaksha Bhavana - Replacing negative emotions with positive ones.		
	  \end{itemize}

\end{frame}

%%%%%%%%%%%%%%%%%%%%%%%%%%%%%%%%%%%%%%%%%%%%%%%%%%%%%%%%%%%
\begin{frame}[fragile]\frametitle{Key Practices and Concepts for Mental Well-being}

      \begin{itemize}
		\item Chitta Vritti Nirodha (चित्त वृत्ति निरोध) - Control of mental fluctuations essential for stability.
		\item Key Practices: 
		\begin{itemize}
			\item Dhyana (ध्यान) - Meditation for concentration and inner peace.
			\item Pranayama (प्राणायाम) - Breath control for emotional regulation.
			\item Asanas (आसन) - Stabilizes mind and body.
			\item Self-Discipline - Adherence to Yamas (यमा:) and Niyamas (नियमा:).
			\item Mindfulness - Awareness of thoughts and emotions.
		\end{itemize}
        \item \textbf{चित्त प्रसादन  Chitta Prasadana}: Cultivating attitudes towards different types of people.
        \begin{itemize}
            \item \textbf{Maitri} (\textbf{मैत्री}): Friendship with happy people.
            \item \textbf{Karuna} (\textbf{करुणा}): Compassion towards unhappy people.
            \item \textbf{Mudita} (\textbf{मुदिता}): Joy for virtuous people.
            \item \textbf{Upeksha} (\textbf{उपेक्षा}): Indifference towards negative or evil people.
        \end{itemize}
      \end{itemize}

\end{frame}


%%%%%%%%%%%%%%%%%%%%%%%%%%%%%%%%%%%%%%%%%%%%%%%%%%%%%%%%%%%%%%%%%%%%%%%%%%%%%%%%%%
\begin{frame}[fragile]\frametitle{}
\begin{center}
{\Large 2.8  Yogic Practices of \textit{Patanjali Yoga} (पातञ्जलि योग): \textit{Bahiranga} (बाहिरंग) and \textit{Antaranga} (अंतरंग) Yoga}
\end{center}
\end{frame}

% %%%%%%%%%%%%%%%%%%%%%%%%%%%%%%%%%%%%%%%%%%%%%%%%%%%%%%%%%%%
% \begin{frame}[fragile]\frametitle{Yogic Practices in \textit{Patanjali Yoga} (पातञ्जलि योग)}

      % \begin{itemize}
		% \item \textit{Patanjali’s Yoga} (पातञ्जलि योग) - Divided into \textbf{Bahiranga} (बाहिरंग) - external and \textbf{Antaranga} (अंतरंग) - internal practices.
		% \item Aims to achieve holistic development and spiritual realization.
		% \item \textbf{Bahiranga} (बाहिरंग) Yoga focuses on external practices.
		% \item \textbf{Antaranga} (अंतरंग) Yoga emphasizes internal, meditative practices.
		% \item Both are essential for achieving the ultimate goal of Yoga.
	  % \end{itemize}

% \end{frame}


% %%%%%%%%%%%%%%%%%%%%%%%%%%%%%%%%%%%%%%%%%%%%%%%%%%%%%%%%%%%
% \begin{frame}[fragile]\frametitle{Bahiranga Yoga}

      % \begin{itemize}
		% \item \textbf{Bahiranga} - External practices of Yoga.
		% \item Includes:
		  % \begin{itemize}
		      % \item \textit{Yamas} (यम) - Ethical restraints (non-violence, truthfulness, non-stealing, etc.).
		      % \item \textit{Niyamas} (नियम) - Personal observances (cleanliness, contentment, self-discipline, etc.).
		      % \item \textit{Asanas} (आसन) - Physical postures to prepare the body for meditation.
		      % \item \textit{Pranayama} (प्राणायाम) - Breath control to regulate vital energy and calm the mind.
		  % \end{itemize}
		% \item Focuses on ethical and physical preparation for deeper practices.
		% \item Establishes a foundation for internal practices.
	  % \end{itemize}

% \end{frame}

% %%%%%%%%%%%%%%%%%%%%%%%%%%%%%%%%%%%%%%%%%%%%%%%%%%%%%%%%%%%
% \begin{frame}[fragile]\frametitle{Antaranga Yoga}

      % \begin{itemize}
		% \item \textbf{Antaranga} - Internal practices of Yoga.
		% \item Includes:
		  % \begin{itemize}
		      % \item \textit{Pratyahara} (प्रत्याहार) - Withdrawal of the senses from external objects.
		      % \item \textit{Dharana} (धारणा) - Concentration on a single point or object.
		      % \item \textit{Dhyana} (ध्यान) - Meditation; sustained and uninterrupted flow of consciousness.
		      % \item \textit{Samadhi} (समाधि) - Enlightenment; a state of profound inner peace and realization.
		  % \end{itemize}
		% \item Focuses on deepening the inner experience and achieving spiritual insight.
		% \item Cultivates mental clarity, inner peace, and ultimate realization of the self.
	  % \end{itemize}

% \end{frame}

% %%%%%%%%%%%%%%%%%%%%%%%%%%%%%%%%%%%%%%%%%%%%%%%%%%%%%%%%%%%
% \begin{frame}[fragile]\frametitle{Integration of Bahiranga and Antaranga Yoga}

      % \begin{itemize}
		% \item \textbf{Bahiranga} and \textbf{Antaranga} practices are interdependent.
		% \item External practices prepare and purify the body and mind for internal practices.
		% \item Internal practices build on the discipline established by external practices.
		% \item Both are necessary for comprehensive development and achieving Yoga’s ultimate goals.
		% \item Harmonizing both aspects leads to a balanced and integrated approach to Yoga.
	  % \end{itemize}

% \end{frame}

%%%%%%%%%%%%%%%%%%%%%%%%%%%%%%%%%%%%%%%%%%%%%%%%%%%%%%%%%%%
\begin{frame}[fragile]\frametitle{Yogic Practices in Patanjali Yoga (पातञ्जलि योग)}

      \begin{itemize}
		\item Patanjali’s Yoga (पातञ्जलि योग) - Divided into Bahiranga (बाहिरंग) - external and Antaranga (अंतरंग) - internal practices.
		\item Aims for holistic development and spiritual realization.
		\item Bahiranga Yoga focuses on external practices:
		  \begin{itemize}
		      \item Yamas (यम) - Ethical restraints (e.g., non-violence, truthfulness).
		      \item Niyamas (नियम) - Personal observances (e.g., cleanliness, contentment).
		      \item Asanas (आसन) - Physical postures to prepare for meditation.
		      \item Pranayama (प्राणायाम) - Breath control to regulate energy and calm the mind.
		  \end{itemize}
		\item Antaranga Yoga emphasizes internal practices:
		  \begin{itemize}
		      \item Pratyahara (प्रत्याहार) - Withdrawal of senses.
		      \item Dharana (धारणा) - Concentration on a single point.
		      \item Dhyana (ध्यान) - Meditation; sustained consciousness.
		      \item Samadhi (समाधि) - Enlightenment; profound inner peace.
		  \end{itemize}
		\item Bahiranga and Antaranga practices are interdependent, establishing a balanced approach to Yoga.
	  \end{itemize}

\end{frame}


%%%%%%%%%%%%%%%%%%%%%%%%%%%%%%%%%%%%%%%%%%%%%%%%%%%%%%%%%%%%%%%%%%%%%%%%%%%%%%%%%%
\begin{frame}[fragile]\frametitle{}
\begin{center}
{\Large 2.9  Concepts of healthy living in Bhagwad Gita}
\end{center}
\end{frame}

%%%%%%%%%%%%%%%%%%%%%%%%%%%%%%%%%%%%%%%%%%%%%%%%%%%%%%%%%%%%%%%%%%%%%%%%%%%%%%%%%%
\begin{frame}[fragile]\frametitle{Introduction}
    \begin{itemize}
        \item In Chapter 16 of the Bhagavad Gita, Krishna describes 26 divine qualities that contribute to a happy and healthy life in society.
        \item These qualities are essential for living harmoniously and practicing spirituality.
    \end{itemize}
    \begin{quote}
        \textbf{अभयम् सत्त्वसंशुद्धिः ज्ञानयोगवस्तितः |}\\
        \textbf{दानं दमश्च यज्ञश्च स्वाध्यायस्तप आर्जवम् ||}
    \end{quote}
    \begin{itemize}
        \item Translation: Fearlessness, purity of mind, steadfastness in spiritual knowledge, charity, control of the senses, performance of sacrifice, study of sacred books, austerity, and straightforwardness.
    \end{itemize}
\end{frame}

%%%%%%%%%%%%%%%%%%%%%%%%%%%%%%%%%%%%%%%%%%%%%%%%%%%%%%%%%%%%%%%%%%%%%%%%%%%%%%%%%%
\begin{frame}[fragile]\frametitle{Qualities of a Divine Nature}
    \begin{itemize}
        \item Fearlessness (\textbf{अभय})
        \item Purity of Mind (\textbf{शुद्धता})
        \item Steadfastness in Spiritual Knowledge (\textbf{स्थैर्य})
        \item Charity (\textbf{दान})
        \item Control of the Senses (\textbf{इंद्रिय नियंत्रण})
        \item Performance of Sacrifice (\textbf{यज्ञ})
        \item Study of Sacred Books (\textbf{स्वाध्याय})
        \item Austerity (\textbf{तप})
        \item Straightforwardness (\textbf{आर्जव})
        \item Nonviolence (\textbf{अहिंसा})
        \item Truthfulness (\textbf{सत्य})
        \item Absence of Anger (\textbf{क्रोध रहित})
        \item Renunciation (\textbf{त्याग})
        \item Peacefulness (\textbf{शांति})
    \end{itemize}
\end{frame}

%%%%%%%%%%%%%%%%%%%%%%%%%%%%%%%%%%%%%%%%%%%%%%%%%%%%%%%%%%%%%%%%%%%%%%%%%%%%%%%%%%
\begin{frame}[fragile]\frametitle{Qualities of a Divine Nature}
    \begin{itemize}
        \item Peacefulness (\textbf{शांति})
        \item Restraint from Fault-Finding (\textbf{दोष निंदानिरोध})
        \item Compassion Towards All Living Beings (\textbf{करुणा})
        \item Absence of Covetousness (\textbf{लोभ रहित})
        \item Gentleness (\textbf{मृदुता})
        \item Modesty (\textbf{लज्जा})
        \item Lack of Fickleness (\textbf{अस्थिरता})
        \item Vigor (\textbf{स्फूर्ति})
        \item Forgiveness (\textbf{क्षमा})
        \item Fortitude (\textbf{धैर्य})
        \item Cleanliness (\textbf{शौच})
        \item Bearing Enmity Towards None (\textbf{द्वेष रहित})
        \item Absence of Vanity (\textbf{अहंकार रहित})
    \end{itemize}
\end{frame}



% %%%%%%%%%%%%%%%%%%%%%%%%%%%%%%%%%%%%%%%%%%%%%%%%%%%%%%%%%%%
% \begin{frame}[fragile]\frametitle{Concepts of Healthy Living in Bhagavad Gita}

      % \begin{itemize}
		% \item \textbf{Bhagavad Gita} - Provides guidance on living a balanced and healthy life.
		% \item Emphasizes the harmony of body, mind, and spirit.
		% \item Encourages living in accordance with \textbf{धर्म} (Dharma) and righteousness.
		% \item Focuses on maintaining balance in daily activities and lifestyle.
		% \item Highlights the importance of self-discipline and moderation.
	  % \end{itemize}

% \end{frame}

% %%%%%%%%%%%%%%%%%%%%%%%%%%%%%%%%%%%%%%%%%%%%%%%%%%%%%%%%%%%
% \begin{frame}[fragile]\frametitle{Principles of Healthy Living}

      % \begin{itemize}
		% \item \textbf{Moderation in Eating} - \textit{मिताहार} (Mitahara): Eat in moderation; balanced diet and mindful eating.
		% \item \textbf{Balanced Lifestyle} - Maintain a balanced routine; avoid extremes in work and rest.
		% \item \textbf{Regular Practice} - Engage in daily practice of Yoga, meditation, or self-discipline.
		% \item \textbf{Right Action} - Perform actions in line with one’s duty and ethical principles.
		% \item \textbf{Mental Peace} - Cultivate a peaceful and steady mind through mindfulness and self-awareness.
	  % \end{itemize}

% \end{frame}

% %%%%%%%%%%%%%%%%%%%%%%%%%%%%%%%%%%%%%%%%%%%%%%%%%%%%%%%%%%%
% \begin{frame}[fragile]\frametitle{Concepts of Diet and Behavior}

      % \begin{itemize}
		% \item \textbf{Healthy Diet} - Food that is:
		  % \begin{itemize}
		      % \item Fresh and wholesome.
		      % \item Prepared with care and respect.
		      % \item Not overly spicy or heavy.
		  % \end{itemize}
		% \item \textbf{Behavior and Attitude} - Approach life with:
		  % \begin{itemize}
		      % \item \textit{संतोष} (Santosha) - Contentment.
		      % \item \textit{वैराग्य} (Vairagya) - Non-attachment.
		      % \item Equanimity in success and failure.
		  % \end{itemize}
		% \item \textbf{Self-Control} - Exercise self-control over desires and impulses.
		% \item \textbf{Spiritual Focus} - Align daily actions with spiritual growth and self-realization.
	  % \end{itemize}

% \end{frame}

% %%%%%%%%%%%%%%%%%%%%%%%%%%%%%%%%%%%%%%%%%%%%%%%%%%%%%%%%%%%
% \begin{frame}[fragile]\frametitle{Living in Harmony with Nature}

      % \begin{itemize}
		% \item \textbf{Natural Rhythm} - Live in harmony with natural cycles and rhythms.
		% \item \textbf{Sattvic Living} - Adopt a lifestyle that promotes purity and tranquility (\textit{सात्त्विक}).
		% \item \textbf{Avoid Excesses} - Avoid excessive indulgence and self-denial.
		% \item \textbf{Holistic Approach} - Integrate physical health, mental peace, and spiritual well-being.
		% \item \textbf{Mindful Living} - Practice mindfulness in all aspects of life.
	  % \end{itemize}

% \end{frame}

%%%%%%%%%%%%%%%%%%%%%%%%%%%%%%%%%%%%%%%%%%%%%%%%%%%%%%%%%%%
\begin{frame}[fragile]\frametitle{Concepts of Healthy Living in Bhagavad Gita}

    \begin{itemize}
        \item \textbf{Guidance:} Balanced life; harmony of body, mind, spirit; align with धर्म (Dharma).
        \item \textbf{Key Principles:}
          \begin{itemize}
              \item \textbf{Moderation (मिताहार):} Balanced diet and mindful eating.
              \item \textbf{Balanced Lifestyle:} Avoid extremes in work and rest.
              \item \textbf{Regular Practice:} Daily Yoga and meditation.
              \item \textbf{Right Action:} Act according to duty and ethics.
              \item \textbf{Mental Peace:} Cultivate mindfulness and self-awareness.
          \end{itemize}
        \item \textbf{Diet and Behavior:}
          \begin{itemize}
              \item \textbf{Healthy Diet:} Fresh, wholesome, balanced.
              \item \textbf{Attitude:} Live with संतोष (Santosha) - contentment, वैराग्य (Vairagya) - non-attachment.
              \item \textbf{Self-Control:} Manage desires.
              \item \textbf{Spiritual Focus:} Align actions with growth.
          \end{itemize}
        \item \textbf{Harmony with Nature:}
          \begin{itemize}
              \item \textbf{Natural Rhythm:} Sync with cycles.
              \item \textbf{Sattvic Living:} Promote purity and tranquility.
              \item \textbf{Avoid Excesses:} Balance indulgence and self-denial.
              \item \textbf{Holistic Approach:} Integrate health, peace, and spirituality.
              \item \textbf{Mindful Living:} Practice mindfulness in all aspects.
          \end{itemize}
    \end{itemize}

\end{frame}



%%%%%%%%%%%%%%%%%%%%%%%%%%%%%%%%%%%%%%%%%%%%%%%%%%%%%%%%%%%%%%%%%%%%%%%%%%%%%%%%%%
\begin{frame}[fragile]\frametitle{}
\begin{center}
{\Large 2.10  Importance of subjective experience in daily Yoga practice}
\end{center}
\end{frame}

% %%%%%%%%%%%%%%%%%%%%%%%%%%%%%%%%%%%%%%%%%%%%%%%%%%%%%%%%%%%%%%%%%%%%%%%%%%%%%%%%%%
% \begin{frame}[fragile]\frametitle{Importance of Subjective Experiences in Daily Yoga Practice}
    % \begin{itemize}
        % \item \textbf{Self-Discipline:} Practicing yoga regularly builds self-discipline. For instance, morning practice helps in waking up early, yielding double benefits.
        % \item \textbf{Physical Harmony:} Asanas bring harmony to the body, ensuring it is strong and disease-free.
        % \item \textbf{Mental Balance:} Pranayama practice fosters calmness and balance in both body and mind.
        % \item \textbf{Self-Study (स्वाध्याय):} Enhances perspective by reading various commentaries and translations of yogic texts.
        % \item \textbf{Attitude of Surrender (ईश्वरप्रणिधान):} Develops an attitude of non-attachment to actions and their results.
        % \item \textbf{Intense Love Towards Knowledge (भक्ति योग):} Cultivates intense love for seeking knowledge.
        % \item \textbf{Eka Tatva Abhyasa (एकतत्त्व अभ्यास):} Focusing on a single principle or concept to deepen practice and understanding.
        % \item \textbf{Benefits:} Helps in developing a deeper connection with the chosen principle, leading to greater clarity and insight.
        % \item \textbf{Patanjali's Guidance:} Emphasizes consistent and dedicated practice (अभ्यास, \textit{Abhyasa}) to achieve mental balance and spiritual growth.
    % \end{itemize}
% \end{frame}

% %%%%%%%%%%%%%%%%%%%%%%%%%%%%%%%%%%%%%%%%%%%%%%%%%%%%%%%%%%%
% \begin{frame}[fragile]\frametitle{Importance of Subjective Experience in Daily Yoga Practice}

      % \begin{itemize}
		% \item \textbf{Subjective Experience} - Personal, internal perception of Yoga practice.
		% \item Focuses on individual feelings, sensations, and inner states.
		% \item Essential for understanding the impact of practice on body and mind.
		% \item Promotes self-awareness and deeper connection to one’s practice.
		% \item Enhances the effectiveness and benefits of Yoga practice.
	  % \end{itemize}

% \end{frame}

% %%%%%%%%%%%%%%%%%%%%%%%%%%%%%%%%%%%%%%%%%%%%%%%%%%%%%%%%%%%
% \begin{frame}[fragile]\frametitle{Self-Awareness and Mindfulness}

      % \begin{itemize}
		% \item \textbf{Self-Awareness} - Being conscious of physical and mental states during practice.
		% \item Encourages observation of subtle changes and progress.
		% \item \textbf{Mindfulness} - Paying attention to the present moment.
		% \item Enhances focus and concentration in practice.
		% \item Supports mental clarity and emotional stability.
	  % \end{itemize}

% \end{frame}

% %%%%%%%%%%%%%%%%%%%%%%%%%%%%%%%%%%%%%%%%%%%%%%%%%%%%%%%%%%%
% \begin{frame}[fragile]\frametitle{Personalization of Practice}

      % \begin{itemize}
		% \item \textbf{Personalization} - Adapting Yoga practices to individual needs and conditions.
		% \item Allows for modifications based on subjective experience and feedback.
		% \item Facilitates alignment with personal goals and limitations.
		% \item Enhances comfort and effectiveness of the practice.
		% \item Supports long-term adherence and progress in Yoga practice.
	  % \end{itemize}

% \end{frame}

% %%%%%%%%%%%%%%%%%%%%%%%%%%%%%%%%%%%%%%%%%%%%%%%%%%%%%%%%%%%
% \begin{frame}[fragile]\frametitle{Integration of Mind and Body}

      % \begin{itemize}
		% \item \textbf{Mind-Body Connection} - Awareness of the interplay between mental and physical aspects.
		% \item Helps in achieving balance and harmony.
		% \item Enhances the holistic benefits of Yoga.
		% \item Supports deeper meditative and reflective states.
		% \item Facilitates a more profound understanding of oneself.
	  % \end{itemize}

% \end{frame}

% %%%%%%%%%%%%%%%%%%%%%%%%%%%%%%%%%%%%%%%%%%%%%%%%%%%%%%%%%%%
% \begin{frame}[fragile]\frametitle{Reflective Practice and Growth}

      % \begin{itemize}
		% \item \textbf{Reflective Practice} - Regularly reviewing personal experiences and progress.
		% \item Encourages continuous learning and improvement.
		% \item Aids in identifying areas for growth and development.
		% \item Supports emotional and spiritual evolution.
		% \item Fosters a deeper commitment to Yoga practice.
	  % \end{itemize}

% \end{frame}

%%%%%%%%%%%%%%%%%%%%%%%%%%%%%%%%%%%%%%%%%%%%%%%%%%%%%%%%%%%%%%%%%%%%%%%%%%%%%%%%%%
\begin{frame}[fragile]\frametitle{Importance of Subjective Experiences in Daily Yoga Practice}

    \begin{itemize}
        \item \textbf{Self-Discipline:} Regular practice builds discipline (e.g., morning routines).
        \item \textbf{Physical Harmony:} Asanas ensure strength and health.
        \item \textbf{Mental Balance:} Pranayama promotes calmness.
        \item \textbf{Self-Study (स्वाध्याय):} Reading enhances understanding.
        \item \textbf{Surrender (ईश्वरप्रणिधान):} Non-attachment to outcomes.
        \item \textbf{Knowledge (भक्ति योग):} Cultivates a love for learning.
        \item \textbf{Eka Tatva Abhyasa (एकतत्त्व अभ्यास):} Focus on one principle for clarity.
        \item \textbf{Patanjali's Guidance:} Consistent practice (अभ्यास) leads to growth.
        \item \textbf{Subjective Experience:} Personal perception of practice.
        \item \textbf{Self-Awareness:} Consciousness of states and changes.
        \item \textbf{Mindfulness:} Present moment focus.
        \item \textbf{Personalization:} Adapt practices to individual needs.
        \item \textbf{Mind-Body Connection:} Awareness of interplay for harmony.
        \item \textbf{Reflective Practice:} Review experiences for growth.
    \end{itemize}

\end{frame}
