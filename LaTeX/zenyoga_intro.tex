%%%%%%%%%%%%%%%%%%%%%%%%%%%%%%%%%%%%%%%%%%%%%%%%%%%%%%%%%%%%%%%%
\begin{frame}[fragile]\frametitle{}
\begin{center}
{\Large Introduction}


{\tiny (Based on ``Zen Yoga'' by P J Saher And Deep Knowledge YouTube Channel by Dr Ashish Shukla)}
\end{center}
\end{frame}


%%%%%%%%%%%%%%%%%%%%%%%%%%%%%%%%%%%%%%%%%%%%%%%%%%%%%%%%%%%%%%%%
\begin{frame}[fragile]\frametitle{Life Challenges and Distractions}
\begin{itemize}
    \item Everyday life often brings distractions.
    \item These distractions pull us away from our spiritual path.
    \item The key is to maintain focus and clarity.
    \item Being aware of mental states helps avoid getting lost in distractions.
\end{itemize}
\end{frame}

%%%%%%%%%%%%%%%%%%%%%%%%%%%%%%%%%%%%%%%%%%%%%%%%%%%%%%%%%%%%%%%%
\begin{frame}[fragile]\frametitle{Living with Purpose}
\begin{itemize}
    \item Life is full of distractions and issues.
    \item To continue on the spiritual path, we must remain focused and not let mundane issues derail us.
    \item Our spiritual journey requires an open mind and understanding of self.
\end{itemize}
\end{frame}

%%%%%%%%%%%%%%%%%%%%%%%%%%%%%%%%%%%%%%%%%%%%%%%%%%%%%%%%%%%%%%%%
\begin{frame}[fragile]\frametitle{Superficial Life vs Profound Experience}
\begin{itemize}
    \item Superficial life focuses on sensory experiences and distractions.
    \item A profound life connects us to the divine essence within.
    \item By focusing inward, we can transform everyday activities into meditative experiences.
    \item This shift in perception prevents life from becoming boring or overwhelming.
\end{itemize}
\end{frame}

%%%%%%%%%%%%%%%%%%%%%%%%%%%%%%%%%%%%%%%%%%%%%%%%%%%%%%%%%%%%%%%%
\begin{frame}[fragile]\frametitle{The Role of Mistakes in Life}
\begin{itemize}
    \item Mistakes are a natural part of life.
    \item The key is to take responsibility for them without blaming others or external circumstances.
    \item Embrace mistakes as learning opportunities for growth and spiritual development.
\end{itemize}
\end{frame}

%%%%%%%%%%%%%%%%%%%%%%%%%%%%%%%%%%%%%%%%%%%%%%%%%%%%%%%%%%%%%%%%
\begin{frame}[fragile]\frametitle{Self-Awareness and Personal Growth}
\begin{itemize}
    \item Personal growth starts with self-awareness.
    \item We must understand our own mind and thought process to evolve.
    \item External circumstances will not change until we change ourselves.
    \item Spiritual practice is about transforming the self, not the external world.
\end{itemize}
\end{frame}

%%%%%%%%%%%%%%%%%%%%%%%%%%%%%%%%%%%%%%%%%%%%%%%%%%%%%%%%%%%%%%%%
\begin{frame}[fragile]\frametitle{Connecting with the Divine Essence}
\begin{itemize}
    \item True progress happens when we connect with our divine essence.
    \item This internal connection shifts our perspective, making challenges easier to handle.
    \item External objects and people no longer define us once we understand our divine self.
\end{itemize}
\end{frame}

%%%%%%%%%%%%%%%%%%%%%%%%%%%%%%%%%%%%%%%%%%%%%%%%%%%%%%%%%%%%%%%%
\begin{frame}[fragile]\frametitle{Change in Perceptions}

\begin{itemize}
\item We believe in what we actually see and hear, but there can be more than that.
\item We can see only between 0.00008 to 0.00009(?) cm. Rest is there but we cant see.
\item Examples: Radio waves which are 20km wavelength, cosmic rays which are $10^{-10}$mm.
\item We can hear only between 30 to 20k per second.
\item Can we still say that we know everything?
\item We are very much Deaf and Blind.
\item By practice of Zenoga, range of perceptions can be increased.
\end{itemize}

\end{frame}

%%%%%%%%%%%%%%%%%%%%%%%%%%%%%%%%%%%%%%%%%%%%%%%%%%%%%%%%%%%%%%%%
\begin{frame}[fragile]\frametitle{}
\begin{center}
{\Large Structure of Mind}


{\tiny (Based on ``Zen Yoga'' by P J Saher And Deep Knowledge YouTube Channel by Dr Ashish Shukla)}
\end{center}
\end{frame}


%%%%%%%%%%%%%%%%%%%%%%%%%%%%%%%%%%%%%%%%%%%%%%%%%%%%%%%%%%%%%%%%
\begin{frame}[fragile]\frametitle{Introduction to Mind Sections}
\begin{itemize}
    \item Our mind has four sections: 
    \item Physical Mind - Controls thinking, understanding, and actions.
    \item Intuition Center - Helps in accessing deeper wisdom.
    \item Para-mental Center - Represents the influence in mental state.
    \item Transcendental Center - Transforms the mind into higher states.
\end{itemize}
\end{frame}

%%%%%%%%%%%%%%%%%%%%%%%%%%%%%%%%%%%%%%%%%%%%%%%%%%%%%%%%%%%%%%%%
\begin{frame}[fragile]
\frametitle{Structure}

\begin{columns}[T] % align columns
\begin{column}{.48\textwidth}
\begin{itemize}
\item Mind is divided into 4 sections
	\begin{itemize}
	\item Section 1 : Brain, Physical Mind
	\item Section 2 : Intuition
	\begin{itemize}
		\item Section 2 a: RAS
		\item Section 2 b: CCS: Critical Certain Stage
	\end{itemize}
	\item Section 3 : Para-mental: Extra Sensory Perceptions
	\item Section 4 : Transcendental: Soul (atman)
	\end{itemize}
\item Lower Dimension (control): 1, 2a
\item Higher Dimension (no-control): 2b, 3, 4
\end{itemize}
\end{column}%
\hfill%
\begin{column}{.48\textwidth}
 \begin{center}
\includegraphics[width=0.9\linewidth,keepaspectratio]{images/zenyoga1}
\end{center}

\end{column}%
\end{columns}
\end{frame}

%%%%%%%%%%%%%%%%%%%%%%%%%%%%%%%%%%%%%%%%%%%%%%%%%%%%%%%%%%%%%%%%
\begin{frame}[fragile]\frametitle{Parallels}
Parallels with Ashtang Yoga अष्टांग योग in Yogasutra योगसूत्र by Patanjali पतंजली 


\begin{itemize}
\item CSS is actually Pratyahara प्रत्याहार stage
\item So, 2b, 3 and 4, (IMO) become dharana धारणा , dhyan ध्यान , samadhi समाधी 
\item and so, 1 and 2a become yama यम  , niyam नियम , asan आसन , pranayam प्राणायाम 
\item Preparing yourself to tackle impulses generated by breath, food/drink, incoming coded by sense organs, sleep and sex via making most of 'I' (intelligence) based action than 'E' (emotion) or 'S' (Sensuality).
\end{itemize}

\end{frame}

%%%%%%%%%%%%%%%%%%%%%%%%%%%%%%%%%%%%%%%%%%%%%%%%%%%%%%%%%%%%%%%%
\begin{frame}[fragile]\frametitle{Mind Section 1 - Physical Mind}
\begin{itemize}
    \item Physical mind allows us to think, understand, and take actions.
    \item Controls voluntary functions.
    \item It is the center of our actions and decisions.
    \item Understanding and controlling this mind section is essential for our development.
\end{itemize}
\end{frame}

%%%%%%%%%%%%%%%%%%%%%%%%%%%%%%%%%%%%%%%%%%%%%%%%%%%%%%%%%%%%%%%%
\begin{frame}[fragile]\frametitle{Mind Section 2 - Intuition Center (2)}
\begin{itemize}
    \item Section 2 controls involuntary activities in our body.
    \item Helps in meditation and concentration.
    \item Functions automatically but can be controlled to some extent.
    \item It’s necessary to enable this section for deeper connection.
\end{itemize}
\end{frame}

%%%%%%%%%%%%%%%%%%%%%%%%%%%%%%%%%%%%%%%%%%%%%%%%%%%%%%%%%%%%%%%%
\begin{frame}[fragile]\frametitle{Mind Section 3 - Para-mental Center}
\begin{itemize}
    \item The Para-mental center influences our mental state.
    \item Helps in shaping our responses and emotional reactions.
    \item Balancing this section can enhance mental clarity.
\end{itemize}
\end{frame}

%%%%%%%%%%%%%%%%%%%%%%%%%%%%%%%%%%%%%%%%%%%%%%%%%%%%%%%%%%%%%%%%
\begin{frame}[fragile]\frametitle{Mind Section 4 - Transcendental Center}
\begin{itemize}
    \item This center helps us reach transcendental states of awareness.
    \item Transforms the mind into higher consciousness.
    \item Necessary for spiritual progression.
\end{itemize}
\end{frame}


%%%%%%%%%%%%%%%%%%%%%%%%%%%%%%%%%%%%%%%%%%%%%%%%%%%%%%%%%%%%%%%%
\begin{frame}[fragile]
\frametitle{Section 2a}
\begin{itemize}
\item Intercepts flow from physical body, senses.
\item Adjusts/tones-down/accentuates and sends to thalamus to cortex for decision making.
\item Thus body can report bodily disorders to brain directly without consulting conscious mind.
\item This covers all unconscious/autonomous functions.
\end{itemize}
\end{frame}

%%%%%%%%%%%%%%%%%%%%%%%%%%%%%%%%%%%%%%%%%%%%%%%%%%%%%%%%%%%%%%%%
\begin{frame}[fragile]
\frametitle{Section 2b}
Concentration, Meditation, Intuition.

 \begin{center}
\includegraphics[width=0.35\linewidth,keepaspectratio]{images/zenyoga2}
\end{center}

Goal: Cross the big gap/divide (no-mans-land) and go to CCS and further. Then Section 3 and 4 are sort of automatic.
We can cross the gap by developing Section 1.

\end{frame}

%%%%%%%%%%%%%%%%%%%%%%%%%%%%%%%%%%%%%%%%%%%%%%%%%%%%%%%%%%%%%%%%
\begin{frame}[fragile]
\frametitle{Centers}

\begin{itemize}
\item Section 1: Physical Mind/Brain is divided into
\begin{itemize}
\item Integrity (I)
\item Emotive (E)
\item Sensual (S)
\item Mobility (M)
\end{itemize}
\item Section 2: Intuition
\item Section 3: Para-mental
\item Section 4: Transcendental
\end{itemize}

\end{frame}


%%%%%%%%%%%%%%%%%%%%%%%%%%%%%%%%%%%%%%%%%%%%%%%%%%%%%%%%%%%%%%%%
\begin{frame}[fragile]
\frametitle{Section 1 Centers}

\begin{itemize}
\item 1. Integrity (I): through intellectual introspection. ``iih'' - will to know, thinking.
\item 2. Emotive (E): Evocative of emotions, feelings, ``rajas''.
\item 3. Sensual (S): Sensuo-vitality, reflexes, drive, longings, ``tamas''.
\item 4. Mobility (M): muscular, ``manas''.
\end{itemize}

For any action/Mobility (ie 4th stage) dominant between the first 3 dictates. General ratio is 2-4-8-2. Aim is to improve I so that it dominates E and S, so that it dictates the action ie M. 

\end{frame}

%%%%%%%%%%%%%%%%%%%%%%%%%%%%%%%%%%%%%%%%%%%%%%%%%%%%%%%%%%%%%%%%
\begin{frame}[fragile]\frametitle{Goal}

\begin{itemize}
\item Find how you take decisions. Is it 'I', 'E' or 'S'.
\item Work on consciously so as to Let 'I' dominate 'E' and 'S'.
\end{itemize}

\end{frame}

%%%%%%%%%%%%%%%%%%%%%%%%%%%%%%%%%%%%%%%%%%%%%%%%%%%%%%%%%%%%%%%%
\begin{frame}[fragile]\frametitle{Introduction to Mind's Centers and Chakras}
  \begin{itemize}
    \item Mind's centers should not be confused with chakras.
    \item Chakras are part of our subtle body and not physical.
    \item Seven chakras align with the subtle body, not the physical one.
    \item The root chakra is connected to the body, but others are outside the body.
    \item Mind's centers and chakras should be understood separately.
  \end{itemize}
\end{frame}

%%%%%%%%%%%%%%%%%%%%%%%%%%%%%%%%%%%%%%%%%%%%%%%%%%%%%%%%%%%%%%%%
\begin{frame}\frametitle{Chakras and the Subtle Body}
  \begin{itemize}
    \item Chakras are aligned with the subtle body, not the physical body.
    \item The root chakra (Muladhara) is physically connected to the body.
    \item Other chakras align with the body in a parallel form.
    \item Understanding chakras requires advanced meditation techniques.
  \end{itemize}
\end{frame}

%%%%%%%%%%%%%%%%%%%%%%%%%%%%%%%%%%%%%%%%%%%%%%%%%%%%%%%%%%%%%%%%
\begin{frame}\frametitle{Vasanas: Desires and Habits}
  \begin{itemize}
    \item Vasanas are deeply connected to physical existence.
    \item There are two types of Vasanas: inherent and acquired.
    \item Vasanas accumulate over time and influence future births.
    \item During death, accumulated Vasanas form a subtle body for reincarnation.
  \end{itemize}
\end{frame}

%%%%%%%%%%%%%%%%%%%%%%%%%%%%%%%%%%%%%%%%%%%%%%%%%%%%%%%%%%%%%%%%
\begin{frame}\frametitle{Vasanas and Reincarnation}
  \begin{itemize}
    \item Vasanas dictate the cycle of birth and death.
    \item Accumulated Vasanas influence the future physical form.
    \item We are trapped in the cycle of birth until Vasanas are cleansed.
    \item To break this cycle, we must work on replacing negative Vasanas.
  \end{itemize}
\end{frame}

%%%%%%%%%%%%%%%%%%%%%%%%%%%%%%%%%%%%%%%%%%%%%%%%%%%%%%%%%%%%%%%%
\begin{frame}\frametitle{Overcoming Vasanas: The Spiritual Path}
  \begin{itemize}
    \item Replacing negative habits with positive ones is key to overcoming Vasanas.
    \item Spirituality can help replace worldly desires with higher aspirations.
    \item This cleansing process is gradual but leads to liberation.
    \item Continuous practice is necessary to transcend Vasanas and move towards spiritual freedom.
  \end{itemize}
\end{frame}

%%%%%%%%%%%%%%%%%%%%%%%%%%%%%%%%%%%%%%%%%%%%%%%%%%%%%%%%%%%%%%%%
\begin{frame}\frametitle{The Role of Desire in Spiritual Evolution}
  \begin{itemize}
    \item Desire (Vasanas) is rooted in the mind.
    \item We must be aware of desires and replace them consciously.
    \item Desires, when purified, guide us on the path of enlightenment.
    \item Replacing worldly desires with spiritual desires accelerates growth.
  \end{itemize}
\end{frame}

%%%%%%%%%%%%%%%%%%%%%%%%%%%%%%%%%%%%%%%%%%%%%%%%%%%%%%%%%%%%%%%%
\begin{frame}\frametitle{The Questions of Enlightenment}
  \begin{itemize}
    \item What is the opposite of birth and life? What is the purpose of life?
    \item Can an ordinary person attain enlightenment, or is it only for masters?
    \item Is free will real, or is life predetermined?
    \item Are we finite beings, or do we carry divine fragments of the infinite within us?
  \end{itemize}
\end{frame}

%%%%%%%%%%%%%%%%%%%%%%%%%%%%%%%%%%%%%%%%%%%%%%%%%%%%%%%%%%%%%%%%
\begin{frame}\frametitle{Free Will and Life's Purpose}
  \begin{itemize}
    \item Is life completely predetermined or do we have the power to shape our own path?
    \item Understanding free will is essential to spiritual growth.
    \item Realizing that we have control over our life’s direction leads to spiritual evolution.
  \end{itemize}
\end{frame}

%%%%%%%%%%%%%%%%%%%%%%%%%%%%%%%%%%%%%%%%%%%%%%%%%%%%%%%%%%%%%%%%
\begin{frame}\frametitle{The Middle Path in Spirituality}
  \begin{itemize}
    \item There is a "middle path" between indulgence and asceticism.
    \item This balanced approach is key to personal growth and spiritual evolution.
    \item Following this path helps in overcoming worldly attachments while progressing spiritually.
  \end{itemize}
\end{frame}

%%%%%%%%%%%%%%%%%%%%%%%%%%%%%%%%%%%%%%%%%%%%%%%%%%%%%%%%%%%%%%%%
\begin{frame}\frametitle{Spiritual Evolution and Liberation}
  \begin{itemize}
    \item The journey to liberation is gradual and may take multiple lifetimes.
    \item One who follows the path of spirituality will eventually reach enlightenment.
    \item Even if liberation doesn’t happen in this life, the journey begins.
    \item The key is persistence and dedication to the spiritual practice.
  \end{itemize}
\end{frame}

%%%%%%%%%%%%%%%%%%%%%%%%%%%%%%%%%%%%%%%%%%%%%%%%%%%%%%%%%%%%%%%%
\begin{frame}\frametitle{Conclusion}
  \begin{itemize}
    \item Recognize and understand your desires (Vasanas) to work towards liberation.
    \item Replace negative Vasanas with positive ones for spiritual growth.
    \item Follow the spiritual path with patience, and enlightenment will eventually come.
    \item The process of cleansing and evolving takes time but is essential for transcendence.
  \end{itemize}
\end{frame}
