%%%%%%%%%%%%%%%%%%%%%%%%%%%%%%%%%%%%%%%%%%%%%%%%%%%%%%%%%%%%%%%%%%%%%%%%%%%%%%%%%%
\begin{frame}[fragile]\frametitle{}
\begin{center}
{\Large Introduction to Yoga and Yogic Practices}
\end{center}
\end{frame}

%%%%%%%%%%%%%%%%%%%%%%%%%%%%%%%%%%%%%%%%%%%%%%%%%%%%%%%%%%%
\begin{frame}[fragile]\frametitle{Syllabus}

\begin{itemize}
\item 1.1  Yoga : Etymology, definitions, aim, objectives and misconceptions. 
\item 1.2  Yoga : Its origin, history and development. 
\item 1.3  Guiding principles to be followed by Yoga practitioners.  
\item 1.4  Principles of Yoga (Triguna, Antahkarana-chatustaya, Tri-Sharira/ Panchakosha). 
\item 1.5  Introduction to major schools of Yoga (Jnana, Bhakti, Karma, Patanjali, Hatha). 
\item 1.6  Introduction to Yoga practices for health and well being. 
\item 1.7  Introduction to Shatkarma: meaning, purpose and their significance in Yoga Sadhana. 
\item 1.8  Introduction to Yogic  Sukshma Vyayama,  Sthula Vyayama and Surya Namaskara.  
\item 1.9  Introduction to Yogasana: meaning, principles, and their health benefits. 
\item 1.10  Introduction to Pranayama and Dhyana and their health benefits. 
\end{itemize}
	  
\end{frame}

%%%%%%%%%%%%%%%%%%%%%%%%%%%%%%%%%%%%%%%%%%%%%%%%%%%%%%%%%%%%%%%%%%%%%%%%%%%%%%%%%%
\begin{frame}[fragile]\frametitle{}
\begin{center}
{\Large Background Terms}
\end{center}
\end{frame}

%%%%%%%%%%%%%%%%%%%%%%%%%%%%%%%%%%%%%%%%%%%%%%%%%%%%%%%%%%%
\begin{frame}[fragile]\frametitle{Understanding Key Concepts}
      \begin{itemize}
        \item \textbf{Brahman}: Absolute reality, infinite, uncaused, eternal, source of being, intelligence, and bliss (Sat, Chit, Ananda).
        \item \textbf{माया}: Creative and illusory power of Brahman, cosmic illusion, Brahman as the unaffected upholder of creation.
        \item \textbf{आत्मन्}: Absolute reality in the individual, one with Brahman, unchanging, refers to the Absolute within.
        \item \textbf{उपाधि}: Limiting adjunct, false identification creating an illusion of limitation, body and mind.
        \item \textbf{अविद्या}: Primordial ignorance, wrong identification with Upadhis.
        \item \textbf{जीव}: Apparent individual soul, self identifying with Upadhi, suffers due to perceived limitations, reborn until realizing true nature (Atman).
        \item \textbf{ईश्वर}: Brahman with Maya, personal God, endowed with omniscience, omnipresence, and unlimited power, creator, preserver, destroyer.
        \item \textbf{पुरुष}: Absolute consciousness, as described in Sankhya Yoga.
        \item \textbf{प्रकृति}: Material cause of creation, three Gunas in balance, potential creation, Maya.
      \end{itemize}
	  
\textbf{Summary}: माया manifests as अविद्या , leading to identification with उपाधि. जीव suffers due to this false identification until realizing आत्मन्.	  
\end{frame}



%%%%%%%%%%%%%%%%%%%%%%%%%%%%%%%%%%%%%%%%%%%%%%%%%%%%%%%%%%%%%%%%%%%%%%%%%%%%%%%%%%
\begin{frame}[fragile]\frametitle{}
\begin{center}
{\Large 1.1 Yoga : Etymology, definitions, aim, objectives and misconceptions}
\end{center}
\end{frame}

%%%%%%%%%%%%%%%%%%%%%%%%%%%%%%%%%%%%%%%%%%%%%%%%%%%%%%%%%%%
\begin{frame}[fragile]\frametitle{Etymology of Yoga}

      \begin{itemize}
		\item Derived from the Sanskrit word \textit{योग} (Yoga).
		\item Rooted in the Sanskrit root \textit{युज्} (Yuj) meaning “to join” or “to unite”.
		\item Represents the union of individual consciousness with universal consciousness.
		\item Ancient practice with origins in the Vedic texts and Upanishads.
		\item Evolved through various philosophies and traditions over centuries.
	  \end{itemize}

\end{frame}

%%%%%%%%%%%%%%%%%%%%%%%%%%%%%%%%%%%%%%%%%%%%%%%%%%%%%%%%%%%
\begin{frame}[fragile]\frametitle{Definitions of Yoga}
      \begin{itemize}
        \item \textbf{Patanjali Yoga Sutras}:      योगश्चित्तवृत्तिनिरोधः || १:२ || 
          To block the patterns of consciousness is Yoga.

        \item \textbf{Yoga Vasishtha}:          मनः प्रशमनोपायः योग इत्यभिधीयते |
          Yoga is called a skilful trick to calm down the mind.
        \item \textbf{Bhagavad Gita Verse 2:48}:         योगस्थः कुरु कर्माणि सङ्गं त्यक्त्वा धनञ्जय |
          सिद्ध्यसिद्ध्योः समो भूत्वा समत्वं योग उच्यते || २:४८ ||
          O Dananjaya, perform action being steadfast in Yoga, abandoning attachment and remaining equanimous in success and failure. Yoga is the equanimity of mind.

        \item \textbf{Bhagavad Gita Verse 2:50}:
          
          बुद्धियुक्तो जहातीह उभे सुकृतदुष्कृते |
          तस्माद्योगाय युज्यस्व योगः कर्मसु कौशलम् || २:५० ||
          Endowed with wisdom of equanimity, cast off in this life both good and bad deeds. Thus, dedicate yourself to Yoga. Skill in Action is Yoga.
          
      \end{itemize}
\end{frame}


%%%%%%%%%%%%%%%%%%%%%%%%%%%%%%%%%%%%%%%%%%%%%%%%%%%%%%%%%%%
\begin{frame}[fragile]\frametitle{Meaning and concepts of Yoga}
Four Ashrams of Vedic Life
          Based on कर्म and धर्म, Vedic life was divided into four ashrams: 
          \begin{itemize}
            \item \textbf{The First Ashrama} - "ब्रह्मचर्य" or the Student Stage
            \item \textbf{The Second Ashrama} - "गृहस्थ" or the Householder Stage
            \item \textbf{The Third Ashrama} - "वानप्रस्थ" or the Hermit Stage
            \item \textbf{The Fourth Ashrama} - "संन्यास" or the Wandering Ascetic Stage
          \end{itemize}
		  
Purusharthas
          
          पुरुषार्थ literally means "aims of man" or "what men live for". It is the prescription for the right way of living, recognizing various inherent urges and satisfying them for a meaningful life.
          
          \begin{itemize}
            \item \textbf{धर्म}: Righteousness
            \item \textbf{अर्थ}: Wealth or material means
            \item \textbf{काम}: Satisfaction of sensual desires
            \item \textbf{मोक्ष}: Freedom or spiritual liberation
          \end{itemize}
\end{frame}

%%%%%%%%%%%%%%%%%%%%%%%%%%%%%%%%%%%%%%%%%%%%%%%%%%%%%%%%%%%
\begin{frame}[fragile]\frametitle{The Gurus and Gurukul System}
      \begin{itemize}
        \item {\textit “To light the candle one needs another burning candle; in the same way, those who are unenlightened need the help of an illumined Guru.”} - Swami Sivananda
          
          \item The word Guru:  गु meaning "Darkness" and रु meaning "to destroy". 
		  \item A Guru is a spiritually enlightened being that dispels the darkness of ignorance and lifts the veil of माया (illusion), rendering the disciple free from the cycle of death and birth.
          
          
          % The journey stops with the finding of the Guru. The Guru is the gateway to the infinite, guiding the disciple towards the personal realization of the eternal truth.
          
          
          % The love between Guru and disciple transcends all human relationships. It is through this relationship that the disciple's potential is given shape and direction, working towards the divine plan.
          
        \item \textbf{The Gurukul System in India}:          गुरुकुल translates to "the home (family) of the preceptor." 
		\item The Guru lives with his family and disciples. Students come to be part of the Guru's family, learning Vedas, Sanskrit, fine arts, administration, and other skills and etiquettes.
          
          
          % Gurukuls were in forests or by riversides. The first 25 years of life, as ब्रह्मचर्य (a celibate student), were spent acquiring training for serving society and personal emancipation.
          
          
          % Life in Gurukul involved rigorous तपस (austerities) to prepare for life's hardships, instilling habits like hard work, enthusiasm, and selfless service. Students performed duties like fetching water, cleaning, and gathering twigs, while practicing meditation and Yoga.
          
          
          % Classes were held under shady banyan trees or in thatched huts. Education was generally free, but the Guru received दक्षिणा (fee) from disciples.
          
      \end{itemize}
\end{frame}


%%%%%%%%%%%%%%%%%%%%%%%%%%%%%%%%%%%%%%%%%%%%%%%%%%%%%%%%%%%
\begin{frame}[fragile]\frametitle{Objectives of Yoga}

      \begin{itemize}
		\item To cultivate \textit{Discipline} and \textit{Self-Control}.
		\item To improve \textit{Mental Focus} and \textit{Concentration}.
		\item To enhance \textit{Emotional Stability} and \textit{Resilience}.
		\item To promote \textit{Physical Fitness} and \textit{Posture}.
		\item To achieve \textit{Holistic Well-Being} and \textit{Harmonious Living}.
	  \end{itemize}

\end{frame}

%%%%%%%%%%%%%%%%%%%%%%%%%%%%%%%%%%%%%%%%%%%%%%%%%%%%%%%%%%%
\begin{frame}[fragile]\frametitle{Misconceptions about Yoga}

      \begin{itemize}
		\item Yoga is only about physical postures (\textit{Asanas}).
		\item Yoga is a religion.
		\item Yoga requires flexibility.
		\item Yoga is just about relaxation.
		\item Yoga is a practice for only young people.
	  \end{itemize}

\end{frame}


%%%%%%%%%%%%%%%%%%%%%%%%%%%%%%%%%%%%%%%%%%%%%%%%%%%%%%%%%%%%%%%%%%%%%%%%%%%%%%%%%%
\begin{frame}[fragile]\frametitle{}
\begin{center}
{\Large 1.2 Yoga : Its origin, history and development}
\end{center}
\end{frame}


%%%%%%%%%%%%%%%%%%%%%%%%%%%%%%%%%%%%%%%%%%%%%%%%%%%%%%%%%%%
\begin{frame}[fragile]\frametitle{Origin of Yoga}

      \begin{itemize}
		\item Originated in ancient India around 5000 BCE.
		\item Rooted in Vedic traditions and texts.
		\item First mentions in the \textit{Rigveda} and \textit{Yajurveda}.
		\item Early practices centered around meditation and ritual.
		\item Integral to early Indian philosophical systems like \textit{Samkhya} and \textit{Vedanta}.
        % \item \textbf{Ancient Roots}: 
          
          % Yoga originated in ancient India. The origin of Yoga is traced to the वेद (Vedas). It is difficult to ascertain a fixed time period for the origin of these ancient texts due to differing historical accounts.
          
        % \item \textbf{Ancient Tradition}: 
          
          % The system of Yoga is an ancient tradition having its origin in India. The practice of Yoga is believed to have started at the very dawn of civilization.
          
        \item \textbf{Yogic Lore}: 
          
          In the Yogic lore, Lord Shiva is considered to be the first yogi or आदियोगी (Adiyogi), and the first Guru or आदि गुरु (Adi Guru).
          		
	  \end{itemize}

\end{frame}

%%%%%%%%%%%%%%%%%%%%%%%%%%%%%%%%%%%%%%%%%%%%%%%%%%%%%%%%%%%
\begin{frame}[fragile]\frametitle{Historical Development of Yoga}

      \begin{itemize}
		\item \textit{Pre-Classical Period}: Upanishads and early Yoga texts (\textit{2000-500 BCE}).
		\item \textit{Classical Period}: \textit{Yoga Sutras of Patanjali} and \textit{Bhagavad Gita} (\textit{500 BCE - 500 CE}).
		\item \textit{Medieval Period}: Development of \textit{Hatha Yoga} and tantric traditions (\textit{500-1500 CE}).
		\item \textit{Modern Period}: Revival and global dissemination in the 19th and 20th centuries.
		\item Major figures: Swami Vivekananda, Sri T. Krishnamacharya, B.K.S. Iyengar.
	  \end{itemize}

\end{frame}

%%%%%%%%%%%%%%%%%%%%%%%%%%%%%%%%%%%%%%%%%%%%%%%%%%%%%%%%%%%
\begin{frame}[fragile]\frametitle{Key Texts and Influences}

      \begin{itemize}
		\item \textit{Vedas} - Early ritualistic and philosophical foundations.
		\item \textit{Upanishads} - Conceptual framework of Yoga.
		\item \textit{Yoga Sutras of Patanjali} - Systematization of Yoga philosophy.
		\item \textit{Bhagavad Gita} - Integration of Yoga with life and duty.
		\item \textit{Hatha Yoga Pradipika} - Practical techniques and practices.
	  \end{itemize}

\end{frame}

%%%%%%%%%%%%%%%%%%%%%%%%%%%%%%%%%%%%%%%%%%%%%%%%%%%%%%%%%%%
\begin{frame}[fragile]\frametitle{Evolution of Yoga Practices}

      \begin{itemize}
		\item Early practices focused on meditation and asceticism.
		\item Development of physical postures (\textit{Asanas}) in the medieval period.
		\item Integration of breath control (\textit{Pranayama}) and energy channels (\textit{Nadis}).
		\item Emergence of different styles: \textit{Hatha}, \textit{Kundalini}, \textit{Raja}, and \textit{Karma Yoga}.
		\item Contemporary practices: \textit{Vinyasa}, \textit{Ashtanga}, and \textit{Power Yoga}.
	  \end{itemize}

\end{frame}



%%%%%%%%%%%%%%%%%%%%%%%%%%%%%%%%%%%%%%%%%%%%%%%%%%%%%%%%%%%%%%%%%%%%%%%%%%%%%%%%%%
\begin{frame}[fragile]\frametitle{}
\begin{center}
{\Large 1.3 Guiding principles to be followed by Yoga practitioners}
\end{center}
\end{frame}

%%%%%%%%%%%%%%%%%%%%%%%%%%%%%%%%%%%%%%%%%%%%%%%%%%%%%%%%%%%
\begin{frame}[fragile]\frametitle{The Five Points of Yoga by Swami Vishnudevananda}
    \begin{itemize}
        \item \textbf{Proper Exercise - Asana}
            
            आसन (Asana) develops the body, broadens mental faculties, and spiritual capacities. Unlike violent physical exercises, yoga postures focus on maintaining the spine’s flexibility and strength.
            
        \item \textbf{Proper Breathing - Pranayama}
            
            प्राणायाम (Pranayama) involves controlling प्राण (Prana) to achieve control over the mind. Proper breathing increases lung capacity and vitality.
            
        \item \textbf{Proper Relaxation - Savasana}
            
            सवासन (Savasana) is the corpse pose, a state of physical, mental, and spiritual relaxation that recharges and balances the body and mind.
            
        \item \textbf{Proper Diet - Vegetarian}
            
            A Yogic diet consists of pure, simple, natural foods that promote health. Vegetarian meals aid digestion and provide balanced nutrition.
            
        \item \textbf{Positive Thinking and Meditation - Vedanta and Dhyana}
            
            ध्यान (Dhyana) is the practice of meditation that focuses the mind internally on the self, leading to lasting happiness and peace.
            
    \end{itemize}
\end{frame}


%%%%%%%%%%%%%%%%%%%%%%%%%%%%%%%%%%%%%%%%%%%%%%%%%%%%%%%%%%%%%%%%%%%%%%%%%%%%%%%%%%
\begin{frame}[fragile]\frametitle{}
\begin{center}
{\Large 1.4 Principles of Yoga (Triguna, Antahkarana-chatustaya, Tri-Sharira/ Panchakosha)}
\end{center}
\end{frame}

%%%%%%%%%%%%%%%%%%%%%%%%%%%%%%%%%%%%%%%%%%%%%%%%%%%%%%%%%%%
\begin{frame}[fragile]\frametitle{Triguna (Three Gunas)}

      \begin{itemize}
		\item \textit{Sattva} - Quality of purity, clarity, and harmony.
		\item \textit{Rajas} - Quality of activity, movement, and restlessness.
		\item \textit{Tamas} - Quality of inertia, darkness, and ignorance.
		\item Balance of Gunas affects mental and emotional states.
		\item Goal: Cultivate \textit{Sattva} for spiritual growth and peace.
	  \end{itemize}

\end{frame}

%%%%%%%%%%%%%%%%%%%%%%%%%%%%%%%%%%%%%%%%%%%%%%%%%%%%%%%%%%%
\begin{frame}[fragile]\frametitle{Antahkarana-chatustaya (Four Aspects of the Inner Instrument)}

      \begin{itemize}
		\item \textit{Manas} (Mind) - Handles thoughts and sensory perceptions.
		\item \textit{Buddhi} (Intellect) - Functions as the decision-making faculty.
		\item \textit{Ahamkara} (Ego) - Sense of individuality and self-identity.
		\item \textit{Chitta} (Memory) - Stores past experiences and impressions.
		\item Harmonizing these aspects aids in mental clarity and self-awareness.
	  \end{itemize}

\end{frame}

%%%%%%%%%%%%%%%%%%%%%%%%%%%%%%%%%%%%%%%%%%%%%%%%%%%%%%%%%%%
\begin{frame}[fragile]\frametitle{Tri-Sharira (Three Bodies)}

      \begin{itemize}
		\item \textit{Sthula Sharira} (Gross Body) - Physical body made of elements.
		\item \textit{Sukshma Sharira} (Subtle Body) - Includes mind, intellect, and ego.
		\item \textit{Karana Sharira} (Causal Body) - The essence of individuality and karma.
		\item Understanding these bodies aids in holistic self-realization.
		\item Goal: Achieve harmony among all three bodies for spiritual growth.
	  \end{itemize}

\end{frame}

%%%%%%%%%%%%%%%%%%%%%%%%%%%%%%%%%%%%%%%%%%%%%%%%%%%%%%%%%%%
\begin{frame}[fragile]\frametitle{Panchakosha (Five Sheaths)}

      \begin{itemize}
		\item \textit{Annamaya Kosha} (Food Sheath) - Physical body nourished by food.
		\item \textit{Pranamaya Kosha} (Vital Air Sheath) - Energy body responsible for life force.
		\item \textit{Manomaya Kosha} (Mental Sheath) - Mind and emotional body.
		\item \textit{Vijnanamaya Kosha} (Wisdom Sheath) - Intellect and discernment.
		\item \textit{Anandamaya Kosha} (Bliss Sheath) - True self, source of bliss and consciousness.
		\item Goal: Transcend the sheaths to realize the true self.
	  \end{itemize}

\end{frame}

%%%%%%%%%%%%%%%%%%%%%%%%%%%%%%%%%%%%%%%%%%%%%%%%%%%%%%%%%%%
\begin{frame}[fragile]\frametitle{Pancha Maha Bhuta: Five Elements}


    \begin{itemize}
        \item \textbf{Akasha (Space/Ether)}
        \item \textbf{Vayu (Air)}
        \item \textbf{Agni (Fire)}
        \item \textbf{Jala (Water)}
        \item \textbf{Prithvi (Earth)}
    \end{itemize}
	
    \textbf{Elemental Sequence:}
	
    \begin{itemize}
        \item From आकाश (Akasha) comes वायु (Vayu)
        \item From वायु (Vayu) comes अग्नि (Agni)
        \item From अग्नि (Agni) comes जल (Jala)
        \item From जल (Jala) comes पृथ्वी (Prithvi)
    \end{itemize}
	
\end{frame}

%%%%%%%%%%%%%%%%%%%%%%%%%%%%%%%%%%%%%%%%%%%%%%%%%%%%%%%%%%%
\begin{frame}[fragile]\frametitle{Pancha Maha Bhuta: Five Elements}

    \textbf{Physical Properties and Functions:}
	
	
        \begin{tabular}{|p{0.21\linewidth}|p{0.1\linewidth}|p{0.1\linewidth}|p{0.1\linewidth}|p{0.1\linewidth}|p{0.1\linewidth}|}
% \begin{tabular}{|l|l|l|l|l|l|}
    \hline
    \textbf{Element} & \textbf{Space/Ether} & \textbf{Air} & \textbf{Fire} & \textbf{Water} & \textbf{Earth} \\
    \hline
    \textbf{Attributes} & Light, Minute, Quick, Clear, Soft & Light, Dry, Rough, Mobile, Cold, Minute, Clear & Dry, Hot, Sharp, Light, Rough, Minute, Clear & Heavy, Moist, Fluid, Slimy, Cold, Thick, Soft & Heavy, Rough, Solid, Massive, Firm, Hard \\
    \hline
    \textbf{Role in the Body} & Body cavities & Movement, pulsation, conduction & Whole body metabolism & All adhesions, joints & Body organs, mass, inertia \\
    \hline
    \end{tabular}

\end{frame}


%%%%%%%%%%%%%%%%%%%%%%%%%%%%%%%%%%%%%%%%%%%%%%%%%%%%%%%%%%%
\begin{frame}[fragile]\frametitle{Tanmatras and Sense Organs}
    \textbf{Tanmatras: Seeding of Elements}
	
	
    \begin{tabular}{|l|l|l|}
    \hline
    \textbf{Element} & \textbf{Sanskrit Word} & \textbf{English Translation} \\
    \hline
    \textbf{Space/Ether} & शब्द (Shabda) & Sound \\
    \hline
    \textbf{Air} & स्पर्श (Sparsha) & Touch \\
    \hline
    \textbf{Fire} & रूप (Rupa) & Shape/Visual/Light \\
    \hline
    \textbf{Water} & रस (Rasa) & Taste \\
    \hline
    \textbf{Earth} & गन्ध (Gandha) & Smell \\
    \hline
    \end{tabular}


    \textbf{Pancha Jnanedriya: Five Sense Organs}
	
    \begin{tabular}{|l|l|}
    \hline
    \textbf{Element} & \textbf{Sense Organ} \\
    \hline
    \textbf{Space/Ether} & Ears \\
    \hline
    \textbf{Air} & Skin \\
    \hline
    \textbf{Fire} & Eyes \\
    \hline
    \textbf{Water} & Tongue \\
    \hline
    \textbf{Earth} & Nose \\
    \hline
    \end{tabular}
\end{frame}

%%%%%%%%%%%%%%%%%%%%%%%%%%%%%%%%%%%%%%%%%%%%%%%%%%%%%%%%%%%
\begin{frame}[fragile]\frametitle{Pancha Karmendriya: Five Action Organs}
    \begin{tabular}{|l|l|l|}
    \hline
    \textbf{Element} & \textbf{Action Organ} & \textbf{Function} \\
    \hline
    \textbf{Space/Ether} & Tongue & Speech \\
    \hline
    \textbf{Air} & Hand & Receiving and Holding \\
    \hline
    \textbf{Fire} & Feet & Movement \\
    \hline
    \textbf{Water} & Genitals & Reproduction \\
    \hline
    \textbf{Earth} & Anus & Discharging the waste \\
    \hline
    \end{tabular}
\end{frame}

%%%%%%%%%%%%%%%%%%%%%%%%%%%%%%%%%%%%%%%%%%%%%%%%%%%%%%%%%%%
\begin{frame}[fragile]\frametitle{Sapta Dhatu: Seven Tissues}
    \begin{itemize}
        \item \textbf{Rasa} - Plasma
        \item \textbf{Rakta} - Blood
        \item \textbf{Mamsa} - Muscle
        \item \textbf{Meda} - Adipose
        \item \textbf{Asthi} - Bone
        \item \textbf{Majja} - Bone Marrow
        \item \textbf{Shukra} - Reproductive Tissues
    \end{itemize}
    \vspace{0.5cm}
    \textbf{Function:} 
    
    धातु (Dhatu) sustains and maintains the body, with each Dhatu providing nourishment to the next.
    
\end{frame}

%%%%%%%%%%%%%%%%%%%%%%%%%%%%%%%%%%%%%%%%%%%%%%%%%%%%%%%%%%%
\begin{frame}[fragile]\frametitle{Tri Doshas: Three Constitutions}
    \begin{tabular}{|l|l|}
    \hline
    \textbf{Dosha} & \textbf{Properties} \\
    \hline
    \textbf{Vata} & Dry, Light, Cold, Rough, Minute, Unsteady \\
    \hline
    \textbf{Pitta} & Unctuous, Hot, Sharp, Light, Bad Smell, Quick in movement, Liquid \\
    \hline
    \textbf{Kapha} & Unctuous, Cold, Massive, Sluggish, Slippery, Soft, Steady \\
    \hline
    \end{tabular}
    \vspace{0.5cm}
    \textbf{Dosha Meaning:}
    
    दोष (Dosha) means 'that which vitiates' and can be seen as a fault or imbalance in the cosmic rhythm.
    
\end{frame}

%%%%%%%%%%%%%%%%%%%%%%%%%%%%%%%%%%%%%%%%%%%%%%%%%%%%%%%%%%
\begin{frame}[fragile]\frametitle{Pancha Prana: Five Pranas in the Body}
    \begin{itemize}
        \item \textbf{Prana} - From the head down to the navel
        \item \textbf{Apana} - From the navel down to the Muladhara chakra
        \item \textbf{Udana} - From the navel to the head
        \item \textbf{Samana} - In the navel region
        \item \textbf{Vyana} - Throughout the body
    \end{itemize}
    \vspace{0.5cm}
    \textbf{Upapranas:}
    \begin{itemize}
        \item \textbf{Naga} - Hiccoughs, Burps
        \item \textbf{Kurma} - Blinking of the eyes
        \item \textbf{Krikal} - Hunger, Thirst, Sneezing, Coughing
        \item \textbf{Devadatta} - Yawning, Drowsiness
        \item \textbf{Dhananjaya} - After death lingering
    \end{itemize}
\end{frame}






%%%%%%%%%%%%%%%%%%%%%%%%%%%%%%%%%%%%%%%%%%%%%%%%%%%%%%%%%%%%%%%%%%%%%%%%%%%%%%%%%
\begin{frame}[fragile]\frametitle{}
\begin{center}
{\Large 1.5 Introduction to major schools of Yoga (Jnana, Bhakti, Karma, Patanjali, Hatha)}
\end{center}
\end{frame}

%%%%%%%%%%%%%%%%%%%%%%%%%%%%%%%%%%%%%%%%%%%%%%%%%%%%%%%%%%
\begin{frame}[fragile]\frametitle{Philosophic Schools in Ancient India}
    \textbf{Types of Schools (Darshana):}
    \begin{itemize}
        \item \textbf{Astik Darshana} - Schools that follow the Vedas
        \item \textbf{Nastik Darshana} - Schools that do not follow the authority of the Vedas
    \end{itemize}
\end{frame}

%%%%%%%%%%%%%%%%%%%%%%%%%%%%%%%%%%%%%%%%%%%%%%%%%%%%%%%%%%
\begin{frame}[fragile]\frametitle{Astik Darshanas: Schools Following the Vedas}
    \begin{tabular}{|l|l|}
    \hline
    \textbf{School Name} & \textbf{Founder} \\
    \hline
    \textbf{Nyaya} & Gautama Rishi \\
    \hline
    \textbf{Vaiseshika} & Kanada Rishi \\
    \hline
    \textbf{Sankhya} & Kapila Muni \\
    \hline
    \textbf{Yoga} & Patanjali Maharishi \\
    \hline
    \textbf{Purva Mimamsa} & Jaimini Rishi \\
    \hline
    \textbf{Uttara Mimamsa or Vedanta} & Vyasa Maharishi \\
    \hline
    \end{tabular}
\end{frame}

%%%%%%%%%%%%%%%%%%%%%%%%%%%%%%%%%%%%%%%%%%%%%%%%%%%%%%%%%%
\begin{frame}[fragile]\frametitle{Nastik Darshanas: Schools Not Following the Vedas}
    \begin{itemize}
        \item \textbf{Charvaka}
        \item \textbf{Buddhism}
        \item \textbf{Jaina}
    \end{itemize}
\end{frame}

%%%%%%%%%%%%%%%%%%%%%%%%%%%%%%%%%%%%%%%%%%%%%%%%%%%%%%%%%%
\begin{frame}[fragile]\frametitle{Jnana Yoga: Yoga of Wisdom}
    \textbf{Definition:} Jnana Yoga is the path of self-realization through discerning the real from the unreal. It is a non-dualistic path that encourages the separation of the real from the illusory.

    
    \textbf{Three Stages of Jnana Yoga Practice:}
    \begin{itemize}
        \item \textbf{Sravana} - Listening or absorbing instructions
        \item \textbf{Manana} - Reflection or contemplation involving reasoning
        \item \textbf{Nidhidyasana} - Repeated meditation and implementation of convictions
    \end{itemize}

    \textbf{Sadhana Chatustaya:}
    \begin{itemize}
        \item \textbf{Manas} - Mind
        \item \textbf{Buddhi} - Intellect
        \item \textbf{Chitta} - Consciousness
        \item \textbf{Ahamkara} - Ego
    \end{itemize}
\end{frame}


%%%%%%%%%%%%%%%%%%%%%%%%%%%%%%%%%%%%%%%%%%%%%%%%%%%%%%%%%%%
\begin{frame}[fragile]\frametitle{Bhakti Yoga: Yoga of Devotion}
    \textbf{Definition:} Bhakti Yoga is the path of unconditional love for God, emphasizing devotion and the union of the lover (the yogi) with the beloved (the Divine).

    
    \textbf{Key Text:} नारद भक्ति सूत्र (Narada Bhakti Sutra) - Discusses the nature of Bhakti and its connection to Prema (divine love).

    
    \textbf{Techniques of Bhakti Yoga:}
    \begin{itemize}
        \item Prayer
        \item Chanting
        \item Japa
        \item Telling stories of God
        \item Pujas
        \item Rituals
    \end{itemize}

    
    \textbf{Types of Bhaktas (Devotees) in Bhagavad Gita:}
    \begin{itemize}
        \item \textbf{Arta} - Distressed
        \item \textbf{Jidnasu} - Inquisitive
        \item \textbf{Artharthee} - Desirer of wealth
    \end{itemize}
\end{frame}

%%%%%%%%%%%%%%%%%%%%%%%%%%%%%%%%%%%%%%%%%%%%%%%%%%%%%%%%%%%
\begin{frame}[fragile]\frametitle{Karma Yoga: Yoga of Action}
          \begin{itemize}
	  \item \textbf{Definition:} Karma Yoga is the path of union through action. It is practiced by those with an outgoing or action-oriented nature. The key is to act selflessly, without personal gain or reward, and to offer the fruits of your actions to God.

 
	  \item \textbf{Core Principles of Karma Yoga (According to Bhagavad Gita):}
    \begin{enumerate}
        \item Work with a sense of duty.
        \item Work without intense attachment to the outcome.
        \item Do not let anxieties about results disturb your mind during the task.
        \item Accept both failure and success with equanimity.
    \end{enumerate}

	  \item  \textbf{Objective:} To sublimate the ego and achieve selfless devotion in all actions.
	      \end{itemize}

\end{frame}

%%%%%%%%%%%%%%%%%%%%%%%%%%%%%%%%%%%%%%%%%%%%%%%%%%%%%%%%%%%
\begin{frame}[fragile]\frametitle{Raja Yoga: The Royal Yoga}
    
      \begin{itemize}
	  \item \textbf{Definition:} Raja Yoga, meaning "royal" or "kingly" Yoga, is considered the culmination of all paths of Yoga. It represents the ultimate state of Self-realization.
	  \item  \textbf{Significance:} According to Swatmarama in the हठयोग प्रदीपिका (Hatha Yoga Pradipika), Hatha Yoga serves as a staircase leading to Raja Yoga. Raja Yoga may not refer to a specific form of Yoga but to the ultimate state of Self-realization.
	  \item \textbf{Objective:} To achieve cessation of mental modifications and restore the Real Self to its pristine purity, as emphasized by Patanjali.
		\item Focuses on the eight limbs of Yoga (\textit{Ashtanga Yoga}).
		\item Aims for mental discipline and spiritual insight.
		\item Key practices: Ethical guidelines, physical postures, breath control, and meditation.
		\item Major text: \textit{Yoga Sutras of Patanjali}.
		\item Emphasizes systematic approach to achieving higher states of consciousness.
	  \end{itemize}
    
\end{frame}



%%%%%%%%%%%%%%%%%%%%%%%%%%%%%%%%%%%%%%%%%%%%%%%%%%%%%%%%%%%
\begin{frame}[fragile]\frametitle{Patanjali Yoga (Raja Yoga)}

      \begin{itemize}
		\item Focuses on the eight limbs of Yoga (\textit{Ashtanga Yoga}).
		\item Aims for mental discipline and spiritual insight.
		\item Key practices: Ethical guidelines, physical postures, breath control, and meditation.
		\item Major text: \textit{Yoga Sutras of Patanjali}.
		\item Emphasizes systematic approach to achieving higher states of consciousness.
	  \end{itemize}

\end{frame}

%%%%%%%%%%%%%%%%%%%%%%%%%%%%%%%%%%%%%%%%%%%%%%%%%%%%%%%%%%%
\begin{frame}[fragile]\frametitle{Hatha Yoga}

      \begin{itemize}
		\item Focuses on physical practices and techniques.
		\item Aims to balance the body and mind through postures (\textit{Asanas}) and breath control (\textit{Pranayama}).
		\item Emphasizes purification of the body to prepare for higher practices.
		\item Major texts: \textit{Hatha Yoga Pradipika}, \textit{Gheranda Samhita}.
		\item Often serves as a preparatory practice for deeper meditative techniques.
	  \end{itemize}

\end{frame}



%%%%%%%%%%%%%%%%%%%%%%%%%%%%%%%%%%%%%%%%%%%%%%%%%%%%%%%%%%%%%%%%%%%%%%%%%%%%%%%%%%
\begin{frame}[fragile]\frametitle{}
\begin{center}
{\Large 1.6 Introduction to Yoga practices for health and well being}
\end{center}
\end{frame}

%%%%%%%%%%%%%%%%%%%%%%%%%%%%%%%%%%%%%%%%%%%%%%%%%%%%%%%%%%%
\begin{frame}[fragile]\frametitle{Health: Meaning and Definition}
    \textbf{World Health Organisation (WHO) Definition:}
    
        Health is defined as a state of complete physical, mental, social, and spiritual well-being and not merely the absence of disease or infirmity.
    

    
    \textbf{Sanskrit Definition:} 
    
        The Sanskrit word for health is स्वास्थ्य (Swasthya), derived from स्व (Swa) meaning "Self" and स्थ (Stha) meaning "abiding." Thus, Swasthya can be translated as "Abiding in one's own Self," reflecting the true nature of every being as सत्-चित्-आनन्द (Sat-Chit-Ananda) or being-Consciousness-Bliss.
    

    
    \textbf{Health according to Yoga:}
    \begin{itemize}
        \item Relaxed Muscles
        \item Loose joints to conserve energy
        \item Low metabolic rate
        \item Efficient utility of energy by the body
        \item Coordinated functioning of organ systems even under stress
    \end{itemize}
\end{frame}

%%%%%%%%%%%%%%%%%%%%%%%%%%%%%%%%%%%%%%%%%%%%%%%%%%%%%%%%%%%
\begin{frame}[fragile]\frametitle{Strength and Balance in Yoga}
    \textbf{Strength:}
    \begin{itemize}
        \item Yoga poses such as नौकासन (Naukasana), उत्कटासन (Utkatasana), and भुजंगासन (Bhujangasana) develop muscle strength similarly to traditional exercises like push-ups, lunges, or squats.
    \end{itemize}

    
    \textbf{Balance:}
    \begin{itemize}
        \item Balance is crucial for fitness and is often overlooked in traditional gym routines. 
        \item Yoga poses such as वृक्षासन (Vrikshasana or Tree Pose) teach practitioners to stay firm on one leg, enhancing overall balance and fitness for daily activities.
    \end{itemize}
\end{frame}

%%%%%%%%%%%%%%%%%%%%%%%%%%%%%%%%%%%%%%%%%%%%%%%%%%%%%%%%%%%
\begin{frame}[fragile]\frametitle{Joint Mobility and Yogic View of Health}
    \textbf{Joint Mobility:}
    \begin{itemize}
        \item Many Yoga poses require moving joints in ways different from linear activities like walking or running.
        \item This practice improves the overall range of motion and helps prevent injuries, reducing stiffness and pain.
    \end{itemize}

    
    \textbf{Yogic View of Health:}
    \begin{itemize}
        \item श्वेताश्वतर उपनिषद् (Shvetasvatara Upanishad): Health in Yoga is characterized by lightness of body, health, a desireless mind, clarity of complexion, a beautiful voice, an agreeable odor, and minimal excretions.
        \item Patanjali’s योग दर्शन (Yoga Darshan): Attributes of a perfect body include beauty, gracefulness, strength, and adamantine hardness. Deep concentration and mastery of सामान वायु (Samana Vayu) lead to radiant effulgence.
    \end{itemize}
\end{frame}

%%%%%%%%%%%%%%%%%%%%%%%%%%%%%%%%%%%%%%%%%%%%%%%%%%%%%%%%%%%
\begin{frame}[fragile]\frametitle{Mental and Emotional Well-being}
    \textbf{Mental Well-being:}
    \begin{itemize}
        \item Conscious process of calming the mind removes weaknesses.
        \item Builds willpower.
        \item Steadfastness becomes the core.
        \item Converts challenges into opportunities.
    \end{itemize}

    
    \textbf{Emotional Well-being:}
    \begin{itemize}
        \item Overcomes emotional conflicts, complexes, and turmoils.
        \item Sharpens and sensitizes emotions systematically.
    \end{itemize}
\end{frame}




%%%%%%%%%%%%%%%%%%%%%%%%%%%%%%%%%%%%%%%%%%%%%%%%%%%%%%%%%%%
\begin{frame}[fragile]\frametitle{Yoga Practices for Physical Health}

      \begin{itemize}
		\item \textit{Asanas} (Postures) - Enhance flexibility, strength, and balance.
		\item \textit{Pranayama} (Breath Control) - Improves lung capacity and energy flow.
		\item \textit{Shavasana} (Corpse Pose) - Promotes relaxation and recovery.
		\item \textit{Kriyas} (Cleansing Techniques) - Detoxify and purify the body.
		\item \textit{Bandhas} (Body Locks) - Support internal organ function and stability.
	  \end{itemize}

\end{frame}

%%%%%%%%%%%%%%%%%%%%%%%%%%%%%%%%%%%%%%%%%%%%%%%%%%%%%%%%%%%
\begin{frame}[fragile]\frametitle{Yoga Practices for Mental Well-Being}

      \begin{itemize}
		\item \textit{Meditation} (Dhyana) - Reduces stress and enhances focus.
		\item \textit{Mindfulness} - Increases present-moment awareness and emotional stability.
		\item \textit{Pranayama} - Balances the nervous system and calms the mind.
		\item \textit{Mantra Chanting} - Provides mental clarity and emotional balance.
		\item \textit{Yoga Nidra} - Deep relaxation practice that improves sleep quality.
	  \end{itemize}

\end{frame}

%%%%%%%%%%%%%%%%%%%%%%%%%%%%%%%%%%%%%%%%%%%%%%%%%%%%%%%%%%%
\begin{frame}[fragile]\frametitle{Yoga Practices for Emotional Balance}

      \begin{itemize}
		\item \textit{Bhakti Yoga} - Cultivates emotional connection and devotion.
		\item \textit{Journaling} - Reflect on emotions and mental patterns.
		\item \textit{Gratitude Practice} - Enhances positive emotional states.
		\item \textit{Group Yoga Classes} - Builds community and support.
		\item \textit{Breath Awareness} - Helps in managing emotional responses.
	  \end{itemize}

\end{frame}

%%%%%%%%%%%%%%%%%%%%%%%%%%%%%%%%%%%%%%%%%%%%%%%%%%%%%%%%%%%
\begin{frame}[fragile]\frametitle{Yoga Practices for Overall Well-Being}

      \begin{itemize}
		\item \textit{Balanced Routine} - Integrate yoga into daily life for consistent benefits.
		\item \textit{Healthy Lifestyle Choices} - Complement yoga with proper nutrition and hydration.
		\item \textit{Holistic Approach} - Address physical, mental, and spiritual aspects.
		\item \textit{Regular Practice} - Ensure regular engagement for sustained well-being.
		\item \textit{Personalized Practice} - Adapt practices to individual needs and goals.
	  \end{itemize}

\end{frame}



%%%%%%%%%%%%%%%%%%%%%%%%%%%%%%%%%%%%%%%%%%%%%%%%%%%%%%%%%%%%%%%%%%%%%%%%%%%%%%%%%%
\begin{frame}[fragile]\frametitle{}
\begin{center}
{\Large 1.7 Introduction to Shatkarma: meaning, purpose and their significance in Yoga Sadhana}
\end{center}
\end{frame}

%%%%%%%%%%%%%%%%%%%%%%%%%%%%%%%%%%%%%%%%%%%%%%%%%%%%%%%%%%%
\begin{frame}[fragile]\frametitle{Introduction to Shatkarma}

      \begin{itemize}
		\item \textit{Shatkarma} - Six purification techniques in Yoga.
		\item Aimed at cleansing the body and mind for enhanced spiritual practice.
		\item Prepares the practitioner for deeper practices like meditation and advanced postures.
		\item Integrates physical, mental, and energetic purification.
		\item Essential for holistic Yoga practice and overall health.
	  \end{itemize}

\end{frame}

%%%%%%%%%%%%%%%%%%%%%%%%%%%%%%%%%%%%%%%%%%%%%%%%%%%%%%%%%%%
\begin{frame}[fragile]\frametitle{Meaning and Purpose of Shatkarma}

      \begin{itemize}
		\item \textit{Kriya} - Techniques to purify and balance the body.
		\item \textit{Purpose} - Remove toxins, enhance vitality, and stabilize the mind.
		\item \textit{Integration} - Facilitates deeper Yoga practices and spiritual growth.
		\item \textit{Holistic Cleansing} - Addresses physical, mental, and energetic levels.
		\item Essential for overcoming physical and mental obstructions.
	  \end{itemize}

\end{frame}

%%%%%%%%%%%%%%%%%%%%%%%%%%%%%%%%%%%%%%%%%%%%%%%%%%%%%%%%%%%
\begin{frame}[fragile]\frametitle{Significance of Shatkarma in Yoga Sadhana}

      \begin{itemize}
		\item \textit{Health Benefits} - Improves digestion, detoxifies, and boosts immunity.
		\item \textit{Mental Clarity} - Reduces stress and mental clutter.
		\item \textit{Energetic Balance} - Regulates the flow of vital energy (\textit{Prana}).
		\item \textit{Spiritual Preparation} - Prepares the practitioner for higher states of consciousness.
		\item \textit{Preventive Measures} - Aids in preventing diseases and imbalances.
	  \end{itemize}

\end{frame}

%%%%%%%%%%%%%%%%%%%%%%%%%%%%%%%%%%%%%%%%%%%%%%%%%%%%%%%%%%%
\begin{frame}[fragile]\frametitle{Overview of Shatkarma Techniques}

      \begin{itemize}
		\item \textit{Kapalabhati} - Skull Shining Breath for clearing nasal passages and energizing.
		\item \textit{Neti} - Nasal cleansing with water or saline solution.
		\item \textit{Vasti} - Colonic cleansing to remove toxins from the intestines.
		\item \textit{Dhauti} - Cleansing of the digestive tract through various methods.
		\item \textit{Trataka} - Concentrated gazing to improve focus and cleanse the eyes.
		\item \textit{Nauli} - Abdominal massage to stimulate digestion and balance energy.
	  \end{itemize}

\end{frame}


%%%%%%%%%%%%%%%%%%%%%%%%%%%%%%%%%%%%%%%%%%%%%%%%%%%%%%%%%%%%%%%%%%%%%%%%%%%%%%%%%%
\begin{frame}[fragile]\frametitle{}
\begin{center}
{\Large 1.8 Introduction to Yogic  Sukshma Vyayama,  Sthula Vyayama and Surya Namaskara}
\end{center}
\end{frame}

%%%%%%%%%%%%%%%%%%%%%%%%%%%%%%%%%%%%%%%%%%%%%%%%%%%%%%%%%%%
\begin{frame}[fragile]\frametitle{Introduction to Sukshma Vyayama}

      \begin{itemize}
		\item \textit{Sukshma Vyayama} - Subtle exercises for the body and mind.
		\item Focuses on gentle movements to prepare the body for more intensive practices.
		\item Enhances joint mobility and flexibility.
		\item Aids in the smooth functioning of internal organs.
		\item Ideal for warming up and increasing energy flow.
	  \end{itemize}

\end{frame}

%%%%%%%%%%%%%%%%%%%%%%%%%%%%%%%%%%%%%%%%%%%%%%%%%%%%%%%%%%%
\begin{frame}[fragile]\frametitle{Purpose and Benefits of Sukshma Vyayama}

      \begin{itemize}
		\item \textit{Improves Circulation} - Enhances blood flow to muscles and joints.
		\item \textit{Increases Flexibility} - Promotes flexibility in joints and muscles.
		\item \textit{Prepares Body} - Warms up the body before more rigorous exercises.
		\item \textit{Reduces Stiffness} - Alleviates joint and muscle stiffness.
		\item \textit{Calms Mind} - Prepares the mind for focused practice.
	  \end{itemize}

\end{frame}

%%%%%%%%%%%%%%%%%%%%%%%%%%%%%%%%%%%%%%%%%%%%%%%%%%%%%%%%%%%
\begin{frame}[fragile]\frametitle{Introduction to Sthula Vyayama}

      \begin{itemize}
		\item \textit{Sthula Vyayama} - Gross or physical exercises for the body.
		\item Includes more intense physical postures and movements.
		\item Aims to build strength, endurance, and overall physical fitness.
		\item Often used in combination with \textit{Sukshma Vyayama} for comprehensive practice.
		\item Focuses on major muscle groups and physical conditioning.
	  \end{itemize}

\end{frame}

%%%%%%%%%%%%%%%%%%%%%%%%%%%%%%%%%%%%%%%%%%%%%%%%%%%%%%%%%%%
\begin{frame}[fragile]\frametitle{Purpose and Benefits of Sthula Vyayama}

      \begin{itemize}
		\item \textit{Strength Building} - Develops muscle strength and endurance.
		\item \textit{Improves Posture} - Enhances overall body alignment and posture.
		\item \textit{Boosts Fitness} - Increases cardiovascular and physical fitness.
		\item \textit{Enhances Vitality} - Promotes general physical health and energy.
		\item \textit{Supports Weight Management} - Aids in maintaining a healthy weight.
	  \end{itemize}

\end{frame}

%%%%%%%%%%%%%%%%%%%%%%%%%%%%%%%%%%%%%%%%%%%%%%%%%%%%%%%%%%%
\begin{frame}[fragile]\frametitle{Introduction to Surya Namaskara}

      \begin{itemize}
		\item \textit{Surya Namaskara} - Sun Salutation, a series of dynamic postures.
		\item Traditionally performed to honor the Sun and its energy.
		• Consists of a sequence of 12 postures.
		\item Integrates movement, breath, and intention.
		\item Enhances overall physical and mental health.
	  \end{itemize}

\end{frame}

%%%%%%%%%%%%%%%%%%%%%%%%%%%%%%%%%%%%%%%%%%%%%%%%%%%%%%%%%%%
\begin{frame}[fragile]\frametitle{Purpose and Benefits of Surya Namaskara}

      \begin{itemize}
		\item \textit{Improves Flexibility} - Stretches and tones the muscles.
		\item \textit{Boosts Circulation} - Enhances blood flow and energy levels.
		\item \textit{Increases Strength} - Builds core strength and endurance.
		\item \textit{Balances Mind} - Calms the mind and prepares for meditation.
		\item \textit{Energizes Body} - Invigorates and revitalizes overall health.
	  \end{itemize}

\end{frame}


%%%%%%%%%%%%%%%%%%%%%%%%%%%%%%%%%%%%%%%%%%%%%%%%%%%%%%%%%%%%%%%%%%%%%%%%%%%%%%%%%%
\begin{frame}[fragile]\frametitle{}
\begin{center}
{\Large 1.9 Introduction to Yogasana: meaning, principles, and their health benefits}
\end{center}
\end{frame}

%%%%%%%%%%%%%%%%%%%%%%%%%%%%%%%%%%%%%%%%%%%%%%%%%%%%%%%%%%%
\begin{frame}[fragile]\frametitle{Introduction to Yogasana}

      \begin{itemize}
		\item \textit{Yogasana} - Physical postures or poses in Yoga.
		\item Derived from the Sanskrit words \textit{Yoga} (union) and \textit{Asana} (seat or posture).
		\item Aims to harmonize body and mind through physical practice.
		\item Forms the foundation for many Yoga practices and techniques.
		\item Enhances physical and mental well-being.
	  \end{itemize}

\end{frame}

%%%%%%%%%%%%%%%%%%%%%%%%%%%%%%%%%%%%%%%%%%%%%%%%%%%%%%%%%%%
\begin{frame}[fragile]\frametitle{Principles of Yogasana}

      \begin{itemize}
		\item \textit{Alignment} - Proper positioning of body parts for effectiveness and safety.
		\item \textit{Breath Awareness} - Coordinating breath with movement to enhance practice.
		\item \textit{Balance} - Achieving physical and mental equilibrium in poses.
		\item \textit{Stability} - Maintaining a steady and comfortable posture.
		\item \textit{Mindfulness} - Being present and focused during practice.
	  \end{itemize}

\end{frame}

%%%%%%%%%%%%%%%%%%%%%%%%%%%%%%%%%%%%%%%%%%%%%%%%%%%%%%%%%%%
\begin{frame}[fragile]\frametitle{Health Benefits of Yogasana}

      \begin{itemize}
		\item \textit{Improves Flexibility} - Enhances range of motion in joints and muscles.
		\item \textit{Builds Strength} - Develops muscle strength and endurance.
		\item \textit{Enhances Posture} - Promotes proper alignment and balance.
		\item \textit{Boosts Circulation} - Improves blood flow and cardiovascular health.
		\item \textit{Reduces Stress} - Calms the mind and reduces anxiety levels.
		\item \textit{Improves Digestion} - Stimulates digestive organs and enhances metabolism.
		\item \textit{Enhances Mental Clarity} - Promotes focus, concentration, and mental well-being.
	  \end{itemize}

\end{frame}



%%%%%%%%%%%%%%%%%%%%%%%%%%%%%%%%%%%%%%%%%%%%%%%%%%%%%%%%%%%%%%%%%%%%%%%%%%%%%%%%%%
\begin{frame}[fragile]\frametitle{}
\begin{center}
{\Large 1.10 Introduction to Pranayama and Dhyana and their health benefits}
\end{center}
\end{frame}

%%%%%%%%%%%%%%%%%%%%%%%%%%%%%%%%%%%%%%%%%%%%%%%%%%%%%%%%%%%
\begin{frame}[fragile]\frametitle{Introduction to Pranayama}

      \begin{itemize}
		\item \textit{Pranayama} - The practice of breath control in Yoga.
		\item Derived from Sanskrit words \textit{Prana} (life force or breath) and \textit{Ayama} (control or extension).
		\item Aims to regulate and expand the breath to enhance life energy.
		\item Integrates breath with physical postures and meditation.
		\item Fundamental for balancing mind and body.
	  \end{itemize}

\end{frame}

%%%%%%%%%%%%%%%%%%%%%%%%%%%%%%%%%%%%%%%%%%%%%%%%%%%%%%%%%%%
\begin{frame}[fragile]\frametitle{Health Benefits of Pranayama}

      \begin{itemize}
		\item \textit{Improves Lung Capacity} - Enhances respiratory efficiency and endurance.
		\item \textit{Balances Nervous System} - Regulates stress and anxiety levels.
		\item \textit{Enhances Concentration} - Increases mental focus and clarity.
		\item \textit{Boosts Immunity} - Strengthens the immune system and overall vitality.
		\item \textit{Aids in Detoxification} - Promotes the removal of toxins from the body.
		\item \textit{Calms the Mind} - Reduces mental agitation and promotes relaxation.
		\item \textit{Regulates Emotions} - Helps in managing emotional responses and stability.
	  \end{itemize}

\end{frame}

%%%%%%%%%%%%%%%%%%%%%%%%%%%%%%%%%%%%%%%%%%%%%%%%%%%%%%%%%%%
\begin{frame}[fragile]\frametitle{Introduction to Dhyana}

      \begin{itemize}
		\item \textit{Dhyana} - Meditation or the practice of focused attention.
		\item Derived from Sanskrit meaning "profound contemplation" or "meditative absorption."
		• Aims to achieve a state of mental stillness and clarity.
		\item Involves sustained concentration and mindfulness.
		\item Integral for achieving higher states of consciousness and inner peace.
	  \end{itemize}

\end{frame}

%%%%%%%%%%%%%%%%%%%%%%%%%%%%%%%%%%%%%%%%%%%%%%%%%%%%%%%%%%%
\begin{frame}[fragile]\frametitle{Health Benefits of Dhyana}

      \begin{itemize}
		\item \textit{Reduces Stress} - Lowers cortisol levels and promotes relaxation.
		\item \textit{Enhances Emotional Well-Being} - Improves mood and emotional resilience.
		\item \textit{Improves Focus} - Increases attention span and cognitive function.
		\item \textit{Balances Blood Pressure} - Helps in maintaining healthy blood pressure levels.
		\item \textit{Promotes Inner Peace} - Fosters a sense of calm and tranquility.
		\item \textit{Aids in Self-Realization} - Encourages deeper self-awareness and understanding.
		\item \textit{Supports Mental Health} - Helps in managing anxiety, depression, and other mental health issues.
	  \end{itemize}

\end{frame}

