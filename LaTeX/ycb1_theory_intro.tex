%%%%%%%%%%%%%%%%%%%%%%%%%%%%%%%%%%%%%%%%%%%%%%%%%%%%%%%%%%%%%%%%%%%%%%%%%%%%%%%%%%
\begin{frame}[fragile]\frametitle{}
\begin{center}
{\Large Introduction to Yoga and Yogic Practices}
\end{center}
\end{frame}

%%%%%%%%%%%%%%%%%%%%%%%%%%%%%%%%%%%%%%%%%%%%%%%%%%%%%%%%%%%
\begin{frame}[fragile]\frametitle{Syllabus}

\begin{itemize}
\item 1.1  Yoga : Etymology, definitions, aim, objectives and misconceptions. 
\item 1.2  Yoga : Its origin, history and development. 
\item 1.3  Guiding principles to be followed by Yoga practitioners.  
\item 1.4  Principles of Yoga (Triguna, Antahkarana-chatustaya, Tri-Sharira/ Panchakosha). 
\item 1.5  Introduction to major schools of Yoga (Jnana, Bhakti, Karma, Patanjali, Hatha). 
\item 1.6  Introduction to Yoga practices for health and well being. 
\item 1.7  Introduction to Shatkarma: meaning, purpose and their significance in Yoga Sadhana. 
\item 1.8  Introduction to Yogic  Sukshma Vyayama,  Sthula Vyayama and Surya Namaskara.  
\item 1.9  Introduction to Yogasana: meaning, principles, and their health benefits. 
\item 1.10  Introduction to Pranayama and Dhyana and their health benefits. 
\end{itemize}
	  
\end{frame}

%%%%%%%%%%%%%%%%%%%%%%%%%%%%%%%%%%%%%%%%%%%%%%%%%%%%%%%%%%%%%%%%%%%%%%%%%%%%%%%%%%
\begin{frame}[fragile]\frametitle{}
\begin{center}
{\Large 1.1 Yoga : Etymology, definitions, aim, objectives and misconceptions}
\end{center}
\end{frame}

%%%%%%%%%%%%%%%%%%%%%%%%%%%%%%%%%%%%%%%%%%%%%%%%%%%%%%%%%%%
\begin{frame}[fragile]\frametitle{Etymology of Yoga}

      \begin{itemize}
		\item Derived from the Sanskrit word \textit{योग} (Yoga).
		\item Rooted in the Sanskrit root \textit{युज्} (Yuj) meaning “to join” or “to unite”.
		\item Represents the union of individual consciousness with universal consciousness.
		\item Ancient practice with origins in the Vedic texts and Upanishads.
		\item Evolved through various philosophies and traditions over centuries.
	  \end{itemize}

\end{frame}

%%%%%%%%%%%%%%%%%%%%%%%%%%%%%%%%%%%%%%%%%%%%%%%%%%%%%%%%%%%
\begin{frame}[fragile]\frametitle{Definitions of Yoga}

      \begin{itemize}
		\item \textit{योगः चित्तवृत्तिनिरोधः} (Yogaḥ cittavṛttinirodhaḥ) - Yoga is the cessation of the fluctuations of the mind. (\textit{Patanjali's Yoga Sutras})
		\item Integration of mind, body, and spirit through practice.
		\item Path to achieving \textit{Sattva} (purity) and balance.
		\item Harmonizing physical and mental disciplines.
		\item Yoga as a method for self-realization and liberation (\textit{Moksha}).
	  \end{itemize}

\end{frame}

%%%%%%%%%%%%%%%%%%%%%%%%%%%%%%%%%%%%%%%%%%%%%%%%%%%%%%%%%%%
\begin{frame}[fragile]\frametitle{Aim of Yoga}

      \begin{itemize}
		\item Achieving \textit{Self-Realization} (\textit{Atma Jnana}).
		\item Attaining \textit{Inner Peace} and \textit{Mental Clarity}.
		\item Developing \textit{Physical Health} and \textit{Flexibility}.
		\item Harmonizing \textit{Mind-Body Connection}.
		\item Reaching \textit{Spiritual Enlightenment} (\textit{Bodhi}).
	  \end{itemize}

\end{frame}

%%%%%%%%%%%%%%%%%%%%%%%%%%%%%%%%%%%%%%%%%%%%%%%%%%%%%%%%%%%
\begin{frame}[fragile]\frametitle{Objectives of Yoga}

      \begin{itemize}
		\item To cultivate \textit{Discipline} and \textit{Self-Control}.
		\item To improve \textit{Mental Focus} and \textit{Concentration}.
		\item To enhance \textit{Emotional Stability} and \textit{Resilience}.
		\item To promote \textit{Physical Fitness} and \textit{Posture}.
		\item To achieve \textit{Holistic Well-Being} and \textit{Harmonious Living}.
	  \end{itemize}

\end{frame}

%%%%%%%%%%%%%%%%%%%%%%%%%%%%%%%%%%%%%%%%%%%%%%%%%%%%%%%%%%%
\begin{frame}[fragile]\frametitle{Misconceptions about Yoga}

      \begin{itemize}
		\item Yoga is only about physical postures (\textit{Asanas}).
		\item Yoga is a religion.
		\item Yoga requires flexibility.
		\item Yoga is just about relaxation.
		\item Yoga is a practice for only young people.
	  \end{itemize}

\end{frame}


%%%%%%%%%%%%%%%%%%%%%%%%%%%%%%%%%%%%%%%%%%%%%%%%%%%%%%%%%%%%%%%%%%%%%%%%%%%%%%%%%%
\begin{frame}[fragile]\frametitle{}
\begin{center}
{\Large 1.2 Yoga : Its origin, history and development}
\end{center}
\end{frame}


%%%%%%%%%%%%%%%%%%%%%%%%%%%%%%%%%%%%%%%%%%%%%%%%%%%%%%%%%%%
\begin{frame}[fragile]\frametitle{Origin of Yoga}

      \begin{itemize}
		\item Originated in ancient India around 5000 BCE.
		\item Rooted in Vedic traditions and texts.
		\item First mentions in the \textit{Rigveda} and \textit{Yajurveda}.
		\item Early practices centered around meditation and ritual.
		\item Integral to early Indian philosophical systems like \textit{Samkhya} and \textit{Vedanta}.
	  \end{itemize}

\end{frame}

%%%%%%%%%%%%%%%%%%%%%%%%%%%%%%%%%%%%%%%%%%%%%%%%%%%%%%%%%%%
\begin{frame}[fragile]\frametitle{Historical Development of Yoga}

      \begin{itemize}
		\item \textit{Pre-Classical Period}: Upanishads and early Yoga texts (\textit{2000-500 BCE}).
		\item \textit{Classical Period}: \textit{Yoga Sutras of Patanjali} and \textit{Bhagavad Gita} (\textit{500 BCE - 500 CE}).
		\item \textit{Medieval Period}: Development of \textit{Hatha Yoga} and tantric traditions (\textit{500-1500 CE}).
		\item \textit{Modern Period}: Revival and global dissemination in the 19th and 20th centuries.
		\item Major figures: Swami Vivekananda, Sri T. Krishnamacharya, B.K.S. Iyengar.
	  \end{itemize}

\end{frame}

%%%%%%%%%%%%%%%%%%%%%%%%%%%%%%%%%%%%%%%%%%%%%%%%%%%%%%%%%%%
\begin{frame}[fragile]\frametitle{Key Texts and Influences}

      \begin{itemize}
		\item \textit{Vedas} - Early ritualistic and philosophical foundations.
		\item \textit{Upanishads} - Conceptual framework of Yoga.
		\item \textit{Yoga Sutras of Patanjali} - Systematization of Yoga philosophy.
		\item \textit{Bhagavad Gita} - Integration of Yoga with life and duty.
		\item \textit{Hatha Yoga Pradipika} - Practical techniques and practices.
	  \end{itemize}

\end{frame}

%%%%%%%%%%%%%%%%%%%%%%%%%%%%%%%%%%%%%%%%%%%%%%%%%%%%%%%%%%%
\begin{frame}[fragile]\frametitle{Evolution of Yoga Practices}

      \begin{itemize}
		\item Early practices focused on meditation and asceticism.
		\item Development of physical postures (\textit{Asanas}) in the medieval period.
		\item Integration of breath control (\textit{Pranayama}) and energy channels (\textit{Nadis}).
		\item Emergence of different styles: \textit{Hatha}, \textit{Kundalini}, \textit{Raja}, and \textit{Karma Yoga}.
		\item Contemporary practices: \textit{Vinyasa}, \textit{Ashtanga}, and \textit{Power Yoga}.
	  \end{itemize}

\end{frame}



%%%%%%%%%%%%%%%%%%%%%%%%%%%%%%%%%%%%%%%%%%%%%%%%%%%%%%%%%%%%%%%%%%%%%%%%%%%%%%%%%%
\begin{frame}[fragile]\frametitle{}
\begin{center}
{\Large 1.3 Guiding principles to be followed by Yoga practitioners}
\end{center}
\end{frame}

%%%%%%%%%%%%%%%%%%%%%%%%%%%%%%%%%%%%%%%%%%%%%%%%%%%%%%%%%%%
\begin{frame}[fragile]\frametitle{Guiding Principles for Yoga Practitioners}

      \begin{itemize}
		\item \textit{Ahimsa} (Non-Violence) - Cultivate compassion towards self and others.
		\item \textit{Satya} (Truthfulness) - Practice honesty in thoughts, speech, and actions.
		\item \textit{Asteya} (Non-Stealing) - Respect others' possessions and time.
		\item \textit{Brahmacharya} (Moderation) - Practice self-discipline and moderation in all aspects.
		\item \textit{Aparigraha} (Non-Possessiveness) - Avoid attachment and possessiveness.
	  \end{itemize}

\end{frame}

%%%%%%%%%%%%%%%%%%%%%%%%%%%%%%%%%%%%%%%%%%%%%%%%%%%%%%%%%%%
\begin{frame}[fragile]\frametitle{Ethical Principles of Yoga}

      \begin{itemize}
		\item \textit{Saucha} (Purity) - Maintain cleanliness of body and mind.
		\item \textit{Santosha} (Contentment) - Cultivate inner peace and satisfaction.
		\item \textit{Tapas} (Discipline) - Practice self-discipline and austerity.
		\item \textit{Svadhyaya} (Self-Study) - Engage in self-reflection and study of sacred texts.
		\item \textit{Ishvara Pranidhana} (Surrender to a Higher Power) - Develop faith and devotion.
	  \end{itemize}

\end{frame}

%%%%%%%%%%%%%%%%%%%%%%%%%%%%%%%%%%%%%%%%%%%%%%%%%%%%%%%%%%%
\begin{frame}[fragile]\frametitle{Practical Guidelines for Practice}

      \begin{itemize}
		\item \textit{Consistency} - Regular and dedicated practice is essential.
		\item \textit{Mindfulness} - Be present and aware during practice.
		\item \textit{Adaptability} - Modify practices according to individual needs and limitations.
		\item \textit{Respect} - Respect your body and its boundaries.
		\item \textit{Patience} - Progress in Yoga takes time; practice with patience and persistence.
	  \end{itemize}

\end{frame}

%%%%%%%%%%%%%%%%%%%%%%%%%%%%%%%%%%%%%%%%%%%%%%%%%%%%%%%%%%%
\begin{frame}[fragile]\frametitle{Behavioral Guidelines for Practitioners}

      \begin{itemize}
		\item \textit{Non-Competition} - Focus on personal growth, not comparison with others.
		\item \textit{Humility} - Approach practice with humility and openness.
		\item \textit{Non-Attachment} - Avoid attachment to specific outcomes or achievements.
		\item \textit{Gratitude} - Cultivate an attitude of gratitude towards practice and teachers.
		\item \textit{Ethical Conduct} - Adhere to ethical conduct both on and off the mat.
	  \end{itemize}

\end{frame}




%%%%%%%%%%%%%%%%%%%%%%%%%%%%%%%%%%%%%%%%%%%%%%%%%%%%%%%%%%%%%%%%%%%%%%%%%%%%%%%%%%
\begin{frame}[fragile]\frametitle{}
\begin{center}
{\Large 1.4 Principles of Yoga (Triguna, Antahkarana-chatustaya, Tri-Sharira/ Panchakosha)}
\end{center}
\end{frame}

%%%%%%%%%%%%%%%%%%%%%%%%%%%%%%%%%%%%%%%%%%%%%%%%%%%%%%%%%%%
\begin{frame}[fragile]\frametitle{Triguna (Three Gunas)}

      \begin{itemize}
		\item \textit{Sattva} - Quality of purity, clarity, and harmony.
		\item \textit{Rajas} - Quality of activity, movement, and restlessness.
		\item \textit{Tamas} - Quality of inertia, darkness, and ignorance.
		\item Balance of Gunas affects mental and emotional states.
		\item Goal: Cultivate \textit{Sattva} for spiritual growth and peace.
	  \end{itemize}

\end{frame}

%%%%%%%%%%%%%%%%%%%%%%%%%%%%%%%%%%%%%%%%%%%%%%%%%%%%%%%%%%%
\begin{frame}[fragile]\frametitle{Antahkarana-chatustaya (Four Aspects of the Inner Instrument)}

      \begin{itemize}
		\item \textit{Manas} (Mind) - Handles thoughts and sensory perceptions.
		\item \textit{Buddhi} (Intellect) - Functions as the decision-making faculty.
		\item \textit{Ahamkara} (Ego) - Sense of individuality and self-identity.
		\item \textit{Chitta} (Memory) - Stores past experiences and impressions.
		\item Harmonizing these aspects aids in mental clarity and self-awareness.
	  \end{itemize}

\end{frame}

%%%%%%%%%%%%%%%%%%%%%%%%%%%%%%%%%%%%%%%%%%%%%%%%%%%%%%%%%%%
\begin{frame}[fragile]\frametitle{Tri-Sharira (Three Bodies)}

      \begin{itemize}
		\item \textit{Sthula Sharira} (Gross Body) - Physical body made of elements.
		\item \textit{Sukshma Sharira} (Subtle Body) - Includes mind, intellect, and ego.
		\item \textit{Karana Sharira} (Causal Body) - The essence of individuality and karma.
		\item Understanding these bodies aids in holistic self-realization.
		\item Goal: Achieve harmony among all three bodies for spiritual growth.
	  \end{itemize}

\end{frame}

%%%%%%%%%%%%%%%%%%%%%%%%%%%%%%%%%%%%%%%%%%%%%%%%%%%%%%%%%%%
\begin{frame}[fragile]\frametitle{Panchakosha (Five Sheaths)}

      \begin{itemize}
		\item \textit{Annamaya Kosha} (Food Sheath) - Physical body nourished by food.
		\item \textit{Pranamaya Kosha} (Vital Air Sheath) - Energy body responsible for life force.
		\item \textit{Manomaya Kosha} (Mental Sheath) - Mind and emotional body.
		\item \textit{Vijnanamaya Kosha} (Wisdom Sheath) - Intellect and discernment.
		\item \textit{Anandamaya Kosha} (Bliss Sheath) - True self, source of bliss and consciousness.
		\item Goal: Transcend the sheaths to realize the true self.
	  \end{itemize}

\end{frame}


%%%%%%%%%%%%%%%%%%%%%%%%%%%%%%%%%%%%%%%%%%%%%%%%%%%%%%%%%%%%%%%%%%%%%%%%%%%%%%%%%%
\begin{frame}[fragile]\frametitle{}
\begin{center}
{\Large 1.5 Introduction to major schools of Yoga (Jnana, Bhakti, Karma, Patanjali, Hatha)}
\end{center}
\end{frame}

%%%%%%%%%%%%%%%%%%%%%%%%%%%%%%%%%%%%%%%%%%%%%%%%%%%%%%%%%%%
\begin{frame}[fragile]\frametitle{Jnana Yoga (Path of Wisdom)}

      \begin{itemize}
		\item Focuses on self-inquiry and knowledge.
		\item Aims to understand the nature of reality and the self.
		\item Emphasizes study of sacred texts and contemplation.
		\item Key practice: Discrimination between the eternal and the temporal.
		\item Major texts: \textit{Upanishads}, \textit{Bhagavad Gita}.
	  \end{itemize}

\end{frame}

%%%%%%%%%%%%%%%%%%%%%%%%%%%%%%%%%%%%%%%%%%%%%%%%%%%%%%%%%%%
\begin{frame}[fragile]\frametitle{Bhakti Yoga (Path of Devotion)}

      \begin{itemize}
		\item Focuses on devotion and love towards a personal deity.
		\item Aims to cultivate a deep, personal relationship with the divine.
		\item Practices include prayer, chanting, and rituals.
		\item Emphasizes surrender and selfless love.
		\item Major texts: \textit{Bhagavad Gita}, \textit{Ramayana}.
	  \end{itemize}

\end{frame}

%%%%%%%%%%%%%%%%%%%%%%%%%%%%%%%%%%%%%%%%%%%%%%%%%%%%%%%%%%%
\begin{frame}[fragile]\frametitle{Karma Yoga (Path of Action)}

      \begin{itemize}
		\item Focuses on selfless action and duty.
		\item Aims to act without attachment to the results.
		\item Emphasizes performing one's duty with dedication and without ego.
		\item Key practice: Offering the fruits of action to the divine.
		\item Major texts: \textit{Bhagavad Gita}.
	  \end{itemize}

\end{frame}

%%%%%%%%%%%%%%%%%%%%%%%%%%%%%%%%%%%%%%%%%%%%%%%%%%%%%%%%%%%
\begin{frame}[fragile]\frametitle{Patanjali Yoga (Raja Yoga)}

      \begin{itemize}
		\item Focuses on the eight limbs of Yoga (\textit{Ashtanga Yoga}).
		\item Aims for mental discipline and spiritual insight.
		\item Key practices: Ethical guidelines, physical postures, breath control, and meditation.
		\item Major text: \textit{Yoga Sutras of Patanjali}.
		\item Emphasizes systematic approach to achieving higher states of consciousness.
	  \end{itemize}

\end{frame}

%%%%%%%%%%%%%%%%%%%%%%%%%%%%%%%%%%%%%%%%%%%%%%%%%%%%%%%%%%%
\begin{frame}[fragile]\frametitle{Hatha Yoga}

      \begin{itemize}
		\item Focuses on physical practices and techniques.
		\item Aims to balance the body and mind through postures (\textit{Asanas}) and breath control (\textit{Pranayama}).
		\item Emphasizes purification of the body to prepare for higher practices.
		\item Major texts: \textit{Hatha Yoga Pradipika}, \textit{Gheranda Samhita}.
		\item Often serves as a preparatory practice for deeper meditative techniques.
	  \end{itemize}

\end{frame}



%%%%%%%%%%%%%%%%%%%%%%%%%%%%%%%%%%%%%%%%%%%%%%%%%%%%%%%%%%%%%%%%%%%%%%%%%%%%%%%%%%
\begin{frame}[fragile]\frametitle{}
\begin{center}
{\Large 1.6 Introduction to Yoga practices for health and well being}
\end{center}
\end{frame}

%%%%%%%%%%%%%%%%%%%%%%%%%%%%%%%%%%%%%%%%%%%%%%%%%%%%%%%%%%%
\begin{frame}[fragile]\frametitle{Yoga Practices for Physical Health}

      \begin{itemize}
		\item \textit{Asanas} (Postures) - Enhance flexibility, strength, and balance.
		\item \textit{Pranayama} (Breath Control) - Improves lung capacity and energy flow.
		\item \textit{Shavasana} (Corpse Pose) - Promotes relaxation and recovery.
		\item \textit{Kriyas} (Cleansing Techniques) - Detoxify and purify the body.
		\item \textit{Bandhas} (Body Locks) - Support internal organ function and stability.
	  \end{itemize}

\end{frame}

%%%%%%%%%%%%%%%%%%%%%%%%%%%%%%%%%%%%%%%%%%%%%%%%%%%%%%%%%%%
\begin{frame}[fragile]\frametitle{Yoga Practices for Mental Well-Being}

      \begin{itemize}
		\item \textit{Meditation} (Dhyana) - Reduces stress and enhances focus.
		\item \textit{Mindfulness} - Increases present-moment awareness and emotional stability.
		\item \textit{Pranayama} - Balances the nervous system and calms the mind.
		\item \textit{Mantra Chanting} - Provides mental clarity and emotional balance.
		\item \textit{Yoga Nidra} - Deep relaxation practice that improves sleep quality.
	  \end{itemize}

\end{frame}

%%%%%%%%%%%%%%%%%%%%%%%%%%%%%%%%%%%%%%%%%%%%%%%%%%%%%%%%%%%
\begin{frame}[fragile]\frametitle{Yoga Practices for Emotional Balance}

      \begin{itemize}
		\item \textit{Bhakti Yoga} - Cultivates emotional connection and devotion.
		\item \textit{Journaling} - Reflect on emotions and mental patterns.
		\item \textit{Gratitude Practice} - Enhances positive emotional states.
		\item \textit{Group Yoga Classes} - Builds community and support.
		\item \textit{Breath Awareness} - Helps in managing emotional responses.
	  \end{itemize}

\end{frame}

%%%%%%%%%%%%%%%%%%%%%%%%%%%%%%%%%%%%%%%%%%%%%%%%%%%%%%%%%%%
\begin{frame}[fragile]\frametitle{Yoga Practices for Overall Well-Being}

      \begin{itemize}
		\item \textit{Balanced Routine} - Integrate yoga into daily life for consistent benefits.
		\item \textit{Healthy Lifestyle Choices} - Complement yoga with proper nutrition and hydration.
		\item \textit{Holistic Approach} - Address physical, mental, and spiritual aspects.
		\item \textit{Regular Practice} - Ensure regular engagement for sustained well-being.
		\item \textit{Personalized Practice} - Adapt practices to individual needs and goals.
	  \end{itemize}

\end{frame}



%%%%%%%%%%%%%%%%%%%%%%%%%%%%%%%%%%%%%%%%%%%%%%%%%%%%%%%%%%%%%%%%%%%%%%%%%%%%%%%%%%
\begin{frame}[fragile]\frametitle{}
\begin{center}
{\Large 1.7 Introduction to Shatkarma: meaning, purpose and their significance in Yoga Sadhana}
\end{center}
\end{frame}

%%%%%%%%%%%%%%%%%%%%%%%%%%%%%%%%%%%%%%%%%%%%%%%%%%%%%%%%%%%
\begin{frame}[fragile]\frametitle{Introduction to Shatkarma}

      \begin{itemize}
		\item \textit{Shatkarma} - Six purification techniques in Yoga.
		\item Aimed at cleansing the body and mind for enhanced spiritual practice.
		\item Prepares the practitioner for deeper practices like meditation and advanced postures.
		\item Integrates physical, mental, and energetic purification.
		\item Essential for holistic Yoga practice and overall health.
	  \end{itemize}

\end{frame}

%%%%%%%%%%%%%%%%%%%%%%%%%%%%%%%%%%%%%%%%%%%%%%%%%%%%%%%%%%%
\begin{frame}[fragile]\frametitle{Meaning and Purpose of Shatkarma}

      \begin{itemize}
		\item \textit{Kriya} - Techniques to purify and balance the body.
		\item \textit{Purpose} - Remove toxins, enhance vitality, and stabilize the mind.
		\item \textit{Integration} - Facilitates deeper Yoga practices and spiritual growth.
		\item \textit{Holistic Cleansing} - Addresses physical, mental, and energetic levels.
		\item Essential for overcoming physical and mental obstructions.
	  \end{itemize}

\end{frame}

%%%%%%%%%%%%%%%%%%%%%%%%%%%%%%%%%%%%%%%%%%%%%%%%%%%%%%%%%%%
\begin{frame}[fragile]\frametitle{Significance of Shatkarma in Yoga Sadhana}

      \begin{itemize}
		\item \textit{Health Benefits} - Improves digestion, detoxifies, and boosts immunity.
		\item \textit{Mental Clarity} - Reduces stress and mental clutter.
		\item \textit{Energetic Balance} - Regulates the flow of vital energy (\textit{Prana}).
		\item \textit{Spiritual Preparation} - Prepares the practitioner for higher states of consciousness.
		\item \textit{Preventive Measures} - Aids in preventing diseases and imbalances.
	  \end{itemize}

\end{frame}

%%%%%%%%%%%%%%%%%%%%%%%%%%%%%%%%%%%%%%%%%%%%%%%%%%%%%%%%%%%
\begin{frame}[fragile]\frametitle{Overview of Shatkarma Techniques}

      \begin{itemize}
		\item \textit{Kapalabhati} - Skull Shining Breath for clearing nasal passages and energizing.
		\item \textit{Neti} - Nasal cleansing with water or saline solution.
		\item \textit{Vasti} - Colonic cleansing to remove toxins from the intestines.
		\item \textit{Dhauti} - Cleansing of the digestive tract through various methods.
		\item \textit{Trataka} - Concentrated gazing to improve focus and cleanse the eyes.
		\item \textit{Nauli} - Abdominal massage to stimulate digestion and balance energy.
	  \end{itemize}

\end{frame}


%%%%%%%%%%%%%%%%%%%%%%%%%%%%%%%%%%%%%%%%%%%%%%%%%%%%%%%%%%%%%%%%%%%%%%%%%%%%%%%%%%
\begin{frame}[fragile]\frametitle{}
\begin{center}
{\Large 1.8 Introduction to Yogic  Sukshma Vyayama,  Sthula Vyayama and Surya Namaskara}
\end{center}
\end{frame}

%%%%%%%%%%%%%%%%%%%%%%%%%%%%%%%%%%%%%%%%%%%%%%%%%%%%%%%%%%%
\begin{frame}[fragile]\frametitle{Introduction to Sukshma Vyayama}

      \begin{itemize}
		\item \textit{Sukshma Vyayama} - Subtle exercises for the body and mind.
		\item Focuses on gentle movements to prepare the body for more intensive practices.
		\item Enhances joint mobility and flexibility.
		\item Aids in the smooth functioning of internal organs.
		\item Ideal for warming up and increasing energy flow.
	  \end{itemize}

\end{frame}

%%%%%%%%%%%%%%%%%%%%%%%%%%%%%%%%%%%%%%%%%%%%%%%%%%%%%%%%%%%
\begin{frame}[fragile]\frametitle{Purpose and Benefits of Sukshma Vyayama}

      \begin{itemize}
		\item \textit{Improves Circulation} - Enhances blood flow to muscles and joints.
		\item \textit{Increases Flexibility} - Promotes flexibility in joints and muscles.
		\item \textit{Prepares Body} - Warms up the body before more rigorous exercises.
		\item \textit{Reduces Stiffness} - Alleviates joint and muscle stiffness.
		\item \textit{Calms Mind} - Prepares the mind for focused practice.
	  \end{itemize}

\end{frame}

%%%%%%%%%%%%%%%%%%%%%%%%%%%%%%%%%%%%%%%%%%%%%%%%%%%%%%%%%%%
\begin{frame}[fragile]\frametitle{Introduction to Sthula Vyayama}

      \begin{itemize}
		\item \textit{Sthula Vyayama} - Gross or physical exercises for the body.
		\item Includes more intense physical postures and movements.
		\item Aims to build strength, endurance, and overall physical fitness.
		\item Often used in combination with \textit{Sukshma Vyayama} for comprehensive practice.
		\item Focuses on major muscle groups and physical conditioning.
	  \end{itemize}

\end{frame}

%%%%%%%%%%%%%%%%%%%%%%%%%%%%%%%%%%%%%%%%%%%%%%%%%%%%%%%%%%%
\begin{frame}[fragile]\frametitle{Purpose and Benefits of Sthula Vyayama}

      \begin{itemize}
		\item \textit{Strength Building} - Develops muscle strength and endurance.
		\item \textit{Improves Posture} - Enhances overall body alignment and posture.
		\item \textit{Boosts Fitness} - Increases cardiovascular and physical fitness.
		\item \textit{Enhances Vitality} - Promotes general physical health and energy.
		\item \textit{Supports Weight Management} - Aids in maintaining a healthy weight.
	  \end{itemize}

\end{frame}

%%%%%%%%%%%%%%%%%%%%%%%%%%%%%%%%%%%%%%%%%%%%%%%%%%%%%%%%%%%
\begin{frame}[fragile]\frametitle{Introduction to Surya Namaskara}

      \begin{itemize}
		\item \textit{Surya Namaskara} - Sun Salutation, a series of dynamic postures.
		\item Traditionally performed to honor the Sun and its energy.
		• Consists of a sequence of 12 postures.
		\item Integrates movement, breath, and intention.
		\item Enhances overall physical and mental health.
	  \end{itemize}

\end{frame}

%%%%%%%%%%%%%%%%%%%%%%%%%%%%%%%%%%%%%%%%%%%%%%%%%%%%%%%%%%%
\begin{frame}[fragile]\frametitle{Purpose and Benefits of Surya Namaskara}

      \begin{itemize}
		\item \textit{Improves Flexibility} - Stretches and tones the muscles.
		\item \textit{Boosts Circulation} - Enhances blood flow and energy levels.
		\item \textit{Increases Strength} - Builds core strength and endurance.
		\item \textit{Balances Mind} - Calms the mind and prepares for meditation.
		\item \textit{Energizes Body} - Invigorates and revitalizes overall health.
	  \end{itemize}

\end{frame}


%%%%%%%%%%%%%%%%%%%%%%%%%%%%%%%%%%%%%%%%%%%%%%%%%%%%%%%%%%%%%%%%%%%%%%%%%%%%%%%%%%
\begin{frame}[fragile]\frametitle{}
\begin{center}
{\Large 1.9 Introduction to Yogasana: meaning, principles, and their health benefits}
\end{center}
\end{frame}

%%%%%%%%%%%%%%%%%%%%%%%%%%%%%%%%%%%%%%%%%%%%%%%%%%%%%%%%%%%
\begin{frame}[fragile]\frametitle{Introduction to Yogasana}

      \begin{itemize}
		\item \textit{Yogasana} - Physical postures or poses in Yoga.
		\item Derived from the Sanskrit words \textit{Yoga} (union) and \textit{Asana} (seat or posture).
		\item Aims to harmonize body and mind through physical practice.
		\item Forms the foundation for many Yoga practices and techniques.
		\item Enhances physical and mental well-being.
	  \end{itemize}

\end{frame}

%%%%%%%%%%%%%%%%%%%%%%%%%%%%%%%%%%%%%%%%%%%%%%%%%%%%%%%%%%%
\begin{frame}[fragile]\frametitle{Principles of Yogasana}

      \begin{itemize}
		\item \textit{Alignment} - Proper positioning of body parts for effectiveness and safety.
		\item \textit{Breath Awareness} - Coordinating breath with movement to enhance practice.
		\item \textit{Balance} - Achieving physical and mental equilibrium in poses.
		\item \textit{Stability} - Maintaining a steady and comfortable posture.
		\item \textit{Mindfulness} - Being present and focused during practice.
	  \end{itemize}

\end{frame}

%%%%%%%%%%%%%%%%%%%%%%%%%%%%%%%%%%%%%%%%%%%%%%%%%%%%%%%%%%%
\begin{frame}[fragile]\frametitle{Health Benefits of Yogasana}

      \begin{itemize}
		\item \textit{Improves Flexibility} - Enhances range of motion in joints and muscles.
		\item \textit{Builds Strength} - Develops muscle strength and endurance.
		\item \textit{Enhances Posture} - Promotes proper alignment and balance.
		\item \textit{Boosts Circulation} - Improves blood flow and cardiovascular health.
		\item \textit{Reduces Stress} - Calms the mind and reduces anxiety levels.
		\item \textit{Improves Digestion} - Stimulates digestive organs and enhances metabolism.
		\item \textit{Enhances Mental Clarity} - Promotes focus, concentration, and mental well-being.
	  \end{itemize}

\end{frame}



%%%%%%%%%%%%%%%%%%%%%%%%%%%%%%%%%%%%%%%%%%%%%%%%%%%%%%%%%%%%%%%%%%%%%%%%%%%%%%%%%%
\begin{frame}[fragile]\frametitle{}
\begin{center}
{\Large 1.10 Introduction to Pranayama and Dhyana and their health benefits}
\end{center}
\end{frame}

%%%%%%%%%%%%%%%%%%%%%%%%%%%%%%%%%%%%%%%%%%%%%%%%%%%%%%%%%%%
\begin{frame}[fragile]\frametitle{Introduction to Pranayama}

      \begin{itemize}
		\item \textit{Pranayama} - The practice of breath control in Yoga.
		\item Derived from Sanskrit words \textit{Prana} (life force or breath) and \textit{Ayama} (control or extension).
		\item Aims to regulate and expand the breath to enhance life energy.
		\item Integrates breath with physical postures and meditation.
		\item Fundamental for balancing mind and body.
	  \end{itemize}

\end{frame}

%%%%%%%%%%%%%%%%%%%%%%%%%%%%%%%%%%%%%%%%%%%%%%%%%%%%%%%%%%%
\begin{frame}[fragile]\frametitle{Health Benefits of Pranayama}

      \begin{itemize}
		\item \textit{Improves Lung Capacity} - Enhances respiratory efficiency and endurance.
		\item \textit{Balances Nervous System} - Regulates stress and anxiety levels.
		\item \textit{Enhances Concentration} - Increases mental focus and clarity.
		\item \textit{Boosts Immunity} - Strengthens the immune system and overall vitality.
		\item \textit{Aids in Detoxification} - Promotes the removal of toxins from the body.
		\item \textit{Calms the Mind} - Reduces mental agitation and promotes relaxation.
		\item \textit{Regulates Emotions} - Helps in managing emotional responses and stability.
	  \end{itemize}

\end{frame}

%%%%%%%%%%%%%%%%%%%%%%%%%%%%%%%%%%%%%%%%%%%%%%%%%%%%%%%%%%%
\begin{frame}[fragile]\frametitle{Introduction to Dhyana}

      \begin{itemize}
		\item \textit{Dhyana} - Meditation or the practice of focused attention.
		\item Derived from Sanskrit meaning "profound contemplation" or "meditative absorption."
		• Aims to achieve a state of mental stillness and clarity.
		\item Involves sustained concentration and mindfulness.
		\item Integral for achieving higher states of consciousness and inner peace.
	  \end{itemize}

\end{frame}

%%%%%%%%%%%%%%%%%%%%%%%%%%%%%%%%%%%%%%%%%%%%%%%%%%%%%%%%%%%
\begin{frame}[fragile]\frametitle{Health Benefits of Dhyana}

      \begin{itemize}
		\item \textit{Reduces Stress} - Lowers cortisol levels and promotes relaxation.
		\item \textit{Enhances Emotional Well-Being} - Improves mood and emotional resilience.
		\item \textit{Improves Focus} - Increases attention span and cognitive function.
		\item \textit{Balances Blood Pressure} - Helps in maintaining healthy blood pressure levels.
		\item \textit{Promotes Inner Peace} - Fosters a sense of calm and tranquility.
		\item \textit{Aids in Self-Realization} - Encourages deeper self-awareness and understanding.
		\item \textit{Supports Mental Health} - Helps in managing anxiety, depression, and other mental health issues.
	  \end{itemize}

\end{frame}

