%%%%%%%%%%%%%%%%%%%%%%%%%%%%%%%%%%%%%%%%%%%%%%%%%%%%%%%%%%%%%%%%%%%%%%%%%%%%%%%%%%
\begin{frame}[fragile]\frametitle{}
\begin{center}
{\Large Introduction to Indian Calendar}
\end{center}
\end{frame}

%%%%%%%%%%%%%%%%%%%%%%%%%%%%%%%%%%%%%%%%%%%%%%%%%%%%%%%%%%%%%%%%%%%%%%%%%%%%%%%%%%
\begin{frame}[fragile]\frametitle{Introduction}
    
    \begin{itemize}
        \item \textbf{Lunar and Solar}: Indian calendars are primarily lunar or lunisolar, with adjustments based on solar movements.
        \item \textbf{Surya Siddhanta}: Ancient text providing principles for solar-based calendars.
        \item \textbf{Tithi and Nakshatra}: Days and lunar asterisms form the basis of Indian calendar calculations.
        \item \textbf{Regional Variations}: Different regions follow different calendar systems based on local traditions.
        \item \textbf{Luni-Solar Festivals}: Festivals like Diwali and Holi are determined based on lunar and solar positions.
    \end{itemize}
\end{frame}

%%%%%%%%%%%%%%%%%%%%%%%%%%%%%%%%%%%%%%%%%%%%%%%%%%%%%%%%%%%%%%%%%%%%%%%%%%%%%%%%%%
\begin{frame}[fragile]\frametitle{Introduction (Cont'd)}
    
    \begin{itemize}
        \item \textbf{Saka Era}: Official civil calendar of India, based on the Saka era starting in 78 CE.
        \item \textbf{Vikram Samvat}: Another prominent Hindu calendar, starting from 57 BCE.
        \item \textbf{Lunar Months}: Indian calendars consist of 12 lunar months, with intercalary months added periodically.
        \item \textbf{Jyotisha}: Astrological calculations play a crucial role in determining auspicious dates.
        \item \textbf{Modern Adaptations}: Some Indian calendars have been adapted to synchronize with the Gregorian calendar.
    \end{itemize}
\end{frame}
